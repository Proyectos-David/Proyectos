\section{Superestructura \texorpdfstring{$\Rh$}{R}}

El objetivo de la superestructura es poder agrupar todas las relaciones
definidas en un conjunto. En este caso $\R$. Debido a que las
$n$-uplas se definen recursivamente de la siguiente manera:
\begin{align*}
  (a) &= a\\
  (a,b) &= \{\{a\}, \{a,b\}\}\\
  (a_1, \dots, a_n, a_{n+1}) &= ((a_1, \dots, a_n), a_{n+1})
\end{align*}
Se puede demostrar que $(a,b) \in A\times B \To
(a,b) \in \Pts{\Pts{A \cup B}}$. Debido a esto, es posible definir un
conjunto el cual contenga todas las relaciones entre números reales. Se
define la colección $\R_n$ de la siguiente manera:
\[
  \begin{siseq}[\{,\}]
    \R_0 &= \R\\
    \R_{n+1} &= \Pts{\bigcup_{k=1}^n \R_k}
  \end{siseq}
\]

Se define entonces, la superestructura $\Rh$ por
\[\Rh = \bigcup_{n\in\N} R_n\]
Los elementos de $\Rh$ serán llamados \emph{entidades} y los elementos
de $\R$ serán llamados \emph{individuos}.

\begin{lemma}[Propiedades de $\R_n$]\label{lema:Rn}~
  \begin{enumerate}
    \item $\Forall{n,m}[n,m\in\N \land n\leq m]{\R_{n+1} \subseteq \R_{m+1}}$
    \item $\ds\Forall{n}[n\in\N]{\bigcup_{k=0}^n \R_k = \R_0 \cup \R_n}$
    \item $\Forall{n,m}[n,m\in\N \land n \leq m]{R_n \in \R_{m+1}}$
    \item $\Forall{n,x,y}[n\in\N \land y\in\R_{n+1} \land x\in y]{x\in\R_0\cup\R_n}$
  \end{enumerate}
\end{lemma}
\begin{demo}~
  \begin{enumerate}
    \item Sean $n,m\in\N$ tales que $n\leq m$.
          \begin{longderivation}<.9>
              \res{x\in\R_{n+1}}\\
            \equiv\\
              \res{x\subseteq \bigcup_{k=0}^n \R_k}\\
            \To\\
              \res{x\subseteq \bigcup_{k=0}^{m}\R_k}\\
            \equiv\\
              \res{x\in\R_{m+1}}
          \end{longderivation}
    \item Sea $n\in\N$, si $n=0$ no hay algo que demostrar. Para $n\in\J$:
          \begin{longderivation}<.9>
              \res{x\in\bigcup_{k=0}^n \R_k}\\
            \equiv\\
              \res{ x\in \R_0 \lor x\in\bigcup_{k=1}^n \R_k }\\
            \why{(i)}\\
              \res{ x\in\R_0 \lor x\in\R_n }\\
            \equiv\\
              \res{x\in\R_0\cup\R_n}
          \end{longderivation}
    \item Sean $n,m \in \N$ tales que $n \leq m$.
          \begin{longderivation}<.9>
              \res{ \R_n \subseteq \bigcup_{k=0}^m\R_k }\\
            \equiv\\
              \res{\R_n \in \Pts{\bigcup_{k=0}^m\R_k}}\\
            \equiv\\
              \res{ \R_n \in \R_{m+1} }
          \end{longderivation}
    \item Sean $n\in\N$, $y\in\R_{n+1}$ y $x\in y$.
          \begin{longderivation}<.9>
              \res{ x \in y \land y \in \R_{n+1} }\\
            \why{(i)}\\
              \res{ x \in y \land y \subseteq \R_0\cup\R_n }\\
            \To\\
              \res{ x \in \R_0\cup\R_n }
          \end{longderivation}
  \end{enumerate}
\end{demo}