\section{Ultraproducto de \texorpdfstring{$\Rh$}{R}}

Sean $I$ un conjunto infinito, $\U$ un ultrafiltro $\delta$-incompleto
sobre $I$, y $\{I_n\}_{n\in\N}$ una partición contable la cual cumple
la definición de $\delta$-incomleto en $\U$.

Considere ahora el conjunto $\IRh$, de las funciones de $I$ en $\Rh$.
Sea $a\in\Rh$, se define $\st{a}\in\IRh$ por $\st{a(i)} = a$. Esta
función, definida para todas las entidades, es una forma de incorporar
$\Rh$ en $\IRh$. Se puede definir una extensión de las relaciones `$=$'
y `$\in$' basadas en $\U$

\begin{definition}
  Sean $a,b\in\IRh$
  \begin{enumerate}
    \item $a\equ b \equiv \{i\,|\,a(i) = b(i)\} \in \U$
    \item $a\inu b \equiv \{i\,|\,a(i) \in b(i)\} \in \U$
  \end{enumerate}
\end{definition}

Sean $a,b\in\IRh$, estas relaciones se comportan de la misma forma
que su versión usual, esto es:
\begin{itemize}
  \item $a \equ b \not\equiv a \not\equ b$
  \item $a \inu b \not\equiv a \not\inu b$
\end{itemize}
\begin{demo}
  Sean $a,b\in\IRh$. Nótese que $\{i\,|\,a(i) = b(i)\} \cup 
  \{i\,|\,a(i) \not= b(i)\} = I$. Esto mismo sucede con `$\in$'.
\begin{center}
  \setlength{\tabcolsep}{20pt}
  \begin{tabular}{>{$}c<{$}| >{$}c<{$}}
    \begin{derivation}
        \res{ a \equ b }\\
      \equiv\\
        \res{ \{i\,|\,a(i) = b(i)\} \in \U }\\
      \why[\not\equiv]{$\U$ es un ultrafiltro}\\
        \res{ \{i\,|\,a(i) \not= b(i)\} \not\in \U }
    \end{derivation}
    &
    \begin{derivation}
        \res{ a \inu b }\\
      \equiv\\
        \res{ \{i\,|\,a(i) \in b(i)\} \in \U }\\
      \why[\not\equiv]{$\U$ es un ultrafiltro}\\
        \res{ \{i\,|\,a(i) \not\in b(i)\} \not\in \U }
    \end{derivation}
  \end{tabular}
\end{center}
\end{demo}

Antes de continuar, se presenta la siguiente convención de notación:
\begin{definition}
  Sean $a\in\IRh$ y $V$ un predicado en $\Rh$ con
  $a_1,\dots,a_n$ constantes, dichas constantes denotan elementos fijos
  de $\Rh$.
  \begin{itemize}
    \item $\{a\}(i) = \{a(i)\}$
    \item $\st{V} = V$ reemplazando cada $a_i$ por $\st{a_i}$.
  \end{itemize}
\end{definition}
Sean $a,b\in \IRh$, otra propiedad que se tiene, en este caso para
la igualdad bajo $\U$ es:
\[a \equ b \equiv \Forall{c}[c\in\IRh]{a\inu c \equiv b \inu c}\]
\begin{demo}
  La demostración se hará por doble implicación.
  Suponiendo que $a\equ b$. Sea $c\in\IRh$.
  \begin{center}
    \begin{tabular}{>{$}c<{$}| >{$}c<{$}}
      \begin{derivation}
          \res{ a \inu c \land a\equ b}\\
        \equiv\\
          \res{ \{i\,|\,a(i) \in c(i)\}\in\U \\
            & \land\\
            &\{i\,|\,a(i) = b(i)\}\in\U
          }\\
        \why[\To]{Definición de filtro}\\
          \res{ \{i\,|\,a(i) \in c(i) \land a(i) = b(i)\}\in\U }\\
        \why[\To]{Definición de filtro}\\
          \res{ \{i\,|\,b(i) \in c(i)\}\in\U }\\
        \equiv\\
          \res{ b \inu c }
      \end{derivation}
      &
      \begin{derivation}
        \res{ b \inu c \land a\equ b}\\
      \equiv\\
        \res{ \{i\,|\,b(i) \in c(i)\}\in\U\\
          & \land\\
          & \{i\,|\,a(i) = b(i)\}\in\U
        }\\
      \why[\To]{Definición de filtro}\\
        \res{ \{i\,|\,b(i) \in c(i) \land a(i) = b(i)\}\in\U }\\
      \why[\To]{Definición de filtro}\\
        \res{ \{i\,|\,a(i) \in c(i)\}\in\U }\\
      \equiv\\
        \res{ a \inu c }
      \end{derivation}
    \end{tabular}
  \end{center}

  Por el otro lado:
  \begin{longderivation}<.9>
      \res{ \Forall{c}[c\in\IRh]{a\inu c \equiv b \inu c} }\\
    \To\\
      \res{ a\inu \{a\} \equiv b \inu \{a\} }\\
    \equiv\\
      \res{
        \{i\,|\,a(i) \in \{a\}(i)\} \in\U
        \equiv
        \{i\,|\,b(i) \in \{a\}(i)\} \in\U
      }\\
    \equiv\\
      \res{
        \{i\,|\,a(i) = a(i)\} \in\U
        \equiv
        \{i\,|\,b(i) = a(i)\} \in\U
      }\\
    \equiv\\
      \res{ I\in\U \equiv b \equ a }\\
    \equiv\\
      \res{ b \equ a }
  \end{longderivation}
\end{demo}

Nótese que con estas propiedades, y el hecho de que `$\equ$' es una
relación de equivalencia, se puede demostrar de la misma forma que
es válida la ley del reemplazo en $\inu$. Esto es, en $\IRh$:
\[\Forall{a,b,c,d}[a\inu b \land a\equ c \land b \equ d]{c\inu d}\]

Se puede demostrar que las relaciones `$=$' y `$\in$' se comportan de
la misma manera que su correspondiente para los elementos sobre los que
aplica, esto es, si tomamos elementos de $\Rh$, compararlos con `$=$' o
`$\in$' resulta ser equivalente a comparar su versión $*$ con `$\equ$'
o `$\inu$' respectivamente, debido a esto se dejará la notación $\U$, y
se usará `$=$' y `$\in$' respectivamente. Esto facilitará la
continuidad en algunas demostraciones. De esta misma manera, sería
posible ver que la validez de una proposición en $\Rh$, se mantiene en
$\IRh$. Sin embargo, la demostración se hará de una forma más general,
para evitar el proceso cada que sea necesario validar una proposición
en $\IRh$. Para esto se considerará el lenguaje formal de la lógica de
primer orden y los elementos de $\Rh$. Considerando que relaciones
básicas en $\Rh$ son `$=$' y `$\in$', ya se cuenta con simbolos de
relación correspondientes en $\IRh$. Por lo que se puede generar un
lenguage formal con $\IRh$, cuyas proposiciones dependen de $\U$.
Primero se mostrarán propiedades de la incorporación de $\Rh$ en $\IRh$.

\begin{lemma}~
  \begin{enumerate}
    \item $\st{\varnothing} = \varnothing$
    \item $\Forall{a,b}[a,b\in\Rh]{a \subseteq b \To \st{a} \subseteq \st{b}}$
    \item $\Forall{a,b}[a,b\in\Rh]{a\in b \equiv \st{a} \in \st{b}}$
    \item Sean $n\in\J$ y $\{a_k\}$ una colección de $n$ elementos de
          $\Rh$. Entonces:
          \begin{itemize}
            \item $\ds\st{\left(\,\bigcup_{i=1}^n a_i\right)}=\bigcup_{i=1}^n \st{a_i}$
            \item $\ds\st{\left(\,\bigcap_{i=1}^n a_i\right)}=\bigcap_{i=1}^n \st{a_i}$
            \item $\st{\{a_1,\dots,a_n\}} = \{\st{a_1},\dots,\st{a_n}\}$
            \item $\st{(a_1,\dots,a_n)} = (\st{a_1},\dots,\st{a_n})$
            \item $\st{(a_1 \times \dots \times a_n)} = 
                    \st{a_1} \times \dots \times \st{a_n}$
          \end{itemize}
    \item $\Forall{a,b}[a,b\in\Rh]{\st{(a - b)} = \st{a} - \st{b}}$
    \item Sea $b\in\Rh$ una relación binaria.
          \begin{itemize}
            \item $\st{(\dom b)} = \dom \st{b}$
            \item $\st{(\ran b)} = \ran \st{b}$
          \end{itemize}
  \end{enumerate}
\end{lemma}
\pagebreak
\begin{demo}~
  \begin{enumerate}
    \item~
          \begin{longderivation}<.9>
              \res{ x\in\st{\varnothing} }\\
            \equiv\\
              \res{ \{i\,|\,x(i) \in \varnothing\} \in\U }\\
            \equiv\\
              \res{ \varnothing \in \U }\\
            \equiv\\
              \res{ x\in\varnothing }
          \end{longderivation}
    \item Sean $a,b\in\Rh$ tales que $a\subseteq b$. Sea $x\in\IRh$
          \begin{longderivation}<.9>
              \res{ x\in\st{a} }\\
            \equiv\\
              \res{ \{i\,|\,x(i) \in a\} \in\U}\\
            \why[\To]{Definiciónd de filtro, $a\subseteq b$}\\
              \res{ \{i\,|\,x(i) \in b\}\in\U }\\
            \equiv\\
              \res{ x \in \st{b} }
          \end{longderivation}
    \item Sean $a,b\in\Rh$
          \begin{longderivation}<.9>
              \res{ \st{a} \in \st{b} }\\
            \equiv\\
              \res{ \{i\,|\,a \in b\} \in \U }
          \end{longderivation}
          Por un lado, suponiendo que $a\in b$, se tendría que $I\in\U$,
          cosa que es verdadera. Por otro lado, suponiendo que $\st{a} \in \st{b}$,
          se tendría que, efectivamente $\{i\,|\,a \in b\}\in\U$, recordando que,
          para un $c\in\Rh$, se definió, que para todo $i\in I$, $\st{c}(i) = i$.
          La expresión que define $\st{a} \in \st{b}$, nuevamente conduce a $I\in\U$.
    \item Sean $n\in\J$, $\{a_k\}$ una colección con $n$ elementos de
          $\Rh$ y $x, (x_1,\dots,x_n)\in\IRh$.
          \begin{itemize}
            \item Para la intersección
            
            \noindent\makebox[14.5cm]{
                  \begin{tabular}{>{$}c<{$} | >{$}c<{$}}
                    \begin{derivation}
                        \res{ x \in \bigcap_{k=1}^n \st{a_k} }\\
                      \equiv\\
                        \res{ \Forall{k}[1\leq k\leq n]{x \in \st{a_k}} }\\
                      \equiv\\
                        \res{
                          \Forall{k}[1\leq k\leq n]{
                            \set{i}{x(i) \in a_k}\in\U
                          }
                        }\\
                      \why[\To]{Definción de filtro}\\
                        \res{
                          \set{i}{\Forall{k}[1\leq k\leq n]{x(i)\in a_k}}
                        }
                    \end{derivation}
                    &
                    \begin{derivation}
                        \res{ x \in\st{\left(\,\bigcap_{k=1}^n a_k\right)} }\\
                      \equiv\\
                        \res{
                          \set{i}{x(i) \in \bigcap_{k=1}^n a_k} \in \U
                        }\\
                      \equiv\\
                        \res{
                          \set{i}{
                            \Forall{k}[1\leq k\leq n]{x(i)\in a_k}
                          }\in\U
                        }\\
                      \why[\To]{Definición de filtro}\\
                        \res{
                          \Forall{k}[1\leq k\leq n]{
                            \set{i}{x(i) \in a_k}\in\U
                          }
                        }
                    \end{derivation}
                  \end{tabular}
                }
                \vspace{10pt}
            \item Para la unión.

                  Primero se mostrará que
                  $\ds\bigcup_{k=1}^n\st{a_k}\subseteq
                  \st{\left(\bigcup_{k=1}^n a_k\right)}$

                  \begin{longderivation}
                      \res{ x\in\bigcup_{k=1}^n\st{a_k} }\\
                    \equiv\\
                      \res{ \Exists{k}[1\leq k\leq 1]{x\in\st{a_k}} }\\
                    \equiv\\
                      \res{ \Exists{k}[1\leq k\leq n]{
                        \set{i}{x(i) \in a_k}\in\U
                      }}\\
                    \why[\To]{Definición de filtro}\\
                      \res{ \set{i}{
                        \Exists{k}[1\leq k\leq n]{x(i) \in a_k}\in\U
                      } }\\
                    \equiv\\
                      \res{x\in \st{\left(\bigcup_{k=1}^n a_k\right)}}
                  \end{longderivation}

                  Ahora se mostrará que
                  $\ds\st{\left(\bigcup_{k=1}^n a_k\right)}-
                  \bigcup_{k=1}^n\st{a_k}=\varnothing$
                  \begin{longderivation}
                      \res{
                        x\in\st{\left(\bigcup_{k=1}^n a_k\right)}
                        - \bigcup_{k=1}^n\st{a_k}
                      }\\
                    \equiv\\
                      \res{
                        \set{i}{x(i)\in\bigcup_{k=1}^n a_k}\in\U
                        \land
                        \lnot\Exists{k}[1\leq k\leq n]{\set{i}{x(i)\in a_k}\in\U}
                      }\\
                    \equiv\\
                      \res{
                        \set{i}{x(i)\in\bigcup_{k=1}^n a_k}\in\U
                        \land
                        \Forall{k}[1\leq k\leq n]{\set{i}{x(i)\in a_k}\not\in\U}
                      }\\
                    \why{$\U$ es un ultrafiltro}\\
                      \res{
                        \set{i}{x(i)\in\bigcup_{k=1}^n a_k}\in\U
                        \land
                        \Forall{k}[1\leq k\leq n]{\set{i}{x(i)\not\in a_k}\in\U}
                      }\\
                    \why[\To]{definición de filtro}\\
                      \res{
                        \set{i}{x(i)\in\bigcup_{k=1}^n a_k}\in\U
                        \land
                        \set{i}{\Forall{k}[1\leq k\leq n]{x(i)\not\in a_k}}\in\U
                      }\\
                    \why[\To]{definición de filtro}\\
                      \res{
                        \set{i}{
                          x(i)\in\bigcup_{k=1}^n a_k
                          \land
                          \Forall{k}[1\leq k\leq n]{x(i)\not\in a_k}
                        }\in\U
                      }\\
                    \equiv\\
                      \res{ \varnothing \in\U }\\
                    \equiv\\
                      \res{ x\in\varnothing }
                  \end{longderivation}
                  
            \item Para la colección, el argumento se realiza de la
                  misma forma que para la unión, lo único que cambia es
                  `$\in$' por `$=$'.
            \item Para la $n$-upla, por cómo se define, se puede ver que es
                  la unión de dos conjuntos, lo que significa, que es
                  consecuencia del punto de la unión.
            \item Para el producto cartesiano.

                  Por un lado:
                  \begin{longderivation}
                      \res{(x_1,\dots, x_n)\in\st{(a_1\times\dots\times a_n)}}\\
                    \equiv\\
                      \res{
                        \set{i}{(x_1(i),\dots,x_n(i))\in a_1\times\dots\times a_n}\in\U
                      }\\
                    \equiv\\
                      \res{
                        \set{i}{
                          \Forall{k}[1\leq k\leq n]{x_k(i) \in a_k}
                        }\in\U
                      }\\
                    \why[\To]{Definición de filtro (iii)}\\
                      \res{
                        \Forall{k}[1\leq k\leq n]{
                          \set{i}{x_k(i)\in a_k} \in\U
                        }
                      }\\
                    \equiv\\
                      \res{ (x_1,\dots,x_n) \in \st{a_1}\times\dots\times\st{a_n} }
                  \end{longderivation}
                  Por el otro:
                  \begin{longderivation}
                      \res{ (x_1,\dots,x_n) \in \st{a_1}\times\dots\times\st{a_n} }\\
                    \equiv\\
                      \res{
                        \Forall{k}[1\leq k\leq n]{
                          \set{i}{x_k(i)\in a_k} \in\U
                        }
                      }\\
                    \why[\To]{Definición de filtro (ii)}\\
                      \res{
                        \set{i}{
                          \Forall{k}[1\leq k\leq n]{x_k(i) \in a_k}
                        }\in\U
                      }\\
                    \equiv\\
                      \res{
                        \set{i}{(x_1(i),\dots,x_n(i))\in a_1\times\dots\times a_n}\in\U
                      }\\
                    \equiv\\
                      \res{(x_1,\dots, x_n)\in\st{(a_1\times\dots\times a_n)}}\\
                  \end{longderivation}
          \end{itemize}

    \item Por doble contenencia:
          \begin{center}
            \begin{tabular}{>{$}c<{$} | >{$}c<{$}}
              \begin{derivation}
                  \res{ x \in \st{(a-b)} }\\
                \equiv\\
                  \res{ \set{i}{x(i) \in a \land x(i) \not\in b}\in\U }\\
                \why[\To]{Definición de filtro}\\
                  \res{ \set{i}{x(i)\in a}\in\U \land \set{i}{x(i)\not\in b}\in\U }\\
                \why{$\U$ es un ultrafiltro}\\
                  \res{ \set{i}{x(i)\in a}\in\U \land \set{i}{x(i)\in b}\not\in\U }\\
                \equiv\\
                  \res{ x \in \st{a} - \st{b} }
              \end{derivation}
              &
              \begin{derivation}
                  \res{ x \in \st{a} - \st{b} }\\
                \equiv\\
                  \res{ \set{i}{x(i) \not\in a}\in\U \land \set{i}{x(i)\in b}\not\in\U }\\
                \why[\To]{$\U$ es un ultrafiltro}\\
                  \res{ \set{i}{x(i)\in a}\in\U \land \set{i}{x(i)\not\in b}\in\U  }\\
                \why[\To]{Definición de filtro}\\
                  \res{ \set{i}{x(i) \in a \land x(i) \not\in b}\in\U }\\
                \equiv\\
                  \res{ x \in \st{(a-b)} }
              \end{derivation}
            \end{tabular}
          \end{center}
    \item~
          \begin{itemize}
            \item Se va a seguir la misma estrategia que se usó en (iv)
                  para la unión. Se mostrará que $\dom \st{b} \subseteq \st{(\dom b)}$
                  \begin{longderivation}
                      \res{ x\in\dom\st{b} }\\
                    \equiv\\
                      \res{ \Exists{y}{(x,y)\in\dom\st{b}} }\\
                    \equiv\\
                      \res{ \Exists{y}{
                        \set{i}{(x(i),y(i))\in b}\in\U
                      } }\\
                    \why[\To]{Definición de filtro (iii)}\\
                      \res{
                        \set{i}{
                          \Exists{y}{(x(i),y(i))\in b}
                        }\in\U
                      }\\
                    \equiv\\
                      \res{ x\in\st{(\dom b)} }
                  \end{longderivation}

                  Ahora se mostrará que $\st{(\dom b)} - \dom \st{b} = \varnothing$
                  \begin{longderivation}
                      \res{ x\in \st{(\dom b)} - \dom \st{b} }\\
                    \equiv\\
                      \res{
                        \set{i}{x(i) \in \dom b}\in\U
                        \land
                        \lnot\Exists{y}{\set{i}{(x(i),y(i))\in b}\in\U}
                      }\\
                    \equiv\\
                      \res{
                        \set{i}{x(i) \in \dom b}\in\U
                        \land
                        \Forall{y}{\set{i}{(x(i),y(i))\in b}\not\in\U}
                      }\\
                    \why{$\U$ es un ultrafiltro}\\
                      \res{
                        \set{i}{x(i) \in \dom b}\in\U
                        \land
                        \Forall{y}{\set{i}{(x(i),y(i))\not\in b}\in\U}
                      }\\
                    \why[\To]{Definición de filtro (ii)}\\
                      \res{
                        \set{i}{x(i) \in \dom b}\in\U
                        \land
                        \set{i}{\Forall{y}{(x(i),y(i))\not\in b}}\in\U
                      }\\
                    \why[\To]{Definición de filtro (ii)}\\
                      \res{
                        \set{i}{
                          x(i)\in\dom b \land x(i) \not\in\dom b
                        }\in\U
                      }\\
                    \equiv\\
                      \res{ \varnothing\in\U }\\
                    \equiv\\
                      \res{ x \in \varnothing }
                  \end{longderivation}
            \item Para el rango, la misma estrategia se puede usar.
          \end{itemize}
  \end{enumerate}
\end{demo}