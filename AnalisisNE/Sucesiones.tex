\section{Sucesiones}

Una sucesión en $\R$ es una función de $\N$ en $\R$. Debido a esto, es
un subconjunto de $\N\times\R$, con lo que es una entidad de $\Rh$.
Una sucesión $\{s_n\}$ en $\R$ se extiende a $\{\st{s_n}\}$ como una función de 
$\st{\N}$ en $\st{\R}$. Esto último se debe al teorema \ref{theo:FT} y
al lema \ref{lema:stR}. Nótese que, para todo $n\in\N$, $\st{s_n} = s_n$,
nuevamente por el lema \ref{lema:stR}, ya que $\st{(\ran s_n)} = \ran \st{s_n}$.
Debido a esto, se dejará la notación $(\st{})$ para hacer referencia a
individuos.

\begin{theorem}
  Sea $\{s_n\}$ una sucesión en $\R$. $\{s_n\}$ ese acotada si y solo si, para todo
  número natural infinito $\omega\in\st{\N}-\N$, $\st{s_\omega}$ es
  finito.
\end{theorem}

\begin{demo}
  Por un lado
  \begin{longderivation}<.9>
      \wff{\Forall{n}[n\in\st{\N}]{\Exists{a}[a\in M_0]{|\st{s_n}| \leq a}}}\\
    \why[\To]{Aplicando $\sd{}$}\\
      \wff{\Forall{n}[n\in\N]{\Exists{a}[a\in M_0]{|s_n| \leq \sd{a}}}}\\
    \equiv\\
      \wff{\{s\} \text{ es acotada en } \R}
  \end{longderivation}
  Por el otro, si $\{s_n\}$ es acotada en $\R$, el rango de $s$ tiene una
  mínima cota superior y una máxima cota inferior. Por el teorema
  \ref{theo:FT}, esto se traduce a que el rango de $\{\st{s_n}\}$ sea acotado
  en $\st{\R}$, con lo que, para todo $n\in\st{\N}$, $\st{s_n}$ es finito.
\end{demo}

Recordando la definición usual para decir que una sucesión $\{s_n\}$
converge a $s$ en $\R$:
\[\Forall{\e}[\e>0]{\Exists{N}[N\in\N]{\Forall{n}[n\in\N \land n\geq N]{|s_n - s| < \e}}}\]
Su análogo en el análisis no estándar es:
\begin{theorem}
  Sean $\{s_n\}$ una sucesión en $\R$ y $s\in\R$. entonces 
  $\lim{n\to\infty} s_n = s$ si y solo si, para todo $\omega\in\st{\N} - \N$,
  $s_\omega =_1 s$.
\end{theorem}

\begin{demo}
  Por un lado, sea $\e>0$
  \begin{longderivation}<1>
      \wff{\Exists{N}[N\in\N]{\Forall{n}[n\in\N \land n\geq N]{|s_n - s| < \e}}}\\
    \equiv\\
      \wff{\Exists{N}[N\in\st{\N}]{\Forall{n}[n\in\st{\N} \land n\geq N]{|\st{s_n} - s| < \e}}}\\
    \why{Definición de infinitesimal}\\
      \wff{\Exists{N}[N\in\st{\N}]{\Forall{n}[n\in\st{\N} \land n\geq N]{\st{s_n} =_1 s}}}\\
    \To\\
      \wff{\Forall{\omega}[\omega\in\st{\N} - \N]{\st{s_\omega} =_1 s}}
  \end{longderivation}
  Por el otro lado:
  \begin{longderivation}<1>
      \wff{\Forall{n}[n\in\st{\N} - \N]{\st{s_n} =_1 s}}\\
    \To\\
      \wff{\Exists{N}[N\in\st{\N}]{\Forall{n}[n\in\st{\N} \land n \geq N]{\st{s_n} =_1 s}}}\\
    \equiv\\
      \wff{
        \Forall{\e}[\e>0]{
          \Exists{N}[N\in\st{N}]{
            \Forall{n}[n\in\st{\N} \land n \geq N]{|\st{s_n} - s| < \e}
          }
        }
      }\\
    \equiv\\
      \wff{
        \Forall{\e}[\e>0]{
          \Exists{N}[N\in\N]{
            \Forall{n}[n\in\N \land n\leq N]{
              |s_n - s| < \e
            }
          }
        }
      }
  \end{longderivation}
\end{demo}

El teorema se puede reescribir de forma equivalente, usando el
homomorfismo de parte estándar:
\begin{longderivation}<.8>
    \wff{\Forall{\omega}[\omega\in\st{\N}-\N]{\st{s_\omega}=_1 s}}\\
  \equiv\\
    \wff{\Forall{\omega}[\omega\in\st{\N}-\N]{\sd{\st{s_\omega}}= s}}\\
\end{longderivation}

Por ejemplo, la sucesión $\{\sqrt[n]{n}\}_{n\in\J}$. Para ver que esta sucesión
converge a $1$, se considera $s_n = \sqrt[n]{n} - 1$.
\begin{longderivation}<.8>
    \wff{s_n = \sqrt[n]{n} - 1}\\
  \equiv\\
    \wff{n = \left(s_n + 1\right)^n}\\
  \equiv\\
    \wff{n = \sum_{k=0}^n \binom{n}{k} s_n^k}\\
  \To\\
    \wff{n \geq \binom{n}{2}s_n^2}\\
  \equiv\\
    \wff{n \geq \frac{n!}{(n-2)!\,2!}s_n^2}\\
  \equiv\\
    \wff{1 \geq \frac{n-1}{2}s_n^2}\\
  \equiv\\
    \wff{\sqrt{\frac{2}{n-1}} \geq s_n}
\end{longderivation}

Se tiene entonces, que para todo $n\in\N$, $0\leq s_n \leq \sqrt{\frac{2}{n-1}}$.
De igual forma, para todo $n\in\st{\N}$, $0\leq \st{s_n} \leq \sqrt{\frac{2}{n-1}}$.
Lo que implica, que para todo $\omega\in\st{\N}-\N$
\begin{longderivation}
    \wff{0\leq\st{s_\omega}\leq\sqrt{\frac{2}{n-1}}}\\
  \why[\To]{Los recíprocos de infinitos son infinitesimales, aplicando $\sd{}$}\\
    \wff{0\leq\sd{\st{s_\omega}}\leq 0}\\
  \equiv\\
    \wff{\sd{\st{s_\omega}} = 0}
\end{longderivation}

\begin{theorem}[Sucesiones de Cauchy]
  Sea $\{s_n\}$ una sucesión. $\{s_n\}$ es una sucesión de Cauchy, si y
  solo si, para todo $\omega,\gamma\in\st{\N}-\N$, $\st{s_\omega =_1 \st{s_\gamma}}$
\end{theorem}

La demostración de este teorema es exáctamente igual que el teorema
anterior.

Por lo mencionado al principio de esta sección, pareciera que el uso de
esta teoría para sucesiones en espacios métricos arbitrarios no tendría
sentido, pues la sucesión podría no hacer parte de $\Rh$. Sin embargo,
solo hace falta ver que todas las definiciones y teoremas que estudian
el comportamiento de las sucesiones en otros espacios métricos, reducen
el problema a nuevas sucesiones las cuales efectivamente, son sucesiones
reales. Por ejemplo, sea $\{p_n\}$ una sucesión en un espacio métrico $X$
el cual tiene como distancia la función $d_X: (X\times X) \to \R^{\geq 0}$.
Se dice que $\{p_n\}$ converge a $p\in X$ cuando
\[\Forall{\e}[\e>0]{\Exists{N}[N\in\N]{\Forall{n}[n\geq N]{d_X(p_n, p)<\e}}}\]
Defínase la sucesión $\{q_n\}$ de $\N$ a $\R$ por:
\[q_n = d_X(p_n,p)\]
La convergencia de $\{p_n\}$ se puede escribir en términos de la
convergencia de $\{q_n\}$ a $0$ de la siguiente manera:
\[\Forall{\e}[\e>0]{\Exists{N}[N\in\N]{\Forall{n}[n\geq N]{|q_n - 0| <\e}}}\]
Como la sucesión $\{q_n\}$ es real, aplican los teoremas anteriores,
con lo que, la convergencia de $\{p_n\}$ se puede expresar en el análisis
no estándar como:
\[\Forall{\omega}[\omega\in\st{\N}-\N]{d_X(p_\omega,p)=_1 0}\]

De la misma forma, la sucesión $\{p_n\}$, es de Cauchy cuando
\[\Forall{\e}[\e>0]{\Exists{N}[N\in\N]{\Forall{n,m}[n,m\geq N]{d_X(p_n,p_m)<\e}}}\]
En este caso, para pasar a una sucesión real, no se tiene como tal una
sucesión, pues su dominio no sería $\N$, sino $\N\times\N$. Esto, sin
embargo, no complica el problema. Defínase $\{s_{n,m}\}$ por
\[s_{n,m} = d_X(p_n,p_m)\]
La proposición se puede reescribir como
\[\Forall{\e}[\e>0]{\Exists{N}[N\in\N]{\Forall{n,m}[n,m\geq N]{s_{n,m}<\e}}}\]
De la misma forma que se realizó en la demostración de los teoremas
para sucesiones reales, se puede obtener un equivalente en el
análisis no estándar:
\[\Forall{\omega,\gamma}[\omega,\gamma\in\st{\N}-\N]{d_X(p_\omega,p_\gamma)=_1 0}\]
