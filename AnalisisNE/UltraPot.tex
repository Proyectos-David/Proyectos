\section{Ultrapotencia de \texorpdfstring{$\Rh$}{R}}

En esta sección se definirá una estructura la cual es una
ultrapotencia de $\Rh$.

Sean $I$ un conjunto infinito, $\U$ un ultrafiltro $\delta$-incompleto
sobre $I$, y $\{I_n\}_{n\in\N}$ una partición contable la cual cumple
la definición de $\delta$-incomleto en $\U$.

Considere ahora el conjunto $\IRh$, de las funciones de $I$ en $\Rh$.
Sea $a\in\Rh$, se define $\st{a}\in\IRh$ por $\st{a(i)} = a$. Esta
función, definida para todas las entidades, es una forma de incorporar
$\Rh$ en $\IRh$. Se puede definir una extensión de las relaciones `$=$'
y `$\in$' basadas en $\U$

\begin{definition}
  Sean $a,b\in\IRh$
  \begin{enumerate}
    \item $a\equ b \equiv \{i\,|\,a(i) = b(i)\} \in \U$
    \item $a\inu b \equiv \{i\,|\,a(i) \in b(i)\} \in \U$
  \end{enumerate}
\end{definition}

Sean $a,b\in\IRh$, estas relaciones se comportan de la misma forma
que su versión usual, esto es:
\begin{itemize}
  \item $a \equ b \not\equiv a \not\equ b$
  \item $a \inu b \not\equiv a \not\inu b$
\end{itemize}
\begin{demo}
  Sean $a,b\in\IRh$. Nótese que $\{i\,|\,a(i) = b(i)\} \cup 
  \{i\,|\,a(i) \not= b(i)\} = I$. Esto mismo sucede con `$\in$'.
\begin{center}
  \setlength{\tabcolsep}{20pt}
  \begin{tabular}{>{$}c<{$}| >{$}c<{$}}
    \begin{derivation}
        \wff{ a \equ b }\\
      \equiv\\
        \wff{ \{i\,|\,a(i) = b(i)\} \in \U }\\
      \why[\not\equiv]{$\U$ es un ultrafiltro}\\
        \wff{ \{i\,|\,a(i) \not= b(i)\} \not\in \U }
    \end{derivation}
    &
    \begin{derivation}
        \wff{ a \inu b }\\
      \equiv\\
        \wff{ \{i\,|\,a(i) \in b(i)\} \in \U }\\
      \why[\not\equiv]{$\U$ es un ultrafiltro}\\
        \wff{ \{i\,|\,a(i) \not\in b(i)\} \not\in \U }
    \end{derivation}
  \end{tabular}
\end{center}
\end{demo}

Antes de continuar, se aclarará algo de la notación. Sean $a\in\IRh$,
$(b_1,\dots,b_m)$ una $m$-upla de elementos de $\IRh$ y $V$ un predicado
en $\Rh$ con $c_1,\dots,c_n$ constantes las cuales denotan elementos
fijos de $\Rh$.
\begin{itemize}
  \item $\{a\}(i) = \{a(i)\}$
  \item $(b_1,\dots,b_m)(i) = (b_1(i),\dots,b_m(i))$
  \item $\st{V} = V$ reemplazando cada $c_i$ por $\st{c_i}$.
\end{itemize}
Sean $a,b\in \IRh$, otra propiedad que se tiene, en este caso para
la igualdad bajo $\U$ es:
\[a \equ b \equiv \Forall{c}[c\in\IRh]{a\inu c \equiv b \inu c}\]
\begin{demo}
  La demostración se hará por doble implicación.
  Suponiendo que $a\equ b$. Sea $c\in\IRh$.
  \begin{center}
    \begin{tabular}{>{$}c<{$}| >{$}c<{$}}
      \begin{derivation}
          \wff{ a \inu c \land a\equ b}\\
        \equiv\\
          \wff{ \{i\,|\,a(i) \in c(i)\}\in\U \\
            & \land\\
            &\{i\,|\,a(i) = b(i)\}\in\U
          }\\
        \why[\To]{Definición de filtro}\\
          \wff{ \{i\,|\,a(i) \in c(i) \land a(i) = b(i)\}\in\U }\\
        \why[\To]{Definición de filtro}\\
          \wff{ \{i\,|\,b(i) \in c(i)\}\in\U }\\
        \equiv\\
          \wff{ b \inu c }
      \end{derivation}
      &
      \begin{derivation}
        \wff{ b \inu c \land a\equ b}\\
      \equiv\\
        \wff{ \{i\,|\,b(i) \in c(i)\}\in\U\\
          & \land\\
          & \{i\,|\,a(i) = b(i)\}\in\U
        }\\
      \why[\To]{Definición de filtro}\\
        \wff{ \{i\,|\,b(i) \in c(i) \land a(i) = b(i)\}\in\U }\\
      \why[\To]{Definición de filtro}\\
        \wff{ \{i\,|\,a(i) \in c(i)\}\in\U }\\
      \equiv\\
        \wff{ a \inu c }
      \end{derivation}
    \end{tabular}
  \end{center}

  Por el otro lado:
  \begin{longderivation}<.9>
      \wff{ \Forall{c}[c\in\IRh]{a\inu c \equiv b \inu c} }\\
    \To\\
      \wff{ a\inu \{a\} \equiv b \inu \{a\} }\\
    \equiv\\
      \wff{
        \{i\,|\,a(i) \in \{a\}(i)\} \in\U
        \equiv
        \{i\,|\,b(i) \in \{a\}(i)\} \in\U
      }\\
    \equiv\\
      \wff{
        \{i\,|\,a(i) = a(i)\} \in\U
        \equiv
        \{i\,|\,b(i) = a(i)\} \in\U
      }\\
    \equiv\\
      \wff{ I\in\U \equiv b \equ a }\\
    \equiv\\
      \wff{ b \equ a }
  \end{longderivation}
\end{demo}

Nótese que con estas propiedades, y el hecho de que `$\equ$' es una
relación de equivalencia, se puede demostrar de la misma forma que
es válida la ley del reemplazo en $\inu$. Esto es, en $\IRh$:
\[\Forall{a,b,c,d}[a\inu b \land a\equ c \land b \equ d]{c\inu d}\]

Se puede demostrar que las relaciones `$=$' y `$\in$' se comportan de
la misma manera que su correspondiente para los elementos sobre los que
aplica, esto es, si tomamos elementos de $\Rh$, compararlos con `$=$' o
`$\in$' resulta ser equivalente a comparar su versión $*$ con `$\equ$'
o `$\inu$' respectivamente, debido a esto se dejará la notación $\U$, y
se usará solamente `$=$' y `$\in$'. Esto facilitará la
continuidad en algunas demostraciones.

De la misma manera en que se demostró la ley del reemplazo para
la extensión de `$\in$', sería posible ver que la validez de una
proposición en $\Rh$, se mantiene en $\IRh$. Sin embargo, la
demostración se hará de una forma más general, para evitar el proceso
cada que sea necesario validar una proposición en $\IRh$. Para esto se
considerará el lenguaje formal de la lógica de primer orden y los
elementos de $\Rh$. Considerando que relaciones
básicas en $\Rh$ son `$=$' y `$\in$', ya se cuenta con símbolos de
relación correspondientes en $\IRh$. Por lo que se puede generar un
lenguaje formal con $\IRh$, cuyas proposiciones dependen de $\U$.
Primero se mostrarán propiedades de la incorporación de $\Rh$ en $\IRh$.

\begin{lemma}\label{lema:stR}~
  \begin{enumerate}
    \item $\st{\varnothing} = \varnothing$
    \item $\Forall{a,b}[a,b\in\Rh]{a \subseteq b \To \st{a} \subseteq \st{b}}$
    \item $\Forall{a,b}[a,b\in\Rh]{a\in b \equiv \st{a} \in \st{b}}$
    \item Sean $n\in\J$ y $\{a_k\}$ una colección de $n$ elementos de
          $\Rh$. Entonces:
          \begin{itemize}
            \item $\st{\left(\,\bigcup_{i=1}^n a_i\right)}=\bigcup_{i=1}^n \st{a_i}$
            \item $\st{\left(\,\bigcap_{i=1}^n a_i\right)}=\bigcap_{i=1}^n \st{a_i}$
            \item $\st{\{a_1,\dots,a_n\}} = \{\st{a_1},\dots,\st{a_n}\}$
            \item $\st{(a_1,\dots,a_n)} = (\st{a_1},\dots,\st{a_n})$
            \item $\st{(a_1 \times \dots \times a_n)} = 
                    \st{a_1} \times \dots \times \st{a_n}$
          \end{itemize}
    \item $\Forall{a,b}[a,b\in\Rh]{\st{(a - b)} = \st{a} - \st{b}}$
    \item Sea $b\in\Rh$ una relación binaria.
          \begin{itemize}
            \item $\st{(\dom b)} = \dom \st{b}$
            \item $\st{(\ran b)} = \ran \st{b}$
          \end{itemize}
  \end{enumerate}
\end{lemma}
\pagebreak
\begin{demo}~
  \begin{enumerate}
    \item~
          \begin{longderivation}<.9>
              \wff{ x\in\st{\varnothing} }\\
            \equiv\\
              \wff{ \{i\,|\,x(i) \in \varnothing\} \in\U }\\
            \equiv\\
              \wff{ \varnothing \in \U }\\
            \equiv\\
              \wff{ x\in\varnothing }
          \end{longderivation}
    \item Sean $a,b\in\Rh$ tales que $a\subseteq b$. Sea $x\in\IRh$
          \begin{longderivation}<.9>
              \wff{ x\in\st{a} }\\
            \equiv\\
              \wff{ \{i\,|\,x(i) \in a\} \in\U}\\
            \why[\To]{Definiciónd de filtro, $a\subseteq b$}\\
              \wff{ \{i\,|\,x(i) \in b\}\in\U }\\
            \equiv\\
              \wff{ x \in \st{b} }
          \end{longderivation}
    \item Sean $a,b\in\Rh$
          \begin{longderivation}<.9>
              \wff{ \st{a} \in \st{b} }\\
            \equiv\\
              \wff{ \{i\,|\,a \in b\} \in \U }
          \end{longderivation}
          Por un lado, suponiendo que $a\in b$, se tendría que $I\in\U$,
          cosa que es verdadera. Por otro lado, suponiendo que $\st{a} \in \st{b}$,
          se tendría que, efectivamente $\{i\,|\,a \in b\}\in\U$, recordando que,
          para un $c\in\Rh$, se definió, que para todo $i\in I$, $\st{c}(i) = i$.
          La expresión que define $\st{a} \in \st{b}$, nuevamente conduce a $I\in\U$.
    \item Sean $n\in\J$, $\{a_k\}$ una colección con $n$ elementos de
          $\Rh$ y $x, (x_1,\dots,x_n)\in\IRh$.
          \begin{itemize}
            \item Para la intersección
            
            \noindent\makebox[14.5cm]{
                  \begin{tabular}{>{$}c<{$} | >{$}c<{$}}
                    \begin{derivation}
                        \wff{ x \in \bigcap_{k=1}^n \st{a_k} }\\
                      \equiv\\
                        \wff{ \Forall{k}[1\leq k\leq n]{x \in \st{a_k}} }\\
                      \equiv\\
                        \wff{
                          \Forall{k}[1\leq k\leq n]{
                            \set{i}{x(i) \in a_k}\in\U
                          }
                        }\\
                      \why[\To]{Definción de filtro (ii)}\\
                        \wff{
                          \set{i}{\Forall{k}[1\leq k\leq n]{x(i)\in a_k}}
                        }
                    \end{derivation}
                    &
                    \begin{derivation}
                        \wff{ x \in\st{\left(\,\bigcap_{k=1}^n a_k\right)} }\\
                      \equiv\\
                        \wff{
                          \set{i}{x(i) \in \bigcap_{k=1}^n a_k} \in \U
                        }\\
                      \equiv\\
                        \wff{
                          \set{i}{
                            \Forall{k}[1\leq k\leq n]{x(i)\in a_k}
                          }\in\U
                        }\\
                      \why[\To]{Definición de filtro (iii)}\\
                        \wff{
                          \Forall{k}[1\leq k\leq n]{
                            \set{i}{x(i) \in a_k}\in\U
                          }
                        }
                    \end{derivation}
                  \end{tabular}
                }
                \vspace{10pt}
            \item Para la unión.

                  Primero se mostrará que
                  $\bigcup_{k=1}^n\st{a_k}\subseteq
                  \st{\left(\bigcup_{k=1}^n a_k\right)}$

                  \begin{longderivation}
                      \wff{ x\in\bigcup_{k=1}^n\st{a_k} }\\
                    \equiv\\
                      \wff{ \Exists{k}[1\leq k\leq 1]{x\in\st{a_k}} }\\
                    \equiv\\
                      \wff{ \Exists{k}[1\leq k\leq n]{
                        \set{i}{x(i) \in a_k}\in\U
                      }}\\
                    \why[\To]{Definición de filtro (iii)}\\
                      \wff{ \set{i}{
                        \Exists{k}[1\leq k\leq n]{x(i) \in a_k}\in\U
                      } }\\
                    \equiv\\
                      \wff{x\in \st{\left(\bigcup_{k=1}^n a_k\right)}}
                  \end{longderivation}

                  Ahora se mostrará que
                  $\st{\left(\bigcup_{k=1}^n a_k\right)}-
                  \bigcup_{k=1}^n\st{a_k}=\varnothing$
                  \begin{longderivation}
                      \wff{
                        x\in\st{\left(\bigcup_{k=1}^n a_k\right)}
                        - \bigcup_{k=1}^n\st{a_k}
                      }\\
                    \equiv\\
                      \wff{
                        \set{i}{x(i)\in\bigcup_{k=1}^n a_k}\in\U
                        \land
                        \lnot\Exists{k}[1\leq k\leq n]{\set{i}{x(i)\in a_k}\in\U}
                      }\\
                    \equiv\\
                      \wff{
                        \set{i}{x(i)\in\bigcup_{k=1}^n a_k}\in\U
                        \land
                        \Forall{k}[1\leq k\leq n]{\set{i}{x(i)\in a_k}\not\in\U}
                      }\\
                    \why{$\U$ es un ultrafiltro}\\
                      \wff{
                        \set{i}{x(i)\in\bigcup_{k=1}^n a_k}\in\U
                        \land
                        \Forall{k}[1\leq k\leq n]{\set{i}{x(i)\not\in a_k}\in\U}
                      }\\
                    \why[\To]{definición de filtro (ii)}\\
                      \wff{
                        \set{i}{x(i)\in\bigcup_{k=1}^n a_k}\in\U
                        \land
                        \set{i}{\Forall{k}[1\leq k\leq n]{x(i)\not\in a_k}}\in\U
                      }\\
                    \why[\To]{definición de filtro (ii)}\\
                      \wff{
                        \set{i}{
                          x(i)\in\bigcup_{k=1}^n a_k
                          \land
                          \Forall{k}[1\leq k\leq n]{x(i)\not\in a_k}
                        }\in\U
                      }\\
                    \equiv\\
                      \wff{ \varnothing \in\U }\\
                    \equiv\\
                      \wff{ x\in\varnothing }
                  \end{longderivation}
                  
            \item Para la colección, el argumento se realiza de la
                  misma forma que para la unión, lo único que cambia es
                  `$\in$' por `$=$'.
            \item Para la $n$-upla, por cómo se define, se puede ver que es
                  la unión de dos conjuntos, lo que significa, que es
                  consecuencia del punto de la unión.
            \item Para el producto cartesiano.

                  Por un lado:
                  \begin{longderivation}
                      \wff{(x_1,\dots, x_n)\in\st{(a_1\times\dots\times a_n)}}\\
                    \equiv\\
                      \wff{
                        \set{i}{(x_1(i),\dots,x_n(i))\in a_1\times\dots\times a_n}\in\U
                      }\\
                    \equiv\\
                      \wff{
                        \set{i}{
                          \Forall{k}[1\leq k\leq n]{x_k(i) \in a_k}
                        }\in\U
                      }\\
                    \why[\To]{Definición de filtro (iii)}\\
                      \wff{
                        \Forall{k}[1\leq k\leq n]{
                          \set{i}{x_k(i)\in a_k} \in\U
                        }
                      }\\
                    \equiv\\
                      \wff{ (x_1,\dots,x_n) \in \st{a_1}\times\dots\times\st{a_n} }
                  \end{longderivation}
                  Por el otro:
                  \begin{longderivation}
                      \wff{ (x_1,\dots,x_n) \in \st{a_1}\times\dots\times\st{a_n} }\\
                    \equiv\\
                      \wff{
                        \Forall{k}[1\leq k\leq n]{
                          \set{i}{x_k(i)\in a_k} \in\U
                        }
                      }\\
                    \why[\To]{Definición de filtro (ii)}\\
                      \wff{
                        \set{i}{
                          \Forall{k}[1\leq k\leq n]{x_k(i) \in a_k}
                        }\in\U
                      }\\
                    \equiv\\
                      \wff{
                        \set{i}{(x_1(i),\dots,x_n(i))\in a_1\times\dots\times a_n}\in\U
                      }\\
                    \equiv\\
                      \wff{(x_1,\dots, x_n)\in\st{(a_1\times\dots\times a_n)}}\\
                  \end{longderivation}
          \end{itemize}

    \item Por doble contenencia:
          \begin{center}
            \begin{tabular}{>{$}c<{$} | >{$}c<{$}}
              \begin{derivation}
                  \wff{ x \in \st{(a-b)} }\\
                \equiv\\
                  \wff{ \set{i}{x(i) \in a \land x(i) \not\in b}\in\U }\\
                \why[\To]{Definición de filtro (iii)}\\
                  \wff{ \set{i}{x(i)\in a}\in\U \land \set{i}{x(i)\not\in b}\in\U }\\
                \why{$\U$ es un ultrafiltro}\\
                  \wff{ \set{i}{x(i)\in a}\in\U \land \set{i}{x(i)\in b}\not\in\U }\\
                \equiv\\
                  \wff{ x \in \st{a} - \st{b} }
              \end{derivation}
              &
              \begin{derivation}
                  \wff{ x \in \st{a} - \st{b} }\\
                \equiv\\
                  \wff{ \set{i}{x(i) \not\in a}\in\U \land \set{i}{x(i)\in b}\not\in\U }\\
                \why[\To]{$\U$ es un ultrafiltro}\\
                  \wff{ \set{i}{x(i)\in a}\in\U \land \set{i}{x(i)\not\in b}\in\U  }\\
                \why[\To]{Definición de filtro (ii)}\\
                  \wff{ \set{i}{x(i) \in a \land x(i) \not\in b}\in\U }\\
                \equiv\\
                  \wff{ x \in \st{(a-b)} }
              \end{derivation}
            \end{tabular}
          \end{center}
    \item~
          \begin{itemize}
            \item Se va a seguir la misma estrategia que se usó en (iv)
                  para la unión. Se mostrará que $\dom \st{b} \subseteq \st{(\dom b)}$
                  \begin{longderivation}
                      \wff{ x\in\dom\st{b} }\\
                    \equiv\\
                      \wff{ \Exists{y}{(x,y)\in\dom\st{b}} }\\
                    \equiv\\
                      \wff{ \Exists{y}{
                        \set{i}{(x(i),y(i))\in b}\in\U
                      } }\\
                    \why[\To]{Definición de filtro (iii)}\\
                      \wff{
                        \set{i}{
                          \Exists{y}{(x(i),y(i))\in b}
                        }\in\U
                      }\\
                    \equiv\\
                      \wff{ x\in\st{(\dom b)} }
                  \end{longderivation}

                  Ahora se mostrará que $\st{(\dom b)} - \dom \st{b} = \varnothing$
                  \begin{longderivation}
                      \wff{ x\in \st{(\dom b)} - \dom \st{b} }\\
                    \equiv\\
                      \wff{
                        \set{i}{x(i) \in \dom b}\in\U
                        \land
                        \lnot\Exists{y}{\set{i}{(x(i),y(i))\in b}\in\U}
                      }\\
                    \equiv\\
                      \wff{
                        \set{i}{x(i) \in \dom b}\in\U
                        \land
                        \Forall{y}{\set{i}{(x(i),y(i))\in b}\not\in\U}
                      }\\
                    \why{$\U$ es un ultrafiltro}\\
                      \wff{
                        \set{i}{x(i) \in \dom b}\in\U
                        \land
                        \Forall{y}{\set{i}{(x(i),y(i))\not\in b}\in\U}
                      }\\
                    \why[\To]{Definición de filtro (ii)}\\
                      \wff{
                        \set{i}{x(i) \in \dom b}\in\U
                        \land
                        \set{i}{\Forall{y}{(x(i),y(i))\not\in b}}\in\U
                      }\\
                    \why[\To]{Definición de filtro (ii)}\\
                      \wff{
                        \set{i}{
                          x(i)\in\dom b \land x(i) \not\in\dom b
                        }\in\U
                      }\\
                    \equiv\\
                      \wff{ \varnothing\in\U }\\
                    \equiv\\
                      \wff{ x \in \varnothing }
                  \end{longderivation}
            \item Para el rango, el argumento es idéntico, cambia el
              	  orden de la pareja ordenada.
          \end{itemize}
  \end{enumerate}
\end{demo}

Ahora, se van a considerar los tipos de elementos que están en $\IRh$.
\pagebreak
\begin{definition}
  $a\in\IRh$ es llamada \emph{entidad interna} cuando existe $n\in\N$
  tal que $a\in\st{\R_n}$. $a$ es llamada \emph{estándar}
  cuando existe $b\in\Rh$ tal que, $a=\st{b}$. Simbólicamente:

  \begin{center}
    \begin{tabular}{>{$}c<{$} | >{$}c<{$}}
      \begin{derivation}
          \wff{\text{$a$ es un entidad interna}}\\
        \equiv\\
          \wff{ \Exists{n}[n\in\N]{a\in\st{\R_n}} }
      \end{derivation}
      &
      \begin{derivation}
        \wff{\text{$a$ es una entidad estándar}}\\
      \equiv\\
        \wff{\Exists{b}[b\in\Rh]{a=\st{b}}}
      \end{derivation}
    \end{tabular}
  \end{center}
  \vspace{20pt}
  El conjunto $\bigcup_{n\in\N}\st{\R_n}$, es llamado la
  ultrapotencia de $\Rh$ con respecto a $\U$, y se denotará como
  $\st{(\Rh)}$.
\end{definition}

Por la definición de elemento interno y estándar, podría parecer que no
existen entidades internas no-estándar. Sin embargo, son estas las que
precisamente logran el objetivo de toda esta construcción.

\begin{theorem}\label{theo:noEst}
  Existen entidades internas no-estándar.
\end{theorem}

\begin{demo}
  Sea $n\in\N$, considere $\R_n$. Nótese que, $\R_n$ es un conjunto
  infinito. Sea $\{a_n\}_{n\in\N}$ una colección de elementos de $\R_n$
  tal que $\Forall{n,m}[n,m\in\N \land n\not= m]{a_n \not= a_m}$.
  Sea $a$ una función de $I$ en $\R_n$ definida por:
  \[a(i) = a_n \quad (i\in I_n)\]
  recordando que $\{I_n\}_{n\in\N}$ es la partición de $I$ que se
  mencionó al principio de esta sección. Primero, se va a mostrar que
  $a$ es una entidad interna. Específicamente, $a\in\st{R_n}$.
  \begin{longderivation}<.9>
      \wff{ a\in\st{\R_n} }\\
    \equiv\\
      \wff{ \set{i}{a(i) \in \R_n}\in\U }\\
    \equiv\\
      \wff{ I \in \U }
  \end{longderivation}

  por contradicción, supóngase que existe $b\in\Rh$ tal que $a=\st{b}$.
  \begin{longderivation}
      \wff{ a=\st{b} }\\
    \equiv\\
      \wff{ \set{i}{a(i) = b} }\\
    \why{$b$ es constante}\\
      \wff{ \Exists{n}[n\in\N]{I_n\in\U} }
  \end{longderivation}
  Esto último, por la definición de $\{I_n\}$, es falso. Así, se tiene
  que $a$ es una entidad interna no-estándar.
\end{demo}

\begin{theorem}~
  \begin{enumerate}
    \item Una entidad es interna si y solo si pertenece a alguna
          entidad estándar.
    \item Los elementos de entidades internas son internos.
  \end{enumerate}
\end{theorem}

\begin{demo}~
  \begin{enumerate}
    \item La demostración se hará por doble implicación. Por un lado,
          sean $n\in\N$, $c\in\R_{n+1}$ y $b=\st{c}$. Esto es,
          $b$ es una entidad estándar, donde $c$ cumple la definición
          de estándar en $b$.
          \begin{longderivation}<.9>
              \wff{a\in b}\\
            \equiv\\
              \wff{ a\in \st{c} }\\
            \equiv\\
              \wff{\set{i}{a(i)\in c}\in\U}\\
            \why[\To]{\hyperref[lema:Rn]{Propiedad de $\R_n$}}\\
              \wff{\set{i}{a(i)\in\R_0\cup\R_n}\in\U}\\
            \why{\hyperref[lema:stR]{Propiedad de ${^*}$}}\\
              \wff{a\in\st{\R_0}\cup\st{\R_n}}\\
            \To\\
              \wff{\Exists{n}[n\in\N]{a\in\st{\R_n}}}
          \end{longderivation}
          Por el otro lado, sean $n\in\N$ y $a\in\st{\R_n}$ una entidad
          interna. Se puede ver inmediatamente que $a$ es elemento de
          una entidad estándar. En efecto, sean $c=\R_n$, $b=\st{c}$.
          Por el \hyperref[lema:Rn]{lema 3.1}, se tiene que
          $c\in\R_{n+1}$, con lo que $b$ resulta ser una entidad estándar.
    \item Sean $n\in\N$, $b\in\R_{n+1}$ y $a\in b$.
          Sea $B=\set{i}{b\in\R_{n+1}}$. Por el
          \hyperref[lema:Rn]{lema 3.1},
          $B=\set{i}{b\subseteq \R_0\cup\R_n}$.
          \begin{longderivation}
              \wff{a\in b}\\
            \To\\
              \wff{\Forall{i}[i\in B]{a(i)\in\R_0\cup\R_n}}\\
            \equiv\\
              \wff{ \set{i}{a(i)\in\R_0\cup\R_n}\in\U }\\
            \why{\hyperref[lema:stR]{lema 4.1}}\\
              \wff{a\in\st{R_0}\cup\st{\R_n}}\\
            \To\\
              \wff{\Exists{n}[n\in\N]{a\in\st{\R_n}}}
          \end{longderivation}
  \end{enumerate}
\end{demo}

Ahora, se va a explorar el comportamiento de las proposiciones al
tomar su incorporación `$({^*})$'.

\begin{theorem}\label{theo:FT}
  Sean $a\in\Rh$, $V(x_1,\dots,x_m)$ un predicado sobre $x_1,\dots,x_m$
  en $\Rh$ y
  \[A=\set{(x_1,\dots,x_m)\in a}{V(x_1,\dots,x_m)}\]
  Entonces $A\in\Rh$ y 
  \[\st{A}=\set{(x_1,\dots,x_m)\in\st{a}}{\st{V(x_1,\dots,x_p)}}\]
\end{theorem}
\begin{demo}
  Sea $n\in\N$ y $a\in\R_{n+1}$. Como $A\subseteq a$, y por definición
  $a\subseteq \R_0\cup\R_n$, se tiene que $A\subseteq\R_0\cup\R_n$, que
  equivale a $A\in\R_{n+1}$. Para el segundo, por doble contenencia:
  \begin{longderivation}
      \wff{(y_1,\dots,y_m) \in\st{A}}\\
    \equiv\\
      \wff{
        (y_1,\dots,y_m)\in\st{
          \set{(x_1,\dots,x_m)\in a}{V(x_1,\dots,x_m)}
        }
      }\\
    \equiv\\
      \wff{
        \set{i}{
          (y_1(i),\dots,y_m(i))\in a \land V(y_1(i),\dots,y_m(i))
        }\in\U
      }\\
    \why[\To]{\hyperref[def:filtro]{definición de filtro (iii)}}\\
      \wff{
        \set{i}{
          (y_1(i),\dots,y_m(i))\in a}\in\U
        \land
        \set{i}{
          V(y_1(i),\dots,y_m(i))}\in\U
      }\\
    \equiv\\
      \wff{
        (y_1,\dots,y_m)\in\st{a} \land \st{V(y_1,\dots,y_m)}
      }\\
    \equiv\\
      \wff{(y_1,\dots,y_m)\in\set{(x_1,\dots,x_m)\in\st{a}}{\st{V(x_1,\dots,x_m)}}}
  \end{longderivation}

  Por el otro lado:
  \begin{longderivation}
      \wff{
        (y_1,\dots,y_m)\in\set{(x_1,\dots,x_m)\in
        \st{a}}{\st{V(x_1,\dots,x_m)}}
      }\\
    \equiv\\
      \wff{
        (y_1,\dots,y_m)\in\st{a} \land \st{V(y_1,\dots,y_m)}
      }\\
    \equiv\\
    \wff{
      \set{i}{
          (y_1(i),\dots,y_m(i))\in a}\in\U
        \land
        \set{i}{
          V(y_1(i),\dots,y_m(i))}\in\U
      }\\
    \why[\To]{\hyperref[def:filtro]{definición de filtro (ii)}}\\
      \wff{
        \set{i}{
          (y_1(i),\dots,y_m(i))\in a \land V(y_1(i),\dots,y_m(i))
        }\in\U
      }\\
    \equiv\\
      \wff{
        (y_1,\dots,y_m)\in\st{
          \set{(x_1,\dots,x_m)\in a}{V(x_1,\dots,x_m)}
        }
      }\\
    \equiv\\
      \wff{
        (y_1,\dots,y_m)\in\st{A}
      }
  \end{longderivation}
\end{demo}

Ya establecida la equivalencia entre predicados del lenguaje formal
en $\Rh$ y la incorporación en $\st{(\Rh)}$, falta abordar las sentencias,
las cuales son proposiciónes sin variables libres. En el siguiente
teorema se establecerá la equivalencia entre sentencias de ambos
lenguajes formales.

\begin{theorem}
  Sea $V$ una sentencia válida en $\Rh$. $V$ es verdadera en $\Rh$ si y
  solo si, $\st{V}$ es verdadera en $\st{(\Rh)}$.
\end{theorem}

\begin{demo}
  De igual forma que se abordó el problema de los predicados tratando
  con conjuntos, se puede abordar este problema. Para esto, hace falta
  entender a qué corresponde una sentencia de $\Rh$ si se piensa
  denotar en términos de un conjunto. Antes de empezar a trabajar con
  cuantificadores, se considerará una sentencia sin cuantificadores.

  Sea $V$ es una sentencia sin cuantificadores y atómica. Sean
  $a_1,\dots,a_m$ las constantes de $V$. Entonces $V$ es de la
  forma $(a_1,\dots,a_{n-1})\in a_n$ o cualquier otra combinación.
  Entonces:
  \begin{longderivation}<.8>
      \wff{\st{(a_1,\dots,a_{n-1})}\in \st{a_n}}\\
    \equiv\\
      \wff{\set{i}{(a_1,\dots,a_{n-1}) \in a_n}\in\U}
  \end{longderivation}
  Está claro que se tiene la equivalencia en este caso. Si $V$ es una
  sentencia sin cuantificadores, pero no es atómica, es una sentencia
  compuesta de varias sentencias atómicas unidas por conectores lógicos,
  en este caso, el conjunto que debe pertenecer a $\U$ cuando $V$ tiene
  conectores lógicos se puede descomponer en uniones, intersecciones o
  complementos, con lo que, para sentencias sin cuantificadores, queda
  demostrado.

  Para una sentencia con cuantificadores. Sea $W(x_1,\dots,x_{n+1})$ un
  predicado sin cuantificadores sobre $x_1,\dots,x_{n+1}$. Sean
  $q_1,\dots,q_{n+1}$ cuantificadores y $V$ una sentencia de la forma
  \[V=(q_{n+1}x_{n+1}\,|:\,(q_n x_n\,|:\,\dots(q_1x_1\,|:\,W(x_1,\dots,x_{n+1}))\dots))\]
  Suponiendo que $q_{n+1}$ es un cuantificador existencial, $V$ se puede escribir en
  términos de un conjunto específico. Sea $b$ el dominio de $q_{n+1}$ y defínase $A$
  como:
  \[A = \set{x}{x\in b \land (q_n x_n\,|:\,\dots(q_1x_1\,|:\,W(x_1,\dots,x_n,x))\dots)}\]
  La sentencia $V$ se puede expresar en términos de $A$ de la siguiente forma:
  \[A\not=\varnothing\]
  Debido a que la definición de $A$ es un predicado sobre $x$, por el teorema
  anterior se tiene que:
  \[
    \st{A} =
    \set{x}{
      x\in\st{b} \land
      (q_n x_n\,|:\,\dots(q_1x_1\,|:\,\st{W(x_1,\dots,x_n,x)})\dots)
    }  
  \]
  Debido a que $A\not=\varnothing \equiv \st{A} \not= \st{\varnothing}$ y que
  $\varnothing=\st{\varnothing}$. Para el caso del cuantificador universal, se
  niega la proposición, obteniendo así un cuantificador existencial. De la
  misma forma que para sentencias atómicas, $V$ es una sentencia con conectores,
  estos se pueden traducir en el conjunto definido a uniones, complementos o
  intersecciones.
\end{demo}

\begin{theorem}
  Sean $V$ un predicado con variables libre $x_1,\dots,x_m$ y
  $a\in\st{(\Rh)}$. Entonces, el conjunto $A$ definido por
  $A=\set{(x_1,\dots,x_n)}{(x_1,\dots,x_n)\in a \land V}$ es interno.
\end{theorem}
\begin{demo}
  Sea $n\in\N$ tal que $a\in\R_{n+1}$. Se puede ver que $A$ es
  subconjunto de $a$. Por definición $a\subseteq \R_0 \cup \R_n$,
  lo que implica que $A\in\R_0\cup\R_n$, que a su vez implica que
  \[\Exists{n}[n\in\N]{A\in\R_n}\]
\end{demo}

Se presentarán dos teoremas que se cumplen en $\R$, los cuales muestran
como ejemplo la interpretación correcta del teorema \ref{theo:FT}.
En estos ejemplos, las sentencias tienen cuantificadores sobre
\textbf{subconjuntos} de $\R$. Para poder aplicar el teorema de forma
correcta, hace falta modificar la expresión para que esté escrita con
las relaciones básicas `$\in$' o `$=$'. La razón de esto, es que cualquier
predicado o sentencia cuantificada sobre subconjuntos de $\R$, no tiene
sentido realmente al considerar el teorema \ref{theo:FT}, pues este hace
referencia a las relaciones básicas. En el caso de los subconjuntos,
la forma de interpretarlo es que se toman subconjuntos $S$ de $\R$, por
lo que la versión estrella tomará subconjuntos $\st{S}$ de $\st{\R}$. Más
adelante se presentarán ejemplos los cuales no pueden aplicarse en
estas sentencias.

\begin{enumerate}
  \item Todo subconjunto no vacío de $\N$ tiene un primer elemento:
        La sentencia se escribe de forma equivalente como
        ``Todo elemento de $\Pts{\N} - \varnothing$ tiene un primer
        elemento''. Al hacer uso del teorema, lo que se obtiene es:

        Todo elemento de $\st{(\Pts{\N}-\varnothing)}$ tiene un primer
        elemento. Nótese que esto se puede transformar a la siguiente
        sentencia:

        Todo subconjunto interno no vacío de $\st{N}$ tiene un primer
        elemento.
  \item Todo subconjunto de $\R$ acotado superiormente tiene una mínima
        cota superior. De igual forma, al usar el teorema, la sentencia
        correspondiente sería:

        Todo subconjunto interno de $\R$ acotado superiormente tiene
        una mínima cota superior.
\end{enumerate}
