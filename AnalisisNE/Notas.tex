\documentclass{article}

\usepackage{logicDG}
\usepackage{calcDG}
\usepackage{upgreek}
\usepackage[T1]{fontenc}
\usepackage[spanish, es-noquoting, es-lcroman]{babel}
\usepackage[a]{esvect}
\usepackage{amssymb}
\usepackage{amsfonts}
\usepackage{mathrsfs}
\usepackage{amsthm}
\usepackage{dsfont}
\usepackage[hidelinks]{hyperref}
\usepackage{enumitem}
\usepackage{setspace}
\usepackage{geometry}
\geometry{
  left=2cm,
  right=2cm,
  bottom=2cm
}
\usepackage{blindtext}

\hyphenpenalty=10000

\newtheoremstyle{definition}%
{10pt} % espaciado arriba
{10pt} % espaciado abajo
{\normalfont} % fuente del cuerpo
{} % indentación
{\bfseries} % fuente del título
{:} % puntuación después del título
{5pt} % espacio después del título
{}
\newtheoremstyle{theorem*}%
{10pt}
{10pt}
{\normalfont}
{}
{\bfseries}
{:}
{\newline}
{}

\theoremstyle{definition}
\newtheorem{definition}{Definición}[section]
\newtheorem{theorem}{Teorema}[section]
\newtheorem{lemma}{Lema}[section]
\newtheorem*{note}{Nota}

\theoremstyle{theorem*}
\newtheorem{lemma*}{Lema}[section]
\newtheorem{theorem*}{Teorema}[section]


\NewDocumentCommand{\F}{}{\mathcal{F}}
\NewDocumentCommand{\U}{}{\mathscr{U}}
\NewDocumentCommand{\N}{}{\mathds{N}}
\NewDocumentCommand{\J}{}{\mathds{J}}
\NewDocumentCommand{\R}{}{\mathds{R}}
\NewDocumentCommand{\Pts}{}{\mathscr{P}}
\NewDocumentCommand{\Rh}{}{\widehat{\R}}

\renewcommand{\labelenumi}{(\roman{enumi})}

\pagecolor[RGB]{28,28,28}
\color[RGB]{212, 211, 210}

\definecolor{bgtitleproof}{RGB}{28,28,28}
\definecolor{bgproof}{RGB}{17, 37, 54}
\definecolor{titleproof}{RGB}{48,48,48}
\definecolor{subproofm}{RGB}{0,0,0}
\definecolor{fontproof}{RGB}{212, 211, 210}
\definecolor{fontprooft}{RGB}{212, 211, 210}

\doublespacing
\begin{document}
\tableofcontents
\clearpage
\section{Filtros}

Sea $I$ un conjunto no vacío.
\begin{definition}
  Un filtro $\F$ sobre $I$ es un conjunto no
  vacío de subconjuntos de $I$, con las siguiente propiedades:
  \begin{enumerate}
    \item $\varnothing \not\in \F$
    \item $A \in \F \land B \in \F \To A\cap B \in \F$
    \item $A \in \F \land A \subseteq B \To B \in \F$
  \end{enumerate}
  La letra $\F$ denotará un filtro sobre $I$, a menos que se especifique
  otra cosa.
\end{definition}

\begin{definition}[Relación de orden] Se dice que
  un filtro $\F_1$ \emph{es más fino} que un filtro $\F_2$ cuando
  $\F_2 \subseteq \F_1$.
\end{definition}

\begin{definition}
  Sea $\mathscr{F}$ el conjunto de filtros sobre $I$. 
  Un ultrafiltro, es un elemento maximal de $\mathscr{F}$. Por definición
  de la relación de orden, esto es, no está contenido propiamente en algún
  otro filtro de $\mathscr{F}$.\\
  La letra $\U$ denotará un ultrafiltro sobre $I$, a menos que se especifique
  otra cosa.
\end{definition}

\begin{theorem}[caracterizaciones de los ultrafiltros]
  Un filtro $\U$ es un ultrafiltro sii
  \begin{enumerate}
    \item $\Forall{A}[A \in I]{A \in \U \not\equiv I - A \in \U}$
    \item Si una unión finita está en $\U$, entonces al menos uno de
          los conjuntos que compone dicha unión, también está en $\U$.
          Esto es, si para un $n\in\N$ hay una colección $\{A_0,\dots,A_n\}$
          tal que $\ds\bigcup_{k=0}^n A_k \in \U$, entonces,
          $\Exists{k}[k\leq n]{A_k \in \U}$
  \end{enumerate}
\end{theorem}

\begin{proofbox}[10]{\textit{Demostración}}
  \begin{enumerate}
    \item Sea $\U$ un filtro que cumple la propiedad a demostrar.
          Se va a suponer que $\U$ no es un ultrafiltro. Así, por
          definición, existe un filtro $\U_2$ más fino que $\U$.
            \begin{longderivation}
                \res{ \U_2 - \U \not= \varnothing }\\
              \equiv\\
                \res{ \Exists{A}{A \in \U_2 - \U} }\\
              \why{ $A$ no está en $\U$, y $\U$ cumple la propiedad mencionada }\\
                \res{ \Exists{A}[A \in \U_2 - \U]{I - A \in \U} }\\
              \why[\To]{$\U \subseteq \U_2$}\\
                \res{ \Exists{A}[A \in \U_2 - \U]{I - A \in \U_2 } }\\
              \equiv\\
                \res{ \Exists{A}[A \not\in \U]{ I - A \in \U_2 \land A \in \U_2 } }\\
              \why[\To]{Definición de filtro, intersección finita}\\
                \res{ \Exists{A}[A \not\in \U]{ \varnothing \in \U_2 } }
            \end{longderivation}
          Esto último es una contradicción con otra de las propiedades en la definición
          de un filtro. En consecuencia, la suposición de que $\U$ no es ultrafiltro es
          incorrecta.
    \item Esta caracterización se va a demostrar usando la anterior. Para mostrar
          su definición de caracterización, se mostrará la equivalencia que tiene con (i).

          Para toda la primera demostación, se considerará un $n \in \N$ y una colección con el
          mismo nombre que la mencionada en la enunciación de la caracterización.
          
          Por una parte, suponiendo que $\ds\bigcup_{k=0}^n A_k \in \U \land
          \Forall{k}[k\leq n]{A_k \not\in \U}$. Por (i), se tiene entonces
          \[\bigcap_{k=1}^n (I-A_k) \not\in \U \land \Forall{k}[k\leq n]{I - A_k \in \U}\]
          Se puede ver que hay una contradicción, pues la intersección de todos los $A_k$
          debe pertenecer a $\U$, pues esta es una intersección finita. Así, (i) $\To$ (ii).

          Por otro lado. Sean $F \not\in \U$, $A_1 = F$ y $A_2 = I-F$. Como
          $\ds\bigcup_{k=1}^{2} A_i = I$, $I \in \U$, entonces al menos uno de los
          $A_k$ debe estar en $\U$. por definición de filtro, se tiene que no pueden ser
          ambos al tiempo, y por (ii), tampoco puede ser que ninguno esté. Así, (ii) $\To$ (i)
  \end{enumerate}
\end{proofbox}

\begin{note}
  $\U$ es un ultrafiltro sii, agregarle otro subconjunto implica que $\varnothing \in \U$
\end{note}


\begin{definition}
  Un filtro $\F$ es llamado $\updelta$-incompleto cuando existe una colección
  $\{F_n\}_{n\in\N}$ en $I$, tal que para todo $n$, $F_n \in \F$ y
  $\ds\bigcap_{n\in\N} F_n \not\in \F$. Un filtro $\F$ es llamado $\updelta$-completo
  cuando no es $\updelta$-incompleto.
\end{definition}
\begin{definition}
  Un filtro $\F$ es llamado libre cuando $\ds\bigcap\F = \varnothing$
\end{definition}

\begin{theorem}
  \begin{enumerate}
    \item Un ultrafiltro $\U$ sobre $I$ es $\updelta$-incompleto sii existe
          una partición contable de $I$ ($I_n$) tal que, para todo $n$,
          $I_n \not\in \U$
    \item Todo ultrafiltro $\updelta$-incompleto es libre
  \end{enumerate}
\end{theorem}

\begin{proofbox}{\textit{Demostración}}
  \begin{enumerate}
    \item Sea $\{F_n\}_{n \in \J}$ una colección de subconjuntos de $I$ que
          cumpla en $\U$ la definición de $\updelta$-incompleto. Como $\U$
          es un ultrafiltro, se tiene entonces que, para todo $n \in \J$,
          $I - F_n \not\in \U$. Por definción de filtro, tampoco
          puede estar en $\U$ alguno de sus subconjuntos. Así mismo,
          como la colección de los $F_n$ cumple la definición de
          $\updelta$-incompleto, se tiene que $\ds\bigcup_{n\in\J}(I-F_n) \in \U$.

          Sea $B_n$ una colección de subconjuntos de $I$ definida por
          $\ds B_k = \bigcup_{n=1}^{k}(I-F_k)$
          Sea $I_n$ una colección de subconjuntos de $I$ definida por
          \[
            \begin{siseq}[\{,\}]
              I_0     &= \bigcap_{k\in\J}F_k\\
              I_{n+1} &= B_{n+1} - B_{n}
            \end{siseq}
          \]

          Se va a mostrar que la colección $\{I_n\}_{n\in\N}$ es una partición de $I$,
          de la cual, para todo $n$, $I_n \not\in \U$

          De entrada se tiene que $I_0 \not\in \U$, por definición de $\updelta$-incompleto.
          Por otro lado, hace falta hallar de forma más explícita qué es $I_n$ para un $n\geq1$.
          
          \begin{align*}
            I_1     &= I - F_1\\
            I_2     &= ((I - F_1) \cup (I-F_2)) - (I-F_1)\\
            I_3     &= ((I-F_1) \cup (I-F_2) \cup (I-F_3)) - ((I - F_1) \cup (I-F_2))\\
            \vdots  & \qquad\vdots\\
          \end{align*}

          Se puede ver inmediatamente que $I_1 \not\in \U$. También se tiene que, para
          un par de subconjuntos de $I$, $A$ y $B$: $A - B = A \cup (I - B)$ y
          $I-(A\cup B) = (I-A)\cap(I-B)$. Una vez aclarado esto:

          \[
            \begin{derivation}
                \res{ I_{n+1} }\\
              =\\
                \res{ B_{n+1} - B_{n} }\\
              =\\
                \res{ \bigcup_{k=1}^{n+1}(I-F_k) - \bigcup_{k=1}^{n}(I-F_k)}\\
              =\\
                \res{ \left((I-F_{n+1}) \cup \bigcup_{k=1}^{n}(I-F_k)\right)
                      \cap \left(I-\bigcup_{k=1}^{n}(I-F_k)\right) }\\
              =\\
                \res{ \left((I-F_{n+1}) \cap
                        \left(I-\bigcup_{k=1}^{n}(I-F_k)\right)
                      \right)
                        \cup
                      \left(
                        \bigcup_{k=1}^{n}(I-F_k) \cap
                        \left(I-\bigcup_{k=1}^{n}(I-F_k)\right)
                      \right) }\\
              =\\
                \res{ (I-F_{n+1}) \cap \bigcap_{k=1}^{n}F_k }
            \end{derivation}
          \]

          Se puede ver que, para todo $n\in\J$, $I_n \subseteq (I-F_n)$, lo que,
          por definición de filtro, y el hecho de que $I-F_n \not\in \U$, implica
          que $I_n \not\in \U$.

          Por la definición de $B_n$, para todo $n \in \J$, $B_n \subseteq B_{n+1}$.
          Esto garantiza que para todo $n,m \in \J \land n\not=m$, $I_n \cup I_m = \varnothing$.

          Falta resolver con $I_0$ y verificar que la unión de todos, en efecto, sea $I$.
          Esto se puede hacer en un mismo paso, calculando la unión de todos los $I_n$
          desde $n=1$. Pero esta unión resulta ser la misma que la unión de todos los
          $I - F_n$. Así,
          \[\bigcup_{n\in\J}I_n = \bigcup_{n\in\J}(I-F_n)\]

          Con esta expresión, se puede ver que, $I_0$, es precisamente el complemento
          en $I$ de esta unión. Con lo que se concluye que la colección $\{I_n\}_{n\in\N}$
          es una partición de $I$ tal que, para todo $n\in\N$, $I_n \not\in \U$.

          En el otro sentio, con ultrafiltro $\U$ sobre $I$ y una partición contable
          de $I$, $\{I_n\}_{n\in\N}$ tal como se define en el enunciado. Se tiene entonces que, para todo $n\in\N$, $I-I_n \in \U$. Por lo que se define $F_n = I - I_n$. solo
          hace falta ver que la intersección de todos los $F_n$ no esté en $\U$.
          \[
            \begin{derivation}
                \res{\bigcap_{n\in\N}F_n}\\
              =\\
                \res{ \bigcap_{n\in\N}(I - I_n) }\\
              =\\
                \res{ I - \bigcup{n\in\N}I_n }\\
              =\\
                \res{ I - I }\\
              =\\
                \res{ \varnothing }\\
              \not\in\\
                \res{\U}
            \end{derivation}
          \]

          Así, se concluye la demostración de esta caracterización de los ultrafiltros
          $\updelta$-incompletos.

    \item Este segundo teorema, resulta ser consecuencia inmediata del anterior.
    Por esta razón, para un ultrafiltro $\updelta$-incompleto, se cuenta con una
    partición contrable $\{I_n\}_{n\in\N}$ que cumple la definición del enunciado.
    
    Lo primero es explorar el significado de $\ds\bigcap\U=\varnothing$

    \begin{longderivation}
        \res{ \bigcap\U = \varnothing }\\
      \equiv\\
        \res{ \lnot \bigcap\U \not= \varnothing }\\
      \equiv\\
        \res{ \lnot\Exists{x}{x\in\bigcap\U} }\\
      \equiv\\
        \res{ \lnot\Exists{x}{\Forall{F}[F\in\U]{x\in F}} }\\
      \equiv\\
        \res{ \Forall{x}{\Exists{F}[F\in\U]{x\not\in F}} }
    \end{longderivation}

    Ya con esta proposición, es más claro lo que hace falta demostrar.
    Por generalización:
    \[
      \begin{derivation}
          \res{ x \in I }\\
        \equiv\\
          \res{ \Exists{n}[n\in\N]{x \in I_n \land I_n \not\in \U} }\\
        \equiv\\
          \res{ \Exists{n}[n\in\N]{x \not\in I - I_n \land (I-I_n)\in\U} }
      \end{derivation}
    \]
    Así, queda demostrado el teorema (ii).
  \end{enumerate}
\end{proofbox}

Junto con el lema de Zorn y la caracterización de los ultrafiltros $\updelta$-incompletos,
se puede demostrar que en cualquier conjunto infinito, existen dichos ultrafiltros.

\section{Súperestructura de \texorpdfstring{$\R$}{R}}

La idea en esta súperestructura, es poder agrupar todas las relaciones
posibles entre número reales. Debido a la definición de las
$n$-uplas de forma recursiva:
\[
  \begin{siseq}[.,.]
    (a) &= a\\
    (a,b) &= \{\{a\}, \{a,b\}\}\\
    (a_1, \dots, a_n, a_{n+1})  &= ((a_1, \dots, a_n), a_{n+1})
  \end{siseq}
\]
y el hecho de que, $(a,b) \in A \times B \To (a,b) \in \Pts(\Pts(A\cup B))$,
se puede lograr definir un conjunto el cual junte todas las $n$-uplas de
números reales.

Se define entones la siguiente colección de conjuntos:
\[
  \begin{siseq}[\{,\}]
    \R_0      &= \R\\
    \R_{n+1}  &= \Pts\left(\bigcup_{k=0}^n \R_k\right)
  \end{siseq}
\]

\begin{definition}
  La súperestructura, denotada por $\Rh$, se definirá como:
  \[\Rh = \bigcup_{n\geq0} \R_n\]
\end{definition}
\begin{definition}
  Los elementos de $\Rh$ serán llamados \emph{entidades}. De estas, los
  elementos de $\R$ serán llamados \emph{individuos}.
\end{definition}

Ahora, se mostrarán algunas propiedades de las entidades. Estas servirán
mucho más adelante para demostrar algunos teoremas a cerca de la construcción
de los número híperreales.
\begin{lemma*}~
  \vspace{-20pt}
  \begin{enumerate}
    \item $\Forall{n,k}[n,k \in \J \land n \geq k]{\R_k \subseteq \R_n}$
    \item $\ds\Forall{n}[n\in\J]{\bigcup_{k=0}^n \R_k = \R_n}$
    \item $\Forall{n,k}[n,k \in\N\land k \leq n]{R_k \in R_n}$
    \item $\Forall{n,x,y}[n\in\N \land y\in\R_{n+1} \land x\in y]{x \in \R_n}$
  \end{enumerate}
\end{lemma*}

\begin{proofbox}{\textit{Demostración}}
  \begin{enumerate}
    \item Sean $n, k \in \J$ tales que $n \geq k$. Se mostrará la contenencia
          De $\R_k$ en $R_n$ tomando un elemento en $\R_k$ y viendo que
          esto implica que esté en $\R_n$.
          \begin{longderivation}
              \res{ x\in\R_k }\\
            \equiv\\
              \res{ x\in\Pts\left(\bigcup_{m=0}^{k-1}R_m\right) }\\
            \equiv\\
              \res{ x \subseteq \bigcup_{m=0}^{k-1}R_m }\\
            \why[\To]{$A\subseteq B \To A \subseteq B \cup C$}\\
              \res{ x \subseteq \bigcup_{m=0}^{n}R_m }\\
            \equiv\\
              \res{ x \in \Pts(\bigcup_{m=0}^{n}R_m) }\\
            \equiv\\
              \res{ x \in \R_n }
          \end{longderivation}
    \item La demostración se hará por inducción:
          \emph{Caso base:}

          \[ \bigcup_{k=0}^1 \R_k = \R_0 \cup \R_1 = \R_1\]

          \emph{Paso inductivo:}
          Suponiendo que, para $n\in\J$, se tiene la propiedad.

          \[
            \begin{derivation}
                \res{ \bigcup_{k=0}^{n}\R_k = \R_0 \cup \R_n }\\
              \why[\To]{$A = B \To A\cup C = B \cup C$}\\
                \res{ \bigcup_{k=0}^{n+1}\R_k = \R_0 \cup \R_n \cup \R_{n+1} }\\
              \why{Propiedad anterior, $A \subseteq B \equiv A\cup B = B$}\\
                \res{\bigcup_{k=0}^{n+1}\R_k = \R_{n+1}}
            \end{derivation}
          \]

    \item Para $n=1$, $k$ debe ser $0$. En este caso, se cumple la propiedad.
          Para $n \geq 1$, se tiene que

          \[
            \begin{derivation}
                \res{ \R_k \in \R_{n+1} }\\
              \equiv\\
                \res{ \R_k \in \Pts\left(\bigcup_{m=0}^{n}\R_m\right) }\\
              \equiv\\
                \res{ \R_k \subseteq \bigcup_{m=0}^{n}\R_m }\\
              \why{Propiedad (ii)}\\
                \res{ \R_k \subseteq \R_n }
            \end{derivation}
          \]

          Esta última expresión, se puede ver por la propiedad (i), es cierta.

    \item Sean $n\in\N$, $y\in\R_{n+1}$, $x\in y$

          \[
            \begin{derivation}
                \res{ x\in y \land y\in\R_{n+1} }\\
              \why{(ii)}\\
                \res{ x\in y \land y \subseteq \R_n}\\
              \To\\
                \res{ x \in \R_n }
            \end{derivation}
          \]
  
    \item 
  \end{enumerate}
\end{proofbox}
\end{document}
