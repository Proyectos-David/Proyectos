\documentclass{article}

\usepackage{logicDG}
\usepackage{calcDG}
\usepackage{upgreek}
\usepackage[T1]{fontenc}
\usepackage[spanish, es-noquoting, es-lcroman]{babel}
\usepackage[a]{esvect}
\usepackage{amsfonts, amssymb}
\usepackage{mathrsfs}
\usepackage{amsthm}
\usepackage[hidelinks]{hyperref}
\usepackage{enumitem}
\usepackage{setspace}
\usepackage{geometry}
\geometry{
  left=2cm,
  right=2cm,
  bottom=2cm
}
\usepackage{blindtext}

\hyphenpenalty=10000

\newtheoremstyle{definition}%
{10pt} % espaciado arriba
{10pt} % espaciado abajo
{\normalfont} % fuente del cuerpo
{} % indentación
{\bfseries} % fuente del título
{:} % puntuación después del título
{5pt} % espacio después del título
{}


\theoremstyle{definition}
\newtheorem{definition}{Definición}[section]
\newtheorem{theorem}{Teorema}[section]
\newtheorem*{note}{Nota}



\NewDocumentCommand{\F}{}{\mathcal{F}}
\NewDocumentCommand{\U}{}{\mathscr{U}}
\NewDocumentCommand{\N}{}{\mathbb{N}}

\renewcommand{\labelenumi}{(\roman{enumi})}

\pagecolor[RGB]{28,28,28}
\color[RGB]{212, 211, 210}

\definecolor{bgtitleproof}{RGB}{28,28,28}
\definecolor{bgproof}{RGB}{17, 37, 54}
\definecolor{titleproof}{RGB}{48,48,48}
\definecolor{subproofm}{RGB}{0,0,0}
\definecolor{fontproof}{RGB}{212, 211, 210}
\definecolor{fontprooft}{RGB}{212, 211, 210}

\doublespacing
\begin{document}
\tableofcontents
\clearpage
\section{Filtros}

Sea $I$ un conjunto no vacío.
\begin{definition}
  Un filtro $\F$ sobre $I$ es un conjunto no
  vacío de subconjuntos de $I$, con las siguiente propiedades:
  \begin{enumerate}
    \item $\varnothing \not\in \F$
    \item $A \in \F \land B \in \F \To A\cap B \in \F$
    \item $A \in \F \land A \subseteq B \To B \in \F$
  \end{enumerate}
  La letra $\F$ denotará un filtro sobre $I$, a menos que se especifique
  otra cosa.
\end{definition}

\begin{definition}[Relación de orden] Se dice que
  un filtro $\F_1$ \emph{es más fino} que un filtro $\F_2$ cuando
  $\F_2 \subseteq \F_1$.
\end{definition}

\begin{definition}
  Sea $\mathcal{F}$ el conjunto de filtros sobre $I$. 
  Un ultrafiltro, es un elemento maximal de $\mathcal{F}$. Por definición
  de la relación de orden, esto es, no está contenido propiamente en algún
  otro filtro de $\mathcal{F}$.\\
  La letra $\U$ denotará un ultrafiltro sobre $I$, a menos que se especifique
  otra cosa.
\end{definition}

\begin{theorem}[caracterizaciones de los ultrafiltros]
  Un filtro $\U$ es un ultrafiltro sii
  \begin{enumerate}
    \item $\Forall{A}[A \in I]{A \in \U \not\equiv I - A \in \U}$
    \item Si una unión finita está en $\U$, entonces al menos uno de
          los conjuntos que compone dicha unión, también está en $\U$.
          Esto es, si para un $n\in\N$ hay una colección $\{A_0,\dots,A_n\}$
          tal que $\ds\bigcup_{k=0}^n A_k \in \U$, entonces,
          $\Exists{k}[k\leq n]{A_k \in \U}$
  \end{enumerate}
\end{theorem}

\begin{proofbox}[10]{\textit{Demostración}}
  \begin{enumerate}
    \item Sea $\U$ un filtro que cumple la propiedad a demostrar.
          Se va a suponer que $\U$ no es un ultrafiltro. Así, por
          definición, existe un filtro $\U_2$ más fino que $\U$.
            \begin{longderivation}
                \res{ \U_2 - \U \not= \varnothing }\\
              \equiv\\
                \res{ \Exists{A}{A \in \U_2 - \U} }\\
              \why{ $A$ no está en $\U$, y $\U$ cumple la propiedad mencionada }\\
                \res{ \Exists{A}[A \in \U_2 - \U]{I - A \in \U} }\\
              \why[\To]{$\U \subseteq \U_2$}\\
                \res{ \Exists{A}[A \in \U_2 - \U]{I - A \in \U_2 } }\\
              \equiv\\
                \res{ \Exists{A}[A \not\in \U]{ I - A \in \U_2 \land A \in \U_2 } }\\
              \why[\To]{Definición de filtro, intersección finita}\\
                \res{ \Exists{A}[A \not\in \U]{ \varnothing \in \U_2 } }
            \end{longderivation}
          Esto último es una contradicción con otra de las propiedades en la definición
          de un filtro. En consecuencia, la suposición de que $\U$ no es ultrafiltro es
          incorrecta.
    \item Esta caracterización se va a demostrar usando la anterior. Para mostrar
          su definición de caracterización, se mostrará la equivalencia que tiene con (i).

          Para toda la primera demostación, se considerará un $n \in \N$ y una colección con el
          mismo nombre que la mencionada en la enunciación de la caracterización.
          
          Por una parte, suponiendo que $\ds\bigcup_{k=0}^n A_k \in \U \land
          \Forall{k}[k\leq n]{A_k \not\in \U}$. Por (i), se tiene entonces
          \[\bigcap_{k=1}^n (I-A_k) \not\in \U \land \Forall{k}[k\leq n]{I - A_k \in \U}\]
          Se puede ver que hay una contradicción, pues la intersección de todos los $A_k$
          debe pertenecer a $\U$, pues esta es una intersección finita. Así, (i) $\To$ (ii).

          Por otro lado. Sean $F \not\in \U$, $A_1 = F$ y $A_2 = I-F$. Como
          $\ds\bigcup_{k=1}^{2} A_i = I$, $I \in \U$, entonces al menos uno de los
          $A_k$ debe estar en $\U$. por definición de filtro, se tiene que no pueden ser
          ambos al tiempo, y por (ii), tampoco puede ser que ninguno esté. Así, (ii) $\To$ (i)
  \end{enumerate}
\end{proofbox}

\begin{note}
  $\U$ es un ultrafiltro sii, agregarle otro elemento implica que $\varnothing \in \U$
\end{note}


\begin{definition}
  Un filtro $\F$ es llamado $\updelta$-incompleto cuando existe una colección
  $\{F_n\}_{n\in\N}$ en $I$, tal que para todo $n$, $F_n \in \F$ y
  $\ds\bigcap_{n\in\N} F_n \not\in \F$. Un filtro $\F$ es llamado $\updelta$-completo
  cuando no es $\updelta$-incompleto.
\end{definition}
\begin{definition}
  Un filtro $\F$ es llamado libre cuando $\ds\bigcap\F = \varnothing$
\end{definition}

\begin{theorem}
  \begin{enumerate}
    \item Un ultrafiltro $\U$ sobre $I$ es $\updelta$-incompleto sii existe
          una partición contable de $I$ ($I_n$) tal que, para todo $n$
          $I_n \not\in \U$
    \item Todo ultrafiltro $\updelta$-incompleto es libre
  \end{enumerate}
\end{theorem}

\begin{proofbox}{\textit{Demostración}}
  \begin{enumerate}
    \item 
  \end{enumerate}
\end{proofbox}
\end{document}
