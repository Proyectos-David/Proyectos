\documentclass{beamer}

\usepackage{logicDG}
\usepackage{calcDG}
\usepackage{graphicx}
\usepackage{dsfont}
\usepackage{amssymb}
\usepackage{mathrsfs}
\usepackage{upgreek}
\usepackage{pgfplots}
\pgfplotsset{compat = newest}
\usepackage[spanish, es-noquoting, es-lcroman, es-noshorthands]{babel}
\usetheme{PaloAlto}
\setbeamercovered{transparent}


% ~~~~ Comandos aux. ~~~~ %
\NewDocumentCommand{\F}{}{\mathcal{F}}
\NewDocumentCommand{\C}{}{\mathds{C}}
\NewDocumentCommand{\U}{}{\mathscr{U}}
\NewDocumentCommand{\N}{}{\mathbb{N}}
\NewDocumentCommand{\J}{}{\mathds{J}}
\NewDocumentCommand{\R}{}{\mathds{R}}
\NewDocumentCommand{\Rh}{}{\widehat{\R}}
\NewDocumentCommand{\e}{}{\upvarepsilon}
\NewDocumentCommand{\B}{}{\mathcal{B}}
\NewDocumentCommand{\equ}{}{=_\U}
\NewDocumentCommand{\inu}{}{\in_\U}
\NewDocumentCommand{\st}{m}{\prescript{*}{}{#1}}
\NewDocumentCommand{\IRh}{}{\prescript{I}{}{\Rh}}
\NewDocumentCommand{\sd}{m}{\mathrm{s\hspace{-0.25pt}t}\!\left(#1\right)}

\NewDocumentCommand{\Pts}{m}{\mathscr{P}\left(#1\right)}
\NewDocumentCommand{\dom}{}{\text{dom}\;}
\NewDocumentCommand{\ran}{}{\text{ran}\;}
\NewDocumentCommand{\set}{m m}{\left\{#1\,\middle|\,#2\right\}}

%  -- Beamer colors --

\definecolor{coso}{HTML}{0038a5}
\setbeamercolor{logo}{bg=coso}
\setbeamercolor{frametitle}{bg=coso} 
\setbeamercolor{sidebar}{bg=coso} 

\graphicspath{./AnalisisNE/}
\logo{\includegraphics[width=1.5cm]{logo-eci-invert.png}}

% -- Titlepage --

\title{Análisis no estándar}

\author[{David G.}]{{David Gómez}}

\institute[ECI]{
    Escuela Colombiana de Ingeniería\\
    Matemáticas
    }
\date[18/12/2023]{18 de Diciembre del 2023}

\begin{document}
\begin{frame}
  \titlepage
\end{frame}

\section{Filtros}
\begin{frame}{Definición}
    Sea $I\not=\varnothing$. Un filtro $\F$ sobre $I$ es un conjunto de
    subconjuntos de $I$ con las siguientes características:
    \begin{enumerate}
      \item $\varnothing\not\in \F$
      \item $A\subseteq B \To (A\in\F \To B\in\F)$
      \item $A,B\in\F \To A\cap B\in\F$
    \end{enumerate}
  \onslide<2>{
    \begin{align*}
      I  &= \{a,b,c\}\\
      \F &= \{\{a,b\},\{a,b,c\}\}
    \end{align*}
  }
\end{frame}

\begin{frame}{Relación de orden}
  Sea $\mathscr{F}$ el conjunto de filtros sobre $I$.
  $\F_1$ es \emph{más fino} que $\F_2$ cuando
  $\F_2 \subseteq \F_1$.
  un filtro $\U$ es llamado ultrafiltro si es un elemento maximal.
  \[\Forall{\F}[\F\in\mathscr{F}-\{\U\}]{\U \not\subset \F}\]
  \onslide<2>{
    \begin{align*}
      I &= \{a,b,c\}\\
      \F_1 &= \{\{a,b\},\{a,b,c\}\}\\
      \F_2 &= \{\{b,c\},\{a,b,c\}\}\\
      \U_1 &= \{\{a\},\{a,b\},\{a,c\},\{a,b,c\}\}\\
      \U_2 &= \{\{b\},\{a,b\},\{b,c\},\{a,b,c\}\}
    \end{align*}
  }
\end{frame}

\begin{frame}{Caracterizaciones}
  Un filtro $\U$ sobre $I$ es un ultrafiltro si y solo si:
  \begin{enumerate}
    \item $\Forall{A}[A\subseteq I]{A\in\U\not\equiv I-A\in\U}$
    \item Si una unión finita es elemento de $\U$, entonces al menos un
          elemento de la unión es elemento de $\U$:
          $\ds\bigcup_{k=0}^n F_n \in \U \To F_k \in \U$ para algún $k$.
  \end{enumerate}
  \onslide<2>{
    \begin{align*}
      I &= \{a,b,c\}\\
      \F &= \{\{a,b\},\{a,b,c\}\}\\
      \U &= \{\{a\},\{a,b\},\{a,c\},\{a,b,c\}\}\\
    \end{align*}
  }
\end{frame}

\begin{frame}{Ulfrafiltros $\delta$-incompletos}
  Sea $I$ un conjunto infinito. $\U$ es un ultrafiltro $\delta$-incompleto
  cuando existe $\{I_n\}_{n\in\N}$, una partición contable de $I$, tal que
  para todo $n$, $I_n \not\in\U$.
  \begin{center}
    \begin{tikzpicture}
      \draw (0,0) -- (5,0);
      \node at (0,0) {$\Biggl($};
      \node at (5,0) {$\Biggr)$};
      \draw (2.5,-0.7) -- (2.5, 0.7);
      \draw[
        domain=0:360,
        samples=500,
        thick,
        color=red!60!black
      ] plot ({0.5*cos(\x)+2.5},{0.2*sin(\x)});
      \draw[
        domain=0:360,
        samples=500,
        thick,
        color=red!60!black
      ] plot ({1*cos(\x)+2.5},{0.4*sin(\x)});
      \draw[
        domain=0:360,
        samples=500,
        thick,
        color=red!60!black
      ] plot ({1.5*cos(\x)+2.5},{0.6*sin(\x)});

      \node at (1.25,0.7) {$I_0$};
      \node at (3.75,0.7) {$I_1$};
    \end{tikzpicture}
  \end{center}
\end{frame}

\section{Superestructura}
\begin{frame}{$n$-uplas}
  \begin{align*}
    (a) &= a\\
    (a,b) &= \{\{a\}, \{a,b\}\}\\
    (a_1,\dots,a_n,a_{n+1}) &= ((a_1,\dots,a_n),a_{n+1})
  \end{align*}
  \begin{align*}
    A\times B &\subseteq \Pts{\Pts{A\cup B}}\\
    A\times B\times C &\subseteq \Pts{\Pts{(A\times B)\cup C}}\\
    & \subseteq \Pts{\Pts{\Pts{\Pts{A\cup B}}\cup C}}\\
    \vdots & \qquad\vdots
  \end{align*}
\end{frame}
\begin{frame}{Superestructura de $\R$}
  \[
    \begin{siseq}[\{,\}]
      \R_0 &= \R\\
      \R_{n+1} &= \Pts{\bigcup_{k=0}^n \R_k}
    \end{siseq}
  \]
  \[\Rh = \bigcup_{n\geq 0} \R_n\]
\end{frame}
\section{Ultrapotencia}
\begin{frame}{Construcción de la ultrapotencia}
  Sea $I$ un conjunto infinito, $\U$ un ultrafiltro sobre $I$, y
  $\{I_n\}_{n\in\N}$ una partición contable de $I$ la cual cumple la
  definición de $\delta$-incompleto en $\U$.
  \[\IRh = \set{f}{\dom f = I \land \ran f \subseteq \Rh}\]
\end{frame}
\begin{frame}{Extensión de la igualdad y la pertenencia}
  Sean $a,b\in\IRh$
  \begin{center}
    \begin{tabular}{>{$}c<{$} | >{$}c<{$}}
      \begin{derivation}
          \res{a\equ b}\\
        \equiv\\
          \res{\set{i}{a(i) = b(i)}\in\U}
      \end{derivation}
      &
      \begin{derivation}
          \res{a\inu b}\\
        \equiv\\
          \res{\set{i}{a(i)\in b(i)}\in\U}
      \end{derivation}
    \end{tabular}
  \end{center}
  \only<1>{
    \[\Forall{a,b}[a,b\in\IRh]{a\equ b \lor a \not\equ b}\]
  }
  \only<2>{
    Se define, para todo $a\in\Rh$, la incorporación a $\IRh$ por la
    función constante $\st{a}(i) = a$ para todo $i\in I$.
  }
\end{frame}
\begin{frame}{Entidades internas y estándar}
  Sea $a\in\IRh$, $a$ es llamada \emph{entidad interna} cuando existe $n\in\N$
  tal que $a\in\st{\R_n}$. $a$ es llamada \emph{estándar}
  cuando existe $b\in\Rh$ tal que, $a=\st{b}$.
  El conjunto $\bigcup_{n\in\N}\st{\R_n}$, es llamado la
  ultrapotencia de $\Rh$ con respecto a $\U$, y se denotará como
  $\st{(\Rh)}$.
  
  Existen entidades internas no-estándar.
\end{frame}
\begin{frame}{Demostración}
  Considerando algún $\R_n$, y una colección $\{a_n\}_{n\in\N}\in\R_n$ tal
  que, si $m\not=n$ entonces $a_m \not= a_n$.
  Se define entonces $a\in\IRh$ por $a(i) = a_n \quad (i\in I_n)$.
  \begin{tabular}{>{$}c<{$} | >{$}c<{$}}
    \begin{derivation}<1>
        \res{a\in\st{\R_n}}\\
      \equiv\\
        \res{\set{i}{a(i)\in\R_n}\in\U}\\
      \equiv\\
        \res{I\in\U}
    \end{derivation}
    &
    \begin{derivation}
        \res{a=\st{b}}\\
      \equiv\\
        \res{\set{i}{a(i)=b}\in\U}\\
      \equiv\\
        \res{\Exists{n}[n\in\N]{I_n\in\U}}
    \end{derivation}
  \end{tabular}
\end{frame}
%TODO parte lógica
\section{Números no estándar}
\begin{frame}{Extensión de otras relaciones y operaciones}
  Sean $a,b,c\in\IRh$
  \begin{align*}
    a + b = c &\equiv \set{i}{a(i) + b(i) = c(i)}\in\U\\
    a \leq b &\equiv \set{i}{a(i) \leq b(i)}\in\U\\
    ||&: \st{R} \to \st{(R^+)}
  \end{align*}
\end{frame}

\begin{frame}{Ejemplo de un número no estádar}
  Defínase $\omega \in \st{\N}$ por
  \[\omega(i) = n\quad (i\in I_n)\]
  Sea $k\in\N$
  \begin{longderivation}
      \res{\omega \leq k}\\
    \equiv\\
      \res{ \set{i}{\omega(i) \leq k}\in\U }\\
    \equiv\\
      \res{\bigcup_{i=0}^k I_i \in\U}
  \end{longderivation}
\end{frame}
\begin{frame}{Números finitos e infinitesimales}
    Un número $a\in\st{\R}$ es llamado \emph{finito} cuando existe
    $r\in\R^+$ tal que $|a|<r$. un número no finito es llamado \emph{infinito}.
    Un número $a\in\st{\R}$ es llamado \emph{infinitesimal} cuando,
    para todo $r\in\R^+$, $|a|<r$.
    \begin{align*}
      M_0 &= \set{a\in\st{\R}}{a \text{ es finito}}\\
      M_1 &= \set{a\in\st{\R}}{a \text{ es infinitesimal}}
    \end{align*}
\end{frame}
\begin{frame}{Isomorfismo con $\R$}
  \begin{enumerate}
    \item $M_0$ es un subanillo de $\st{\R}$.
    \item $M_1$ es un subanillo de $M_0$.
    \item $M_1$ es un ideal maximal de $M_0$.
  \end{enumerate}
  \begin{itemize}
    \only<1>{
      \item Con la relación $=_1$ en $\st{R}$, definida por
            $a=_1b \equiv a-b\in M_1$, se define el anillo cociente
            $M_0/M_1$.
      \item $M_0/M_1$ es isomorfo a $\R$.
    }
    \onslide<2>{
      \item El homomorfismo de $M_0$ a $\R$ con kernel $M_1$ se denotará
            como $\sd{}$
    }
  \end{itemize}
\end{frame}
\begin{frame}{Sistema numérico no estándar $\st{\C}$}
  \only<1>{
    \begin{itemize}
      \item Superestructura $\widehat{\R\times\R}$.
      \item Ultrapotencia $\st{\left(\widehat{\R\times\R}\right)}$.
      \item $\st{\C} = \st{\R}\times\st{\R}$.
    \end{itemize}
  }
  \only<2>{
    Un número $z\in\st{\C}$ es llamado \emph{finito} cuado existe
  $r\in\R^+$ tal que $\|z\| < r$. Un número no finito es llamado
  \emph{infinito}.

  Un número $z\in\st{\C}$ es llamado \emph{infinitesimal} cuando, para
  todo $r\in\R^+$, $\|z\| < r$.

  $z = a + b\,i$ es infinitesimal si y solo si $a,b$ son infinitesimales.
  }
\end{frame}

\section{Sucesiones}
\begin{frame}{Sucesiones Reales}
  Al ser funciones de $\N$ en $\R$ son subconjuntos de $\N\times\R$, por
  lo que son entidades de $\Rh$. La extensión de una sucesión $\{s_n\}$
  es $\{\st{s_n}\}$ la cual es una función de $\st{\N}$ a $\st{\R}$.

  \[\st{(\dom \{s_n\})} = \dom \st{\{s_n\}}\]
\end{frame}
\begin{frame}{Convergencia y Sucesiones de Cauchy}
  \[\{s_n\} \to s\]
  \only<1>{
      \begin{enumerate}
      \item Convergencia:
      \begin{itemize}
        \item Dado $\e>0$, existe $N\in\N$ tal que:
        \[\Forall{n}[n\in\N \land n\geq N]{|s_n - s| < \e}\]
        \item De forma análoga:
        \[\Forall{n}[n\in\st{\N} - \N]{\st{s_n} =_1 s}\]
      \end{itemize}
      \end{enumerate}
    }
    \only<2>{
      \begin{enumerate}
      \setcounter{enumi}{1}
      \item Sucesión de Cauchy
      \begin{itemize}
        \item Dado $\e>0$, existe $N\in\N$ tal que:
        \[\Forall{n,m}[n,m\in\N \land n,m \geq N]{|s_n - s_m| < \e}\]
        \item De form análoga:
        \[\Forall{n,m}[n,m\in\st{\N}-\N]{s_n =_1 s_m}\]
      \end{itemize}
      \end{enumerate}
    }
\end{frame}
\begin{frame}{Sucesiones en otros espacios métricos}
  Sea $\{s_n\}$ una sucesión en un espacio métrico $X$ con función
  distancia $d_X$.
  Sea $\e>0$, existe $N\in\N$ tal que:
  \begin{itemize}
    \item $\{s_n\} \to s$:
    \only<1-2>{\[\Forall{n}[n\geq N]{d_X(s_n, s)<\e}\]}
    \only<3>{\[\Forall{n}[n\geq N]{|q_n - 0|<\e}\]}
    \only<4>{\[\Forall{n}[n\in\st{\N}-\N]{d_X(s_n,s)=_1 0}\]}
    \item $\{s_n\}$ es una sucesción de Cauchy:
    \only<1-2>{\[\Forall{n,m}[n,m\geq N]{d_X(s_n,s_m)<\e}\]}
    \only<3>{\[\Forall{n,m}[n,m\geq N]{|p_{n,m} - 0|<\e}\]}
    \only<4>{\[\Forall{n,m}[n,m\in\st{\N}-\N]{d_X(s_n,s_m)=_1 0}\]}
  \end{itemize}
  \pause
  \begin{align*}
    q_n &= d_X(s_n,s)\\
    p_{n,m} &= d_X(s_n,s_m)
  \end{align*}
\end{frame}
\end{document}
