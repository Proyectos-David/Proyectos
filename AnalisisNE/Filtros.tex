\section{Filtros}

Los filtros son, como se mencionó, objetos de la teoría de conjuntos.
Como su nombre indica, son objetos que filtran, de forma análoga a lo que
puede hacer un colador. Para esta sección, se considerará $I$ como un
conjunto no vacío, esto último es necesario para la definición de filtro.

\begin{definition}\label{def:filtro}
  Un filtro $\F$ sobre $I$, es un conjunto no vacío de subconjuntos de $I$.
  $\F$ cumple las siguientes características:
  \begin{enumerate}
    \item $\varnothing \not\in \F$
    \item $\Forall{A,B}[A,B \in \F]{A\cap B \in \F}$
    \item $\Forall{A,B}[A\in \F \land A \subseteq B]{B \in \F}$
  \end{enumerate}

  Para esta sección, la letra $\F$ denotará un filtro sobre $I$
\end{definition}

Por ejemplo, considere el conjunto $X=\{a,b,c\}$. Un filtro $G$ sobre $X$
puede ser 
\[G = \{\{a\}, \{a,b\}, \{a,c\}, \{a,b,c\}\}\]
Nótese que por la definición de filtro, el conjunto sobre el que este se
define, siempre debe ser un elemento del filtro.

Así como un colador puede ser más fino que otro, en el sentido que
deja pasar menos cosas, también se pueden comparar a los filtros definidos
sobre un conjunto.

\begin{definition}[Relación de orden]
  Un filtro $\F_1$ es más fino que un filtro $\F_2$ cuando $\F_2 \subseteq \F_1$
\end{definition}

Volvamos al conjunto $X$ definido para el ejemplo anterior. Sean $G_1$, $G_2$
filtros sobre $X$, y
\[G_1 = \{\{a\}, \{a,b\},\{a,c\},\{a,b,c\}\}\text{ , } G_2 = \{\{a,b,c\}\}\]
Se puede ver que $G_2 \subseteq G_1$.

Se puede ver que hay filtros que no se pueden comparar, incluso en conjuntos tan
simples como $X$. Si consideramos ahora un nuevo filtro\\
$G_3 = \{\{b\}, \{b,a\},\{b,c\},\{a,b,c\}\}$, está claro que no se pueden comparar
$G_3$ y $G_1$.

\begin{definition}
  Sea $\mathscr{F}$ el conjunto de filtros definidos sobre $I$. Se define
  el concepto de ultrafiltro como un elemento maximal de $\mathscr{F}$ con
  la relación de orden definida anteriormente. Simbólicamente,
  un filtro $\U$ sobre $I$, es un ultrafiltro cuando
  \[\Forall{\F}[\F\in\mathscr{F}]{\U\not\subset\F}\]

  Para esta sección, la letra $\U$ denotará un ultrafiltro sobre $I$.
\end{definition}

\begin{theorem}[Caracterizaciones]
  Sea $\U$ un ultrafiltro sobre $I$. $\U$ es un ultrafiltro si y solo si:
  \begin{enumerate}
    \item $\Forall{A}[A\subseteq I]{A\in\U \not\equiv I-A \in \U}$
    \item Sean $n\in\N$, $\{A_k\}$ una colección de $n$ subconjuntos de $I$ tal que
          \[\bigcup_{k=0}^n A_k \in \U\]
          entonces
          \[\Exists{k}[k\leq n]{A_k\in\U}\]
  \end{enumerate}
\end{theorem}

\begin{demo}[i] Por un lado, se va a mostrar que si $\U$ es un ultrafiltro,
  entonces se tiene la propiedad. Por contradicción, se va a suponer
  que $\U$ es un ultrafiltro, y se tiene un subconjunto $A$ de $I$, tal
  que $A \not\in \U \,\land\, I-A\not\in\U$. Una forma equivalente de escribir
  el punto (ii) de la \hyperref[def:filtro]{definición de filtro} es
  \[\Forall{A,B}[A\subseteq B]{B\not\in\F \To A\not\in\F}\]
  Con esto se puede ver que ningún subconjunto, tanto de $A$ como de $I-A$
  es elemento de $\U$.

  Sea $\U_2 = \{B \subseteq I \,|\, B \cup A \in \U\}$, se puede ver que
  $\U \subset \U_2$, en efecto
  \begin{longderivation}<1>
      \res{ B \in \U }\\
    \To\\
      \res{ B \cup A \in \U }\\
    \equiv\\
      \res{ B \in \U_2 }
  \end{longderivation}

  No son iguales, pues, por ejemplo, $I-A \cup A = I$, $I-A \in \U_2$.\\
  Hace falta ver que $\U_2$ es un filtro.
  \begin{enumerate}
    \item Por contradicción, es inmediato:
          \begin{longderivation}<0.7>
              \res{ \varnothing \in \U_2 }\\
            \equiv\\
              \res{ A \in \U }
          \end{longderivation}
    \item Sean $X, Y \in \U_2$
          \begin{longderivation}<0.7>
              \res{ X \cap Y \in \U_2 }\\
            \equiv\\
              \res{ (X \cap Y) \cup A \in \U }\\
            \equiv\\
              \res{ (X \cup A) \cap (Y \cup A) \in \U }\\
          \end{longderivation}
          Como $X,Y\in\U_2$, se tiene que ambos términos de la intersección
          son elementos de $\U$. Como $\U$ es un filtro, por definición, esta
          intersección también es elemento de $\U$.
    \item Sean $X\in\U_2$ y $Y\supseteq X$
          \begin{longderivation}
              \res{ X \subseteq  Y }\\
            \To\\
              \res{ X \cup A \subseteq Y \cup A }\\
            \why[\To]{$X\in\U_2$ y $\U$ es un filtro}\\
              \res{ Y \cup A \in \U }\\
            \equiv\\
              \res{ Y \in \U_2 }
          \end{longderivation}
  \end{enumerate}

  Entonces, se tiene que $\U \subset \U_2$ y $\U_2$ es un filtro sobre
  $I$. Lo cual contradice la hipótesis de que $\U$ es un ultrafiltro.

  Por el otro lado, de igual forma por contradicción. Se va a suponer que
  $\U$ es un filtro con la propiedad (i) y que $\U$ no es un ultrafiltro.

  Como $\U$ no es un ultrafiltro, existe un filtro $\U_2$ tal que
  $\U \subset \U_2$.
  \begin{longderivation}
      \res{ \U_2 - \U \not= \varnothing }\\
    \equiv\\
      \res{ \Exists{A}{A \in \U_2 - \U} }\\
    \why{ $\U$ cumple (i) y $A \not\in \U$ }\\
      \res{ \Exists{A}[A\in\U_2-\U]{I-A\in\U} }\\
    \why[\To]{$\U\subset\U_2$}\\
      \res{ \Exists{A}[A\in\U_2-\U]{I-A\in \U_2} }\\
    \equiv\\
      \res{ \Exists{A}[A\not\in\U]{A\in\U_2 \land I-A\in\U_2} }\\
    \why[\To]{Definición de filtro}\\
      \res{ \Exists{A}[A\not\in\U]{\varnothing\in\U_2} }
  \end{longderivation}
  Esto último contradice la definición de filtro, mostrando así que la
  suposición de que $\U$ no es un ultrafiltro, es incorrecta.
\end{demo}
\begin{demo}[ii]
  
\end{demo}
