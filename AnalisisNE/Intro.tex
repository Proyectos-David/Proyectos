\section{Introduccion}

Cuando Newton y Leibniz trabajaron en la fundamentación del cálculo, una
de sus diferencias fue la definición de límite. Mientras Newton lo
definió de la forma en la que se ha enseñado principalmente, la
definición $\e$ y $\delta$. Leibniz, lo definia de una forma que
incluso parece una versión más amigable que la de Newton. Leibniz
consideraba números infinitamente pequeños, de tal forma que fueran
menores que cualquier número positivo pero mayores a $0$. Junto con
números infinitamente grandes, mayores que cualquier número positivo.

El problema de esta definición, se encontró cuando se intentó fundamentar
formalmente. Cosa que Leibniz ni sus discípulos lograron demostrar.
La definición de Newton recurre a los mismos números reales ya usados.
La definición de Leibniz, recurre a una nueva especie de números, los cuales
deben ser comparables y se deben poder operar con los reales. La idea entonces
con esta nueva especie de números, es poder operar con estos, para
posteriormente tomar el resultado y recuperar la información que interesa,
la que corresponde a un valor real estándar. Esta nueva especie de números
resulta tener aplicaciones en más áreas que el cálculo de límites, sin
embargo, no hacen parte del objetivo de este proyecto, el cual consta de
presentar esta idea en relación al análisis estándar, específicamente,
el análisis diferencial.

Los pasos para la construcción de estos números, recurre a los filtros,
un objeto de la teoría de conjuntos sobre el que se hablará en el
documento. Con estos, se puede lograr la construcción de esta nueva especie
de números sobre los reales. 
