\section{Introduccion}

En el análisis, el concepto de límite es uno de sus fundamentos,
el cual se usa para definir todo tipo de operaciones o hallar
resultados sobre algún objeto de la teoría, como puede ser una
función. La definición usual de límite es la conocida
``$\e$ , $\delta$''. Sin embargo, en los inicios del cálculo,
Leibniz intentó fundamentar otra definición, la cual consiste en
usar \emph{infinitesimales}, que son números infinitamente pequeños
o infinitamente grandes. Dichos números debian comportarse,
respectivamente, de la siguiente manera:
un número infinitamente pequeño es menor que todo número real positivo
pero mayor a $0$, y un número infinitamente grande es mayor que todo
número real. Leibniz sostenía que dichos números debían mantener las
mismas propiedades que los números reales, en cierta manera.

El problema de esta definición, se encontró cuando se intentó fundamentar
formalmente. Cosa que ni Leibniz, ni sus discípulos, lograron.
Como se mencionó, esta definición recurre a una nueva especie de números, los cuales
deben ser comparables y se deben poder operar con los reales. La idea,
entonces, con esta nueva especie de números, es poder operar con estos,
para posteriormente, tomar el resultado y recuperar la información que interesa,
la que corresponde a un valor real estándar. Esta nueva especie de números
resulta tener aplicaciones en más áreas que el cálculo de límites, sin
embargo, no hacen parte del objetivo de este proyecto, el cual consta de
presentar esta idea en relación al análisis estándar, específicamente,
el análisis diferencial.

Los pasos para la construcción de estos números recurre a los filtros,
objetos de la teoría de conjuntos sobre los que se hablará en el
documento. Con estos, se puede lograr la construcción de esta nueva especie
de números sobre los reales.
