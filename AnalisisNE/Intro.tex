\section{Introduccion}

La definición clásica de límite, presentada en el análisis clásico, es
la \\formalización de las ideas con las que Newton construyó el cálculo.
En esta definición se considera, por ejemplo, una función $f:\R \to \R$,
un valor $c \in \R$ al que se hace tender la variable de $f$, y un
valor de resultado $a \in \R$. Se formula entonces que:
\[
    \begin{derivation}
            \res{ \Lim{x}{c} f(x) = a}\\
        \equiv\\
            \res{ (
                    \forall \e \sthat \e > 0 \sat
                    (
                        \exists \updelta \sthat \updelta > 0 \sat
                        (
                            \forall x \sthat 0< |x - c|<\updelta :
                            |f(x) - c| < \e
                        )
                    )
                ) }
    \end{derivation}
\]

Esta definición, claro, está dada por la naturaleza de $f$, ya que este
concepto de límite se puede dar para cualquier función $g:A \to B$,
donde $A$ y $B$ son espacios métricos.

La definición de límite con la que Leibniz construyó el cálculo, es
diferente, y no del todo conocida, pues ni él, ni alguno de sus
sucesores pudieron construirla formalmente. Leibniz hablaba de números
infinitamente pequeños o infinitamente grandes. Bajo esta idea, el
límite del ejemplo consistiría tomar un número $c'$ tal que $|c' - c|$ es
infinitamente pequeño, y demostrar entonces que \\$|f(c') - a|$ también
es infinitamente pequeño. Esta explicación se nota un poco más intuitiva
de entrada. Sin embargo, tomando los conociemitos del análisis clásico,
esta definición de $c'$ solo sería posible si $c'=c$, cosa que no
tendría sentido, si se quiere tener información del límite. En este
artículo, presentaré los resultados obtenidos principalmente por
Abraham Robinson, y la formalización de las ideas de los %TODO ref%
infinitesimales.
