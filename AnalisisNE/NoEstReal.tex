\section{El sistema numérico no estándar \texorpdfstring{$\st{\R}$}{R}}

Por el \hyperref[theo:FT]{teorema 4.3}, las mismas propiedades que aplican
en $\R$, aplican en $\st{\R}$ hasta donde apliquen en ambos simultaneamente.
Para simplificar la notación, se mantendrá el mismo símbolo para
denotar operaciones usuales en $\R$ y en $\st{\R}$.
Por ejemplo, en $\st{\R}$, $a + b = c$ es equivalente a
$\set{i}{a(i) + b(i) = c(i)}\in\U$. De la misma manera para la
resta, multiplicación. De igual forma, en $\st{\R}$, $a \leq b$ es
equivalente a $\set{i}{a(i)\leq b(i)}\in\U$. Un ejemplo de la
aplicación del \hyperref[theo:FT]{teorema 4.3} es la siguiente sentencia.
\[\Forall{x,y}[x,y\in\R]{x<y \;\lor\; x>y \;\lor\; x=y}\]
Dado que esta sentencia es verdadera en $\R$, se tiene entonces que la
sentencia
\[\Forall{x,y}[x,y\in\st{\R}]{x<y \;\lor\; x>y \;\lor\; x=y}\]
es verdadera en $\st{\R}$.

Nótese que de la misma manera, es inmediato que el elemento neutro del
producto en $\st{\R}$ es $\st{1}$. Por conveniencia, se omitirá la notación
$({^*})$ para elementos de $\st{R}$ y por lo mencionado antes,
se tomará $\R \subset \st{R}$.

Aclarando más acerca de algunas funciones comunes en $\R$, el valor
absoluto, definido por:
\[
  |r| =
  \begin{syseq}(\{,\})
    r   &&\quad (r > 0)\\
    0   &&\quad (r = 0)\\
    -r  &&\quad (r < 0)
  \end{syseq}
\]
es una función de $\R$ en $\R^{\geq 0}$. Debido a la incorporación
del sistema formal de $\R$ en $\st{\R}$, se tiene una función
correspondiente en $\st{\R}$, la cual iría de $\st{\R}$ a
$\st{\R^{\geq 0}}$. Por el \hyperref[theo:FT]{teorema 4.3}, esta función
cumple que:
\[
  \st{|r|} =
  \begin{syseq}(\{,\})
    r   &&\quad (r > 0)\\
    0   &&\quad (r = 0)\\
    -r  &&\quad (r < 0)
  \end{syseq}
\]
por supuesto, para $r\in\st{\R}$. Debido a esto, se denotará el valor
absoluto en $\st{\R}$ de la forma usual. Por la misma razón, se denotará
de manera usual las funciones $\min$ y $\max$.

Sea $S\subseteq\R$, al pasar a $\st{\R}$, $\st{S}$ corresponde a un
subconjunto de $\st{\R}$, el cual mantienen las mismas propiedades que $S$
hasta donde tenga sentido en ambos simultaneamente. Así como se definió
la superestructura $\Rh$, se puede definir una para $S$, $\widehat{S}$, y,
de igual forma, se puede definir una ultrapotencia de $\widehat{S}$ denotada
por $\st{(\widehat{S})}$. De igual forma, por lo mencionado anteriormente,
se tomará $S\subseteq\st{S}$. Nótese, que al dejar la notación $({^*})$,
por el \hyperref[lema:stR]{lema 4.1}, se tiene que $S=\st{S}$ si y solo
si $S$ es finito.

Un resultado del álgebra nos dice que todo cuerpo arquimediano es
isomorfo con un subcuerpo de $\R$. Debido a la existencia de los
infinitesimales en $\st{R}$, se tiene que $\st{\R}$ no es arquimediano.
Sin embargo, $\st{R}$ cumple las mismas propiedades que $\R$, lo que
parecería ser una contradicción. La razón de esto, es que la propiedad
arquimediana es relativa a un conjunto, en el caso de $\R$, es con $\N$.
explícitamente, la propiedad es la siguiente:
\[\Forall{x,y}[x,y\in\R\land0<x\leq y]{\Exists{n}[n\in\N]{y\leq n\,x}}\]
Al hacer uso del \hyperref[theo:FT]{teorema 4.3}, se tiene que
\[\Forall{x,y}[x,y\in\st{\R}\land 0<x\leq y]{\Exists{n}[n\in\st{\N}]{y\leq n\,x}}\]
lo que significa, que $\st{\R}$ es arquimediano con respecto a $\st{N}$.

Recordando el \hyperref[theo:noEst]{teorema 4.1}, este nos dice que existen
entidades internas no-estándar. Lo que nos conduce a el siguiente teorema:

\begin{theorem}~
  \begin{enumerate}
    \item Existe $\omega\in\st{\N}$ tal que, para todo $r\in\R$,
          $\omega>|r|$.
    \item Existe $\e\in\st{\R}$ tal que, $0<\e$ y para todo $r\in\R$,
          $\e < |r|$.
  \end{enumerate}
\end{theorem}

\begin{demo}~
  \item Por la propiedad arquimediana de $\R$ con respecto a $\N$, solo hace falta
        demostrar que para todo $n\in\N$, $\omega > n$.
        Defínase la función $\omega$ de $I$ en $\N$ por:
        \[\omega(i) = n\quad (i\in I_n)\]
        Defínase para todo número natural el conjunto $A_n=\set{i}{\omega(i)\leq k}$.
        Nótese que, para todo $n$, $A_n\not\in\U$. En efecto, sea $k\in\N$
        \begin{longderivation}<.8>
            \wff{A_k}\\
          =\\
            \wff{\set{i}{\omega(i)\leq k}}\\
          =\\
            \wff{\bigcup_{i=0}^k I_i}
        \end{longderivation}
        Como $\U$ es un ultrafiltro, si $\bigcup_{i=0}^k I_i\in\U$, por las
        caracterizaciones de los ultrafiltros, al menos uno de los
        elementos de la unión debe estar en $\U$, lo que resultaría en
        que existe un $n\in\N$ tal que $I_n\in\U$. Esto es una
        contradicción por la definición de $\U$ y $\{I_n\}$. Así,
        \begin{longderivation}
            \wff{
              \Forall{n}[n\in\N]{
                \set{i}{\omega(i)\leq n}\not\in\U
              }
            }\\
          \why{$\U$ es un ultrafiltro}\\
            \wff{
              \Forall{n}[n\in\N]{
                \set{i}{\omega(i) > n}\in\U
              }
            }\\
          \equiv\\
            \wff{\Forall{n}[n\in\N]{\omega > n}}
        \end{longderivation}
  \item Por la propiedad arquimediana de $\R$ con respecto a $\N$, solo
        hace falta demostrar que para todo
        $n\in\N$, $\e < \frac{1}{n+1}$. Defínase la función $\e$ de $I$
        en $\R$ por:
        \[\e(i)=\frac{1}{n+1} \quad (i\in I_n)\]
        De la misma manera, se define para todo número natural el
        conjunto $ B_n=\set{i}{\e(i)\geq \frac{1}{k+1}}$. Sea $k\in\N$
        \begin{longderivation}<.8>
            \wff{B_k}\\
          =\\
            \wff{\set{i}{\e(i) \geq \frac{1}{k+1}}}\\
          =\\
            \wff{\bigcup_{i=0}^k I_i}
        \end{longderivation}
        Nuevamente se repite el argumento y se concluye que:
        \[\Forall{n}[n\in\N]{\e < \frac{1}{n+1}}\]
\end{demo}

\begin{definition}
  Un número $a\in\st{\R}$ es llamado \emph{finito} cuando existe
  $r\in\R^+$ tal que $|a| < r$. un número no finito es llamado
  \emph{infinto}.

  Un número $a\in\st{\R}$ es llamado \emph{infinitesimal} cuando, para
  todo $r\in\R^+$, $|a|<r$.

  El conjunto de todos los números finitos se denotará como $M_0$ y el
  conjunto de todos los infinitesimales como $M_1$.
\end{definition}

Nótese que $\R\subseteq M_0$, $M_1\subseteq M_0$ y $\R \cap M_1 = \{0\}$,
esto es, $0$ es el único infinitesimal estándar. Se puede ver que el
recíproco de un número infinito es infinitesimal, y el recíproco de
un número infinitesimal distinto de $0$, es infinito.

\begin{theorem}
  Sea $n\in\st{\N}$, $n$ es finito si y solo si $n$ es un número natural
  estándar. Esto es, $\st{\N} \cap M_0 = \N$.
\end{theorem}

\begin{demo}
  La demostración se hará por doble implicación. Por un lado, si
  $n\in\N$, se puede considerar $n+1$. Este número cumple la definición
  de que $n$ es finito. Por el otro lado, si $n\in\st{\N}$ es finito,
  entonces, existe $r\in\R^+$ tal que $n<r$. En $\R$, se cumple que
  $n<r$ es equivalente a que $n\in\set{m\in\N}{m<r}$, el cual es un
  conjunto no vacío para todo $r>0$. Debido a esto, la propiedad
  se cumple en $\st{\R}$ por el \hyperref[theo:FT]{teorema 4.3}.
\end{demo}

Considerando $M_0$. Este es un subanillo de $\st{R}$, en efecto, es
cerrado bajo la suma y el producto. Asímismo, $M_1$ es un subanillo de
$M_0$. La razón es la misma, sin embargo puede no parecer tan intuitivo.
Considere un par de funciones de $I$ en $\st{R}$ las cuales definen un
par de números infinitesimales, la suma y el producto de estas resulta
en un infinitesimal, cosa que se puede corroborar fácilmente realizando
un proceso como el que se usó para mostrar que existen los
infinitesimales.

\begin{lemma}
  $M_1$ es un ideal maximal de $M_0$.
\end{lemma}
\begin{demo}
  Lo primero es mostrar que $M_1$ es un ideal de $M_0$. En efecto, sean
  $h\in M_1$ y $a\in M_0$. Por definición de infinitesimal:
  \begin{longderivation}<1.1>
      \wff{\Forall{r}[r\in\R^+]{ |h|<r}}\\
    \equiv\\
      \wff{\Forall{r}[r\in\R^+]{ |a\,h| < |a|\,r}}\\
    \why[\To]{$|a|\,r \in \R^+$}\\
      \wff{\Forall{r'}[r'\in\R^+]{|a\,h| < r'}}\\
    \equiv\\
      \wff{a\,h \in M_1}
  \end{longderivation}
  Sea $a\in M_0 - M_1$, por definición, se tiene que:
  \begin{longderivation}<.9>
      \wff{\Exists{r_1,r_2}[r_1,r_2\in\R^+]{r_1 < |a| < r_2}}\\
    \equiv\\
      \wff{\Exists{r_1,r_2}[r_1,r_2\in\R^+]{\frac{1}{r_1} > \frac{1}{|a|} > \frac{1}{r_2}}}\\
    \equiv\\
      \wff{\frac{1}{a}\in M_0 - M_1}
  \end{longderivation}
  Sea $X$ un ideal de $M_0$ tal que $M_1\subset X$. entonces existe un
  $b\in M_0 - M_1$ tal que $b\in X$. Sin embargo, como se mostró,
  $\frac{1}{b}\in M_0$, con lo que, por definición de ideal,
  $b\,\frac{1}{b}=1\in X$, lo que a su vez implica que $X=M_0$. Así,
  queda demostrado que $M_1$ es un ideal maximal de $M_0$.
\end{demo}

Defínase la relación de equivalencia $=_1$ en $\st{\R}$ por:
$a=_1 b \equiv a - b \in M_1$. Considere entonces el anillo cociente
$M_0/M_1$.

\begin{theorem}
  El anillo cociente $M_0/M_1$ is isomorfo al cuerpo $\R$ de los números
  reales estándar.
\end{theorem}

\begin{demo}
  Por un lado, a cada real se le puede asignar una clase de
  equivalencia. En efecto, en cada clase de equivalencia no pueden
  haber dos números reales estándar distintos. Sean $r_1$, $r_2$ dos
  números reales usuales en la misma clase de equivalencia.
  \begin{longderivation}
      \wff{r_1 - r_2 \in M_1 \land |r_1 - r_2| \not= 0}\\
    \To\\
      \wff{ |r_1 - r_2| < |r_1 - r_2| }
  \end{longderivation}
  Esto último es una contradicción, claramente. Por otro lado, para todo
  $a\in M_0$, existe un único número real $r$ tal que $a-r=_1 0$. En
  efecto, sea $a\in M_0$, defínanse $D=\set{x\in\R}{x \leq a}$ y
  $D'=\R - D$. La pareja $(D,D')$ define una cortadura de Dedekin en $\R$.
  Sea $r\in\R$ tal que $r$ define la misma cortadura, se puede ver que
  esto implica que $a=_1 r$, pues de lo contrario $a$ o $r$ estaría en
  alguno de los conjuntos que definen la cortadura. Así, $M_0/M_1$ es
  isomorfo con $\R$.
\end{demo}

De lo descrito anteriormente, se puede entonces tratar con los
elementos de $M_0/M_1$ como si fueran los mismos elementos de $\R$.

\begin{definition}
  El homomorfismo de $M_0$ a $\R$ con kernel $M_1$ será llamado
  \emph{homomorfismo de parte estándar} y se denotará por $\mathrm{s\hspace{-0.25pt}t}$.
\end{definition}

A continuación se mostrarán algunas propiedades de este homomorfismo.

\begin{lemma}[Propiedades] Para todo $a,b\in M_0$
  \begin{enumerate}
    \item $\sd{a+b} = \sd{a} + \sd{b}$
    \item $\sd{a\,b} = \sd{a}\,\sd{b}$
    \item $a \leq b \To \sd{a} \leq \sd{b}$
    \item $\sd{|a|} = |\sd{a}|$
    \item $\sd{\max\{a,b\}} = \max\{\sd{a},\sd{b}\}$
    \item $\sd{\min\{a,b\}} = \min\{\sd{a},\sd{b}\}$
    \item $a\in M_1 \equiv \sd{a} = 0$
    \item $a\in\R \To \sd{a} = a$
    \item $\sd{a} \geq 0 \To |a| =_1 \sd{a}$
    \item $a=_1 b \equiv \sd{a} = \sd{b}$
  \end{enumerate}
\end{lemma}

Se puede ver esto debido a cómo se definió el homomorfismo, y el hecho
de que en cada clase de equivalencia de $M_0/M_1$, hay exáctamente un
número real estándar.
\begin{definition}
  Las clases de equivalencia de $M_0$ respecto a $M_1$ serán llamadas
  \emph{mónadas} de los números estándar que las determinan, y serán
  denotadas, para un $r\in\R$, como $\mu(r)$. Por ejemplo,
  $\mu(0) = M_1$.
\end{definition}

Una breve introducción al sistema numérico no estándar:

Siguiendo el hecho de que los números complejos son construidos usando
parejas ordenadas, partiendo de la misma idea, se construye la
superestructura $\widehat{\R\times\R}$ definida por $\R\times\R$. De la
misma forma se llega al conjunto $\st{\left(\widehat{\R\times\R}\right)}$.
La diferencia en este caso, es la estructura algebráica a considerar.

Se toma entonces $\st{\C} = \st{\R}\times\st{\R}$, el cual hereda las
propiedades de $\C$. En $\st{\C}$, se puede definir el concepto de
\emph{finito} e \emph{infinitesimal} de la siguiente manera:
\begin{definition}
  Un número $z\in\st{\C}$ es llamado \emph{finito} cuado existe
  $r\in\R^+$ tal que $\|z\| < r$. Un número no finito es llamado
  \emph{infinito}.

  Un número $z\in\st{\C}$ es llamado \emph{infinitesimal} cuando, para
  todo $r\in\R^+$, $\|z\| < r$.
\end{definition}

Nótese que, bajo estas definiciones, es fácil ver que un número complejo
$z = a + b\,i$ es infinitesimal si y solo si $a,b$ son infinitesimales.
