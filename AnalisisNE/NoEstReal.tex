\section{El sistema no estándar \texorpdfstring{$\st{\R}$}{R}}

Por el \hyperref[theo:FT]{teorema 4.3}, las mismas propiedades que aplican
en $\R$, aplican en $\st{\R}$ hasta donde apliquen en ambos simultaneamente.
Para simplificar la notación, se mantendrá el mismo símbolo para
denotar operaciones usuales en $\R$ y en $\st{\R}$.
Por ejemplo, en $\st{\R}$, $a + b = c$ es equivalente a
$\set{i}{a(i) + b(i) = c(i)}\in\U$. De la misma manera para la
resta, multiplicación. De igual forma, en $\st{\R}$, $a \leq b$ es
equivalente a $\set{i}{a(i)\leq b(i)}\in\U$. Un ejemplo de la
aplicación del \hyperref[theo:FT]{teorema 4.3} es la siguiente sentencia.
\[\Forall{x,y}[x,y\in\R]{x<y \;\lor\; x>y \;\lor\; x=y}\]
Dado que esta sentencia es verdadera en $\R$, se tiene entonces que la
sentencia
\[\Forall{x,y}[x,y\in\st{\R}]{x<y \;\lor\; x>y \;\lor\; x=y}\]
es verdadera en $\st{\R}$.

Nótese que de la misma manera, es inmediato que el elemento neutro del
producto en $\st{\R}$ es $\st{1}$. Por conveniencia, se omitirá la notación
$({^*})$ para elementos de $\st{R}$ y por lo mencionado antes,
se tomará $\R \subset \st{R}$.

Aclarando más acerca de algunas funciones comunes en $\R$, el valor
absoluto, definido por:
\[
  |r| =
  \begin{siseq}[\{,\}]
    r   &&\quad (r > 0)\\
    0   &&\quad (r = 0)\\
    -r  &&\quad (r < 0)
  \end{siseq}
\]
es una función de $\R$ en $\R^{\geq 0}$. Debido a la incorporación
del sistema formal de $\R$ en $\st{\R}$, se tiene una función
correspondiente en $\st{\R}$, la cual iría de $\st{\R}$ a
$\st{\R^{\geq 0}}$. Por el \hyperref[theo:FT]{teorema 4.3}, esta función
cumple que:
\[
  \st{|r|} =
  \begin{siseq}[\{,\}]
    r   &&\quad (r > 0)\\
    0   &&\quad (r = 0)\\
    -r  &&\quad (r < 0)
  \end{siseq}
\]
por supuesto, para $r\in\st{\R}$. Debido a esto, se denotará el valor
absoluto en $\st{\R}$ de la forma usual. Por la misma razón, se denotará
de manera usual las funciones $\min$ y $\max$.

Sea $S\subseteq\R$, al pasar a $\st{\R}$, $\st{S}$ corresponde a un
subconjunto de $\st{\R}$, el cual mantienen las mismas propiedades que $S$
hasta donde tenga sentido en ambos simultaneamente. Así como se definió
la superestructura $\Rh$, se puede definir una para $S$, $\widehat{S}$, y,
de igual forma, se puede definir una ultrapotencia de $\widehat{S}$ denotada
por $\st{(\widehat{S})}$. De igual forma, por lo mencionado anteriormente,
se tomará $S\subseteq\st{S}$. Nótese, que al dejar la notación $({^*})$,
por el \hyperref[lema:stR]{lema 4.1}, se tiene que $S=\st{S}$ si y solo
si $S$ es finito.

Un resultado del álgebra nos dice que todo cuerpo arquimediano es
isomorfo con un subcuerpo de $\R$. Debido a que $\R\subset\st{R}$, se
tiene que $\st{\R}$ no es arquimediano. Sin embargo, $\st{R}$ cumple
las mismas propiedades que $\R$, lo que parecería ser una contradicción.
La razón de esto, es que la propiedad arquimediana es relativa a un
conjunto, en el caso de $\R$, es con $\N$. explícitamente, la propiedad
es la siguiente:
\[\Forall{x,y}[x,y\in\R\land0<x\leq y]{\Exists{n}[n\in\N]{y\leq n\,x}}\]
Al hacer uso del \hyperref[theo:FT]{teorema 4.3}, se tiene que
\[\Forall{x,y}[x,y\in\st{\R}\land 0<x\leq y]{\Exists{n}[n\in\st{\N}]{y\leq n\,x}}\]
lo que significa, que $\st{\R}$ es arquimediano con respecto a $\st{N}$.

Recordando el \hyperref[theo:noEst]{teorema 4.1}, este nos dice que existen
entidades internas no-estándar. Lo que nos conduce a el siguiente teorema:

\begin{theorem}~
  \begin{enumerate}
    \item Existe $\omega\in\st{\N}$ tal que, para todo $r\in\R$,
          $\omega>|r|$.
    \item Existe $\e\in\st{\R}$ tal que, $0<\e$ y para todo $r\in\R$,
          $\e < |r|$.
  \end{enumerate}
\end{theorem}

\begin{demo}~
  \item Por la propiedad arquimediana de $\R$ con respecto a $\N$, solo hace falta
        demostrar que para todo $n\in\N$, $\omega > n$.
        Defínase la función $\omega$ de $I$ en $\N$ por:
        \[\omega(i) = n\quad (i\in I_n)\]
        Defínase para todo número natural el conjunto $A_n=\set{i}{\omega(i)\leq k}$.
        Nótese que, para todo $n$, $A_n\not\in\U$. En efecto, sea $k\in\N$
        \begin{longderivation}<.8>
            \res{A_k}\\
          =\\
            \res{\set{i}{\omega(i)\leq k}}\\
          =\\
            \res{\bigcup_{i=0}^k I_i}
        \end{longderivation}
        Como $\U$ es un ultrafiltro, si $\ds\bigcup_{i=0}^k I_i\in\U$, por las
        caracterizaciones de los ultrafiltros, al menos uno de los
        elementos de la unión debe estar en $\U$, lo que resultaría en
        que existe un $n\in\N$ tal que $I_n\in\U$. Esto es una
        contradicción por la definición de $\U$ y $\{I_n\}$. Así,
        \begin{longderivation}
            \res{
              \Forall{n}[n\in\N]{
                \set{i}{\omega(i)\leq n}\not\in\U
              }
            }\\
          \why{$\U$ es un ultrafiltro}\\
            \res{
              \Forall{n}[n\in\N]{
                \set{i}{\omega(i) > n}\in\U
              }
            }\\
          \equiv\\
            \res{\Forall{n}[n\in\N]{\omega > n}}
        \end{longderivation}
  \item Por la propiedad arquimediana de $\R$ con respecto a $\N$, solo
        hace falta demostrar que para todo
        $n\in\N$, $\ds\e < \frac{1}{n+1}$. Defínase la función $\e$ de $I$
        en $\R$ por:
        \[\e(i)=\frac{1}{n+1} \quad (i\in I_n)\]
        De la misma manera, se define para todo número natural el
        conjunto $\ds B_n=\set{i}{\e(i)\geq \frac{1}{k+1}}$. Sea $k\in\N$
        \begin{longderivation}<.8>
            \res{B_k}\\
          =\\
            \res{\set{i}{\e(i) \geq \frac{1}{k+1}}}\\
          =\\
            \res{\bigcup_{i=0}^k I_i}
        \end{longderivation}
        Nuevamente se repite el argumento y se concluye que:
        \[\Forall{n}[n\in\N]{\e < \frac{1}{n+1}}\]
\end{demo}

\begin{definition}
  Un número $a\in\st{\R}$ es llamado \emph{finito} cuando existe
  $r\in\R^+$ tal que $|a| < r$. un número no finito es llamado
  \emph{infinto}.

  Un número $a\in\st{\R}$ es llamado \emph{infinitesimal} cuando, para
  todo $r\in\R^+$, $|a|<r$.

  El conjunto de todos los números finitos se denotará como $M_0$ y el
  conjunto de todos los infinitesimales como $M_1$.
\end{definition}

Nótese que $\R\subseteq M_0$, $M_1\subseteq M_0$ y $\R \cap M_1 = \{0\}$,
esto es, $0$ es el único infinitesimal estándar. Se puede ver que el
recíproco de un número infinito es infinitesimal, y el recíproco de
un número infinitesimal distinto de $0$, es infinito.

\begin{theorem}
  Sea $n\in\st{\N}$, $n$ es finito si y solo si $n$ es un número natural
  estándar. Esto es, $\st{\N} \cap M_0 = \N$.
\end{theorem}

\begin{demo}
  Si $n$ es finito, entonces existe $r\in\R$ tal que $n< |r|$.
\end{demo}
