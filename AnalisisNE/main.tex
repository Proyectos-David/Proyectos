\documentclass[leqno, 12pt]{article}

\usepackage{logicDG, calcDG}
\usepackage{amsfonts}
\usepackage[hidelinks]{hyperref}
\usepackage{lmodern}
\usepackage[T1]{fontenc}
\usepackage{setspace}
\usepackage[spanish, es-noquoting, es-lcroman, es-noshorthands]{babel}
\usepackage{fancyhdr}
\usepackage{graphicx}
\usepackage{mathrsfs}
\usepackage{amssymb}
\usepackage{dsfont}
\usepackage{upgreek}
\usepackage{amsthm}
\usepackage{geometry}
\geometry{
  left=2cm,
  right=2cm,
  bottom=4cm,
  a4paper
}
\usepackage{pdfrender}
\pdfrender{TextRenderingMode=2,LineWidth=.5pt}

\pagestyle{fancy}
\fancyhf{}
\setlength{\headheight}{70.38103pt}

% ~~~~~~ Autor/es ~~~~~~ %
\rhead{\textit{David G.}}
% ~~~~~~ Esquina ~~~~~~ %

\lhead{\includegraphics[width = 4cm]{\logo}}
\lfoot{Página \thepage}

% ~~~~~~ Título ~~~~~~ %
\rfoot{Análisis N.E.}
% ~~~~~~ Esquina ~~~~~~ %
\renewcommand{\headrule}{\hbox to \headwidth{\color{rojoEci}\leaders\hrule height \headrulewidth\hfill}}
\renewcommand{\footrulewidth}{0.4pt}

\hyphenpenalty=10000

% ~~~~ Ruta a Imágenes ~~~~ %
\graphicspath{{./AnalisisNE/}}
\newcommand{\logo}{"logo-eci-normal.png"}
% ~~~~ Ruta a Imágenes ~~~~ %


% ~~~~~~ Autor/es y Titulo ~~~~~~ %
\newcommand{\titlename}{Análisis No Estandar}
\renewcommand{\author}{{David Gómez}}
% ~~~~~~ Autor/es y Titulo ~~~~~~ %

\definecolor{rojoEci}{RGB}{225, 70, 49}

% Enumi func

\renewcommand{\labelenumi}{(\roman{enumi})}

% ~~~~ Nombres de conjuntos y demás ~~~~~ %
\newtheoremstyle{definition}%
{10pt} % espaciado arriba
{20pt} % espaciado abajo
{\normalfont} % fuente del cuerpo
{} % indentación
{\bfseries} % fuente del título
{:} % puntuación después del título
{5pt} % espacio después del título
{}
\newtheoremstyle{theorem*}%
{10pt}
{10pt}
{\normalfont}
{}
{\bfseries}
{:}
{\newline}
{}

\newtheoremstyle{proof}%
{5pt}
{10pt}
{\normalfont}
{\parindent}
{\itshape\bfseries}
{:}
{5pt}
{}


\theoremstyle{definition}
\newtheorem{definition}{Definición}[section]
\newtheorem{theorem}{Teorema}[section]
\newtheorem{lemma}{Lema}[section]
\newtheorem*{note}{Nota}

\theoremstyle{theorem*}
\newtheorem{lemma*}{Lema}[section]
\newtheorem{theorem*}{Teorema}[section]

\theoremstyle{proof}
\newtheorem*{demo}{Demostración}


% ~~~~ Nombres de conjuntos y demás ~~~~~ %
\NewDocumentCommand{\F}{}{\mathcal{F}}
\NewDocumentCommand{\U}{}{\mathscr{U}}
\NewDocumentCommand{\N}{}{\mathbb{N}}
\NewDocumentCommand{\J}{}{\mathds{J}}
\NewDocumentCommand{\R}{}{\mathds{R}}
\NewDocumentCommand{\Pts}{}{\mathscr{P}}
\NewDocumentCommand{\Rh}{}{\widehat{\R}}
\NewDocumentCommand{\e}{}{\upvarepsilon}
\NewDocumentCommand{\B}{}{\mathcal{B}}
\NewDocumentCommand{\equ}{}{=_\U}
\NewDocumentCommand{\inu}{}{\in_\U}
\NewDocumentCommand{\st}{m}{\prescript{*}{}{#1}}
\NewDocumentCommand{\IRh}{}{\prescript{I}{}{\Rh}}

\NewDocumentCommand{\dom}{}{\text{dom}\;}
\NewDocumentCommand{\ran}{}{\text{ran}\;}
\NewDocumentCommand{\set}{m m}{\left\{#1\,\middle|\,#2\right\}}

\renewcommand{\labelitemi}{$\bullet$}

\doublespacing
\begin{document}
\begin{titlepage}
    \begin{center}
        \vspace{1cm}

        \textbf{\Huge{\titlename}}

        \vspace{1.5cm}

        \textbf{\large{\author}}

        \vspace{3cm}

        \includegraphics[width=0.8\textwidth]{\logo}
        
        \vfill

        Física de Calor y Ondas

        Escuela Colombiana de Ingeniería Julio Garavito

        \today
    \end{center}
\end{titlepage}

\clearpage
\tableofcontents

\section{Introduccion}

Cuando Newton y Leibniz trabajaron en la fundamentación del cálculo, una
de sus diferencias fue la definición de límite. Mientras Newton lo
definió de la forma en la que se ha enseñado principalmente, la
definición $\e$ y $\delta$. Leibniz, lo definia de una forma que
incluso parece una versión más amigable que la de Newton. Leibniz
consideraba números infinitamente pequeños, de tal forma que fueran
menores que cualquier número positivo pero mayores a $0$. Junto con
números infinitamente grandes, mayores que cualquier número positivo.

El problema de esta definición, se encontró cuando se intentó fundamentar
formalmente. Cosa que Leibniz ni sus discípulos lograron demostrar.
La definición de Newton recurre a los mismos números reales ya usados.
La definición de Leibniz, recurre a una nueva especie de números, los cuales
deben ser comparables y se deben poder operar con los reales. La idea entonces
con esta nueva especie de números, es poder operar con estos, para
posteriormente tomar el resultado y recuperar la información que interesa,
la que corresponde a un valor real estándar. Esta nueva especie de números
resulta tener aplicaciones en más áreas que el cálculo de límites, sin
embargo, no hacen parte del objetivo de este proyecto, el cual consta de
presentar esta idea en relación al análisis estándar, específicamente,
el análisis diferencial.

Los pasos para la construcción de estos números, recurre a los filtros,
un objeto de la teoría de conjuntos sobre el que se hablará en el
documento. Con estos, se puede lograr la construcción de esta nueva especie
de números sobre los reales. 


\section{Filtros}

Los filtros, como se mencionó, son objetos de la teoría de conjuntos,
que, como su nombre indica, filtran de forma análoga a lo que
puede hacer un colador. Para esta sección, se considerará $I$ como un
conjunto no vacío, esto último es necesario para la definición de filtro.

\begin{definition}\label{def:filtro}
  Un filtro $\F$ sobre $I$ es un conjunto no vacío de subconjuntos de $I$
  el cual cumple las siguientes características:
  \begin{enumerate}
    \item $\varnothing \not\in \F$
    \item $\Forall{A,B}[A,B \in \F]{A\cap B \in \F}$
    \item $\Forall{A,B}[A\in \F \land A \subseteq B]{B \in \F}$
  \end{enumerate}

  Para esta sección, la letra $\F$ denotará un filtro sobre $I$.
\end{definition}

Por ejemplo, considere el conjunto $X=\{a,b,c\}$. Un filtro $G$ sobre $X$
puede ser 
\[G = \{\{a\}, \{a,b\}, \{a,c\}, \{a,b,c\}\}\]
Nótese que por la definición de filtro, el conjunto sobre el que este se
define, siempre debe ser un elemento del filtro.

\begin{definition}
  Sea $\mathcal{B}$ una colección no vacía de subconjuntos de $I$ tal
  que
  \[
    \Forall{A,B}[A,B \in \mathcal{B}]{
      \Exists{C}[C\in\mathcal{B}]{C \subseteq A \cap B}
      }
  \]
  Se dice entonces, que $\mathcal{B}$ es una base para algún filtro sobre $I$.
  Nótese que toda base genera un filtro, como puede ser
  \[\F = \{A \subseteq I \,|\, \Exists{B}[B\in\mathcal{B}]{B\subseteq A}\}\]
\end{definition}

\begin{definition}
  Sea $\mathcal{S}$ una colección de subconjuntos de $I$ tal que
  $\mathcal{S}$ cumpla la propiedad de las intersecciones finitas. Esto es,
  para toda colección finita de elementos de $S$, su intersección es
  distinta de $\varnothing$. Se dice entonces que $\mathcal{S}$ es una
  sub-base para algun filtro sobre $I$.
\end{definition}

Así como un colador puede ser más fino que otro, en el sentido que
deja pasar menos cosas, también se pueden comparar a los filtros definidos
sobre un conjunto.

\begin{definition}[Relación de orden]
  Sea $\mathscr{F}$ el conjunto de filtros sobre $I$, la relación de contenencia
  ordena parcialmente a $\mathscr{F}$, y entre dos elementos se define de la siguiente
  manera:

  Un filtro $\F_1$ es \emph{más fino} que un filtro $\F_2$ cuando $\F_2 \subseteq \F_1$.
\end{definition}

Como ejemplo, volvamos al conjunto $X$ definido para el ejemplo anterior. Sean $G_1$, $G_2$
filtros sobre $X$, donde
\[G_1 = \{\{a\}, \{a,b\},\{a,c\},\{a,b,c\}\}\text{ , } G_2 = \{\{a,b,c\}\}\]
Se puede ver que $G_2 \subseteq G_1$. Sin embargo, hay filtros que no
se pueden comparar, incluso en conjuntos tan simples como podría ser
$X$. Si consideramos ahora un nuevo filtro\\
$G_3 = \{\{b\}, \{b,a\},\{b,c\},\{a,b,c\}\}$, está claro que no se pueden comparar
$G_3$ y $G_1$.

\begin{definition}
  Sea $\mathscr{F}$ el conjunto de filtros definidos sobre $I$. Se define
  el concepto de ultrafiltro como un elemento maximal de $\mathscr{F}$ con
  la relación de orden definida anteriormente. Simbólicamente,
  un filtro $\U$ sobre $I$, es un ultrafiltro cuando
  \[\Forall{\F}[\F\in\mathscr{F}\land\F \not=\U]{\U\not\subset\F}\]

  Para esta sección, la letra $\U$ denotará un ultrafiltro sobre $I$.\\
  Como ejemplo, se pueden tomar los filtros $G_1$ y $G_3$ de antes.
\end{definition}
\pagebreak
\begin{theorem}[Caracterizaciones]
  Sea $\U$ un ultrafiltro sobre $I$. $\U$ es un ultrafiltro si y solo si:
  \begin{enumerate}
    \item $\Forall{A}[A\subseteq I]{A\in\U \not\equiv I-A \in \U}$
    \item Sean $n\in\J$, $\{A_k\}$ una colección de $n$ subconjuntos de $I$ tal que
          \[\bigcup_{k=1}^n A_k \in \U\]
          entonces
          \[\Exists{k}[k\leq n]{A_k\in\U}\]
  \end{enumerate}
\end{theorem}

\begin{demo}[i] Por un lado, se va a mostrar que si $\U$ es un ultrafiltro,
  entonces se tiene la propiedad. Por contradicción, se va a suponer
  que $\U$ es un ultrafiltro, y se tiene un subconjunto $A$ de $I$, tal
  que $A \not\in \U \,\land\, I-A\not\in\U$. Una forma equivalente de escribir
  el punto (iii) de la \hyperref[def:filtro]{definición de filtro} es
  \[\Forall{A,B}[A\subseteq B]{B\not\in\F \To A\not\in\F}\]
  Con esto se puede ver que ningún subconjunto, tanto de $A$ como de $I-A$
  es elemento de $\U$.

  Sea $\U_2 = \{B \subseteq I \,|\, B \cup A \in \U\}$, se puede ver que
  $\U \subset \U_2$, en efecto
  \begin{longderivation}<1>
      \res{ B \in \U }\\
    \To\\
      \res{ B \cup A \in \U }\\
    \equiv\\
      \res{ B \in \U_2 }
  \end{longderivation}

  No son iguales, pues, por ejemplo, $I-A \cup A = I$, $I-A \in \U_2$.\\
  Hace falta ver que $\U_2$ es un filtro.
  \begin{enumerate}
    \item Por contradicción, es inmediato:
          \begin{longderivation}<0.7>
              \res{ \varnothing \in \U_2 }\\
            \equiv\\
              \res{ A \in \U }
          \end{longderivation}
    \item Sean $X, Y \in \U_2$
          \begin{longderivation}<0.7>
              \res{ X \cap Y \in \U_2 }\\
            \equiv\\
              \res{ (X \cap Y) \cup A \in \U }\\
            \equiv\\
              \res{ (X \cup A) \cap (Y \cup A) \in \U }\\
          \end{longderivation}
          Como $X,Y\in\U_2$, se tiene que ambos términos de la intersección
          son elementos de $\U$. Como $\U$ es un filtro, por definición, esta
          intersección también es elemento de $\U$.
    \item Sean $X\in\U_2$ y $Y\supseteq X$
          \begin{longderivation}
              \res{ X \subseteq  Y }\\
            \To\\
              \res{ X \cup A \subseteq Y \cup A }\\
            \why[\To]{$X\in\U_2$ y $\U$ es un filtro}\\
              \res{ Y \cup A \in \U }\\
            \equiv\\
              \res{ Y \in \U_2 }
          \end{longderivation}
  \end{enumerate}

  Entonces, se tiene que $\U \subset \U_2$ y $\U_2$ es un filtro sobre
  $I$. Lo cual contradice la hipótesis de que $\U$ es un ultrafiltro.

  Por el otro lado, de igual forma por contradicción, se va a suponer que
  $\U$ es un filtro con la propiedad (i) y que $\U$ no es un ultrafiltro.

  Como $\U$ no es un ultrafiltro, por el lema del ultrafiltro, existe
  un filtro $\U_2$ tal que $\U \subset \U_2$.
  \begin{longderivation}
      \res{ \U_2 - \U \not= \varnothing }\\
    \equiv\\
      \res{ \Exists{A}{A \in \U_2 - \U} }\\
    \why{ $\U$ cumple (i) y $A \not\in \U$ }\\
      \res{ \Exists{A}[A\in\U_2-\U]{I-A\in\U} }\\
    \why[\To]{$\U\subset\U_2$}\\
      \res{ \Exists{A}[A\in\U_2-\U]{I-A\in \U_2} }\\
    \equiv\\
      \res{ \Exists{A}[A\not\in\U]{A\in\U_2 \land I-A\in\U_2} }\\
    \why[\To]{Definición de filtro}\\
      \res{ \Exists{A}[A\not\in\U]{\varnothing\in\U_2} }
  \end{longderivation}
  Esto último contradice la definición de filtro, mostrando así que la
  suposición de que $\U$ no es un ultrafiltro, es incorrecta.
\end{demo}
\begin{demo}[ii]
  Como ya se demostró la equivalencia entre la definición de
  ultrafiltro y (i), se va a usar esta última.

  Por un lado, se va a mostrar que (i) $\To$ (ii).
  Sean $\U$ un ultrafiltro sobre $I$, $n\in\N$ y $\{A_k\}$ una colección
  de conjuntos de $I$ tal que su unión esté en $\U$. Se va a mostrar
  que existe un elemento de esta colección que está en $\U$. Para
  esto, se va a suponer que no existe dicho elemento, es decir:
  \begin{longderivation}
      \res{ \bigcup_{k=1}^{n} A_k \in \U \land
            \Forall{k}[1\leq k\leq n]{A_k \not\in \U} }\\
    \why{(i)}\\
      \res{ \bigcap_{k=0}^{n} (I - A_k) \not\in\U \land
            \Forall{k}[1\leq k\leq n]{I - A_k \in \U} }
  \end{longderivation}
  Esta última expresión es contradictoria, pues por definición, las
  intersecciones finitas de elementos de filtros son elementos de los
  filtros (\hyperref[def:filtro]{propiedad (ii) de la definición}).
  Justamente se tiene que todos los elementos de una intersección finita
  son elementos de $\U$, y su intersección no es elemento del filtro, Así
  (i) $\To$ (ii).

  Por otro lado, se va a mostrar que (ii) $\To$ (i). Sea $\U$ un
  ultrafiltro sobre $I$. Suponiendo que se cuenta con (ii). Sean
  $F \in \U$, $A_1 = F$ y $A_2 = I-F$. Como $\ds\bigcup_{k=1}^2 A_k = I$
  y $I\in\U$, por (ii), al menos uno de los $A_k$ debe pertenecer a $\U$.
  Por definición de filtro, no puede ser que ambos sean elementos de $\U$
  y (ii) garantiza que no puede ser que ninguno sea elemento de $\U$. Es decir,
  $A_1\in\U \not\equiv A_2\in\U$, que reemplazando, es (i). Así
  (ii) $\To$ (i).
\end{demo}

\begin{definition}
  Un filtro $\F$ es llamado $\delta$-incompleto cuando existe una colección
  contable de subconjuntos de $I$, tal que todos sus elementos estén en $\U$ y
  su intersección no. Es decir, si existe $\{F_n\}_{n\in\N}$ tal que,
  para todo $n\in\N$, $F_n\in\U$ y $\ds\bigcap_{n\in\N}F_n \not\in \U$. Un filtro
  es llamado $\delta$-completo cuando no es $\delta$-incompleto.
\end{definition}

\begin{theorem}[Caracterización]
  Un ultrafiltro $\U$ es $\delta$-incompleto si y solo si, existe
  $\{I_n\}$, una partición contable de $I$, tal que, para todo $n$,
  $I_n\not\in\U$.
\end{theorem}

\begin{demo}
  Sean $\U$ un ultrafiltro $\delta$-incompleto sobre $I$ y
  $\{F_n\}_{n\in\J}$ una colección de subconjuntos de $I$, la cual
  cumple la definición de $\delta$-incompleto en $\U$. Se tiene
  entonces que
  \begin{longderivation}
      \res{ \Forall{n}[n\in\J]{F_n\in\U} \land \bigcap_{n\in\J} F_n \not\in \U}\\
    \why{ $\U$ es un ultrafiltro }\\
      \res{ \Forall{n}[n\in\J]{I - F_n \not\in \U} \land
            \bigcup_{n\in\J} (I - F_n) \in \U }
  \end{longderivation}

  Sea $\{B_n\}_{n\in\J}$ una colección de subconjuntos de $I$, definida por
  $\ds B_n = \bigcup_{k=1}^n (I - F_k)$. Nótese que, los elementos de
  $B_n$ están contenidos consecutivamente, esto es
  \[\Forall{n,m}[n,m\in\J \land n \leq m]{B_n \subseteq B_m}\]

  Sea $\{I_n\}_{n\in\N}$ una colección de subconjuntos de $I$ definida por
  \[
    \begin{siseq}[\{,\}]
      I_0     &= \bigcap_{k\in\J} F_k\\
      I_{n+1} &= B_{n+1} - B_n
    \end{siseq}
  \]

  Se va a mostrar que $\{I_n\}_{n\in\N}$ es una partición de $I$ tal que,
  para todo $n$, $I_n\not\in\U$. Es evidente que $I_0\not\in\U$, pues se
  tiene en la definición de $\{F_n\}$.
  \begin{longderivation}
      \res{ I_{n+1} }\\
    =\\
      \res{ B_{n+1} - B_{n} }\\
    =\\
      \res{ \bigcup_{k=1}^{n+1}(I-F_k) - \bigcup_{k=1}^{n}(I-F_k)}\\
    =\\
      \res{ \left((I-F_{n+1}) \cup \bigcup_{k=1}^{n}(I-F_k)\right)
            \cap \left(I-\bigcup_{k=1}^{n}(I-F_k)\right) }\\
    =\\
      \res{ \left((I-F_{n+1}) \cap
              \left(I-\bigcup_{k=1}^{n}(I-F_k)\right)
            \right)
              \cup
            \left(
              \bigcup_{k=1}^{n}(I-F_k) \cap
              \left(I-\bigcup_{k=1}^{n}(I-F_k)\right)
            \right) }\\
    =\\
      \res{ (I-F_{n+1}) \cap \bigcap_{k=1}^{n}F_k }
  \end{longderivation}

  Se puede ver que, para todo $n\in\J$, $I_n \subseteq I - F_n$. Por la
  \hyperref[def:filtro]{definición de filtro}, nuevamente el punto (iii),
  para todo $k\in\J$, nunguno de los subconjuntos de $I_k$ es elemento de $\U$,
  así, para todo $n\in\N$, $I_n\not\in\U$.
  
  Por como se definió $I_n$ para $n\in\J$, se puede ver que sus
  elementos son disjuntos. Asimismo, al ser estos subconjuntos de
  $I - F_n$ respectivamente, son disjuntos con $I_0$. Esto
  último se verá de forma más clara al corroborar que la unión de la
  colección sea efectivamente $I$.
  \begin{longderivation}
      \res{ \bigcup_{n\in\J}I_n }\\
    =\\
      \res{ \bigcup_{n\in\J}B_n }\\
    =\\
      \res{ \bigcup_{n\in\J}(I - F_n) }
  \end{longderivation}

  Para finalizar la demostración, se tomará el complemento de esta unión,
  la cual es disjunta con dicha unión y además, su unión es $I$.
  \begin{longderivation}
      \res{ I - \bigcup_{n\in\J}(I - F_n) }\\
    =\\
      \res{ \bigcap_{n\in\J}F_n }\\
    =\\
      \res{ I_0 }
  \end{longderivation}
\end{demo}

Nótese que la existencia de un ultrafiltro $\delta$-incompleto sobre
un conjunto, requiere que dicho conjunto sea infinito.

\begin{theorem}
  Sea $I$ un conjunto infinito, entonces, existe un ultrafiltro
  $\delta$-incompleto sobre $I$.
\end{theorem}

\begin{demo}
  Sea $\{I_n\}_{n\in\N}$ una partición contable de $I$. Defínanse $A$ y $\B$
  como:
  \begin{align*}
    A &= \{I_n \,|\, n\in\N\} \cup \{S \subseteq I \,|\, S \text{ es finito}\}\\
    \B &= \{I - S \,|\, S \in A\}
  \end{align*}

  Si $\B$ es una sub-base, entonces existe un filtro asociado $\F$, el cual
  debe estar contenido en un ultrafiltro $\U$ por el lema del ultrafiltro. por
  como se definió $\B$, se tendría que, para todo $n$, $I-I_n\in\U$.

  La condición necesaria para que $\B$ sea una sub-base, es que cumpla la
  propiedad de las intersecciones finitas.
  Sean $P_1$, $P_2$ subconjuntos finitos de $\N$, y $\{H_n\}_{n\in P_2}$ una
  colección de subconjuntos finitos de $I$.

  \begin{longderivation}
      \res{ \bigcap_{n\in P}(I - I_n) \cap \bigcap_{n\in P_2} (I - H_n) }\\
    =\\
      \res{ I - \left(\bigcup_{n\in P_1} I_n \cup \bigcup_{n\in P_2} H_n\right) }
  \end{longderivation}

  Nótese que, $\ds\bigcup_{n\in I - P_1} I_n$ es un subconjunto infinito de $I$.
  También, se tiene que $\ds\bigcup_{n\in P_2}H_n$ es un subconjunto finito de $I$.
  Con esto se demuestra que la intersección presentada es distinta de $\varnothing$.
  Con lo cual, $\B$ es una sub-base.
\end{demo}


\section{Superestructura \texorpdfstring{$\Rh$}{R}}

El objetivo de la superestructura es poder agrupar todas las relaciones
definidas en un conjunto. En este caso $\R$. Debido a que las
$n$-uplas se definen recursivamente de la siguiente manera:
\begin{align*}
  (a) &= a\\
  (a,b) &= \{\{a\}, \{a,b\}\}\\
  (a_1, \dots, a_n, a_{n+1}) &= ((a_1, \dots, a_n), a_{n+1})
\end{align*}
Se puede demostrar que $(a,b) \in A\times B \To
(a,b) \in \Pts{\Pts{A \cup B}}$. Debido a esto, es posible definir un
conjunto el cual contenga todas las relaciones entre números reales. Se
define la colección $\R_n$ de la siguiente manera:
\[
  \begin{syseq}(\{,\})
    \R_0 &= \R\\
    \R_{n+1} &= \Pts{\bigcup_{k=1}^n \R_k}
  \end{syseq}
\]

Se define entonces, la superestructura $\Rh$ por
\[\Rh = \bigcup_{n\in\N} R_n\]
Los elementos de $\Rh$ serán llamados \emph{entidades} y los elementos
de $\R$ serán llamados \emph{individuos}.

\begin{lemma}[Propiedades de $\R_n$]\label{lema:Rn}~
  \begin{enumerate}
    \item $\Forall{n,m}[n,m\in\N \land n\leq m]{\R_{n+1} \subseteq \R_{m+1}}$
    \item $\Forall{n}[n\in\N]{\bigcup_{k=0}^n \R_k = \R_0 \cup \R_n}$
    \item $\Forall{n,m}[n,m\in\N \land n \leq m]{R_n \in \R_{m+1}}$
    \item $\Forall{n,x,y}[n\in\N \land y\in\R_{n+1} \land x\in y]{x\in\R_0\cup\R_n}$
  \end{enumerate}
\end{lemma}
\begin{demo}~
  \begin{enumerate}
    \item Sean $n,m\in\N$ tales que $n\leq m$.
          \begin{longderivation}<.9>
              \wff{x\in\R_{n+1}}\\
            \equiv\\
              \wff{x\subseteq \bigcup_{k=0}^n \R_k}\\
            \To\\
              \wff{x\subseteq \bigcup_{k=0}^{m}\R_k}\\
            \equiv\\
              \wff{x\in\R_{m+1}}
          \end{longderivation}
    \item Sea $n\in\N$, si $n=0$ no hay algo que demostrar. Para $n\in\J$:
          \begin{longderivation}<.9>
              \wff{x\in\bigcup_{k=0}^n \R_k}\\
            \equiv\\
              \wff{ x\in \R_0 \lor x\in\bigcup_{k=1}^n \R_k }\\
            \why{(i)}\\
              \wff{ x\in\R_0 \lor x\in\R_n }\\
            \equiv\\
              \wff{x\in\R_0\cup\R_n}
          \end{longderivation}
    \item Sean $n,m \in \N$ tales que $n \leq m$.
          \begin{longderivation}<.9>
              \wff{ \R_n \subseteq \bigcup_{k=0}^m\R_k }\\
            \equiv\\
              \wff{\R_n \in \Pts{\bigcup_{k=0}^m\R_k}}\\
            \equiv\\
              \wff{ \R_n \in \R_{m+1} }
          \end{longderivation}
    \item Sean $n\in\N$, $y\in\R_{n+1}$ y $x\in y$.
          \begin{longderivation}<.9>
              \wff{ x \in y \land y \in \R_{n+1} }\\
            \why{(i)}\\
              \wff{ x \in y \land y \subseteq \R_0\cup\R_n }\\
            \To\\
              \wff{ x \in \R_0\cup\R_n }
          \end{longderivation}
  \end{enumerate}
\end{demo}

\section{Ultraproducto de \texorpdfstring{$\Rh$}{R}}

Sean $I$ un conjunto infinito, $\U$ un ultrafiltro $\delta$-incompleto
sobre $I$, y $\{I_n\}_{n\in\N}$ una partición contable la cual cumple
la definición de $\delta$-incomleto en $\U$.

Considere ahora el conjunto $\IRh$, de las funciones de $I$ en $\Rh$.
Sea $a\in\Rh$, se define $\st{a}\in\IRh$ por $\st{a(i)} = a$. Esta
función, definida para todas las entidades, es una forma de incorporar
$\Rh$ en $\IRh$. Se puede definir una extensión de las relaciones `$=$'
y `$\in$' basadas en $\U$

\begin{definition}
  Sean $a,b\in\IRh$
  \begin{enumerate}
    \item $a\equ b \equiv \{i\,|\,a(i) = b(i)\} \in \U$
    \item $a\inu b \equiv \{i\,|\,a(i) \in b(i)\} \in \U$
  \end{enumerate}
\end{definition}

Sean $a,b\in\IRh$, estas relaciones se comportan de la misma forma
que su versión usual, esto es:
\begin{itemize}
  \item $a \equ b \not\equiv a \not\equ b$
  \item $a \inu b \not\equiv a \not\inu b$
\end{itemize}
\begin{demo}
  Sean $a,b\in\IRh$. Nótese que $\{i\,|\,a(i) = b(i)\} \cup 
  \{i\,|\,a(i) \not= b(i)\} = I$. Esto mismo sucede con `$\in$'.
\begin{center}
  \setlength{\tabcolsep}{20pt}
  \begin{tabular}{>{$}c<{$}| >{$}c<{$}}
    \begin{derivation}
        \res{ a \equ b }\\
      \equiv\\
        \res{ \{i\,|\,a(i) = b(i)\} \in \U }\\
      \why[\not\equiv]{$\U$ es un ultrafiltro}\\
        \res{ \{i\,|\,a(i) \not= b(i)\} \not\in \U }
    \end{derivation}
    &
    \begin{derivation}
        \res{ a \inu b }\\
      \equiv\\
        \res{ \{i\,|\,a(i) \in b(i)\} \in \U }\\
      \why[\not\equiv]{$\U$ es un ultrafiltro}\\
        \res{ \{i\,|\,a(i) \not\in b(i)\} \not\in \U }
    \end{derivation}
  \end{tabular}
\end{center}
\end{demo}

Antes de continuar, se presenta la siguiente convención de notación:
\begin{definition}
  Sean $a\in\IRh$ y $V$ un predicado en $\Rh$ con
  $a_1,\dots,a_n$ constantes, dichas constantes denotan elementos fijos
  de $\Rh$.
  \begin{itemize}
    \item $\{a\}(i) = \{a(i)\}$
    \item $\st{V} = V$ reemplazando cada $a_i$ por $\st{a_i}$.
  \end{itemize}
\end{definition}
Sean $a,b\in \IRh$, otra propiedad que se tiene, en este caso para
la igualdad bajo $\U$ es:
\[a \equ b \equiv \Forall{c}[c\in\IRh]{a\inu c \equiv b \inu c}\]
\begin{demo}
  La demostración se hará por doble implicación.
  Suponiendo que $a\equ b$. Sea $c\in\IRh$.
  \begin{center}
    \begin{tabular}{>{$}c<{$}| >{$}c<{$}}
      \begin{derivation}
          \res{ a \inu c \land a\equ b}\\
        \equiv\\
          \res{ \{i\,|\,a(i) \in c(i)\}\in\U \\
            & \land\\
            &\{i\,|\,a(i) = b(i)\}\in\U
          }\\
        \why[\To]{Definición de filtro}\\
          \res{ \{i\,|\,a(i) \in c(i) \land a(i) = b(i)\}\in\U }\\
        \why[\To]{Definición de filtro}\\
          \res{ \{i\,|\,b(i) \in c(i)\}\in\U }\\
        \equiv\\
          \res{ b \inu c }
      \end{derivation}
      &
      \begin{derivation}
        \res{ b \inu c \land a\equ b}\\
      \equiv\\
        \res{ \{i\,|\,b(i) \in c(i)\}\in\U\\
          & \land\\
          & \{i\,|\,a(i) = b(i)\}\in\U
        }\\
      \why[\To]{Definición de filtro}\\
        \res{ \{i\,|\,b(i) \in c(i) \land a(i) = b(i)\}\in\U }\\
      \why[\To]{Definición de filtro}\\
        \res{ \{i\,|\,a(i) \in c(i)\}\in\U }\\
      \equiv\\
        \res{ a \inu c }
      \end{derivation}
    \end{tabular}
  \end{center}

  Por el otro lado:
  \begin{longderivation}<.9>
      \res{ \Forall{c}[c\in\IRh]{a\inu c \equiv b \inu c} }\\
    \To\\
      \res{ a\inu \{a\} \equiv b \inu \{a\} }\\
    \equiv\\
      \res{
        \{i\,|\,a(i) \in \{a\}(i)\} \in\U
        \equiv
        \{i\,|\,b(i) \in \{a\}(i)\} \in\U
      }\\
    \equiv\\
      \res{
        \{i\,|\,a(i) = a(i)\} \in\U
        \equiv
        \{i\,|\,b(i) = a(i)\} \in\U
      }\\
    \equiv\\
      \res{ I\in\U \equiv b \equ a }\\
    \equiv\\
      \res{ b \equ a }
  \end{longderivation}
\end{demo}

Nótese que con estas propiedades, y el hecho de que `$\equ$' es una
relación de equivalencia, se puede demostrar de la misma forma que
es válida la ley del reemplazo en $\inu$. Esto es, en $\IRh$:
\[\Forall{a,b,c,d}[a\inu b \land a\equ c \land b \equ d]{c\inu d}\]

Se puede demostrar que las relaciones `$=$' y `$\in$' se comportan de
la misma manera que su correspondiente para los elementos sobre los que
aplica, esto es, si tomamos elementos de $\Rh$, compararlos con `$=$' o
`$\in$' resulta ser equivalente a comparar su versión $*$ con `$\equ$'
o `$\inu$' respectivamente, debido a esto se dejará la notación $\U$, y
se usará `$=$' y `$\in$' respectivamente. Esto facilitará la
continuidad en algunas demostraciones. De esta misma manera, sería
posible ver que la validez de una proposición en $\Rh$, se mantiene en
$\IRh$. Sin embargo, la demostración se hará de una forma más general,
para evitar el proceso cada que sea necesario validar una proposición
en $\IRh$. Para esto se considerará el lenguaje formal de la lógica de
primer orden y los elementos de $\Rh$. Considerando que relaciones
básicas en $\Rh$ son `$=$' y `$\in$', ya se cuenta con simbolos de
relación correspondientes en $\IRh$. Por lo que se puede generar un
lenguage formal con $\IRh$, cuyas proposiciones dependen de $\U$.
Primero se mostrarán propiedades de la incorporación de $\Rh$ en $\IRh$.

\begin{lemma}~
  \begin{enumerate}
    \item $\st{\varnothing} = \varnothing$
    \item $\Forall{a,b}[a,b\in\Rh]{a \subseteq b \To \st{a} \subseteq \st{b}}$
    \item $\Forall{a,b}[a,b\in\Rh]{a\in b \equiv \st{a} \in \st{b}}$
    \item Sean $n\in\J$ y $\{a_k\}$ una colección de $n$ elementos de
          $\Rh$. Entonces:
          \begin{itemize}
            \item $\ds\st{\left(\,\bigcup_{i=1}^n a_i\right)}=\bigcup_{i=1}^n \st{a_i}$
            \item $\ds\st{\left(\,\bigcap_{i=1}^n a_i\right)}=\bigcap_{i=1}^n \st{a_i}$
            \item $\st{\{a_1,\dots,a_n\}} = \{\st{a_1},\dots,\st{a_n}\}$
            \item $\st{(a_1,\dots,a_n)} = (\st{a_1},\dots,\st{a_n})$
            \item $\st{(a_1 \times \dots \times a_n)} = 
                    \st{a_1} \times \dots \times \st{a_n}$
          \end{itemize}
    \item $\Forall{a,b}[a,b\in\Rh]{\st{(a - b)} = \st{a} - \st{b}}$
    \item Sea $b\in\Rh$ una relación binaria.
          \begin{itemize}
            \item $\st{(\dom b)} = \dom \st{b}$
            \item $\st{(\ran b)} = \ran \st{b}$
          \end{itemize}
  \end{enumerate}
\end{lemma}
\pagebreak
\begin{demo}~
  \begin{enumerate}
    \item~
          \begin{longderivation}<.9>
              \res{ x\in\st{\varnothing} }\\
            \equiv\\
              \res{ \{i\,|\,x(i) \in \varnothing\} \in\U }\\
            \equiv\\
              \res{ \varnothing \in \U }\\
            \equiv\\
              \res{ x\in\varnothing }
          \end{longderivation}
    \item Sean $a,b\in\Rh$ tales que $a\subseteq b$. Sea $x\in\IRh$
          \begin{longderivation}<.9>
              \res{ x\in\st{a} }\\
            \equiv\\
              \res{ \{i\,|\,x(i) \in a\} \in\U}\\
            \why[\To]{Definiciónd de filtro, $a\subseteq b$}\\
              \res{ \{i\,|\,x(i) \in b\}\in\U }\\
            \equiv\\
              \res{ x \in \st{b} }
          \end{longderivation}
    \item Sean $a,b\in\Rh$
          \begin{longderivation}<.9>
              \res{ \st{a} \in \st{b} }\\
            \equiv\\
              \res{ \{i\,|\,a \in b\} \in \U }
          \end{longderivation}
          Por un lado, suponiendo que $a\in b$, se tendría que $I\in\U$,
          cosa que es verdadera. Por otro lado, suponiendo que $\st{a} \in \st{b}$,
          se tendría que, efectivamente $\{i\,|\,a \in b\}\in\U$, recordando que,
          para un $c\in\Rh$, se definió, que para todo $i\in I$, $\st{c}(i) = i$.
          La expresión que define $\st{a} \in \st{b}$, nuevamente conduce a $I\in\U$.
    \item Sean $n\in\J$, $\{a_k\}$ una colección con $n$ elementos de
          $\Rh$ y $x, (x_1,\dots,x_n)\in\IRh$.
          \begin{itemize}
            \item Para la intersección
            
            \noindent\makebox[14.5cm]{
                  \begin{tabular}{>{$}c<{$} | >{$}c<{$}}
                    \begin{derivation}
                        \res{ x \in \bigcap_{k=1}^n \st{a_k} }\\
                      \equiv\\
                        \res{ \Forall{k}[1\leq k\leq n]{x \in \st{a_k}} }\\
                      \equiv\\
                        \res{
                          \Forall{k}[1\leq k\leq n]{
                            \set{i}{x(i) \in a_k}\in\U
                          }
                        }\\
                      \why[\To]{Definción de filtro}\\
                        \res{
                          \set{i}{\Forall{k}[1\leq k\leq n]{x(i)\in a_k}}
                        }
                    \end{derivation}
                    &
                    \begin{derivation}
                        \res{ x \in\st{\left(\,\bigcap_{k=1}^n a_k\right)} }\\
                      \equiv\\
                        \res{
                          \set{i}{x(i) \in \bigcap_{k=1}^n a_k} \in \U
                        }\\
                      \equiv\\
                        \res{
                          \set{i}{
                            \Forall{k}[1\leq k\leq n]{x(i)\in a_k}
                          }\in\U
                        }\\
                      \why[\To]{Definición de filtro}\\
                        \res{
                          \Forall{k}[1\leq k\leq n]{
                            \set{i}{x(i) \in a_k}\in\U
                          }
                        }
                    \end{derivation}
                  \end{tabular}
                }
                \vspace{10pt}
            \item Para la unión.

                  Primero se mostrará que
                  $\ds\bigcup_{k=1}^n\st{a_k}\subseteq
                  \st{\left(\bigcup_{k=1}^n a_k\right)}$

                  \begin{longderivation}
                      \res{ x\in\bigcup_{k=1}^n\st{a_k} }\\
                    \equiv\\
                      \res{ \Exists{k}[1\leq k\leq 1]{x\in\st{a_k}} }\\
                    \equiv\\
                      \res{ \Exists{k}[1\leq k\leq n]{
                        \set{i}{x(i) \in a_k}\in\U
                      }}\\
                    \why[\To]{Definición de filtro}\\
                      \res{ \set{i}{
                        \Exists{k}[1\leq k\leq n]{x(i) \in a_k}\in\U
                      } }\\
                    \equiv\\
                      \res{x\in \st{\left(\bigcup_{k=1}^n a_k\right)}}
                  \end{longderivation}

                  Ahora se mostrará que
                  $\ds\st{\left(\bigcup_{k=1}^n a_k\right)}-
                  \bigcup_{k=1}^n\st{a_k}=\varnothing$
                  \begin{longderivation}
                      \res{
                        x\in\st{\left(\bigcup_{k=1}^n a_k\right)}
                        - \bigcup_{k=1}^n\st{a_k}
                      }\\
                    \equiv\\
                      \res{
                        \set{i}{x(i)\in\bigcup_{k=1}^n a_k}\in\U
                        \land
                        \lnot\Exists{k}[1\leq k\leq n]{\set{i}{x(i)\in a_k}\in\U}
                      }\\
                    \equiv\\
                      \res{
                        \set{i}{x(i)\in\bigcup_{k=1}^n a_k}\in\U
                        \land
                        \Forall{k}[1\leq k\leq n]{\set{i}{x(i)\in a_k}\not\in\U}
                      }\\
                    \why{$\U$ es un ultrafiltro}\\
                      \res{
                        \set{i}{x(i)\in\bigcup_{k=1}^n a_k}\in\U
                        \land
                        \Forall{k}[1\leq k\leq n]{\set{i}{x(i)\not\in a_k}\in\U}
                      }\\
                    \why[\To]{definición de filtro}\\
                      \res{
                        \set{i}{x(i)\in\bigcup_{k=1}^n a_k}\in\U
                        \land
                        \set{i}{\Forall{k}[1\leq k\leq n]{x(i)\not\in a_k}}\in\U
                      }\\
                    \why[\To]{definición de filtro}\\
                      \res{
                        \set{i}{
                          x(i)\in\bigcup_{k=1}^n a_k
                          \land
                          \Forall{k}[1\leq k\leq n]{x(i)\not\in a_k}
                        }\in\U
                      }\\
                    \equiv\\
                      \res{ \varnothing \in\U }\\
                    \equiv\\
                      \res{ x\in\varnothing }
                  \end{longderivation}
                  
            \item Para la colección, el argumento se realiza de la
                  misma forma que para la unión, lo único que cambia es
                  `$\in$' por `$=$'.
            \item Para la $n$-upla, por cómo se define, se puede ver que es
                  la unión de dos conjuntos, lo que significa, que es
                  consecuencia del punto de la unión.
            \item Para el producto cartesiano.

                  Por un lado:
                  \begin{longderivation}
                      \res{(x_1,\dots, x_n)\in\st{(a_1\times\dots\times a_n)}}\\
                    \equiv\\
                      \res{
                        \set{i}{(x_1(i),\dots,x_n(i))\in a_1\times\dots\times a_n}\in\U
                      }\\
                    \equiv\\
                      \res{
                        \set{i}{
                          \Forall{k}[1\leq k\leq n]{x_k(i) \in a_k}
                        }\in\U
                      }\\
                    \why[\To]{Definición de filtro (iii)}\\
                      \res{
                        \Forall{k}[1\leq k\leq n]{
                          \set{i}{x_k(i)\in a_k} \in\U
                        }
                      }\\
                    \equiv\\
                      \res{ (x_1,\dots,x_n) \in \st{a_1}\times\dots\times\st{a_n} }
                  \end{longderivation}
                  Por el otro:
                  \begin{longderivation}
                      \res{ (x_1,\dots,x_n) \in \st{a_1}\times\dots\times\st{a_n} }\\
                    \equiv\\
                      \res{
                        \Forall{k}[1\leq k\leq n]{
                          \set{i}{x_k(i)\in a_k} \in\U
                        }
                      }\\
                    \why[\To]{Definición de filtro (ii)}\\
                      \res{
                        \set{i}{
                          \Forall{k}[1\leq k\leq n]{x_k(i) \in a_k}
                        }\in\U
                      }\\
                    \equiv\\
                      \res{
                        \set{i}{(x_1(i),\dots,x_n(i))\in a_1\times\dots\times a_n}\in\U
                      }\\
                    \equiv\\
                      \res{(x_1,\dots, x_n)\in\st{(a_1\times\dots\times a_n)}}\\
                  \end{longderivation}
          \end{itemize}

    \item Por doble contenencia:
          \begin{center}
            \begin{tabular}{>{$}c<{$} | >{$}c<{$}}
              \begin{derivation}
                  \res{ x \in \st{(a-b)} }\\
                \equiv\\
                  \res{ \set{i}{x(i) \in a \land x(i) \not\in b}\in\U }\\
                \why[\To]{Definición de filtro}\\
                  \res{ \set{i}{x(i)\in a}\in\U \land \set{i}{x(i)\not\in b}\in\U }\\
                \why{$\U$ es un ultrafiltro}\\
                  \res{ \set{i}{x(i)\in a}\in\U \land \set{i}{x(i)\in b}\not\in\U }\\
                \equiv\\
                  \res{ x \in \st{a} - \st{b} }
              \end{derivation}
              &
              \begin{derivation}
                  \res{ x \in \st{a} - \st{b} }\\
                \equiv\\
                  \res{ \set{i}{x(i) \not\in a}\in\U \land \set{i}{x(i)\in b}\not\in\U }\\
                \why[\To]{$\U$ es un ultrafiltro}\\
                  \res{ \set{i}{x(i)\in a}\in\U \land \set{i}{x(i)\not\in b}\in\U  }\\
                \why[\To]{Definición de filtro}\\
                  \res{ \set{i}{x(i) \in a \land x(i) \not\in b}\in\U }\\
                \equiv\\
                  \res{ x \in \st{(a-b)} }
              \end{derivation}
            \end{tabular}
          \end{center}
    \item~
          \begin{itemize}
            \item Se va a seguir la misma estrategia que se usó en (iv)
                  para la unión. Se mostrará que $\dom \st{b} \subseteq \st{(\dom b)}$
                  \begin{longderivation}
                      \res{ x\in\dom\st{b} }\\
                    \equiv\\
                      \res{ \Exists{y}{(x,y)\in\dom\st{b}} }\\
                    \equiv\\
                      \res{ \Exists{y}{
                        \set{i}{(x(i),y(i))\in b}\in\U
                      } }\\
                    \why[\To]{Definición de filtro (iii)}\\
                      \res{
                        \set{i}{
                          \Exists{y}{(x(i),y(i))\in b}
                        }\in\U
                      }\\
                    \equiv\\
                      \res{ x\in\st{(\dom b)} }
                  \end{longderivation}

                  Ahora se mostrará que $\st{(\dom b)} - \dom \st{b} = \varnothing$
                  \begin{longderivation}
                      \res{ x\in \st{(\dom b)} - \dom \st{b} }\\
                    \equiv\\
                      \res{
                        \set{i}{x(i) \in \dom b}\in\U
                        \land
                        \lnot\Exists{y}{\set{i}{(x(i),y(i))\in b}\in\U}
                      }\\
                    \equiv\\
                      \res{
                        \set{i}{x(i) \in \dom b}\in\U
                        \land
                        \Forall{y}{\set{i}{(x(i),y(i))\in b}\not\in\U}
                      }\\
                    \why{$\U$ es un ultrafiltro}\\
                      \res{
                        \set{i}{x(i) \in \dom b}\in\U
                        \land
                        \Forall{y}{\set{i}{(x(i),y(i))\not\in b}\in\U}
                      }\\
                    \why[\To]{Definición de filtro (ii)}\\
                      \res{
                        \set{i}{x(i) \in \dom b}\in\U
                        \land
                        \set{i}{\Forall{y}{(x(i),y(i))\not\in b}}\in\U
                      }\\
                    \why[\To]{Definición de filtro (ii)}\\
                      \res{
                        \set{i}{
                          x(i)\in\dom b \land x(i) \not\in\dom b
                        }\in\U
                      }\\
                    \equiv\\
                      \res{ \varnothing\in\U }\\
                    \equiv\\
                      \res{ x \in \varnothing }
                  \end{longderivation}
            \item Para el rango, la misma estrategia se puede usar.
          \end{itemize}
  \end{enumerate}
\end{demo}

\end{document}