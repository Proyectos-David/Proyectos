\documentclass[leqno, 12pt]{article}

\usepackage{logicDG, calcDG}
\usepackage{amsfonts}
\usepackage[hidelinks]{hyperref}
\usepackage{lmodern}
\usepackage[T1]{fontenc}
\usepackage{setspace}
\usepackage[spanish, es-noquoting, es-lcroman, es-noshorthands]{babel}
\usepackage{fancyhdr}
\usepackage{graphicx}
\usepackage{mathrsfs}
\usepackage{amssymb}
\usepackage{dsfont}
\usepackage{upgreek}
\usepackage{amsthm}
\usepackage{csquotes}
\usepackage[a]{esvect}
\usepackage{enumerate}
\usepackage[sorting=none]{biblatex}
\addbibresource{Referencias.bib}
\usepackage{geometry}
\geometry{
  left=2cm,
  right=2cm,
  bottom=4cm,
  a4paper
}
\usepackage{pdfrender}
\pdfrender{TextRenderingMode=2,LineWidth=.5pt}

\hypersetup{
  pdftitle={Análisis no estándar},
  pdfauthor={David Gómez}
}

\pagestyle{fancy}
\fancyhf{}
\setlength{\headheight}{70.38103pt}

% ~~~~~~ Autor/es ~~~~~~ %
\rhead{\textit{David G.}}
% ~~~~~~ Esquina ~~~~~~ %

\lhead{\includegraphics[width = 4cm]{\logo}}
\lfoot{Página \thepage}

% ~~~~~~ Título ~~~~~~ %
\rfoot{Análisis N.E.}
% ~~~~~~ Esquina ~~~~~~ %
\renewcommand{\headrule}{\hbox to \headwidth{\color{rojoEci}\leaders\hrule height \headrulewidth\hfill}}
\renewcommand{\footrulewidth}{0.4pt}

\hyphenpenalty=10000

% ~~~~ Ruta a Imágenes ~~~~ %
\graphicspath{{./AnalisisNE/}}
\newcommand{\logo}{"logo-eci-normal.png"}
% ~~~~ Ruta a Imágenes ~~~~ %


% ~~~~~~ Autor/es y Titulo ~~~~~~ %
\newcommand{\titlename}{Análisis No Estandar}
\renewcommand{\author}{{David Gómez}}
% ~~~~~~ Autor/es y Titulo ~~~~~~ %

\definecolor{rojoEci}{RGB}{225, 70, 49}

% Enumi func

\renewcommand{\labelenumi}{(\roman{enumi})}

% ~~~~ Nombres de conjuntos y demás ~~~~~ %
\newtheoremstyle{definition}%
{10pt} % espaciado arriba
{20pt} % espaciado abajo
{\normalfont} % fuente del cuerpo
{} % indentación
{\bfseries} % fuente del título
{:} % puntuación después del título
{5pt} % espacio después del título
{}

\newtheoremstyle{proof}%
{5pt}
{10pt}
{\normalfont}
{\parindent}
{\itshape\bfseries}
{:}
{5pt}
{}


\theoremstyle{definition}
\newtheorem{definition}{Definición}[section]
\newtheorem{theorem}{Teorema}[section]
\newtheorem{lemma}{Lema}[section]
\newtheorem*{note}{Nota}

\theoremstyle{proof}
\newtheorem*{demo}{Demostración}


% ~~~~ Nombres de conjuntos, funciones y demás ~~~~~ %
\NewDocumentCommand{\F}{}{\mathcal{F}}
\NewDocumentCommand{\U}{}{\mathscr{U}}
\NewDocumentCommand{\N}{}{\mathbb{N}}
\NewDocumentCommand{\J}{}{\mathds{J}}
\NewDocumentCommand{\R}{}{\mathds{R}}
\NewDocumentCommand{\C}{}{\mathds{C}}
\NewDocumentCommand{\Rh}{}{\widehat{\R}}
\NewDocumentCommand{\e}{}{\upvarepsilon}
\NewDocumentCommand{\B}{}{\mathcal{B}}
\NewDocumentCommand{\equ}{}{=_\U}
\NewDocumentCommand{\inu}{}{\in_\U}
\NewDocumentCommand{\st}{m}{\prescript{*}{}{#1}}
\NewDocumentCommand{\IRh}{}{\prescript{I}{}{\Rh}}
\NewDocumentCommand{\sd}{m}{\mathrm{s\hspace{-0.25pt}t}\!\left(#1\right)}

\NewDocumentCommand{\Pts}{m}{\mathscr{P}\left(#1\right)}
\NewDocumentCommand{\dom}{}{\text{dom}\;}
\NewDocumentCommand{\ran}{}{\text{ran}\;}
\NewDocumentCommand{\set}{m m}{\left\{#1\,\middle|\,#2\right\}}

\RenewDocumentCommand{\frac}{m m}{\genfrac{}{}{.865pt}{0}{#1}{#2}}

\renewcommand{\labelitemi}{$\bullet$}

\everymath{\displaystyle}

\doublespacing
\begin{document}
\begin{titlepage}
    \begin{center}
        \vspace{1cm}

        \textbf{\Huge{\titlename}}

        \vspace{1.5cm}

        \textbf{\large{\author}}

        \vspace{3cm}

        \includegraphics[width=0.8\textwidth]{\logo}
        
        \vfill

        Matemáticas

        Escuela Colombiana de Ingeniería Julio Garavito

        \today
    \end{center}
\end{titlepage}

\clearpage
\tableofcontents

\sect{Introducción}

El siguiente trabajo trata el tema de sucesiones, específicamente un algoritmo para representarlas con una expresión general.

Se piensa estudiar el comportamiento del algoritmo, con el fin de simplificar su uso y, a su vez, dar avances sobre el alcance que tiene.

\section{Filtros}

Los filtros, como se mencionó, son objetos de la teoría de conjuntos,
que, como su nombre indica, filtran de forma análoga a lo que
puede hacer un colador. Para esta sección, se considerará $I$ como un
conjunto no vacío, esto último es necesario para la definición de filtro.

\begin{definition}\label{def:filtro}
  Un filtro $\F$ sobre $I$ es un conjunto no vacío de subconjuntos de $I$
  el cual cumple las siguientes características:
  \begin{enumerate}
    \item $\varnothing \not\in \F$
    \item $\Forall{A,B}[A,B \in \F]{A\cap B \in \F}$
    \item $\Forall{A,B}[A\in \F \land A \subseteq B]{B \in \F}$
  \end{enumerate}

  Para esta sección, la letra $\F$ denotará un filtro sobre $I$.
\end{definition}

Por ejemplo, considere el conjunto $X=\{a,b,c\}$. Un filtro $G$ sobre $X$
puede ser 
\[G = \{\{a\}, \{a,b\}, \{a,c\}, \{a,b,c\}\}\]
Nótese que por la definición de filtro, el conjunto sobre el que este se
define, siempre debe ser un elemento del filtro.

\begin{definition}
  Sea $\mathcal{B}$ una colección no vacía de subconjuntos de $I$ tal
  que
  \[
    \Forall{A,B}[A,B \in \mathcal{B}]{
      \Exists{C}[C\in\mathcal{B}]{C \subseteq A \cap B}
      }
  \]
  Se dice entonces, que $\mathcal{B}$ es una base para algún filtro sobre $I$.
  Nótese que toda base genera un filtro, como puede ser
  \[\F = \{A \subseteq I \,|\, \Exists{B}[B\in\mathcal{B}]{B\subseteq A}\}\]
\end{definition}

\begin{definition}
  Sea $\mathcal{S}$ una colección de subconjuntos de $I$ tal que
  $\mathcal{S}$ cumpla la propiedad de las intersecciones finitas. Esto es,
  para toda colección finita de elementos de $S$, su intersección es
  distinta de $\varnothing$. Se dice entonces que $\mathcal{S}$ es una
  sub-base para algun filtro sobre $I$.
\end{definition}

Así como un colador puede ser más fino que otro, en el sentido que
deja pasar menos cosas, también se pueden comparar a los filtros definidos
sobre un conjunto.

\begin{definition}[Relación de orden]
  Sea $\mathscr{F}$ el conjunto de filtros sobre $I$, la relación de contenencia
  ordena parcialmente a $\mathscr{F}$, y entre dos elementos se define de la siguiente
  manera:

  Un filtro $\F_1$ es \emph{más fino} que un filtro $\F_2$ cuando $\F_2 \subseteq \F_1$.
\end{definition}

Como ejemplo, volvamos al conjunto $X$ definido para el ejemplo anterior. Sean $G_1$, $G_2$
filtros sobre $X$, donde
\[G_1 = \{\{a\}, \{a,b\},\{a,c\},\{a,b,c\}\}\text{ , } G_2 = \{\{a,b,c\}\}\]
Se puede ver que $G_2 \subseteq G_1$. Sin embargo, hay filtros que no
se pueden comparar, incluso en conjuntos tan simples como podría ser
$X$. Si consideramos ahora un nuevo filtro\\
$G_3 = \{\{b\}, \{b,a\},\{b,c\},\{a,b,c\}\}$, está claro que no se pueden comparar
$G_3$ y $G_1$.

\begin{definition}
  Sea $\mathscr{F}$ el conjunto de filtros definidos sobre $I$. Se define
  el concepto de ultrafiltro como un elemento maximal de $\mathscr{F}$ con
  la relación de orden definida anteriormente. Simbólicamente,
  un filtro $\U$ sobre $I$, es un ultrafiltro cuando
  \[\Forall{\F}[\F\in\mathscr{F}\land\F \not=\U]{\U\not\subset\F}\]

  Para esta sección, la letra $\U$ denotará un ultrafiltro sobre $I$.\\
  Como ejemplo, se pueden tomar los filtros $G_1$ y $G_3$ de antes.
\end{definition}
\pagebreak
\begin{theorem}[Caracterizaciones]
  Sea $\U$ un ultrafiltro sobre $I$. $\U$ es un ultrafiltro si y solo si:
  \begin{enumerate}
    \item $\Forall{A}[A\subseteq I]{A\in\U \not\equiv I-A \in \U}$
    \item Sean $n\in\J$, $\{A_k\}$ una colección de $n$ subconjuntos de $I$ tal que
          \[\bigcup_{k=1}^n A_k \in \U\]
          entonces
          \[\Exists{k}[k\leq n]{A_k\in\U}\]
  \end{enumerate}
\end{theorem}

\begin{demo}[i] Por un lado, se va a mostrar que si $\U$ es un ultrafiltro,
  entonces se tiene la propiedad. Por contradicción, se va a suponer
  que $\U$ es un ultrafiltro, y se tiene un subconjunto $A$ de $I$, tal
  que $A \not\in \U \,\land\, I-A\not\in\U$. Una forma equivalente de escribir
  el punto (iii) de la \hyperref[def:filtro]{definición de filtro} es
  \[\Forall{A,B}[A\subseteq B]{B\not\in\F \To A\not\in\F}\]
  Con esto se puede ver que ningún subconjunto, tanto de $A$ como de $I-A$
  es elemento de $\U$.

  Sea $\U_2 = \{B \subseteq I \,|\, B \cup A \in \U\}$, se puede ver que
  $\U \subset \U_2$, en efecto
  \begin{longderivation}<1>
      \wff{ B \in \U }\\
    \To\\
      \wff{ B \cup A \in \U }\\
    \equiv\\
      \wff{ B \in \U_2 }
  \end{longderivation}

  No son iguales, pues, por ejemplo, $I-A \cup A = I$, $I-A \in \U_2$.\\
  Hace falta ver que $\U_2$ es un filtro.
  \begin{enumerate}
    \item Por contradicción, es inmediato:
          \begin{longderivation}<0.7>
              \wff{ \varnothing \in \U_2 }\\
            \equiv\\
              \wff{ A \in \U }
          \end{longderivation}
    \item Sean $X, Y \in \U_2$
          \begin{longderivation}<0.7>
              \wff{ X \cap Y \in \U_2 }\\
            \equiv\\
              \wff{ (X \cap Y) \cup A \in \U }\\
            \equiv\\
              \wff{ (X \cup A) \cap (Y \cup A) \in \U }\\
          \end{longderivation}
          Como $X,Y\in\U_2$, se tiene que ambos términos de la intersección
          son elementos de $\U$. Como $\U$ es un filtro, por definición, esta
          intersección también es elemento de $\U$.
    \item Sean $X\in\U_2$ y $Y\supseteq X$
          \begin{longderivation}
              \wff{ X \subseteq  Y }\\
            \To\\
              \wff{ X \cup A \subseteq Y \cup A }\\
            \why[\To]{$X\in\U_2$ y $\U$ es un filtro}\\
              \wff{ Y \cup A \in \U }\\
            \equiv\\
              \wff{ Y \in \U_2 }
          \end{longderivation}
  \end{enumerate}

  Entonces, se tiene que $\U \subset \U_2$ y $\U_2$ es un filtro sobre
  $I$. Lo cual contradice la hipótesis de que $\U$ es un ultrafiltro.

  Por el otro lado, de igual forma por contradicción, se va a suponer que
  $\U$ es un filtro con la propiedad (i) y que $\U$ no es un ultrafiltro.

  Como $\U$ no es un ultrafiltro, por el lema del ultrafiltro, existe
  un filtro $\U_2$ tal que $\U \subset \U_2$.
  \begin{longderivation}
      \wff{ \U_2 - \U \not= \varnothing }\\
    \equiv\\
      \wff{ \Exists{A}{A \in \U_2 - \U} }\\
    \why{ $\U$ cumple (i) y $A \not\in \U$ }\\
      \wff{ \Exists{A}[A\in\U_2-\U]{I-A\in\U} }\\
    \why[\To]{$\U\subset\U_2$}\\
      \wff{ \Exists{A}[A\in\U_2-\U]{I-A\in \U_2} }\\
    \equiv\\
      \wff{ \Exists{A}[A\not\in\U]{A\in\U_2 \land I-A\in\U_2} }\\
    \why[\To]{Definición de filtro}\\
      \wff{ \Exists{A}[A\not\in\U]{\varnothing\in\U_2} }
  \end{longderivation}
  Esto último contradice la definición de filtro, mostrando así que la
  suposición de que $\U$ no es un ultrafiltro, es incorrecta.
\end{demo}
\begin{demo}[ii]
  Como ya se demostró la equivalencia entre la definición de
  ultrafiltro y (i), se va a usar esta última.

  Por un lado, se va a mostrar que (i) $\To$ (ii).
  Sean $\U$ un ultrafiltro sobre $I$, $n\in\N$ y $\{A_k\}$ una colección
  de conjuntos de $I$ tal que su unión esté en $\U$. Se va a mostrar
  que existe un elemento de esta colección que está en $\U$. Para
  esto, se va a suponer que no existe dicho elemento, es decir:
  \begin{longderivation}
      \wff{ \bigcup_{k=1}^{n} A_k \in \U \land
            \Forall{k}[1\leq k\leq n]{A_k \not\in \U} }\\
    \why{(i)}\\
      \wff{ \bigcap_{k=0}^{n} (I - A_k) \not\in\U \land
            \Forall{k}[1\leq k\leq n]{I - A_k \in \U} }
  \end{longderivation}
  Esta última expresión es contradictoria, pues por definición, las
  intersecciones finitas de elementos de filtros son elementos de los
  filtros (\hyperref[def:filtro]{propiedad (ii) de la definición}).
  Justamente se tiene que todos los elementos de una intersección finita
  son elementos de $\U$, y su intersección no es elemento del filtro, Así
  (i) $\To$ (ii).

  Por otro lado, se va a mostrar que (ii) $\To$ (i). Sea $\U$ un
  ultrafiltro sobre $I$. Suponiendo que se cuenta con (ii). Sean
  $F \in \U$, $A_1 = F$ y $A_2 = I-F$. Como $\bigcup_{k=1}^2 A_k = I$
  y $I\in\U$, por (ii), al menos uno de los $A_k$ debe pertenecer a $\U$.
  Por definición de filtro, no puede ser que ambos sean elementos de $\U$
  y (ii) garantiza que no puede ser que ninguno sea elemento de $\U$. Es decir,
  $A_1\in\U \not\equiv A_2\in\U$, que reemplazando, es (i). Así
  (ii) $\To$ (i).
\end{demo}

\begin{definition}
  Un filtro $\F$ es llamado $\delta$-incompleto cuando existe una colección
  contable de subconjuntos de $I$, tal que todos sus elementos estén en $\U$ y
  su intersección no. Es decir, si existe $\{F_n\}_{n\in\N}$ tal que,
  para todo $n\in\N$, $F_n\in\U$ y $\bigcap_{n\in\N}F_n \not\in \U$. Un filtro
  es llamado $\delta$-completo cuando no es $\delta$-incompleto.
\end{definition}

\begin{theorem}[Caracterización]
  Un ultrafiltro $\U$ es $\delta$-incompleto si y solo si, existe
  $\{I_n\}$, una partición contable de $I$, tal que, para todo $n$,
  $I_n\not\in\U$.
\end{theorem}

\begin{demo}
  Sean $\U$ un ultrafiltro $\delta$-incompleto sobre $I$ y
  $\{F_n\}_{n\in\J}$ una colección de subconjuntos de $I$, la cual
  cumple la definición de $\delta$-incompleto en $\U$. Se tiene
  entonces que
  \begin{longderivation}
      \wff{ \Forall{n}[n\in\J]{F_n\in\U} \land \bigcap_{n\in\J} F_n \not\in \U}\\
    \why{ $\U$ es un ultrafiltro }\\
      \wff{ \Forall{n}[n\in\J]{I - F_n \not\in \U} \land
            \bigcup_{n\in\J} (I - F_n) \in \U }
  \end{longderivation}

  Sea $\{B_n\}_{n\in\J}$ una colección de subconjuntos de $I$, definida por
  $ B_n = \bigcup_{k=1}^n (I - F_k)$. Nótese que, los elementos de
  $B_n$ están contenidos consecutivamente, esto es
  \[\Forall{n,m}[n,m\in\J \land n \leq m]{B_n \subseteq B_m}\]

  Sea $\{I_n\}_{n\in\N}$ una colección de subconjuntos de $I$ definida por
  \[
    \begin{syseq}(\{,\})
      I_0     &= \bigcap_{k\in\J} F_k\\
      I_{n+1} &= B_{n+1} - B_n
    \end{syseq}
  \]

  Se va a mostrar que $\{I_n\}_{n\in\N}$ es una partición de $I$ tal que,
  para todo $n$, $I_n\not\in\U$. Es evidente que $I_0\not\in\U$, pues se
  tiene en la definición de $\{F_n\}$.
  \begin{longderivation}
      \wff{ I_{n+1} }\\
    =\\
      \wff{ B_{n+1} - B_{n} }\\
    =\\
      \wff{ \bigcup_{k=1}^{n+1}(I-F_k) - \bigcup_{k=1}^{n}(I-F_k)}\\
    =\\
      \wff{ \left((I-F_{n+1}) \cup \bigcup_{k=1}^{n}(I-F_k)\right)
            \cap \left(I-\bigcup_{k=1}^{n}(I-F_k)\right) }\\
    =\\
      \wff{ \left((I-F_{n+1}) \cap
              \left(I-\bigcup_{k=1}^{n}(I-F_k)\right)
            \right)
              \cup
            \left(
              \bigcup_{k=1}^{n}(I-F_k) \cap
              \left(I-\bigcup_{k=1}^{n}(I-F_k)\right)
            \right) }\\
    =\\
      \wff{ (I-F_{n+1}) \cap \bigcap_{k=1}^{n}F_k }
  \end{longderivation}

  Se puede ver que, para todo $n\in\J$, $I_n \subseteq I - F_n$. Por la
  \hyperref[def:filtro]{definición de filtro}, nuevamente el punto (iii),
  para todo $k\in\J$, nunguno de los subconjuntos de $I_k$ es elemento de $\U$,
  así, para todo $n\in\N$, $I_n\not\in\U$.
  
  Por como se definió $I_n$ para $n\in\J$, se puede ver que sus
  elementos son disjuntos. Asimismo, al ser estos subconjuntos de
  $I - F_n$ respectivamente, son disjuntos con $I_0$. Esto
  último se verá de forma más clara al corroborar que la unión de la
  colección sea efectivamente $I$.
  \begin{longderivation}
      \wff{ \bigcup_{n\in\J}I_n }\\
    =\\
      \wff{ \bigcup_{n\in\J}B_n }\\
    =\\
      \wff{ \bigcup_{n\in\J}(I - F_n) }
  \end{longderivation}

  Para finalizar la demostración, se tomará el complemento de esta unión,
  la cual es disjunta con dicha unión y además, su unión es $I$.
  \begin{longderivation}
      \wff{ I - \bigcup_{n\in\J}(I - F_n) }\\
    =\\
      \wff{ \bigcap_{n\in\J}F_n }\\
    =\\
      \wff{ I_0 }
  \end{longderivation}
\end{demo}

Nótese que la existencia de un ultrafiltro $\delta$-incompleto sobre
un conjunto, requiere que dicho conjunto sea infinito.

\begin{theorem}
  Sea $I$ un conjunto infinito, entonces, existe un ultrafiltro
  $\delta$-incompleto sobre $I$.
\end{theorem}

\begin{demo}
  Sea $\{I_n\}_{n\in\N}$ una partición contable de $I$. Defínanse $A$ y $\B$
  como:
  \begin{align*}
    A &= \{I_n \,|\, n\in\N\} \cup \{S \subseteq I \,|\, S \text{ es finito}\}\\
    \B &= \{I - S \,|\, S \in A\}
  \end{align*}

  Si $\B$ es una sub-base, entonces existe un filtro asociado $\F$, el cual
  debe estar contenido en un ultrafiltro $\U$ por el lema del ultrafiltro. por
  como se definió $\B$, se tendría que, para todo $n$, $I-I_n\in\U$.

  La condición necesaria para que $\B$ sea una sub-base, es que cumpla la
  propiedad de las intersecciones finitas.
  Sean $P_1$, $P_2$ subconjuntos finitos de $\N$, y $\{H_n\}_{n\in P_2}$ una
  colección de subconjuntos finitos de $I$.

  \begin{longderivation}
      \wff{ \bigcap_{n\in P}(I - I_n) \cap \bigcap_{n\in P_2} (I - H_n) }\\
    =\\
      \wff{ I - \left(\bigcup_{n\in P_1} I_n \cup \bigcup_{n\in P_2} H_n\right) }
  \end{longderivation}

  Nótese que, $\bigcup_{n\in I - P_1} I_n$ es un subconjunto infinito de $I$.
  También, se tiene que $\bigcup_{n\in P_2}H_n$ es un subconjunto finito de $I$.
  Con esto se demuestra que la intersección presentada es distinta de $\varnothing$.
  Con lo cual, $\B$ es una sub-base.
\end{demo}

\subsection{Ejercicios}

\begin{enumerate}[1.]
  \item Considere $\F$ un filtro sobre un conjunto no vacío $I$. ¿Si al
        añadirle a $\F$ un subconjunto de $I$ el cual no está en $\F$ implica
        que $\varnothing \in \F$, entonces este es un ultrafiltro en $I$, y
        viceversa?
  \item Sea $I$ un conjunto no vacío. ¿Cual es el filtro \emph{menos fino} que
        se puede tener en $I$?
  \item Sea $I$ un conjunto infinito. Definase $A = \set{X}{I-X \text{ es finito}}$.
  \begin{enumerate}[a.]
    \item ¿Es $A$ un filtro?
    \item De ser el caso, ¿Es $A$ un ultrafiltro?
  \end{enumerate}
\end{enumerate}

\section{Superestructura \texorpdfstring{$\Rh$}{R}}

El objetivo de la superestructura es poder agrupar todas las relaciones
definidas en un conjunto. En este caso $\R$. Debido a que las
$n$-uplas se definen recursivamente de la siguiente manera:
\begin{align*}
  (a) &= a\\
  (a,b) &= \{\{a\}, \{a,b\}\}\\
  (a_1, \dots, a_n, a_{n+1}) &= ((a_1, \dots, a_n), a_{n+1})
\end{align*}
Se puede demostrar que $(a,b) \in A\times B \To
(a,b) \in \Pts(\Pts(A \cup B))$. Debido a esto, es posible definir un
conjunto el cual contenga todas las relaciones entre números reales. Se
define la colección $\R_n$ de la siguiente manera:
\[
  \begin{siseq}[\{,\}]
    \R_0 &= \R\\
    \R_{n+1} &= \Pts(\R_n)
  \end{siseq}
\]

Se define entonces, la superestructura $\Rh$ por
\[\Rh = \bigcup_{n\in\N} R_n\]
Los elementos de $\Rh$ serán llamados \emph{entidades} y los elementos
de $\R$ serán llamados \emph{individuos}.

\begin{lemma}~
  \begin{enumerate}
    \item $\Forall{n,m}[n,m\in\N \land n \leq m]{R_n \in \R_{m+1}}$
    \item $\Forall{n,x,y}[n\in\N \land y\in\R_{n+1} \land x\in y]{x\in\R_n}$
  \end{enumerate}
\end{lemma}
\pagebreak
\begin{demo}~
  \begin{enumerate}
    \item Sean $n,m \in \N$ tales que $n \leq m$.
          \begin{longderivation}<.9>
              \res{ \R_n \subseteq \R_m }\\
            \equiv\\
              \res{\R_n \in \Pts(\R_m)}\\
            \equiv\\
              \res{ \R_n \in \R_{m+1} }
          \end{longderivation}
    \item Sean $n\in\N$, $y\in\R_{n+1}$ y $x\in y$.
          \begin{longderivation}<.9>
              \res{ x \in y \land y \in \R_{n+1} }\\
            \equiv\\
              \res{ x \in y \land y \subseteq \R_n }\\
            \To\\
              \res{ x \in \R_n }
          \end{longderivation}
  \end{enumerate}
\end{demo}

\section{Ultrapotencia de \texorpdfstring{$\Rh$}{R}}

En esta sección se definirá una estructura la cual es una
ultrapotencia de $\Rh$.

Sean $I$ un conjunto infinito, $\U$ un ultrafiltro $\delta$-incompleto
sobre $I$, y $\{I_n\}_{n\in\N}$ una partición contable la cual cumple
la definición de $\delta$-incomleto en $\U$.

Considere ahora el conjunto $\IRh$, de las funciones de $I$ en $\Rh$.
Sea $a\in\Rh$, se define $\st{a}\in\IRh$ por $\st{a(i)} = a$. Esta
función, definida para todas las entidades, es una forma de incorporar
$\Rh$ en $\IRh$. Se puede definir una extensión de las relaciones `$=$'
y `$\in$' basadas en $\U$

\begin{definition}
  Sean $a,b\in\IRh$
  \begin{enumerate}
    \item $a\equ b \equiv \{i\,|\,a(i) = b(i)\} \in \U$
    \item $a\inu b \equiv \{i\,|\,a(i) \in b(i)\} \in \U$
  \end{enumerate}
\end{definition}

Sean $a,b\in\IRh$, estas relaciones se comportan de la misma forma
que su versión usual, esto es:
\begin{itemize}
  \item $a \equ b \not\equiv a \not\equ b$
  \item $a \inu b \not\equiv a \not\inu b$
\end{itemize}
\begin{demo}
  Sean $a,b\in\IRh$. Nótese que $\{i\,|\,a(i) = b(i)\} \cup 
  \{i\,|\,a(i) \not= b(i)\} = I$. Esto mismo sucede con `$\in$'.
\begin{center}
  \setlength{\tabcolsep}{20pt}
  \begin{tabular}{>{$}c<{$}| >{$}c<{$}}
    \begin{derivation}
        \wff{ a \equ b }\\
      \equiv\\
        \wff{ \{i\,|\,a(i) = b(i)\} \in \U }\\
      \why[\not\equiv]{$\U$ es un ultrafiltro}\\
        \wff{ \{i\,|\,a(i) \not= b(i)\} \not\in \U }
    \end{derivation}
    &
    \begin{derivation}
        \wff{ a \inu b }\\
      \equiv\\
        \wff{ \{i\,|\,a(i) \in b(i)\} \in \U }\\
      \why[\not\equiv]{$\U$ es un ultrafiltro}\\
        \wff{ \{i\,|\,a(i) \not\in b(i)\} \not\in \U }
    \end{derivation}
  \end{tabular}
\end{center}
\end{demo}

Antes de continuar, se aclarará algo de la notación. Sean $a\in\IRh$,
$(b_1,\dots,b_m)$ una $m$-upla de elementos de $\IRh$ y $V$ un predicado
en $\Rh$ con $c_1,\dots,c_n$ constantes las cuales denotan elementos
fijos de $\Rh$.
\begin{itemize}
  \item $\{a\}(i) = \{a(i)\}$
  \item $(b_1,\dots,b_m)(i) = (b_1(i),\dots,b_m(i))$
  \item $\st{V} = V$ reemplazando cada $c_i$ por $\st{c_i}$.
\end{itemize}
Sean $a,b\in \IRh$, otra propiedad que se tiene, en este caso para
la igualdad bajo $\U$ es:
\[a \equ b \equiv \Forall{c}[c\in\IRh]{a\inu c \equiv b \inu c}\]
\begin{demo}
  La demostración se hará por doble implicación.
  Suponiendo que $a\equ b$. Sea $c\in\IRh$.
  \begin{center}
    \begin{tabular}{>{$}c<{$}| >{$}c<{$}}
      \begin{derivation}
          \wff{ a \inu c \land a\equ b}\\
        \equiv\\
          \wff{ \{i\,|\,a(i) \in c(i)\}\in\U \\
            & \land\\
            &\{i\,|\,a(i) = b(i)\}\in\U
          }\\
        \why[\To]{Definición de filtro}\\
          \wff{ \{i\,|\,a(i) \in c(i) \land a(i) = b(i)\}\in\U }\\
        \why[\To]{Definición de filtro}\\
          \wff{ \{i\,|\,b(i) \in c(i)\}\in\U }\\
        \equiv\\
          \wff{ b \inu c }
      \end{derivation}
      &
      \begin{derivation}
        \wff{ b \inu c \land a\equ b}\\
      \equiv\\
        \wff{ \{i\,|\,b(i) \in c(i)\}\in\U\\
          & \land\\
          & \{i\,|\,a(i) = b(i)\}\in\U
        }\\
      \why[\To]{Definición de filtro}\\
        \wff{ \{i\,|\,b(i) \in c(i) \land a(i) = b(i)\}\in\U }\\
      \why[\To]{Definición de filtro}\\
        \wff{ \{i\,|\,a(i) \in c(i)\}\in\U }\\
      \equiv\\
        \wff{ a \inu c }
      \end{derivation}
    \end{tabular}
  \end{center}

  Por el otro lado:
  \begin{longderivation}<.9>
      \wff{ \Forall{c}[c\in\IRh]{a\inu c \equiv b \inu c} }\\
    \To\\
      \wff{ a\inu \{a\} \equiv b \inu \{a\} }\\
    \equiv\\
      \wff{
        \{i\,|\,a(i) \in \{a\}(i)\} \in\U
        \equiv
        \{i\,|\,b(i) \in \{a\}(i)\} \in\U
      }\\
    \equiv\\
      \wff{
        \{i\,|\,a(i) = a(i)\} \in\U
        \equiv
        \{i\,|\,b(i) = a(i)\} \in\U
      }\\
    \equiv\\
      \wff{ I\in\U \equiv b \equ a }\\
    \equiv\\
      \wff{ b \equ a }
  \end{longderivation}
\end{demo}

Nótese que con estas propiedades, y el hecho de que `$\equ$' es una
relación de equivalencia, se puede demostrar de la misma forma que
es válida la ley del reemplazo en $\inu$. Esto es, en $\IRh$:
\[\Forall{a,b,c,d}[a\inu b \land a\equ c \land b \equ d]{c\inu d}\]

Se puede demostrar que las relaciones `$=$' y `$\in$' se comportan de
la misma manera que su correspondiente para los elementos sobre los que
aplica, esto es, si tomamos elementos de $\Rh$, compararlos con `$=$' o
`$\in$' resulta ser equivalente a comparar su versión $*$ con `$\equ$'
o `$\inu$' respectivamente, debido a esto se dejará la notación $\U$, y
se usará solamente `$=$' y `$\in$'. Esto facilitará la
continuidad en algunas demostraciones.

De la misma manera en que se demostró la ley del reemplazo para
la extensión de `$\in$', sería posible ver que la validez de una
proposición en $\Rh$, se mantiene en $\IRh$. Sin embargo, la
demostración se hará de una forma más general, para evitar el proceso
cada que sea necesario validar una proposición en $\IRh$. Para esto se
considerará el lenguaje formal de la lógica de primer orden y los
elementos de $\Rh$. Considerando que relaciones
básicas en $\Rh$ son `$=$' y `$\in$', ya se cuenta con símbolos de
relación correspondientes en $\IRh$. Por lo que se puede generar un
lenguaje formal con $\IRh$, cuyas proposiciones dependen de $\U$.
Primero se mostrarán propiedades de la incorporación de $\Rh$ en $\IRh$.

\begin{lemma}\label{lema:stR}~
  \begin{enumerate}
    \item $\st{\varnothing} = \varnothing$
    \item $\Forall{a,b}[a,b\in\Rh]{a \subseteq b \To \st{a} \subseteq \st{b}}$
    \item $\Forall{a,b}[a,b\in\Rh]{a\in b \equiv \st{a} \in \st{b}}$
    \item Sean $n\in\J$ y $\{a_k\}$ una colección de $n$ elementos de
          $\Rh$. Entonces:
          \begin{itemize}
            \item $\st{\left(\,\bigcup_{i=1}^n a_i\right)}=\bigcup_{i=1}^n \st{a_i}$
            \item $\st{\left(\,\bigcap_{i=1}^n a_i\right)}=\bigcap_{i=1}^n \st{a_i}$
            \item $\st{\{a_1,\dots,a_n\}} = \{\st{a_1},\dots,\st{a_n}\}$
            \item $\st{(a_1,\dots,a_n)} = (\st{a_1},\dots,\st{a_n})$
            \item $\st{(a_1 \times \dots \times a_n)} = 
                    \st{a_1} \times \dots \times \st{a_n}$
          \end{itemize}
    \item $\Forall{a,b}[a,b\in\Rh]{\st{(a - b)} = \st{a} - \st{b}}$
    \item Sea $b\in\Rh$ una relación binaria.
          \begin{itemize}
            \item $\st{(\dom b)} = \dom \st{b}$
            \item $\st{(\ran b)} = \ran \st{b}$
          \end{itemize}
  \end{enumerate}
\end{lemma}
\pagebreak
\begin{demo}~
  \begin{enumerate}
    \item~
          \begin{longderivation}<.9>
              \wff{ x\in\st{\varnothing} }\\
            \equiv\\
              \wff{ \{i\,|\,x(i) \in \varnothing\} \in\U }\\
            \equiv\\
              \wff{ \varnothing \in \U }\\
            \equiv\\
              \wff{ x\in\varnothing }
          \end{longderivation}
    \item Sean $a,b\in\Rh$ tales que $a\subseteq b$. Sea $x\in\IRh$
          \begin{longderivation}<.9>
              \wff{ x\in\st{a} }\\
            \equiv\\
              \wff{ \{i\,|\,x(i) \in a\} \in\U}\\
            \why[\To]{Definiciónd de filtro, $a\subseteq b$}\\
              \wff{ \{i\,|\,x(i) \in b\}\in\U }\\
            \equiv\\
              \wff{ x \in \st{b} }
          \end{longderivation}
    \item Sean $a,b\in\Rh$
          \begin{longderivation}<.9>
              \wff{ \st{a} \in \st{b} }\\
            \equiv\\
              \wff{ \{i\,|\,a \in b\} \in \U }
          \end{longderivation}
          Por un lado, suponiendo que $a\in b$, se tendría que $I\in\U$,
          cosa que es verdadera. Por otro lado, suponiendo que $\st{a} \in \st{b}$,
          se tendría que, efectivamente $\{i\,|\,a \in b\}\in\U$, recordando que,
          para un $c\in\Rh$, se definió, que para todo $i\in I$, $\st{c}(i) = i$.
          La expresión que define $\st{a} \in \st{b}$, nuevamente conduce a $I\in\U$.
    \item Sean $n\in\J$, $\{a_k\}$ una colección con $n$ elementos de
          $\Rh$ y $x, (x_1,\dots,x_n)\in\IRh$.
          \begin{itemize}
            \item Para la intersección
            
            \noindent\makebox[14.5cm]{
                  \begin{tabular}{>{$}c<{$} | >{$}c<{$}}
                    \begin{derivation}
                        \wff{ x \in \bigcap_{k=1}^n \st{a_k} }\\
                      \equiv\\
                        \wff{ \Forall{k}[1\leq k\leq n]{x \in \st{a_k}} }\\
                      \equiv\\
                        \wff{
                          \Forall{k}[1\leq k\leq n]{
                            \set{i}{x(i) \in a_k}\in\U
                          }
                        }\\
                      \why[\To]{Definción de filtro (ii)}\\
                        \wff{
                          \set{i}{\Forall{k}[1\leq k\leq n]{x(i)\in a_k}}
                        }
                    \end{derivation}
                    &
                    \begin{derivation}
                        \wff{ x \in\st{\left(\,\bigcap_{k=1}^n a_k\right)} }\\
                      \equiv\\
                        \wff{
                          \set{i}{x(i) \in \bigcap_{k=1}^n a_k} \in \U
                        }\\
                      \equiv\\
                        \wff{
                          \set{i}{
                            \Forall{k}[1\leq k\leq n]{x(i)\in a_k}
                          }\in\U
                        }\\
                      \why[\To]{Definición de filtro (iii)}\\
                        \wff{
                          \Forall{k}[1\leq k\leq n]{
                            \set{i}{x(i) \in a_k}\in\U
                          }
                        }
                    \end{derivation}
                  \end{tabular}
                }
                \vspace{10pt}
            \item Para la unión.

                  Primero se mostrará que
                  $\bigcup_{k=1}^n\st{a_k}\subseteq
                  \st{\left(\bigcup_{k=1}^n a_k\right)}$

                  \begin{longderivation}
                      \wff{ x\in\bigcup_{k=1}^n\st{a_k} }\\
                    \equiv\\
                      \wff{ \Exists{k}[1\leq k\leq 1]{x\in\st{a_k}} }\\
                    \equiv\\
                      \wff{ \Exists{k}[1\leq k\leq n]{
                        \set{i}{x(i) \in a_k}\in\U
                      }}\\
                    \why[\To]{Definición de filtro (iii)}\\
                      \wff{ \set{i}{
                        \Exists{k}[1\leq k\leq n]{x(i) \in a_k}\in\U
                      } }\\
                    \equiv\\
                      \wff{x\in \st{\left(\bigcup_{k=1}^n a_k\right)}}
                  \end{longderivation}

                  Ahora se mostrará que
                  $\st{\left(\bigcup_{k=1}^n a_k\right)}-
                  \bigcup_{k=1}^n\st{a_k}=\varnothing$
                  \begin{longderivation}
                      \wff{
                        x\in\st{\left(\bigcup_{k=1}^n a_k\right)}
                        - \bigcup_{k=1}^n\st{a_k}
                      }\\
                    \equiv\\
                      \wff{
                        \set{i}{x(i)\in\bigcup_{k=1}^n a_k}\in\U
                        \land
                        \lnot\Exists{k}[1\leq k\leq n]{\set{i}{x(i)\in a_k}\in\U}
                      }\\
                    \equiv\\
                      \wff{
                        \set{i}{x(i)\in\bigcup_{k=1}^n a_k}\in\U
                        \land
                        \Forall{k}[1\leq k\leq n]{\set{i}{x(i)\in a_k}\not\in\U}
                      }\\
                    \why{$\U$ es un ultrafiltro}\\
                      \wff{
                        \set{i}{x(i)\in\bigcup_{k=1}^n a_k}\in\U
                        \land
                        \Forall{k}[1\leq k\leq n]{\set{i}{x(i)\not\in a_k}\in\U}
                      }\\
                    \why[\To]{definición de filtro (ii)}\\
                      \wff{
                        \set{i}{x(i)\in\bigcup_{k=1}^n a_k}\in\U
                        \land
                        \set{i}{\Forall{k}[1\leq k\leq n]{x(i)\not\in a_k}}\in\U
                      }\\
                    \why[\To]{definición de filtro (ii)}\\
                      \wff{
                        \set{i}{
                          x(i)\in\bigcup_{k=1}^n a_k
                          \land
                          \Forall{k}[1\leq k\leq n]{x(i)\not\in a_k}
                        }\in\U
                      }\\
                    \equiv\\
                      \wff{ \varnothing \in\U }\\
                    \equiv\\
                      \wff{ x\in\varnothing }
                  \end{longderivation}
                  
            \item Para la colección, el argumento se realiza de la
                  misma forma que para la unión, lo único que cambia es
                  `$\in$' por `$=$'.
            \item Para la $n$-upla, por cómo se define, se puede ver que es
                  la unión de dos conjuntos, lo que significa, que es
                  consecuencia del punto de la unión.
            \item Para el producto cartesiano.

                  Por un lado:
                  \begin{longderivation}
                      \wff{(x_1,\dots, x_n)\in\st{(a_1\times\dots\times a_n)}}\\
                    \equiv\\
                      \wff{
                        \set{i}{(x_1(i),\dots,x_n(i))\in a_1\times\dots\times a_n}\in\U
                      }\\
                    \equiv\\
                      \wff{
                        \set{i}{
                          \Forall{k}[1\leq k\leq n]{x_k(i) \in a_k}
                        }\in\U
                      }\\
                    \why[\To]{Definición de filtro (iii)}\\
                      \wff{
                        \Forall{k}[1\leq k\leq n]{
                          \set{i}{x_k(i)\in a_k} \in\U
                        }
                      }\\
                    \equiv\\
                      \wff{ (x_1,\dots,x_n) \in \st{a_1}\times\dots\times\st{a_n} }
                  \end{longderivation}
                  Por el otro:
                  \begin{longderivation}
                      \wff{ (x_1,\dots,x_n) \in \st{a_1}\times\dots\times\st{a_n} }\\
                    \equiv\\
                      \wff{
                        \Forall{k}[1\leq k\leq n]{
                          \set{i}{x_k(i)\in a_k} \in\U
                        }
                      }\\
                    \why[\To]{Definición de filtro (ii)}\\
                      \wff{
                        \set{i}{
                          \Forall{k}[1\leq k\leq n]{x_k(i) \in a_k}
                        }\in\U
                      }\\
                    \equiv\\
                      \wff{
                        \set{i}{(x_1(i),\dots,x_n(i))\in a_1\times\dots\times a_n}\in\U
                      }\\
                    \equiv\\
                      \wff{(x_1,\dots, x_n)\in\st{(a_1\times\dots\times a_n)}}\\
                  \end{longderivation}
          \end{itemize}

    \item Por doble contenencia:
          \begin{center}
            \begin{tabular}{>{$}c<{$} | >{$}c<{$}}
              \begin{derivation}
                  \wff{ x \in \st{(a-b)} }\\
                \equiv\\
                  \wff{ \set{i}{x(i) \in a \land x(i) \not\in b}\in\U }\\
                \why[\To]{Definición de filtro (iii)}\\
                  \wff{ \set{i}{x(i)\in a}\in\U \land \set{i}{x(i)\not\in b}\in\U }\\
                \why{$\U$ es un ultrafiltro}\\
                  \wff{ \set{i}{x(i)\in a}\in\U \land \set{i}{x(i)\in b}\not\in\U }\\
                \equiv\\
                  \wff{ x \in \st{a} - \st{b} }
              \end{derivation}
              &
              \begin{derivation}
                  \wff{ x \in \st{a} - \st{b} }\\
                \equiv\\
                  \wff{ \set{i}{x(i) \not\in a}\in\U \land \set{i}{x(i)\in b}\not\in\U }\\
                \why[\To]{$\U$ es un ultrafiltro}\\
                  \wff{ \set{i}{x(i)\in a}\in\U \land \set{i}{x(i)\not\in b}\in\U  }\\
                \why[\To]{Definición de filtro (ii)}\\
                  \wff{ \set{i}{x(i) \in a \land x(i) \not\in b}\in\U }\\
                \equiv\\
                  \wff{ x \in \st{(a-b)} }
              \end{derivation}
            \end{tabular}
          \end{center}
    \item~
          \begin{itemize}
            \item Se va a seguir la misma estrategia que se usó en (iv)
                  para la unión. Se mostrará que $\dom \st{b} \subseteq \st{(\dom b)}$
                  \begin{longderivation}
                      \wff{ x\in\dom\st{b} }\\
                    \equiv\\
                      \wff{ \Exists{y}{(x,y)\in\dom\st{b}} }\\
                    \equiv\\
                      \wff{ \Exists{y}{
                        \set{i}{(x(i),y(i))\in b}\in\U
                      } }\\
                    \why[\To]{Definición de filtro (iii)}\\
                      \wff{
                        \set{i}{
                          \Exists{y}{(x(i),y(i))\in b}
                        }\in\U
                      }\\
                    \equiv\\
                      \wff{ x\in\st{(\dom b)} }
                  \end{longderivation}

                  Ahora se mostrará que $\st{(\dom b)} - \dom \st{b} = \varnothing$
                  \begin{longderivation}
                      \wff{ x\in \st{(\dom b)} - \dom \st{b} }\\
                    \equiv\\
                      \wff{
                        \set{i}{x(i) \in \dom b}\in\U
                        \land
                        \lnot\Exists{y}{\set{i}{(x(i),y(i))\in b}\in\U}
                      }\\
                    \equiv\\
                      \wff{
                        \set{i}{x(i) \in \dom b}\in\U
                        \land
                        \Forall{y}{\set{i}{(x(i),y(i))\in b}\not\in\U}
                      }\\
                    \why{$\U$ es un ultrafiltro}\\
                      \wff{
                        \set{i}{x(i) \in \dom b}\in\U
                        \land
                        \Forall{y}{\set{i}{(x(i),y(i))\not\in b}\in\U}
                      }\\
                    \why[\To]{Definición de filtro (ii)}\\
                      \wff{
                        \set{i}{x(i) \in \dom b}\in\U
                        \land
                        \set{i}{\Forall{y}{(x(i),y(i))\not\in b}}\in\U
                      }\\
                    \why[\To]{Definición de filtro (ii)}\\
                      \wff{
                        \set{i}{
                          x(i)\in\dom b \land x(i) \not\in\dom b
                        }\in\U
                      }\\
                    \equiv\\
                      \wff{ \varnothing\in\U }\\
                    \equiv\\
                      \wff{ x \in \varnothing }
                  \end{longderivation}
            \item Para el rango, el argumento es idéntico, cambia el
              	  orden de la pareja ordenada.
          \end{itemize}
  \end{enumerate}
\end{demo}

Ahora, se van a considerar los tipos de elementos que están en $\IRh$.
\pagebreak
\begin{definition}
  $a\in\IRh$ es llamada \emph{entidad interna} cuando existe $n\in\N$
  tal que $a\in\st{\R_n}$. $a$ es llamada \emph{estándar}
  cuando existe $b\in\Rh$ tal que, $a=\st{b}$. Simbólicamente:

  \begin{center}
    \begin{tabular}{>{$}c<{$} | >{$}c<{$}}
      \begin{derivation}
          \wff{\text{$a$ es un entidad interna}}\\
        \equiv\\
          \wff{ \Exists{n}[n\in\N]{a\in\st{\R_n}} }
      \end{derivation}
      &
      \begin{derivation}
        \wff{\text{$a$ es una entidad estándar}}\\
      \equiv\\
        \wff{\Exists{b}[b\in\Rh]{a=\st{b}}}
      \end{derivation}
    \end{tabular}
  \end{center}
  \vspace{20pt}
  El conjunto $\bigcup_{n\in\N}\st{\R_n}$, es llamado la
  ultrapotencia de $\Rh$ con respecto a $\U$, y se denotará como
  $\st{(\Rh)}$.
\end{definition}

Por la definición de elemento interno y estándar, podría parecer que no
existen entidades internas no-estándar. Sin embargo, son estas las que
precisamente logran el objetivo de toda esta construcción.

\begin{theorem}\label{theo:noEst}
  Existen entidades internas no-estándar.
\end{theorem}

\begin{demo}
  Sea $n\in\N$, considere $\R_n$. Nótese que, $\R_n$ es un conjunto
  infinito. Sea $\{a_n\}_{n\in\N}$ una colección de elementos de $\R_n$
  tal que $\Forall{n,m}[n,m\in\N \land n\not= m]{a_n \not= a_m}$.
  Sea $a$ una función de $I$ en $\R_n$ definida por:
  \[a(i) = a_n \quad (i\in I_n)\]
  recordando que $\{I_n\}_{n\in\N}$ es la partición de $I$ que se
  mencionó al principio de esta sección. Primero, se va a mostrar que
  $a$ es una entidad interna. Específicamente, $a\in\st{R_n}$.
  \begin{longderivation}<.9>
      \wff{ a\in\st{\R_n} }\\
    \equiv\\
      \wff{ \set{i}{a(i) \in \R_n}\in\U }\\
    \equiv\\
      \wff{ I \in \U }
  \end{longderivation}

  por contradicción, supóngase que existe $b\in\Rh$ tal que $a=\st{b}$.
  \begin{longderivation}
      \wff{ a=\st{b} }\\
    \equiv\\
      \wff{ \set{i}{a(i) = b} }\\
    \why{$b$ es constante}\\
      \wff{ \Exists{n}[n\in\N]{I_n\in\U} }
  \end{longderivation}
  Esto último, por la definición de $\{I_n\}$, es falso. Así, se tiene
  que $a$ es una entidad interna no-estándar.
\end{demo}

\begin{theorem}~
  \begin{enumerate}
    \item Una entidad es interna si y solo si pertenece a alguna
          entidad estándar.
    \item Los elementos de entidades internas son internos.
  \end{enumerate}
\end{theorem}

\begin{demo}~
  \begin{enumerate}
    \item La demostración se hará por doble implicación. Por un lado,
          sean $n\in\N$, $c\in\R_{n+1}$ y $b=\st{c}$. Esto es,
          $b$ es una entidad estándar, donde $c$ cumple la definición
          de estándar en $b$.
          \begin{longderivation}<.9>
              \wff{a\in b}\\
            \equiv\\
              \wff{ a\in \st{c} }\\
            \equiv\\
              \wff{\set{i}{a(i)\in c}\in\U}\\
            \why[\To]{\hyperref[lema:Rn]{Propiedad de $\R_n$}}\\
              \wff{\set{i}{a(i)\in\R_0\cup\R_n}\in\U}\\
            \why{\hyperref[lema:stR]{Propiedad de ${^*}$}}\\
              \wff{a\in\st{\R_0}\cup\st{\R_n}}\\
            \To\\
              \wff{\Exists{n}[n\in\N]{a\in\st{\R_n}}}
          \end{longderivation}
          Por el otro lado, sean $n\in\N$ y $a\in\st{\R_n}$ una entidad
          interna. Se puede ver inmediatamente que $a$ es elemento de
          una entidad estándar. En efecto, sean $c=\R_n$, $b=\st{c}$.
          Por el \hyperref[lema:Rn]{lema 3.1}, se tiene que
          $c\in\R_{n+1}$, con lo que $b$ resulta ser una entidad estándar.
    \item Sean $n\in\N$, $b\in\R_{n+1}$ y $a\in b$.
          Sea $B=\set{i}{b\in\R_{n+1}}$. Por el
          \hyperref[lema:Rn]{lema 3.1},
          $B=\set{i}{b\subseteq \R_0\cup\R_n}$.
          \begin{longderivation}
              \wff{a\in b}\\
            \To\\
              \wff{\Forall{i}[i\in B]{a(i)\in\R_0\cup\R_n}}\\
            \equiv\\
              \wff{ \set{i}{a(i)\in\R_0\cup\R_n}\in\U }\\
            \why{\hyperref[lema:stR]{lema 4.1}}\\
              \wff{a\in\st{R_0}\cup\st{\R_n}}\\
            \To\\
              \wff{\Exists{n}[n\in\N]{a\in\st{\R_n}}}
          \end{longderivation}
  \end{enumerate}
\end{demo}

Ahora, se va a explorar el comportamiento de las proposiciones al
tomar su incorporación `$({^*})$'.

\begin{theorem}\label{theo:FT}
  Sean $a\in\Rh$, $V(x_1,\dots,x_m)$ un predicado sobre $x_1,\dots,x_m$
  en $\Rh$ y
  \[A=\set{(x_1,\dots,x_m)\in a}{V(x_1,\dots,x_m)}\]
  Entonces $A\in\Rh$ y 
  \[\st{A}=\set{(x_1,\dots,x_m)\in\st{a}}{\st{V(x_1,\dots,x_p)}}\]
\end{theorem}
\begin{demo}
  Sea $n\in\N$ y $a\in\R_{n+1}$. Como $A\subseteq a$, y por definición
  $a\subseteq \R_0\cup\R_n$, se tiene que $A\subseteq\R_0\cup\R_n$, que
  equivale a $A\in\R_{n+1}$. Para el segundo, por doble contenencia:
  \begin{longderivation}
      \wff{(y_1,\dots,y_m) \in\st{A}}\\
    \equiv\\
      \wff{
        (y_1,\dots,y_m)\in\st{
          \set{(x_1,\dots,x_m)\in a}{V(x_1,\dots,x_m)}
        }
      }\\
    \equiv\\
      \wff{
        \set{i}{
          (y_1(i),\dots,y_m(i))\in a \land V(y_1(i),\dots,y_m(i))
        }\in\U
      }\\
    \why[\To]{\hyperref[def:filtro]{definición de filtro (iii)}}\\
      \wff{
        \set{i}{
          (y_1(i),\dots,y_m(i))\in a}\in\U
        \land
        \set{i}{
          V(y_1(i),\dots,y_m(i))}\in\U
      }\\
    \equiv\\
      \wff{
        (y_1,\dots,y_m)\in\st{a} \land \st{V(y_1,\dots,y_m)}
      }\\
    \equiv\\
      \wff{(y_1,\dots,y_m)\in\set{(x_1,\dots,x_m)\in\st{a}}{\st{V(x_1,\dots,x_m)}}}
  \end{longderivation}

  Por el otro lado:
  \begin{longderivation}
      \wff{
        (y_1,\dots,y_m)\in\set{(x_1,\dots,x_m)\in
        \st{a}}{\st{V(x_1,\dots,x_m)}}
      }\\
    \equiv\\
      \wff{
        (y_1,\dots,y_m)\in\st{a} \land \st{V(y_1,\dots,y_m)}
      }\\
    \equiv\\
    \wff{
      \set{i}{
          (y_1(i),\dots,y_m(i))\in a}\in\U
        \land
        \set{i}{
          V(y_1(i),\dots,y_m(i))}\in\U
      }\\
    \why[\To]{\hyperref[def:filtro]{definición de filtro (ii)}}\\
      \wff{
        \set{i}{
          (y_1(i),\dots,y_m(i))\in a \land V(y_1(i),\dots,y_m(i))
        }\in\U
      }\\
    \equiv\\
      \wff{
        (y_1,\dots,y_m)\in\st{
          \set{(x_1,\dots,x_m)\in a}{V(x_1,\dots,x_m)}
        }
      }\\
    \equiv\\
      \wff{
        (y_1,\dots,y_m)\in\st{A}
      }
  \end{longderivation}
\end{demo}

Ya establecida la equivalencia entre predicados del lenguaje formal
en $\Rh$ y la incorporación en $\st{(\Rh)}$, falta abordar las sentencias,
las cuales son proposiciónes sin variables libres. En el siguiente
teorema se establecerá la equivalencia entre sentencias de ambos
lenguajes formales.

\begin{theorem}
  Sea $V$ una sentencia válida en $\Rh$. $V$ es verdadera en $\Rh$ si y
  solo si, $\st{V}$ es verdadera en $\st{(\Rh)}$.
\end{theorem}

\begin{demo}
  De igual forma que se abordó el problema de los predicados tratando
  con conjuntos, se puede abordar este problema. Para esto, hace falta
  entender a qué corresponde una sentencia de $\Rh$ si se piensa
  denotar en términos de un conjunto. Antes de empezar a trabajar con
  cuantificadores, se considerará una sentencia sin cuantificadores.

  Sea $V$ es una sentencia sin cuantificadores y atómica. Sean
  $a_1,\dots,a_m$ las constantes de $V$. Entonces $V$ es de la
  forma $(a_1,\dots,a_{n-1})\in a_n$ o cualquier otra combinación.
  Entonces:
  \begin{longderivation}<.8>
      \wff{\st{(a_1,\dots,a_{n-1})}\in \st{a_n}}\\
    \equiv\\
      \wff{\set{i}{(a_1,\dots,a_{n-1}) \in a_n}\in\U}
  \end{longderivation}
  Está claro que se tiene la equivalencia en este caso. Si $V$ es una
  sentencia sin cuantificadores, pero no es atómica, es una sentencia
  compuesta de varias sentencias atómicas unidas por conectores lógicos,
  en este caso, el conjunto que debe pertenecer a $\U$ cuando $V$ tiene
  conectores lógicos se puede descomponer en uniones, intersecciones o
  complementos, con lo que, para sentencias sin cuantificadores, queda
  demostrado.

  Para una sentencia con cuantificadores. Sea $W(x_1,\dots,x_{n+1})$ un
  predicado sin cuantificadores sobre $x_1,\dots,x_{n+1}$. Sean
  $q_1,\dots,q_{n+1}$ cuantificadores y $V$ una sentencia de la forma
  \[V=(q_{n+1}x_{n+1}\,|:\,(q_n x_n\,|:\,\dots(q_1x_1\,|:\,W(x_1,\dots,x_{n+1}))\dots))\]
  Suponiendo que $q_{n+1}$ es un cuantificador existencial, $V$ se puede escribir en
  términos de un conjunto específico. Sea $b$ el dominio de $q_{n+1}$ y defínase $A$
  como:
  \[A = \set{x}{x\in b \land (q_n x_n\,|:\,\dots(q_1x_1\,|:\,W(x_1,\dots,x_n,x))\dots)}\]
  La sentencia $V$ se puede expresar en términos de $A$ de la siguiente forma:
  \[A\not=\varnothing\]
  Debido a que la definición de $A$ es un predicado sobre $x$, por el teorema
  anterior se tiene que:
  \[
    \st{A} =
    \set{x}{
      x\in\st{b} \land
      (q_n x_n\,|:\,\dots(q_1x_1\,|:\,\st{W(x_1,\dots,x_n,x)})\dots)
    }  
  \]
  Debido a que $A\not=\varnothing \equiv \st{A} \not= \st{\varnothing}$ y que
  $\varnothing=\st{\varnothing}$. Para el caso del cuantificador universal, se
  niega la proposición, obteniendo así un cuantificador existencial. De la
  misma forma que para sentencias atómicas, $V$ es una sentencia con conectores,
  estos se pueden traducir en el conjunto definido a uniones, complementos o
  intersecciones.
\end{demo}

\begin{theorem}
  Sean $V$ un predicado con variables libre $x_1,\dots,x_m$ y
  $a\in\st{(\Rh)}$. Entonces, el conjunto $A$ definido por
  $A=\set{(x_1,\dots,x_n)}{(x_1,\dots,x_n)\in a \land V}$ es interno.
\end{theorem}
\begin{demo}
  Sea $n\in\N$ tal que $a\in\R_{n+1}$. Se puede ver que $A$ es
  subconjunto de $a$. Por definición $a\subseteq \R_0 \cup \R_n$,
  lo que implica que $A\in\R_0\cup\R_n$, que a su vez implica que
  \[\Exists{n}[n\in\N]{A\in\R_n}\]
\end{demo}

Se presentarán dos teoremas que se cumplen en $\R$, los cuales muestran
como ejemplo la interpretación correcta del teorema \ref{theo:FT}.
En estos ejemplos, las sentencias tienen cuantificadores sobre
\textbf{subconjuntos} de $\R$. Para poder aplicar el teorema de forma
correcta, hace falta modificar la expresión para que esté escrita con
las relaciones básicas `$\in$' o `$=$'. La razón de esto, es que cualquier
predicado o sentencia cuantificada sobre subconjuntos de $\R$, no tiene
sentido realmente al considerar el teorema \ref{theo:FT}, pues este hace
referencia a las relaciones básicas. En el caso de los subconjuntos,
la forma de interpretarlo es que se toman subconjuntos $S$ de $\R$, por
lo que la versión estrella tomará subconjuntos $\st{S}$ de $\st{\R}$. Más
adelante se presentarán ejemplos los cuales no pueden aplicarse en
estas sentencias.

\begin{enumerate}
  \item Todo subconjunto no vacío de $\N$ tiene un primer elemento:
        La sentencia se escribe de forma equivalente como
        ``Todo elemento de $\Pts{\N} - \varnothing$ tiene un primer
        elemento''. Al hacer uso del teorema, lo que se obtiene es:

        Todo elemento de $\st{(\Pts{\N}-\varnothing)}$ tiene un primer
        elemento. Nótese que esto se puede transformar a la siguiente
        sentencia:

        Todo subconjunto interno no vacío de $\st{N}$ tiene un primer
        elemento.
  \item Todo subconjunto de $\R$ acotado superiormente tiene una mínima
        cota superior. De igual forma, al usar el teorema, la sentencia
        correspondiente sería:

        Todo subconjunto interno de $\R$ acotado superiormente tiene
        una mínima cota superior.
\end{enumerate}


\section{El sistema numérico no estándar \texorpdfstring{$\st{\R}$}{R}}

Por el \hyperref[theo:FT]{teorema 4.3}, las mismas propiedades que aplican
en $\R$, aplican en $\st{\R}$ hasta donde apliquen en ambos simultaneamente.
Para simplificar la notación, se mantendrá el mismo símbolo para
denotar operaciones usuales en $\R$ y en $\st{\R}$.
Por ejemplo, en $\st{\R}$, $a + b = c$ es equivalente a
$\set{i}{a(i) + b(i) = c(i)}\in\U$. De la misma manera para la
resta, multiplicación. De igual forma, en $\st{\R}$, $a \leq b$ es
equivalente a $\set{i}{a(i)\leq b(i)}\in\U$. Un ejemplo de la
aplicación del \hyperref[theo:FT]{teorema 4.3} es la siguiente sentencia.
\[\Forall{x,y}[x,y\in\R]{x<y \;\lor\; x>y \;\lor\; x=y}\]
Dado que esta sentencia es verdadera en $\R$, se tiene entonces que la
sentencia
\[\Forall{x,y}[x,y\in\st{\R}]{x<y \;\lor\; x>y \;\lor\; x=y}\]
es verdadera en $\st{\R}$.

Nótese que de la misma manera, es inmediato que el elemento neutro del
producto en $\st{\R}$ es $\st{1}$. Por conveniencia, se omitirá la notación
$({^*})$ para elementos de $\st{R}$ y por lo mencionado antes,
se tomará $\R \subset \st{R}$.

Aclarando más acerca de algunas funciones comunes en $\R$, el valor
absoluto, definido por:
\[
  |r| =
  \begin{syseq}(\{,\})
    r   &&\quad (r > 0)\\
    0   &&\quad (r = 0)\\
    -r  &&\quad (r < 0)
  \end{syseq}
\]
es una función de $\R$ en $\R^{\geq 0}$. Debido a la incorporación
del sistema formal de $\R$ en $\st{\R}$, se tiene una función
correspondiente en $\st{\R}$, la cual iría de $\st{\R}$ a
$\st{\R^{\geq 0}}$. Por el \hyperref[theo:FT]{teorema 4.3}, esta función
cumple que:
\[
  \st{|r|} =
  \begin{syseq}(\{,\})
    r   &&\quad (r > 0)\\
    0   &&\quad (r = 0)\\
    -r  &&\quad (r < 0)
  \end{syseq}
\]
por supuesto, para $r\in\st{\R}$. Debido a esto, se denotará el valor
absoluto en $\st{\R}$ de la forma usual. Por la misma razón, se denotará
de manera usual las funciones $\min$ y $\max$.

Sea $S\subseteq\R$, al pasar a $\st{\R}$, $\st{S}$ corresponde a un
subconjunto de $\st{\R}$, el cual mantienen las mismas propiedades que $S$
hasta donde tenga sentido en ambos simultaneamente. Así como se definió
la superestructura $\Rh$, se puede definir una para $S$, $\widehat{S}$, y,
de igual forma, se puede definir una ultrapotencia de $\widehat{S}$ denotada
por $\st{(\widehat{S})}$. De igual forma, por lo mencionado anteriormente,
se tomará $S\subseteq\st{S}$. Nótese, que al dejar la notación $({^*})$,
por el \hyperref[lema:stR]{lema 4.1}, se tiene que $S=\st{S}$ si y solo
si $S$ es finito.

Un resultado del álgebra nos dice que todo cuerpo arquimediano es
isomorfo con un subcuerpo de $\R$. Debido a la existencia de los
infinitesimales en $\st{R}$, se tiene que $\st{\R}$ no es arquimediano.
Sin embargo, $\st{R}$ cumple las mismas propiedades que $\R$, lo que
parecería ser una contradicción. La razón de esto, es que la propiedad
arquimediana es relativa a un conjunto, en el caso de $\R$, es con $\N$.
explícitamente, la propiedad es la siguiente:
\[\Forall{x,y}[x,y\in\R\land0<x\leq y]{\Exists{n}[n\in\N]{y\leq n\,x}}\]
Al hacer uso del \hyperref[theo:FT]{teorema 4.3}, se tiene que
\[\Forall{x,y}[x,y\in\st{\R}\land 0<x\leq y]{\Exists{n}[n\in\st{\N}]{y\leq n\,x}}\]
lo que significa, que $\st{\R}$ es arquimediano con respecto a $\st{N}$.

Recordando el \hyperref[theo:noEst]{teorema 4.1}, este nos dice que existen
entidades internas no-estándar. Lo que nos conduce a el siguiente teorema:

\begin{theorem}~
  \begin{enumerate}
    \item Existe $\omega\in\st{\N}$ tal que, para todo $r\in\R$,
          $\omega>|r|$.
    \item Existe $\e\in\st{\R}$ tal que, $0<\e$ y para todo $r\in\R$,
          $\e < |r|$.
  \end{enumerate}
\end{theorem}

\begin{demo}~
  \item Por la propiedad arquimediana de $\R$ con respecto a $\N$, solo hace falta
        demostrar que para todo $n\in\N$, $\omega > n$.
        Defínase la función $\omega$ de $I$ en $\N$ por:
        \[\omega(i) = n\quad (i\in I_n)\]
        Defínase para todo número natural el conjunto $A_n=\set{i}{\omega(i)\leq k}$.
        Nótese que, para todo $n$, $A_n\not\in\U$. En efecto, sea $k\in\N$
        \begin{longderivation}<.8>
            \wff{A_k}\\
          =\\
            \wff{\set{i}{\omega(i)\leq k}}\\
          =\\
            \wff{\bigcup_{i=0}^k I_i}
        \end{longderivation}
        Como $\U$ es un ultrafiltro, si $\bigcup_{i=0}^k I_i\in\U$, por las
        caracterizaciones de los ultrafiltros, al menos uno de los
        elementos de la unión debe estar en $\U$, lo que resultaría en
        que existe un $n\in\N$ tal que $I_n\in\U$. Esto es una
        contradicción por la definición de $\U$ y $\{I_n\}$. Así,
        \begin{longderivation}
            \wff{
              \Forall{n}[n\in\N]{
                \set{i}{\omega(i)\leq n}\not\in\U
              }
            }\\
          \why{$\U$ es un ultrafiltro}\\
            \wff{
              \Forall{n}[n\in\N]{
                \set{i}{\omega(i) > n}\in\U
              }
            }\\
          \equiv\\
            \wff{\Forall{n}[n\in\N]{\omega > n}}
        \end{longderivation}
  \item Por la propiedad arquimediana de $\R$ con respecto a $\N$, solo
        hace falta demostrar que para todo
        $n\in\N$, $\e < \frac{1}{n+1}$. Defínase la función $\e$ de $I$
        en $\R$ por:
        \[\e(i)=\frac{1}{n+1} \quad (i\in I_n)\]
        De la misma manera, se define para todo número natural el
        conjunto $ B_n=\set{i}{\e(i)\geq \frac{1}{k+1}}$. Sea $k\in\N$
        \begin{longderivation}<.8>
            \wff{B_k}\\
          =\\
            \wff{\set{i}{\e(i) \geq \frac{1}{k+1}}}\\
          =\\
            \wff{\bigcup_{i=0}^k I_i}
        \end{longderivation}
        Nuevamente se repite el argumento y se concluye que:
        \[\Forall{n}[n\in\N]{\e < \frac{1}{n+1}}\]
\end{demo}

\begin{definition}
  Un número $a\in\st{\R}$ es llamado \emph{finito} cuando existe
  $r\in\R^+$ tal que $|a| < r$. un número no finito es llamado
  \emph{infinto}.

  Un número $a\in\st{\R}$ es llamado \emph{infinitesimal} cuando, para
  todo $r\in\R^+$, $|a|<r$.

  El conjunto de todos los números finitos se denotará como $M_0$ y el
  conjunto de todos los infinitesimales como $M_1$.
\end{definition}

Nótese que $\R\subseteq M_0$, $M_1\subseteq M_0$ y $\R \cap M_1 = \{0\}$,
esto es, $0$ es el único infinitesimal estándar. Se puede ver que el
recíproco de un número infinito es infinitesimal, y el recíproco de
un número infinitesimal distinto de $0$, es infinito.

\begin{theorem}
  Sea $n\in\st{\N}$, $n$ es finito si y solo si $n$ es un número natural
  estándar. Esto es, $\st{\N} \cap M_0 = \N$.
\end{theorem}

\begin{demo}
  La demostración se hará por doble implicación. Por un lado, si
  $n\in\N$, se puede considerar $n+1$. Este número cumple la definición
  de que $n$ es finito. Por el otro lado, si $n\in\st{\N}$ es finito,
  entonces, existe $r\in\R^+$ tal que $n<r$. En $\R$, se cumple que
  $n<r$ es equivalente a que $n\in\set{m\in\N}{m<r}$, el cual es un
  conjunto no vacío para todo $r>0$. Debido a esto, la propiedad
  se cumple en $\st{\R}$ por el \hyperref[theo:FT]{teorema 4.3}.
\end{demo}

Considerando $M_0$. Este es un subanillo de $\st{R}$, en efecto, es
cerrado bajo la suma y el producto. Asímismo, $M_1$ es un subanillo de
$M_0$. La razón es la misma, sin embargo puede no parecer tan intuitivo.
Considere un par de funciones de $I$ en $\st{R}$ las cuales definen un
par de números infinitesimales, la suma y el producto de estas resulta
en un infinitesimal, cosa que se puede corroborar fácilmente realizando
un proceso como el que se usó para mostrar que existen los
infinitesimales.

\begin{lemma}
  $M_1$ es un ideal maximal de $M_0$.
\end{lemma}
\begin{demo}
  Lo primero es mostrar que $M_1$ es un ideal de $M_0$. En efecto, sean
  $h\in M_1$ y $a\in M_0$. Por definición de infinitesimal:
  \begin{longderivation}<1.1>
      \wff{\Forall{r}[r\in\R^+]{ |h|<r}}\\
    \equiv\\
      \wff{\Forall{r}[r\in\R^+]{ |a\,h| < |a|\,r}}\\
    \why[\To]{$|a|\,r \in \R^+$}\\
      \wff{\Forall{r'}[r'\in\R^+]{|a\,h| < r'}}\\
    \equiv\\
      \wff{a\,h \in M_1}
  \end{longderivation}
  Sea $a\in M_0 - M_1$, por definición, se tiene que:
  \begin{longderivation}<.9>
      \wff{\Exists{r_1,r_2}[r_1,r_2\in\R^+]{r_1 < |a| < r_2}}\\
    \equiv\\
      \wff{\Exists{r_1,r_2}[r_1,r_2\in\R^+]{\frac{1}{r_1} > \frac{1}{|a|} > \frac{1}{r_2}}}\\
    \equiv\\
      \wff{\frac{1}{a}\in M_0 - M_1}
  \end{longderivation}
  Sea $X$ un ideal de $M_0$ tal que $M_1\subset X$. entonces existe un
  $b\in M_0 - M_1$ tal que $b\in X$. Sin embargo, como se mostró,
  $\frac{1}{b}\in M_0$, con lo que, por definición de ideal,
  $b\,\frac{1}{b}=1\in X$, lo que a su vez implica que $X=M_0$. Así,
  queda demostrado que $M_1$ es un ideal maximal de $M_0$.
\end{demo}

Defínase la relación de equivalencia $=_1$ en $\st{\R}$ por:
$a=_1 b \equiv a - b \in M_1$. Considere entonces el anillo cociente
$M_0/M_1$.

\begin{theorem}
  El anillo cociente $M_0/M_1$ is isomorfo al cuerpo $\R$ de los números
  reales estándar.
\end{theorem}

\begin{demo}
  Por un lado, a cada real se le puede asignar una clase de
  equivalencia. En efecto, en cada clase de equivalencia no pueden
  haber dos números reales estándar distintos. Sean $r_1$, $r_2$ dos
  números reales usuales en la misma clase de equivalencia.
  \begin{longderivation}
      \wff{r_1 - r_2 \in M_1 \land |r_1 - r_2| \not= 0}\\
    \To\\
      \wff{ |r_1 - r_2| < |r_1 - r_2| }
  \end{longderivation}
  Esto último es una contradicción, claramente. Por otro lado, para todo
  $a\in M_0$, existe un único número real $r$ tal que $a-r=_1 0$. En
  efecto, sea $a\in M_0$, defínanse $D=\set{x\in\R}{x \leq a}$ y
  $D'=\R - D$. La pareja $(D,D')$ define una cortadura de Dedekin en $\R$.
  Sea $r\in\R$ tal que $r$ define la misma cortadura, se puede ver que
  esto implica que $a=_1 r$, pues de lo contrario $a$ o $r$ estaría en
  alguno de los conjuntos que definen la cortadura. Así, $M_0/M_1$ es
  isomorfo con $\R$.
\end{demo}

De lo descrito anteriormente, se puede entonces tratar con los
elementos de $M_0/M_1$ como si fueran los mismos elementos de $\R$.

\begin{definition}
  El homomorfismo de $M_0$ a $\R$ con kernel $M_1$ será llamado
  \emph{homomorfismo de parte estándar} y se denotará por $\mathrm{s\hspace{-0.25pt}t}$.
\end{definition}

A continuación se mostrarán algunas propiedades de este homomorfismo.

\begin{lemma}[Propiedades] Para todo $a,b\in M_0$
  \begin{enumerate}
    \item $\sd{a+b} = \sd{a} + \sd{b}$
    \item $\sd{a\,b} = \sd{a}\,\sd{b}$
    \item $a \leq b \To \sd{a} \leq \sd{b}$
    \item $\sd{|a|} = |\sd{a}|$
    \item $\sd{\max\{a,b\}} = \max\{\sd{a},\sd{b}\}$
    \item $\sd{\min\{a,b\}} = \min\{\sd{a},\sd{b}\}$
    \item $a\in M_1 \equiv \sd{a} = 0$
    \item $a\in\R \To \sd{a} = a$
    \item $\sd{a} \geq 0 \To |a| =_1 \sd{a}$
    \item $a=_1 b \equiv \sd{a} = \sd{b}$
  \end{enumerate}
\end{lemma}

Se puede ver esto debido a cómo se definió el homomorfismo, y el hecho
de que en cada clase de equivalencia de $M_0/M_1$, hay exáctamente un
número real estándar.
\begin{definition}
  Las clases de equivalencia de $M_0$ respecto a $M_1$ serán llamadas
  \emph{mónadas} de los números estándar que las determinan, y serán
  denotadas, para un $r\in\R$, como $\mu(r)$. Por ejemplo,
  $\mu(0) = M_1$.
\end{definition}

Una breve introducción al sistema numérico no estándar:

Siguiendo el hecho de que los números complejos son construidos usando
parejas ordenadas, partiendo de la misma idea, se construye la
superestructura $\widehat{\R\times\R}$ definida por $\R\times\R$. De la
misma forma se llega al conjunto $\st{\left(\widehat{\R\times\R}\right)}$.
La diferencia en este caso, es la estructura algebráica a considerar.

Se toma entonces $\st{\C} = \st{\R}\times\st{\R}$, el cual hereda las
propiedades de $\C$. En $\st{\C}$, se puede definir el concepto de
\emph{finito} e \emph{infinitesimal} de la siguiente manera:
\begin{definition}
  Un número $z\in\st{\C}$ es llamado \emph{finito} cuado existe
  $r\in\R^+$ tal que $\|z\| < r$. Un número no finito es llamado
  \emph{infinito}.

  Un número $z\in\st{\C}$ es llamado \emph{infinitesimal} cuando, para
  todo $r\in\R^+$, $\|z\| < r$.
\end{definition}

Nótese que, bajo estas definiciones, es fácil ver que un número complejo
$z = a + b\,i$ es infinitesimal si y solo si $a,b$ son infinitesimales.


\section{Sucesiones en \texorpdfstring{$\R$}{R}}

Una sucesión en $\R$ es una función de $\N$ en $\R$. Debido a esto, es
un subconjunto de $\N\times\R$, con lo que es una entidad de $\Rh$.
Una sucesión $s$ en $\R$ se extiende a $\st{s}$ como una función de 
$\st{\N}$ en $\st{\R}$. Esto último se debe al teorema \ref{theo:FT} y
al lema \ref{lema:stR}.


\section{Continuidad y Diferenciabilidad}

Una función de variable real y valor real, es un subconjunto de
$\R\times\R$, por lo que, es una entidad de $\Rh$. De igual forma, una
función $f$ se extiende a $\st{f}$ la cual, por el teorema \ref{theo:FT}
y el lema \ref{lema:stR} es un subconjunto de $\st{\R}\times\st{\R}$.
De la misma forma, si $A=\dom f$, para todo $x\in A$, $f(a) = \st{f(a)}$.
Cabe resaltar, que las propiedades de $f$ se mantienen en $\st{\R}$
mientras estas se puedan expresar en ambos lenguajes.

La definición de $\lim_{x\to c} f(x) = a$ en el análisis estándar es
\[
  \Forall{\e}[\e>0]{
    \Exists{\delta}[\delta>0]{
      \Forall{x}[x\in\R\land 0< |x-c|<\delta]{|f(x)-a|<\e}
    }
  }  
\]

De la misma forma que se hizo con las sucesiones, se obtiene una
definición análoga para el análisis no estándar:
\[\Forall{x}[x\in\mu(c)-\{c\}]{\st{f(x)} =_1 a}\]
Definiendo la suma de conjuntos como:
$A+B = \set{a+b}{a\in A \land b\in B}$, se tiene que, para todo
$r\in\R$, $\mu(r) = M_1 + \{r\}$. Usando esto, se puede
reescribir la definición de límite de la siguiente forma
\[\Forall{x}[x\in M_1-\{0\}]{\st{f(c+x)}=_1 a}\]

Nótese que para la continuidad de $f$ en $c$, se tiene respectivamente
la siguiente definición:
\[\Forall{x}[x\in M_1]{\st{f(c+x)}=_1 f(c)}\]

La derivada, en su definición, tiene dos ``versiones'', las cuales son
completamente equivalentes. La derivada de $f$ en $x$ existe si y solo
si los siguientes límites existen (basta con uno, pues son el mismo límite):

\begin{enumerate}
  \item $\lim{h\to 0}\frac{f(x + h) - f(x)}{h}$
  \item $\lim{t\to x}\frac{f(t) - f(x)}{t - x}$
\end{enumerate}

Por simplicidad se considerará el resultado de estos límites como $L$.
En el análisis no estándar, análogamente, la derivada de $f$ en $x$
existe si y solo si, para todo $h\in M_1-\{0\}$:
\[\frac{\st{f(x + h)} - f(x)}{h} =_1 L\]

De igual forma, se puede definir $f'(x)$ como la parte estándar del
cociente que define el límite:
\[f'(x) = \sd{\frac{\st{f(x + h)} - f(x)}{h}}\]

Como ejemplo, se mostrará el uso de estas definiciones para demostrar que
la diferenciabilidad en un punto implica continuidad en ese punto.
Sean $x\in\R$, $f$ una función derivable en $x$ y $h\in M_1-\{0\}$.
\begin{longderivation}<.8>
    \wff{\st{f(x+h)} - f(x)}\\
  =_1\\
    \wff{\frac{\st{f(x + h)} - f(x)}{h}\,h}\\
  =_1\\
    \wff{f'(x)\,h}\\
  =_1\\
    \wff{0}
\end{longderivation}

Nótese que la demostración se puede replicar en el análisis estándar,
ya que la relación `$=_1$' es análoga a decir que algo tiende a un
resultado bajo cierta condición, esta condición vienen a ser los
hiperreales que están operando.


\nocite{Ultrafilters}
\nocite{Robinson}
\printbibliography[heading=bibintoc]

\end{document}