\sect{Eliminación de la recursión}

El algoritmo es fácil e incluso rápido de usar para pocos números. Sin embargo
a medida que los valores deseados aumenten, hace que el proceso sea mucho más lento,
pues para cada iteración, es necesario tener el paso anterior y operar con este en un
$n$ específico. Fuera de ese detalle es sencillo eliminar la recursión, pues tenemos un
número $y_k$, el cual multiplica un polinomio que varía junto con la iteración.
Esto nos dice que entonces, si se tiene una forma general de estos coeficientes, se podría dar
el resultado de una iteración $k$ de la siguiente forma:
\[a_{k}(n) = \sum_{i=0}^{k} y_i \prod_{c=0}^{i-1}(n - c)\]
El único inconveniente para escribirlo de esta forma es precisamente $y_i\ $, pues este valor depende
de la iteración anterior evaluada en $i$. Por lo que se debe encontrar una sucesión
la cual represente el valor de $y$.

Veamos algunos de los valores de $y_k$ con $S = \left\langle x_0, x_1, \dots \right\rangle$

$$
    \begin{matrix*}[l]
        a_0(n) = x_0                &\flechita& y_0 = x_0\\
        a_1(n) = (x_1 - x_0)n + x_0 &\flechita& y_1 = x_1 - x_0\\
        a_2(n) = \dfrac{1}{2} [x_2 - 2x_1 + x_0]n(n-1) + (x_1 - x_0)n + x_0
        & \flechita &y_2 = x_2 - 2x_1  + x_0\\
        \vdots&\vdots&\vdots
    \end{matrix*}
$$

Sabemos desde antes que
\[y_{k + 1} = \dfrac{1}{(k + 1)!}\left[x_{k + 1} - a_{k}(k+1)\right]\]
De esto, el valor que se debe generalizar, es lo que se encuentra entre corchetes. Para visualizar
mejor el resultado de lo anterior, se omitirá la fracción con el factorial y se alinearán
los $x$ correspondientes a su subíndice.

\begin{proofbox}{Valores desde $k=0$ hasta $k=6$}
    $
    \begin{matrix*}[l]
        k = 0 & \flechita& x_0\\
        k = 1 & \flechita& -x_0 + x_1\\
        k = 2 & \flechita& x_0 - 2x_1 + x_2\\
        k = 3 & \flechita& -x_0 + 3x_1 - 3x_2 + x_3\\
        k = 4 & \flechita& x_0 - 4x_1 + 6x_2 - 4x_3 + x_4\\
        k = 5 & \flechita& -x_0 + 5x_1 - 10x_2 + 10x_3 - 5x_4 + x_5\\
        k = 6 & \flechita& x_0 -6x_1 + 15x_2 - 20x_3 + 15x_4 - 6x_5 + x_6
    \end{matrix*}
    $
\end{proofbox}

Ahora, como los valores de $S$ no deben afectar el comportamiento de cada $y$, y para visualizar
aún mejor el comportamiento de estos valores, vamos a organizarlos en una matriz,
la cual tenga por columna el índice de cada $x$ y como fila los valores de $k$. De esta forma,
podremos ver los coeficientes de los valores deseados en cada $y_k$

\[
\begin{blockarray}{cccccccc}
        & x_0     & x_1   & x_2   & x_3   & x_4   & x_5   & x_6\\
    \begin{block}{c(ccccccc)}
    0   &   1     & 0     & 0     & 0     & 0     & 0     & 0\\
    1   &   -1    & 1     & 0     & 0     & 0     & 0     & 0\\
    2   &   1     & -2    & 1     & 0     & 0     & 0     & 0\\
    3   &   -1    & 3     & -3    & 1     & 0     & 0     & 0\\
    4   &   1     & -4    & 6     & -4    & 1     & 0     & 0\\
    5   &   -1    & 5     & -10   & 10    & -5    & 1     & 0\\
    6   &   1     & -6    & 15    & -20   & 15    & -6    & 1\\
    \end{block}
\end{blockarray}
\]

\newpage

Fijándonos en cada columna por separado, podemos ver que hay cierto
patrón. Para hallarlo, usamos la fórmula recursiva, ya demostrada.\\
Llamaremos $\phi_j(n)$ a la sucesión la cual relaciona $k$ con los valores
de una columna $j$. Tras llegar a un punto en el que las siguientes
iteraciones resultan iguales a una anterior, se halló entonces que:

\[\phi_j(i) = \dfrac{(-1)^{i + j}}{j!}\prod_{c=0}^{j-1}(i - c)\]

Bajo está conjetura, cada $y_k$ se podría obtener directamente con las funciones que
expresan cada columna de la matriz. Nótese que el subíndice de cada función $\phi$
corresponde al mismo subíndice de $x$, y el valor en el que se evalúa, corresponde
a la iteración.

Por ejemplo:

\[
    \begin{derivation}<1.5>
            \res{ y_3 = 
            \dfrac{1}{3!}(\,x_0\,\phi_0(3) + x_1\,\phi_1(3)  + x_2\,\phi_2(3) + x_3\,\phi_3(3)\,) }\\
        \why{ Def.($\phi_j(i)$) }\\
            \res{ y_3 = 
            \dfrac{1}{3!}(-x_0 + 3x_1 - 3x_2 + x_3) }
    \end{derivation}
\]

Por ejemplo, si los valores de la matriz son válidos, se podría expresar la iteración $a_3(n)$ de la siguiente forma:

\[
    \begin{derivation}
            \res{ a_3(n) = \sum_{i = 0}^{3} y_i \prod_{c=0}^{i-1}(n-c)}\\
        \why*{}\\
            \res{ a_3(n) = y_0 + y_1\,n + y_2\,n(n-1) + y_3\,n(n-1)(n-2) }\\
        \why{ Conjetura de $y_k$ }\\
            \res{
                \begin{aligned}
                    a_3(n) = &x_0 + (-x_0 + x_1)\,n 
                    + \dfrac{1}{2}(x_0 -2x_1 + x_2)\,n(n-1)\\
                    & + \dfrac{1}{3!}(-x_0 + 3x_1 - 3x_2 + x_3)\,n(n-1)(n-2)
                \end{aligned}
            }
    \end{derivation}
\]

Entonces lo que hace falta demostrar es la siguiente igualdad\\
(el caso base, de $a_0(n)$, es inmediato):

\[a_{k+1}(n) = \sum_{i=0}^{k}y_i\prod_{c=0}^{i-1}(n-c)\]

Antes de seguir con dicha demostración, cabe mencionar que se facilita la lectura\\
de ambas expresiones, en términos de binomios:

\begin{proofbox}{Expresiones con binomios}
    \begin{enumerate}[label=(\Roman*)]
        \item Recursiva
        
        \[\left\{
            \begin{aligned}
                a_0(n)      &= x_0\\
                a_{k+1}(n)  &= \binom{n}{k+1}\left(x_{k+1} - a_k(k+1)\right) + a_k(k+1)
            \end{aligned}
        \right\}\]

        \item Conjetura
        
        \begin{align*}
            a_k(n)  &= \sum_{i=0}^{k}\binom{n}{i}
            \left(x_0\binom{i}{0}(-1)^i + \cdots x_k \binom{i}{k}(-1)^{i+k}\right)\\
                    &= \sum_{i=0}^{k}\binom{n}{i}\sum_{j=0}^{i}\binom{i}{j}(-1)^{i+j}x_j            
        \end{align*}

    \end{enumerate}
\end{proofbox}

La expresión obtenida para la conjetura viene de que el valor de las
funciones $\phi$ solo son diferentes a $0$ desde que son evaluadas
en un valor superior a su subíndice, y este coincide con el binomio
entre ambos valores. A su vez, desde la presentación de la matriz, se puede
ver la relación directa entre los coeficientes y el triángulo de pascal
alternado.

Procediendo con la demostración, llamaremos $a^{(r)}$ a la función bajo la
recursión, y $a^{(c)}$ a la función bajo la conjetura y siendo este un paso
inductivo, suponiendo que la conjetura se cumple para $a_k(n)$:

\[
    \begin{derivation}<2>
            \res{ a_{k+1}^{(c)}(n) = a_{k+1}^{(r)}(n) }\\
        \why{ Igualdad con binomios }\\
            \res{ \sum_{i=0}^{k+1}\binom{n}{i}\sum_{j=0}^{i}\binom{i}{j}(-1)^{i+j}x_j
            = \binom{n}{k+1}(x_{k+1} - a_k(k+1)) + a_k(n) }\\
        \why{ (izq.) La suma hasta $k$ es $a_k(n)$, Hipótesis de inducción }\\
            \res{ \binom{n}{k+1} \sum_{j=0}^{k+1}\binom{k+1}{j}(-1)^{k+1+j}x_j
            =  \binom{n}{k+1}\left(x_{k+1} 
            - \sum_{i=0}^{k}\binom{k+1}{i}\sum_{j=0}^{i}\binom{i}{j}(-1)^{i+j}x_j\right)}\\
        \why*{}\\
            \res{ (-1)^{k+1}\sum_{j=0}^{k+1}\binom{k+1}{j}(-1)^{j}x_j
            = x_{k+1} - \sum_{i=0}^{k}\binom{k+1}{i}\sum_{j=0}^{i}\binom{i}{j}(-1)^{i+j}x_j}\\
        \why{ (izq.) El último término de la suma, es $x_{k+1}$ }\\
            \res{ (-1)^{k+1}\sum_{j=0}^{k}\binom{k+1}{j}(-1)^{j}x_j
            = - \sum_{i=0}^{k}\binom{k+1}{i}\sum_{j=0}^{i}\binom{i}{j}(-1)^{i+j}x_j }\\
        \why*{}\\
            \res{ (-1)^{k}\sum_{j=0}^{k}\binom{k+1}{j}(-1)^{j}x_j
            = \sum_{i=0}^{k}\binom{k+1}{i}\sum_{j=0}^{i}\binom{i}{j}(-1)^{i+j}x_j }
    \end{derivation}
\]

    \vspace{1cm}
    Dado que todo $x_i$ es un valor arbitrario, y su coeficiente no depende en lo absoluto
    del valor que tome este, el problema se reduce a igualar los coeficientes de un $x_m$
    donde $0 \leq m \leq k$.
    Para la suma de sumas en $a_k^{(c)}$, se puede ver que
    \begin{enumerate}[label=(\Roman*)]
        \item $x_m$ empieza a acumular coeficientes desde $i=m$
        \item Los demás elementos de la segunda suma no afectan el resultado para $x_m$, pues
        corresponden a los siguientes $x_i$
    \end{enumerate}
    Por estos resultados, la igualdad de coeficientes queda de la siguiente manera:
    \vspace{1cm}
        \begin{longderivation}<2>
                \res{ (-1)^{k + m}\binom{k+1}{m}
                = \sum_{i=m}^{k}\binom{k+1}{i}\binom{i}{m} (-1)^{i + m}}\\
            \why{$\binom{a}{b}\binom{b}{c} = \binom{a}{c}\binom{a-c}{b - c}$}\\
                \res{ (-1)^{k+m}\binom{k+1}{m}
                = \sum_{i=m}^{k}\binom{k+1}{m}\binom{k+1-m}{i-m}(-1)^{i+m} }\\
            \why*{}\\
                \res{ (-1)^{k+m} = \sum_{i=m}^{k}\binom{k+1-m}{i-m}(-1)^{i+m}}\\
            \why{ Cambio de límites en la suma, $(-1)^{a + 2b} = (-1)^a$ }\\
                \res{ (-1)^{k+m} = \sum_{i=0}^{k-m}\binom{k+1-m}{i}(-1)^i }\\
            \why{ Añadiendo y restando la expresión de la suma en $i=k+1 - m$}\\
                \res{ (-1)^{k+m} = -\binom{k+1-m}{k+1-m}(-1)^{k+1+m}
                + \sum_{i=0}^{k+1-m}\binom{k+1-m}{i}(-1)^{i} }\\
            \why{ Teorema del binomio: $(1 - 1)^{k+1-m}$, $(-1)^a = (-1)^{a - 2}$}\\
                \res{ (-1)^{k+m} = (-1)^{k+m} + (1-1)^{k+1-m} }\\
            \why*{}\\
                \res{ (-1)^{k+m} = (-1)^{k+m} }\\
            \why*{}\\
                \res{ \textit{true} }
        \end{longderivation}

Esto nos demuestra que para cualquier sucesión arbitraria, el término general está dado
por la suma presentada como conjetura.

\newpage

Veamos, el mismo ejemplo presentado anteriormente, la suma de cuadrados.

\[S = \left<0,1,5,14,30,55,\dots\right>\]

Tomando hasta $a_4$, deberíamos obtener un polinomio de grado $4$, por lo que
si el polinomio resultante es de grado menor, podemos dar pausa al algoritmo.


    \begin{longderivation}
            \res{ a_4(n) = \sum_{i=0}^{4}\binom{n}{i}\sum_{j=0}^{i}\binom{i}{j}(-1)^{i+j}x_j }\\
        \why{ Los valores de $x$ corresponden a $S$ }\\
            \res{
                \begin{aligned}
                    a_4(n) =    & 0 + n(-0 + 1) + \frac{n(n-1)}{2}(0 - 2(1) + 5)\\[5pt]
                                & + \frac{n(n-1)(n-2)}{6}(-0 + 3(1) - 3(5) + 14)\\[5pt]
                                & + \frac{n(n-1)(n-2)}{24}(0 - 4(1) + 6(5) - 4(14) + 30)
                \end{aligned}
            }\\
        \why*{}\\
            \res{ a_4(n) = n + \frac{3n(n-1)}{2} + \frac{2n(n-1)(n-2)}{3} + 0 }\\
        \why*{}\\
            \res{ a_4(n) = \frac{n(n+1)(2n+1)}{6} }
    \end{longderivation}

