\sect{Eliminación de la recursión}

El algoritmo es fácil e incluso rápido de usar para pocos números. Sin embargo
a medida que los valores deseados aumenten, hace que el proceso sea mucho más lento,
pues para cada iteración, es necesario tener el paso anterior y operar con este en un
$n$ específico. Fuera de ese detalle es sencillo eliminar la recursión, pues tenemos un
número $y_k$, el cual multiplica un polinomio que varía junto con la iteración.
Esto nos dice que entonces, si se tiene una forma general de estos coeficientes, se podría dar
el resultado de una iteración $k$ de la siguiente forma:
\[a_{k}(n) = \sum_{i=0}^{k} y_i \prod_{c=0}^{i-1}(n - c)\]
El único inconveniente para escribirlo de esta forma, es precisamente $y_i$. Pues este valor depende
de la iteración anterior evaluada en $i$. Por lo que se debe encontrar una sucesión
la cual represente el valor de $y_k$.

Veamos algunos de los valores de $y_k$ con $S = \left\langle x_0, x_1, \dots \right\rangle$

$$
    \begin{matrix*}[l]
        a_0(n) = x_0                &\flechita& y_0 = x_0\\
        a_1(n) = (x_1 - x_0)n + x_0 &\flechita& y_1 = x_1 - x_0\\
        a_2(n) = \dfrac{1}{2} [x_2 - 2x_1 + x_0]n(n-1) + (x_1 - x_0)n + x_0
        & \flechita &y_2 = x_2 - 2x_1  + x_0\\
        \vdots&\vdots&\vdots
    \end{matrix*}
$$

Sabemos desde antes que
\[y_{k + 1} = \dfrac{1}{(k + 1)!}\left[x_{k + 1} - a_{k}(k+1)\right]\]
De esto, el valor que se debe generalizar, es lo que se encuentra entre corchetes. Para visualizar
mejor el resultado de esto, se omitirá la fracción con el factorial.

\begin{proofbox}{Valores desde $k=0$ hasta $k=5$}
    $
    \begin{matrix*}[l]
        k = 0 & \flechita& x_0\\
        k = 1 & \flechita& -x_0 + x_1\\
        k = 2 & \flechita& x_0 - 2x_1 + x_2\\
        k = 3 & \flechita& -x_0 + 3x_1 - 3x_2 + x_3\\
        k = 4 & \flechita& x_0 - 4x_1 + 6x_2 - 4x_3 + x_4\\
        k = 5 & \flechita& -x_0 + 5x_1 - 10x_2 + 10x_3 - 5x_4 + x_5\\
        k = 6 & \flechita& x_0 -6x_1 + 15x_2 - 20x_3 + 15x_4 - 6x_5 + x_6
    \end{matrix*}
    $
\end{proofbox}

Ahora, como los valores de $S$ no deben afectar el comportamiento de cada $y$, y para visualizar
aún mejor el comportamiento de estos valores, vamos a organizar los valores en una matriz,
la cual tenga por columna el índice de cada $x$ y como fila los valores de $k$. De esta forma,
podremos ver los coeficientes de los valores deseados en cada $y_k$

\[
\begin{blockarray}{cccccccc}
        & x_0     & x_1   & x_2   & x_3   & x_4   & x_5   & x_6\\
    \begin{block}{c(ccccccc)}
    0   &   1     & 0     & 0     & 0     & 0     & 0     & 0\\
    1   &   -1    & 1     & 0     & 0     & 0     & 0     & 0\\
    2   &   1     & -2    & 1     & 0     & 0     & 0     & 0\\
    3   &   -1    & 3     & -3    & 1     & 0     & 0     & 0\\
    4   &   1     & -4    & 6     & -4    & 1     & 0     & 0\\
    5   &   -1    & 5     & -10   & 10    & -5    & 1     & 0\\
    6   &   1     & -6    & 15    & -15   & 15    & -6    & 1\\
    \end{block}
\end{blockarray}
\]

Fijándonos en cada columna por separado, podemos ver que hay cierto
patrón. Para hallarlo, usamos la fórmula recursiva, ya demostrada.

Llamaremos $\phi_j(n)$ a la sucesión la cual relaciona $k$ con los valores
de una columna $j$.

Tras llegar a un punto en el que las siguientes iteraciones resultan iguales a una
anterior, se halló entonces que:

\[\phi_j(i) = \dfrac{(-1)^{i + j}}{j!}\prod_{c=0}^{j-1}(i - c)\]
