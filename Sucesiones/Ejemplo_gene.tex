\sect{Ejemplo de uso}
Como sabemos, el algoritmo puede generar expresiones polinómicas.\\
Un caso particular de esto sería la suma de $n$ números naturales elevados
al cuadrado. Partimos de unos cuantos términos, hallados manualmente:

\[S = \left<0, 1, 5, 14, 30, 55, \dots \right>\]

Entonces, hacemos iteraciones hasta poder conjeturar un resultado:

\[
    \RenewDocumentCommand{\arraystretch}{}{2}
    \begin{matrix*}[l]
        a_0(n) &=& 0\\
        a_1(n) &=& (1-0)n = n\\
        a_2(n) &=& \dfrac{1}{2}(5 - 2)n(n-1) + n = \dfrac{n(3n - 1)}{2}\\
        a_3(n) &=& \dfrac{1}{6}(14 - 12)n(n-1)(n-2) + \dfrac{n(3n - 1)}{2} 
        = \dfrac{n(n+1)(2n+1)}{6}\\
        a_4(n) &=& \dfrac{1}{24}(30 - 30)n(n-1)(n-2)(n-3) + \dfrac{n(n+1)(2n+1)}{6}
        = a_3(n)
    \end{matrix*}
\]

En el momento en el que una iteración $k$ funciona para el término deseado en $k+1$,
es momento de detenerse. Es bueno ver si en siguientes valores también cumple, y de ser
el caso, establecer la conjetura. Posteriormente, usar lo obtenido y demostrar la validez
de la fórmula hallada. En nuestro ejemplo, la iteración $a_3$ efectivamente corresponde
al resultado de sumar $n$ números naturales al cuadrado.
