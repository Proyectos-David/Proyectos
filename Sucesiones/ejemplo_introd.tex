\sect{Obtención del algoritmo}

\subsect{Ejemplo introductorio}
Suponga que tiene los siguientes números: $36, 28, 15$\\
Suponga que se les asignan respectivamente los siguientes índices: $0, 1, 2$\\
¿Cómo se relacionan estos índices con los números dados?\\
¿Existe una expresión que represente esta relación?

Para esto, hacemos iteraciones sobre cómo debería ser el término general hasta un índice específico, de la siguiente manera:

\vspace{10pt}
Llamemos $a_0(n)$ el término general que relaciona los índices hasta el $0$ con los números dados.\\
Lo más sencillo es hacer

\[a_0(n) = 36\]

Ahora veamos cómo podríamos obtener $a_1(n)$: tenemos información de que
$a_0(n)$ funciona para $n = 0$, entonces forzamos un término nuevo junto
a este, que no altere a $a_0$ cuando $n=0$ pero que en $n=1$ nos dé el
resultado deseado.
Suponemos un $y$ que cumpla el siguiente sistema:
\begin{equation*}
    \left\{
        \begin{aligned}
            a_1(n) &= y\cdot n + 36\\
            28 &= y(1) + 36
        \end{aligned}
    \right\}
\end{equation*}
Nuevamente, ¿Por qué?, $a_1$ se define de esta manera debido a que en $n=0$ el término añadido es nulo, y queda únicamente lo que ya servía ($a_0$) y en $n=1$ queremos que $y$ sea tal que se fuerce el valor de $28$.
Resolviendo, se obtiene que $y = -8$ y
\[a_1(n) = -8n + 36\]

De igual forma, para el siguiente valor, tenemos información de que en $n=0$ y en $n=1$ , $a_1$ puede relacionar los índices con los valores dados. Entonces forzamos un tercer término para poder relacionar el siguiente valor ($15$) sin alterar lo que ya encontramos. De igual forma suponemos un $y$ tal que:
\begin{equation*}
    \left\{
        \begin{aligned}
            a_2(n) &= y\cdot n(n-1) -8n + 36\\
            15 &= y (2)(2-1) - 8(2) + 36
        \end{aligned}
    \right\}
\end{equation*}

{\noindent Resolviendo, se obtiene que $y = -2.5$ y}
\[a_2(n) = -2.5n(n-1) - 8n + 36\]