\documentclass{article}

\usepackage{geometry}
\usepackage{graphicx, fancyhdr}
\usepackage[hidelinks]{hyperref}
\usepackage{setspace}
\usepackage{mathtools}
\usepackage{logicDG}
\usepackage[spanish, es-noshorthands, es-noquoting]{babel}


\ExplSyntaxOn

\NewDocumentCommand{\sect}{m}{\addcontentsline{toc}{section}{#1} \section*{#1}}%

\NewDocumentCommand{\subsect}{m}{\addcontentsline{toc}{subsection}{#1} \subsection*{#1}}%

\NewDocumentCommand{\subsubsect}{m}{\addcontentsline{toc}{subsubsection}{#1} \subsection*{#1}}%
\ExplSyntaxOff


\pagestyle{fancy}
\fancyhf{}
\setlength{\headheight}{70.38103pt}
\rhead{\textit{\author}}
\lhead{\includegraphics[width = 4cm]{\logo}}
\lfoot{Página \thepage}
\rfoot{\titlename}
\renewcommand{\headrule}{\hbox to \headwidth{\color{rojoEci}\leaders\hrule height \headrulewidth\hfill}}
\renewcommand{\footrulewidth}{0.4pt}

\hyphenpenalty=10000

\newcommand{\logo}{"C:/Users/usuario/OneDrive/Documentos/U/logo-eci.jpg"}
\setlength{\parindent}{0pt}

\newcommand{\titlename}{Sucesiones}%
\renewcommand{\author}{{David Gómez, Laura Rincón}}

\definecolor{rojoEci}{RGB}{225, 70, 49}

\doublespacing

\begin{document}
\begin{titlepage}
    \begin{center}
        \vspace*{1cm}

        \textbf{\Huge{\titlename}}

        \vspace{1.5cm}

        \textbf{{\author}}

        \vspace{3cm}

        \includegraphics[width=0.8\textwidth]{\logo}

        \vspace{3cm}

        Matemáticas\linebreak
        Escuela Colombiana de Ingeniería Julio Garavito\linebreak
        Colombia\linebreak
        \today

    \end{center}
\end{titlepage}
\clearpage
\tableofcontents
\clearpage

\sect{Introducción}

El siguiente trabajo trata el tema de sucesiones, específicamente un algoritmo para representarlas con una expresión general.

Se piensa estudiar el comportamiento del algoritmo, con el fin de simplificar su uso y, a su vez, dar avances sobre el alcance que tiene.

\sect{Obtención del algoritmo}
\subsect{Ejemplo introductorio}
Suponga que tiene los siguientes números: $36, 28, 15$

Suponga que se les asignan respectivamente los siguientes índices: $0, 1, 2$

¿Cómo se relacionan estos índices con los números dados?
¿Existe una expresión que represente esta relación?

Para esto, hacemos iteraciones sobre cómo debería ser el término general hasta un índice específico, de la siguiente manera:

Llamemos $a_0(n)$ el término general que relaciona los índices hasta el $0$ con los números dados.

Lo más sencillo es hacer
\[a_0(n) = 36\]
Ahora veamos cómo podríamos obtener $a_1(n)$

Tenemos información de que $a_0(n)$ funciona para $n = 0$, entonces forzamos un término nuevo junto a este, que no altere a $a_0$ cuando $n=0$ pero que en $n=1$ nos dé el resultado deseado.
Suponemos un $y$ tal que se cumpla el siguiente sistema:
\begin{equation*}
    \left\{
        \begin{aligned}
            a_1(n) &= y\cdot n + 36\\
            28 &= y(1) + 36
        \end{aligned}
    \right\}
\end{equation*}
Nuevamente, ¿Por qué?, $a_1$ se define de esta manera debido a que en $n=0$ el término añadido es nulo, y queda únicamente lo que ya servía ($a_0$) y en $n=1$ queremos que $y$ sea tal que se fuerce el valor de $28$
Resolviendo, se obtiene que $y = -8$ y
\[a_1(n) = -8n + 36\]

De igual forma, para el siguiente valor, tenemos información de que en $n=0$ y en $n=1$ , $a_1$ puede relacionar los índices con los valores dados. Entonces forzamos un tercer término para poder relacionar el siguiente valor ($15$) sin alterar lo que ya encontramos. De igual forma suponemos un $y$ tal que:
\begin{equation*}
    \left\{
        \begin{aligned}
            a_2(n) &= y\cdot n(n-1) -8n + 36\\
            15 &= y (2)(2-1) - 8(2) + 36
        \end{aligned}
    \right\}
\end{equation*}

Resolviendo, se obtiene que $y = -2.5$ y
\[a_2(n) = -2.5n(n-1) - 8n + 36\]

\subsect{Obtención general}

Suponga ahora la siguiente sucesión $S$ y sus índices $I$:
\begin{align*}
    S &= \langle x_0, x_1, x_2, \dots x_n\rangle\\
    I &= \langle 0, 1, 2, \dots n \rangle
\end{align*}

El objetivo es obtener el término general de $S$ con índices $I$ .

Procedemos de igual forma que en el ejemplo:

\[a_0(n) = x_0\]

Se tomará la Notación $y_k$ para representar el coeficiente de la iteración $k$ al resolver la ecuación que se mostró en el ejemplo, evitando hacer mención de la suposición de dicho $y$ en cada procedimiento.

\begin{equation*}
    \left\{
        \begin{aligned}
            a_1(n) &= y_1 \cdot n + x_0\\
            x_1 &= y_1(1) + x_0
        \end{aligned}
    \right\}
\end{equation*}

Nuevamente, para $a_2(n)$
\begin{equation*}
    \left\{
        \begin{aligned}
            a_2(n) &= y_2 \cdot n(n-1) + (x_1 - x_0)n + x_0\\
            x_2 &= y_2(2)(1) + (x_1 - x_0)(2) + x_0
        \end{aligned}  
    \right\}
\end{equation*}

Para $a_3(n)$

\begin{equation*}
    \left\{
        \begin{aligned}
            a_3(n) &= y_3 \cdot n(n-1)(n-2) + \dfrac{1}{2}[x_2 - ((x_1  - x_0)(2) + x_0)]n(n-1) + (x_1 - x_0)n + x_0\\
            x_3 &= y_3 (3)(2)(1) + \dfrac{1}{2}[x_2 - ((x_1  - x_0)(2) + x_0)](3)(2) + (x_1 - x_0)(3) + x_0
        \end{aligned}
    \right\}
\end{equation*}

A partir de esta información, se conjetura lo siguiente:

\begin{proofbox}[10]{Algorítmo}
    El término general de una sucesión de $k$ términos, está dada por
    \begin{center}
        \begin{derivation}<1.5>
                \res{ \left\{
                    \begin{aligned}
                        a_k(n) &= \dfrac{1}{k!}[x_k - a_{k-1}(k)]n(n-1)(n-2)\dots(n-(k-1)) + a_{k-1}(n)\\
                        a_0 &= x_0
                    \end{aligned}
                \right\} }\\
            \why{Notación}\\
                \res{ \left\{
                    \begin{aligned}
                        a_k(n) &= \dfrac{1}{k!}[x_k - a_{k-1}(k)]\displaystyle\prod_{c=0}^{k-1}(n-c) + a_{k-1}(n)\\
                        a_0 &= x_0
                    \end{aligned}
                \right\} }
        \end{derivation}
    \end{center}
    Por su puesto, hace falta demostrarlo:
\end{proofbox}
\clearpage
\begin{subproofbox}{demostración}
    Por medio de la inducción, se toma un caso base, el cual será $k=0$. Esto significa que la expresión obtenida funciona para la sucesión $S = \langle x_0\rangle$
    y usando la fórmula conjeturada, se obtiene que
    \[a_0(n) = x_0 \]
    Que efectivamente, cumple.
    Ahora se supone que existe un $b$ tal que la fórmula se cumple para todo número natural menor que o igual a $b$. Queremos ver qué sucede si queremos hallar una sucesión que también pueda seguir el valor de $x_{b+1}$

    Tenemos $S = \langle x_0, x_1, \dots x_b, x_{b+1}\rangle$
    y que $a_b(n)$ relaciona a los naturales con los $x_i$ de $S$ hasta $x_b$

    Queremos hallar una sucesión que pueda relacionar desde $x_0$ hasta $x_{b+1}$, a los naturales. Tomando la información anterior, suponemos que hay un $y_{b+1}$ tal que:

    \begin{equation*}
        \left\{
            \begin{aligned}
                a_{b+1}(n) &= y_{b+1} \displaystyle\prod_{c=0}^{b}(n-c) + a_b(n)\\
                x_{b+1} &= y_{b+1} \displaystyle\prod_{c=0}^{b}(b+1 - c) + a_b(b + 1)                
            \end{aligned}
        \right\}
    \end{equation*}
    Se define de esta manera por la misma razón que se mencionó al principio. Sabemos que $a_b(n)$ sirve para todos los naturales antes que $b+1$ y relaciona cada uno respectivamente con un elemento de $S$. Entonces añadimos un nuevo término el cual se anule en todos los valores de $n$ hasta $b$ de forma que podamos seguir usando $a_b$ y forzando a que en $n=b+1$ el valor obtenido sea $x_{b+1}$ usando $y_{b+1}$ .

    Desarrollando la segunda igualdad.

    \begin{align*}
        y_{b+1} \displaystyle\prod_{c=0}^{b}(b+1 - c) &= x_{b+1} - a_b(b+1)\\
        y_{b+1} (b+1)(b)(b-1)\cdots 1 &= x_{b+1} - a_b(b+1)\\
        y_{b+1} (b+1)! &= x_{b+1} - a_b(b+1)\\
        y_{b+1} &= \dfrac{1}{(b+1)!}[x_{b+1} - a_b(b+1)]
    \end{align*}

    Reemplazando en la primera igualdad, se obtiene lo siguiente:

    \[a_{b+1}(n) = \dfrac{1}{(b+1)!}[x_{b+1} - a_b(b+1)]\displaystyle\prod_{c=0}^{b}(n-c) + a_b(n)\]

    Con este resultado se demuestra la validez de la fórmula.
\end{subproofbox}

\end{document}