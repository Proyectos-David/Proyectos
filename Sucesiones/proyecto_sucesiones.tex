\documentclass{article}

\usepackage{geometry}
\usepackage{graphicx, fancyhdr}
\usepackage[hidelinks]{hyperref}
\usepackage{setspace}
\usepackage{mathtools}
\usepackage{logicDG}
\usepackage{blkarray}
\usepackage{enumitem}
\usepackage[spanish, es-noshorthands, es-noquoting]{babel}



\NewDocumentCommand{\sect}{m}{\addcontentsline{toc}{section}{#1} \section*{#1}}%

\NewDocumentCommand{\subsect}{m}{\addcontentsline{toc}{subsection}{#1} \subsection*{#1}}%

\NewDocumentCommand{\subsubsect}{m}{\addcontentsline{toc}{subsubsection}{#1} \subsection*{#1}}%

% Página

\pagestyle{fancy}
\fancyhf{}
\setlength{\headheight}{70.38103pt}
\rhead{\textit{David G., Laura R., Daniel P.}}
\lhead{\includegraphics[width = 4cm]{\logo}}
\lfoot{Página \thepage}
\rfoot{Sucesiones Arbitrarias}
\renewcommand{\headrule}{\hbox to \headwidth{\color{rojoEci}\leaders\hrule height \headrulewidth\hfill}}
\renewcommand{\footrulewidth}{0.4pt}

\hyphenpenalty=10000

\newcommand{\logo}{"C:/Users/usuario/OneDrive/Documentos/U/logo-eci.png"}
\setlength{\parindent}{0pt}

\newcommand{\titlename}{Sucesiones \\[10pt] Arbitrarias}%
\renewcommand{\author}{{David Gómez, Laura Rincón, Daniel Pérez}}

\definecolor{rojoEci}{RGB}{225, 70, 49}


\doublespacing

%flechita

\usepackage{scalerel}

\NewDocumentCommand{\flechita}{}{\scalerel*{
    \includegraphics{C:/Users/usuario/OneDrive/Documentos/U/flechita}
    }{X}}

\NewDocumentCommand{\vdflechita}{}{\rotatebox[origin=c]{270}{\flechita}}

\NewDocumentCommand{\vuflechita}{}{\rotatebox[origin=c]{90}{\flechita}}


\begin{document}
\begin{titlepage}
    \begin{center}
        \vspace*{1cm}

        \textbf{\Huge{\titlename}}

        \vspace{1.5cm}

        \textbf{\Large{\author}}

        \vspace{3cm}

        \includegraphics[width=0.8\textwidth]{\logo}

        \vfill

        Matemáticas\linebreak
        Escuela Colombiana de Ingeniería Julio Garavito\linebreak
        Colombia\linebreak
        \today

    \end{center}
\end{titlepage}

\clearpage
\tableofcontents
\clearpage

\section{Introduccion}

Cuando Newton y Leibniz trabajaron en la fundamentación del cálculo, una
de sus diferencias fue la definición de límite. Mientras Newton lo
definió de la forma en la que se ha enseñado principalmente, la
definición $\e$ y $\delta$. Leibniz, lo definia de una forma que
incluso parece una versión más amigable que la de Newton. Leibniz
consideraba números infinitamente pequeños, de tal forma que fueran
menores que cualquier número positivo pero mayores a $0$. Junto con
números infinitamente grandes, mayores que cualquier número positivo.

El problema de esta definición, se encontró cuando se intentó fundamentar
formalmente. Cosa que Leibniz ni sus discípulos lograron demostrar.
La definición de Newton recurre a los mismos números reales ya usados.
La definición de Leibniz, recurre a una nueva especie de números, los cuales
deben ser comparables y se deben poder operar con los reales. La idea entonces
con esta nueva especie de números, es poder operar con estos, para
posteriormente tomar el resultado y recuperar la información que interesa,
la que corresponde a un valor real estándar. Esta nueva especie de números
resulta tener aplicaciones en más áreas que el cálculo de límites, sin
embargo, no hacen parte del objetivo de este proyecto, el cual consta de
presentar esta idea en relación al análisis estándar, específicamente,
el análisis diferencial.

Los pasos para la construcción de estos números, recurre a los filtros,
un objeto de la teoría de conjuntos sobre el que se hablará en el
documento. Con estos, se puede lograr la construcción de esta nueva especie
de números sobre los reales. 


\sect{Obtención del algoritmo}

\subsect{Ejemplo introductorio}
Suponga que tiene los siguientes números: $36, 28, 15$\\
Suponga que se les asignan respectivamente los siguientes índices: $0, 1, 2$\\
¿Cómo se relacionan estos índices con los números dados?\\
¿Existe una expresión que represente esta relación?

Para esto, hacemos iteraciones sobre cómo debería ser el término general hasta un índice específico, de la siguiente manera:

\vspace{10pt}
Llamemos $a_0(n)$ el término general que relaciona los índices hasta el $0$ con los números dados.\\
Lo más sencillo es hacer

\[a_0(n) = 36\]

Ahora veamos cómo podríamos obtener $a_1(n)$: tenemos información de que
$a_0(n)$ funciona para $n = 0$, entonces forzamos un término nuevo junto
a este, que no altere a $a_0$ cuando $n=0$ pero que en $n=1$ nos dé el
resultado deseado.
Suponemos un $y$ que cumpla el siguiente sistema:
\begin{equation*}
    \left\{
        \begin{aligned}
            a_1(n) &= y\cdot n + 36\\
            28 &= y(1) + 36
        \end{aligned}
    \right\}
\end{equation*}
Nuevamente, ¿Por qué?, $a_1$ se define de esta manera debido a que en $n=0$ el término añadido es nulo, y queda únicamente lo que ya servía ($a_0$) y en $n=1$ queremos que $y$ sea tal que se fuerce el valor de $28$.
Resolviendo, se obtiene que $y = -8$ y
\[a_1(n) = -8n + 36\]

De igual forma, para el siguiente valor, tenemos información de que en $n=0$ y en $n=1$ , $a_1$ puede relacionar los índices con los valores dados. Entonces forzamos un tercer término para poder relacionar el siguiente valor ($15$) sin alterar lo que ya encontramos. De igual forma suponemos un $y$ tal que:
\begin{equation*}
    \left\{
        \begin{aligned}
            a_2(n) &= y\cdot n(n-1) -8n + 36\\
            15 &= y (2)(2-1) - 8(2) + 36
        \end{aligned}
    \right\}
\end{equation*}

{\noindent Resolviendo, se obtiene que $y = -2.5$ y}
\[a_2(n) = -2.5n(n-1) - 8n + 36\]

\subsect{Obtención general}

Suponga ahora la siguiente sucesión $S$ y sus índices $I$:
\begin{align*}
    S &= \langle x_0, x_1, x_2, \dots x_n\rangle\\
    I &= \langle 0, 1, 2, \dots n \rangle
\end{align*}

El objetivo es obtener el término general de $S$ con índices $I$ .

Procedemos de igual forma que en el ejemplo:

\[a_0(n) = x_0\]

Se tomará la Notación $y_k$ para representar el coeficiente de la iteración $k$ al resolver la ecuación que se mostró en el ejemplo, evitando hacer mención de la suposición de dicho $y$ en cada procedimiento.

\begin{equation*}
    \left\{
        \begin{aligned}
            a_1(n) &= y_1 \cdot n + x_0\\
            x_1 &= y_1(1) + x_0
        \end{aligned}
    \right\}
\end{equation*}

Nuevamente, para $a_2(n)$
\begin{equation*}
    \left\{
        \begin{aligned}
            a_2(n) &= y_2 \cdot n(n-1) + (x_1 - x_0)n + x_0\\
            x_2 &= y_2(2)(1) + (x_1 - x_0)(2) + x_0
        \end{aligned}  
    \right\}
\end{equation*}

Para $a_3(n)$

\begin{equation*}
    \left\{
        \begin{aligned}
            a_3(n) &= y_3 \cdot n(n-1)(n-2) + \dfrac{1}{2}[x_2 - ((x_1  - x_0)(2) + x_0)]n(n-1) + (x_1 - x_0)n + x_0\\
            x_3 &= y_3 (3)(2)(1) + \dfrac{1}{2}[x_2 - ((x_1  - x_0)(2) + x_0)](3)(2) + (x_1 - x_0)(3) + x_0
        \end{aligned}
    \right\}
\end{equation*}

A partir de esta información, se conjetura lo siguiente:

\begin{proofbox}[10]{Algorítmo}
    El término general de una sucesión de $k$ términos, está dada por
    \begin{center}
        \begin{derivation}<1.5>
                \res{ \left\{
                    \begin{aligned}
                        a_k(n) &= \dfrac{1}{k!}[x_k - a_{k-1}(k)]n(n-1)(n-2)\dots(n-(k-1)) + a_{k-1}(n)\\
                        a_0 &= x_0
                    \end{aligned}
                \right\} }\\
            \why{Notación}\\
                \res{ \left\{
                    \begin{aligned}
                        a_k(n) &= \dfrac{1}{k!}[x_k - a_{k-1}(k)]\displaystyle\prod_{c=0}^{k-1}(n-c) + a_{k-1}(n)\\
                        a_0 &= x_0
                    \end{aligned}
                \right\} }
        \end{derivation}
    \end{center}
    Por su puesto, hace falta demostrarlo:
\end{proofbox}
\clearpage
\begin{subproofbox}{demostración}
    Por medio de la inducción, se toma un caso base, el cual será $k=0$. Esto significa que la expresión obtenida funciona para la sucesión $S = \langle x_0\rangle$
    y usando la fórmula conjeturada, se obtiene que
    \[a_0(n) = x_0 \]
    Que efectivamente, cumple.
    Ahora se supone que existe un $b$ tal que la fórmula se cumple para todo número natural menor que o igual a $b$. Queremos ver qué sucede si queremos hallar una sucesión que también pueda seguir el valor de $x_{b+1}$

    Tenemos $S = \langle x_0, x_1, \dots x_b, x_{b+1}\rangle$
    y que $a_b(n)$ relaciona a los naturales con los $x_i$ de $S$ hasta $x_b$

    Queremos hallar una sucesión que pueda relacionar desde $x_0$ hasta $x_{b+1}$, a los naturales. Tomando la información anterior, suponemos que hay un $y_{b+1}$ tal que:

    \begin{equation*}
        \left\{
            \begin{aligned}
                a_{b+1}(n) &= y_{b+1} \displaystyle\prod_{c=0}^{b}(n-c) + a_b(n)\\
                x_{b+1} &= y_{b+1} \displaystyle\prod_{c=0}^{b}(b+1 - c) + a_b(b + 1)                
            \end{aligned}
        \right\}
    \end{equation*}
    Se define de esta manera por la misma razón que se mencionó al principio. Sabemos que $a_b(n)$ sirve para todos los naturales antes que $b+1$ y relaciona cada uno respectivamente con un elemento de $S$. Entonces añadimos un nuevo término el cual se anule en todos los valores de $n$ hasta $b$ de forma que podamos seguir usando $a_b$ y forzando a que en $n=b+1$ el valor obtenido sea $x_{b+1}$ usando $y_{b+1}$ .

    Desarrollando la segunda igualdad.

    \begin{align*}
        y_{b+1} \displaystyle\prod_{c=0}^{b}(b+1 - c) &= x_{b+1} - a_b(b+1)\\
        y_{b+1} (b+1)(b)(b-1)\cdots 1 &= x_{b+1} - a_b(b+1)\\
        y_{b+1} (b+1)! &= x_{b+1} - a_b(b+1)\\
        y_{b+1} &= \dfrac{1}{(b+1)!}[x_{b+1} - a_b(b+1)]
    \end{align*}

    Reemplazando en la primera igualdad, se obtiene lo siguiente:

    \[a_{b+1}(n) = \dfrac{1}{(b+1)!}[x_{b+1} - a_b(b+1)]\displaystyle\prod_{c=0}^{b}(n-c) + a_b(n)\]

    Con este resultado se demuestra la validez de la fórmula.
\end{subproofbox}

\sect{Ejemplo de uso}
Se puede ver del algoritmo, que el uso de este algoritmo viene bastante
bien para la obtención de expresiones polinómicas.

Un caso particular de esto sería la suma de $n$ números naturales elevados
al cuadrado. Partimos de unos cuantos términos, hallados manualmente:

\[S = \left<0, 1, 5, 14, 30, 55, \dots \right>\]

Entonces, hacemos iteraciones hasta poder conjeturar un resultado:

\begin{center}
    \begin{derivation}<1.5>
        a_0(n) &=& 0\\
        a_1(n) &=& (1-0)n = n\\
        a_2(n) &=& \dfrac{1}{2}(5 - 2)n(n-1) + n = \dfrac{n(3n - 1)}{2}\\
        a_3(n) &=& \dfrac{1}{6}(14 - 12)n(n-1)(n-2) + \dfrac{n(3n - 1)}{2} 
        = \dfrac{n(n+1)(2n+1)}{6}\\
        a_4(n) &=& \dfrac{1}{24}(30 - 30)n(n-1)(n-2)(n-3) + \dfrac{n(n+1)(2n+1)}{6}
        = a_3(n)
    \end{derivation}
\end{center}

En el momento en el que una iteración $k$ funciona para el término deseado en $k+1$,
es momento de detenerse. Es bueno ver si en siguientes valores también cumple, y de ser
el caso, establecer la conjetura. Posteriormente, usar lo obtenido y demostrar la validez
de la fórmula hallada. En nuestro ejemplo, la iteración $a_3$ efectivamente corresponde
al resultado de sumar $n$ números naturales al cuadrado.


\sect{Eliminación de la recursión}

El algoritmo es fácil e incluso rápido de usar para pocos números. Sin embargo
a medida que los valores deseados aumenten, hace que el proceso sea mucho más lento,
pues para cada iteración, es necesario tener el paso anterior y operar con este en un
$n$ específico. Fuera de ese detalle es sencillo eliminar la recursión, pues tenemos un
número $y_k$, el cual multiplica un polinomio que varía junto con la iteración.
Esto nos dice que entonces, si se tiene una forma general de estos coeficientes, se podría dar
el resultado de una iteración $k$ de la siguiente forma:
\[a_{k}(n) = \sum_{i=0}^{k} y_i \prod_{c=0}^{i-1}(n - c)\]
El único inconveniente para escribirlo de esta forma, es precisamente $y_i$. Pues este valor depende
de la iteración anterior evaluada en $i$. Por lo que se debe encontrar una sucesión
la cual represente el valor de $y_k$.

Veamos algunos de los valores de $y_k$ con $S = \left\langle x_0, x_1, \dots \right\rangle$

$$
    \begin{matrix*}[l]
        a_0(n) = x_0                &\flechita& y_0 = x_0\\
        a_1(n) = (x_1 - x_0)n + x_0 &\flechita& y_1 = x_1 - x_0\\
        a_2(n) = \dfrac{1}{2} [x_2 - 2x_1 + x_0]n(n-1) + (x_1 - x_0)n + x_0
        & \flechita &y_2 = x_2 - 2x_1  + x_0\\
        \vdots&\vdots&\vdots
    \end{matrix*}
$$

Sabemos desde antes que
\[y_{k + 1} = \dfrac{1}{(k + 1)!}\left[x_{k + 1} - a_{k}(k+1)\right]\]
De esto, el valor que se debe generalizar, es lo que se encuentra entre corchetes. Para visualizar
mejor el resultado de esto, se omitirá la fracción con el factorial.

\begin{proofbox}{Valores desde $k=0$ hasta $k=6$}
    $
    \begin{matrix*}[l]
        k = 0 & \flechita& x_0\\
        k = 1 & \flechita& -x_0 + x_1\\
        k = 2 & \flechita& x_0 - 2x_1 + x_2\\
        k = 3 & \flechita& -x_0 + 3x_1 - 3x_2 + x_3\\
        k = 4 & \flechita& x_0 - 4x_1 + 6x_2 - 4x_3 + x_4\\
        k = 5 & \flechita& -x_0 + 5x_1 - 10x_2 + 10x_3 - 5x_4 + x_5\\
        k = 6 & \flechita& x_0 -6x_1 + 15x_2 - 20x_3 + 15x_4 - 6x_5 + x_6
    \end{matrix*}
    $
\end{proofbox}

Ahora, como los valores de $S$ no deben afectar el comportamiento de cada $y$, y para visualizar
aún mejor el comportamiento de estos valores, vamos a organizar los valores en una matriz,
la cual tenga por columna el índice de cada $x$ y como fila los valores de $k$. De esta forma,
podremos ver los coeficientes de los valores deseados en cada $y_k$

\[
\begin{blockarray}{cccccccc}
        & x_0     & x_1   & x_2   & x_3   & x_4   & x_5   & x_6\\
    \begin{block}{c(ccccccc)}
    0   &   1     & 0     & 0     & 0     & 0     & 0     & 0\\
    1   &   -1    & 1     & 0     & 0     & 0     & 0     & 0\\
    2   &   1     & -2    & 1     & 0     & 0     & 0     & 0\\
    3   &   -1    & 3     & -3    & 1     & 0     & 0     & 0\\
    4   &   1     & -4    & 6     & -4    & 1     & 0     & 0\\
    5   &   -1    & 5     & -10   & 10    & -5    & 1     & 0\\
    6   &   1     & -6    & 15    & -20   & 15    & -6    & 1\\
    \end{block}
\end{blockarray}
\]

Fijándonos en cada columna por separado, podemos ver que hay cierto
patrón. Para hallarlo, usamos la fórmula recursiva, ya demostrada.

Llamaremos $\phi_j(n)$ a la sucesión la cual relaciona $k$ con los valores
de una columna $j$.

Tras llegar a un punto en el que las siguientes iteraciones resultan iguales a una
anterior, se halló entonces que:

\[\phi_j(i) = \dfrac{(-1)^{i + j}}{j!}\prod_{c=0}^{j-1}(i - c)\]

Bajo está conjetura, cada $y_k$ se podría obtener directamente con las funciones que
expresan cada columna de la matriz. Nótese que el subíndice de cada función $\phi$
corresponde al mismo subíndice de $x$, y el valor en el que se evalúa, corresponde
a la iteración.

Por ejemplo:

\[
    \begin{derivation}<1.5>
            \res{ y_3 = 
            \dfrac{1}{3!}(\,x_0\,\phi_0(3) + x_1\,\phi_1(3)  + x_2\,\phi_2(3) + x_3\,\phi_3(3)\,) }\\
        \why{ Def.($\phi_j(i)$) }\\
            \res{ y_3 = 
            \dfrac{1}{3!}(-x_0 + 3x_1 - 3x_2 + x_3) }
    \end{derivation}
\]

Por ejemplo, si los valores de la matriz son válidos, se podría expresar la iteración $a_3(n)$ de la siguiente forma:

\[
    \begin{derivation}
            \res{ a_3(n) = \sum_{i = 0}^{3} y_i \prod_{c=0}^{i-1}(n-c)}\\
        \why*{}\\
            \res{ a_3(n) = y_0 + y_1\,n + y_2\,n(n-1) + y_3\,n(n-1)(n-2) }\\
        \why{ Conjetura de $y_k$ }\\
            \res{ a_3(n) = x_0 + (-x_0 + x_1)\,n + \dfrac{1}{2}(x_0 -2x_1 + x_2)\,n(n-1) }\\
            &\qquad\qquad + \dfrac{1}{3!}(-x_0 + 3x_1 - 3x_2 + x_3)\,n(n-1)(n-2)
    \end{derivation}
\]

Entonces lo que hace falta demostrar es la siguiente igualdad (el caso base, de $a_0(n)$, es inmediato):

\[a_{k+1}(n) = \sum_{i=0}^{k}y_i\prod_{c=0}^{i-1}(n-c)\]

Antes de seguir con dicha demostración, cabe mencionar que se facilita la lectura\\
de ambas expresiones, en términos de binomios:

\begin{proofbox}{Expresiones con binomios}
    \begin{enumerate}[label=(\Roman*)]
        \item Recursiva
        
        \[\left\{
            \begin{aligned}
                a_0(n)      &= x_0\\
                a_{k+1}(n)  &= \binom{n}{k+1}\left(x_{k+1} - a_k(k+1)\right) + a_k(k+1)
            \end{aligned}
        \right\}\]

        \item Conjetura
        
        \begin{align*}
            a_k(n)  &= \sum_{i=0}^{k}\binom{n}{i}
            \left(x_0\binom{i}{0}(-1)^i + \cdots x_k \binom{i}{k}(-1)^{i+k}\right)\\
                    &= \sum_{i=0}^{k}\binom{n}{i}\sum_{j=0}^{i}\binom{i}{j}(-1)^{i+j}x_j            
        \end{align*}

        La expresión obtenida para la conjetura viene de que el valor de las funciones $\phi$
        solo son diferentes a $0$ desde que son evaluadas en un valor superior a su subíndice,
        y este coincide con el binomio entre ambos valores. A su vez, desde la presentación de
        la matriz, se puede ver la relación directa entre los coeficientes y el triángulo
        de pascal alternado.
    \end{enumerate}
\end{proofbox}

Procediendo con la demostración, llamaremos $a^{(r)}$ a la función bajo la recursión,
y $a^{(c)}$ a la función bajo la conjetura y siendo este un paso inductivo, suponiendo
que la conjetura se cumple para $a_k(n)$:

\[
    \begin{derivation}<2>
            \res{ a_{k+1}^{(c)}(n) = a_{k+1}^{(r)}(n) }\\
        \why{ Igualdad con binomios }\\
            \res{ \sum_{i=0}^{k+1}\binom{n}{k+1}\sum_{j=0}^{i}\binom{i}{j}(-1)^{i+j}x_j
            = \binom{n}{k+1}(x_{k+1} - a_k(k+1)) + a_k(n) }\\
        \why{ (izq.) La suma hasta $k$ es $a_k(n)$, Hipótesis de inducción }\\
            \res{ \binom{n}{k+1} \sum_{j=0}^{k+1}\binom{k+1}{j}(-1)^{k+1+j}x_j
            =  \binom{n}{k+1}\left(x_{k+1} 
            - \sum_{i=0}^{k}\binom{k+1}{i}\sum_{j=0}^{i}\binom{i}{j}(-1)^{i+j}x_j\right)}\\
        \why*{}\\
            \res{ (-1)^{k+1}\sum_{j=0}^{k+1}\binom{k+1}{j}(-1)^{j}x_j
            = x_{k+1} - \sum_{i=0}^{k}\binom{k+1}{i}\sum_{j=0}^{i}\binom{i}{j}(-1)^{i+j}x_j}\\
        \why{ (izq.) El último término de la suma, es $x_{k+1}$ }\\
            \res{ (-1)^{k+1}\sum_{j=0}^{k}\binom{k+1}{j}(-1)^{j}x_j
            = - \sum_{i=0}^{k}\binom{k+1}{i}\sum_{j=0}^{i}\binom{i}{j}(-1)^{i+j}x_j }\\
        \why*{}\\
            \res{ (-1)^{k}\sum_{j=0}^{k}\binom{k+1}{j}(-1)^{j}x_j
            = \sum_{i=0}^{k}\binom{k+1}{i}\sum_{j=0}^{i}\binom{i}{j}(-1)^{i+j}x_j }
    \end{derivation}
\]

    \vspace{1cm}
    Dado que todo $x_i$ es un valor arbitrario, y su coeficiente no depende en lo absoluto
    del valor que tome este, el problema se reduce a igualar los coeficientes de un $x_m$
    donde $0 \leq m \leq k$.
    Para la suma de sumas en $a_k^{(c)}$, se puede ver que
    \begin{enumerate}[label=(\Roman*)]
        \item $x_m$ empieza a acumular coeficientes desde $i=m$
        \item Los demás elementos de la segunda suma no afectan el resultado para $x_m$
    \end{enumerate}
    Por estos resultados, la igualdad de coeficientes queda de la siguiente manera:
    \vspace{1cm}
        \begin{longderivation}<2>
                \res{ (-1)^{k + m}\binom{k+1}{m}
                = \sum_{i=m}^{k}\binom{k+1}{i}\binom{i}{m} (-1)^{i + m}}\\
            \why{$\binom{a}{b}\binom{b}{c} = \binom{a}{c}\binom{a-c}{b - c}$}\\
                \res{ (-1)^{k+m}\binom{k+1}{m}
                = \sum_{i=m}^{k}\binom{k+1}{m}\binom{k+1-m}{i-m}(-1)^{i+m} }\\
            \why*{}\\
                \res{ (-1)^{k+m} = \sum_{i=m}^{k}\binom{k+1-m}{i-m}(-1)^{i+m}}\\
            \why{ Cambio de límites en la suma, $(-1)^{a + 2b} = (-1)^a$ }\\
                \res{ (-1)^{k+m} = \sum_{i=0}^{k-m}\binom{k+1-m}{i}(-1)^i }\\
            \why{ Añadiendo y restando la expresión de la suma en $i=k+1$}\\
                \res{ (-1)^{k+m} = -\binom{k+1-m}{k+1-m}(-1)^{k+1+m}
                + \sum_{i=0}^{k+1-m}\binom{k+1-m}{i}(-1)^{i} }\\
            \why{ Teorema del binomio: $(1 - 1)^{k+1-m}$, $(-1)^a = (-1)^{a - 2}$}\\
                \res{ (-1)^{k+m} = (-1)^{k+m} + (1-1)^{k+1-m} }\\
            \why*{}\\
                \res{ (-1)^{k+m} = (-1)^{k+m} }\\
            \why*{}\\
                \res{ \textit{true} }
        \end{longderivation}

Esto nos demuestra que para cualquier sucesión arbitraria, el término general está dado
por la suma presentada como conjetura.

Veamos, el mismo ejemplo presentado anteriormente, la suma de cuadrados.

\[S = \left<0,1,5,14,30,55,\dots\right>\]

Tomando hasta $a_4$, deberíamos obtener un polinomio de grado $4$, por lo que
si el polinomio resultante es de grado menor, podemos dar pausa al algoritmo.

\[
    \begin{derivation}
            \res{ a_4(n) = \sum_{i=0}^{4}\binom{n}{i}\sum_{j=0}^{i}\binom{i}{j}(-1)^{i+j}x_j }\\
        \why{ Los valores de $x$ corresponden a $S$ }\\
            \res{ a_4(n) = 0 + n(-0 + 1) + \frac{n(n-1)}{2}(0 - 2(1) + 5)\\
            &\displaystyle\qquad+ \frac{n(n-1)(n-2)}{6}(-0 + 3(1) - 3(5) + 14)\\
            &\displaystyle\qquad + \frac{n(n-1)(n-2)}{24}(0 - 4(1) + 6(5) - 4(14) + 30)}\\
        \why*{}\\
            \res{ n + \frac{3n(n-1)}{2} + \frac{2n(n-1)(n-2)}{3} + 0 }\\
        \why*{}\\
            \res{ \frac{n(n+1)(2n+1)}{6} }
    \end{derivation}
\]


\end{document}