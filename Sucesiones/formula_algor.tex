\subsect{Obtención general}

Suponga ahora la siguiente sucesión $S$ y sus índices $I$:
\begin{align*}
    S &= \langle x_0, x_1, x_2, \dots , x_n\rangle\\
    I &= \langle 0, 1, 2, \dots , n \rangle
\end{align*}

El objetivo es obtener el término general de $S$ con índices $I$ .

Procedemos de igual forma que en el ejemplo:

\[a_0(n) = x_0\]

Se tomará la Notación $y_k$ para representar el coeficiente de la iteración $k$ al resolver la ecuación que se mostró en el ejemplo, evitando hacer mención de la suposición de dicho $y$ en cada procedimiento.

\begin{equation*}
    \left\{
        \begin{aligned}
            a_1(n) &= y_1 \cdot n + x_0\\
            x_1 &= y_1(1) + x_0
        \end{aligned}
    \right\}
\end{equation*}

Nuevamente, para $a_2(n)$
\begin{equation*}
    \left\{
        \begin{aligned}
            a_2(n) &= y_2 \cdot n(n-1) + (x_1 - x_0)n + x_0\\
            x_2 &= y_2(2)(1) + (x_1 - x_0)(2) + x_0
        \end{aligned}  
    \right\}
\end{equation*}

Para $a_3(n)$

\begin{equation*}
    \left\{
        \begin{aligned}
            a_3(n) &= y_3 \cdot n(n-1)(n-2) + \dfrac{1}{2}[x_2 - ((x_1  - x_0)(2) + x_0)]n(n-1) + (x_1 - x_0)n + x_0\\
            x_3 &= y_3 (3)(2)(1) + \dfrac{1}{2}[x_2 - ((x_1  - x_0)(2) + x_0)](3)(2) + (x_1 - x_0)(3) + x_0
        \end{aligned}
    \right\}
\end{equation*}

A partir de esta información, se conjetura lo siguiente:

\begin{proofbox}[10]{Algorítmo}
    El término general de una sucesión de $k$ términos, está dada por
    \begin{center}
        \begin{derivation}<1.5>
                \res{ \left\{
                    \begin{aligned}
                        a_{k+1}(n) &= \dfrac{1}{(k+1)!}[x_k - a_{k}(k+1)]n(n-1)(n-2)\dots(n-k) + a_{k}(n)\\
                        a_0 &= x_0
                    \end{aligned}
                \right\} }\\
            \why{Notación}\\
                \res{ \left\{
                    \begin{aligned}
                        a_{k+1}(n) &= \dfrac{1}{(k+1)!}[x_k - a_{k-1}(k)]\displaystyle\prod_{c=0}^{k}(n-c)
                        + a_{k}(n)\\
                        a_0 &= x_0
                    \end{aligned}
                \right\} }
        \end{derivation}
    \end{center}
    Por su puesto, hace falta demostrarlo:
\end{proofbox}
\clearpage


Por medio de la inducción, se toma un caso base, el cual será $k=0$. Esto significa que la expresión obtenida funciona para la sucesión $S = \langle x_0\rangle$
y usando la fórmula conjeturada, se obtiene que
\[a_0(n) = x_0 \]
Que efectivamente, cumple.
Ahora se supone que existe un $b$ tal que la fórmula se cumple para todo número natural menor que o igual a $b$. Queremos ver qué sucede si queremos hallar una sucesión que también pueda seguir el valor de $x_{b+1}$

Tenemos $S = \langle x_0, x_1, \dots x_b, x_{b+1}\rangle$
y que $a_b(n)$ relaciona a los naturales con los $x_i$ de $S$ hasta $x_b$

Queremos hallar una sucesión que pueda relacionar desde $x_0$ hasta $x_{b+1}$, a los naturales. Tomando la información anterior, suponemos que hay un $y_{b+1}$ tal que:

\begin{equation*}
    \left\{
        \begin{aligned}
            a_{b+1}(n) &= y_{b+1} \displaystyle\prod_{c=0}^{b}(n-c) + a_b(n)\\
            x_{b+1} &= y_{b+1} \displaystyle\prod_{c=0}^{b}(b+1 - c) + a_b(b + 1)                
        \end{aligned}
    \right\}
\end{equation*}
Se define de esta manera por la misma razón que se mencionó al principio. Sabemos que $a_b(n)$ sirve para todos los naturales antes que $b+1$ y relaciona cada uno respectivamente con un elemento de $S$. Entonces añadimos un nuevo término el cual se anule en todos los valores de $n$ hasta $b$ de forma que podamos seguir usando $a_b$ y forzando a que en $n=b+1$ el valor obtenido sea $x_{b+1}$ usando $y_{b+1}$ .

Desarrollando la segunda igualdad.
\begin{center}
    \begin{derivation}
            \res{ y_{b+1} \displaystyle\prod_{c=0}^{b}(b+1 - c) = 
            x_{b+1} - a_b(b+1) }\\
        \why{Notación}\\
            \res{ y_{b+1} (b+1)(b)(b-1)\cdots 1 = x_{b+1} - a_b(b+1) }\\
        \why{Notación}\\
            \res{ y_{b+1} (b+1)! = x_{b+1} - a_b(b+1) }\\
        \why{Aritmética}\\
            \res{ y_{b+1} = \dfrac{1}{(b+1)!}[x_{b+1} - a_b(b+1)] }
    \end{derivation}
\end{center}
\vspace*{10pt}
Reemplazando en la primera igualdad, se obtiene lo siguiente:

\[a_{b+1}(n) = \dfrac{1}{(b+1)!}[x_{b+1} - a_b(b+1)]\displaystyle\prod_{c=0}^{b}(n-c) + a_b(n)\]

Con este resultado se demuestra la validez de la fórmula.
