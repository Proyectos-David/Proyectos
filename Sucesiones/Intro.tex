\sect{Introducción}

Una extrapolación polinómica es una forma de estimar datos fuera de un
intervalo dado. La idea es muy similar a la interpolación, sin embargo,
la interpolación tiene como objetivo ver datos que se encuentren dentro
del intervalo. Para fines de este artículo, esa diferencia no es realmente
notoria. El objetivo de ambos es obtener una expresión polinómica la cual
permita trabajar con un conjunto de datos dados.\\
Por lo general, se toman como variables tanto el dominio como el rango del
polinomio en cuestión. Esto es, los algoritmos están diseñados para poder
usar un conjunto arbitrario de parejas ordenadas.
Sin embargo, en muchos casos se necesita que el dominio esté contenido en
los naturales, o se obtenga linealmente de los naturales
(ej.: $\{0.1,\ 0.2,\ 0.3,\ \dots\}$). 

En el presente artículo, se pretende proponer un algoritmo para extrapolar
conjuntos de puntos cuya primera coordenada esté en un conjunto como los
mencionados. Inicialmente, se plantea una
función recursiva para posteriormente, presentar una solución directa (sin recursión).\\
Este algoritmo, con una pequeña variación, fue presentado de manera informal
por el matemático Eduardo Sáenz de Cabezón, a modo de divulgación.
Este algoritmo, no es más que una aplicación de la interpolación de Lagrange.\\

\begin{center}
    Recordando que la interpolación de Lagrange se define de la siguiente manera:\\[20pt]
$
\left\{
\begin{aligned}
    f_n(x) &= \sum_{i=0}^{n-1}f(x_1)L_i(x)\\
    L_i(x) &= \prod_{j = 0 \land j \not= i}^{n} \frac{x - x_j}{x_i - x_j}
\end{aligned}
\right\}
$

    Donde $f_n$ es la función que interpola $n$ puntos
\end{center}

Tras explorar el funcionamiento de este algoritmo, se encontró una relación
con el triángulo de Pascal, bajo el cual se logró eliminar la recursión de
este caso particular de la interpolación de Lagrange. 