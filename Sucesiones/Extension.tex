\sect{Extensión a otros dominios}

El algoritmo presentado, sirve para obtener el término general de una
secuencia finita y extrapolarla. Sin embargo, el alcance de este
no se limita a funciones con dominio en los naturales.

Suponga que se tiene una función finita $f: \Lambda \to \Upsilon$, y $|f| = k$
% con $k \in \mathbb{N}$.
Ahora se generan dos sucesiones $S_\Lambda$ y $S_\Upsilon$ a partir de $f$:

Defínase $S_\Lambda$ recursivamente como:
\[
\left\{
    \begin{aligned}
        S_\Lambda(0) &= \texttt{min}(\Lambda)\\
        S_\Lambda(n + 1) &= 
        \texttt{min}\left(\Lambda - \bigcup_{i=0}^{n}\left\{S_\Lambda(i)\right\} \right)
    \end{aligned}
\right\}
\]

Y a partir de esta se define $S_\Upsilon$ de la siguiente forma:

\[S_\Upsilon(n) = f(S_\Lambda(n))\]

Al ser finitas, ambas sucesiones pueden trabajarse bajo el algoritmo presentado.
Dado que $f$ es función, $S_\Lambda$ es una función inyectiva, y por ende, invertible.\\
Por otro lado, se puede ver que ambas sucesiones tienen el mismo número de parejas, esto es:
$\left|S_\Lambda \right| = \left|S_\Upsilon \right|$.
A partir de esto, se puede hacer que la extrapolación se pueda trabajar con cualquier conjunto
de parejas (siempre y cuando este conjunto sea una función): tomando el término general de $S_\Upsilon$ y el de $S_\Lambda$ como $S_\Upsilon(n)$ y $S_\Lambda(n)$ respectivamente, la extrapolación será:

\[S_\Upsilon\left(S_\Lambda^{-1}(x)\right)\]