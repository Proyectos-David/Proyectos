\documentclass{beamer}
\usepackage{logicDG}
\usepackage{blkarray}
\usepackage[spanish, es-noquoting]{babel}
\usetheme{PaloAlto}
\setbeamercovered{transparent}

%  -- Beamer colors --

\definecolor{coso}{HTML}{0038a5}
\setbeamercolor{logo}{bg=white}
\setbeamercolor{frametitle}{bg=coso} 
\setbeamercolor{sidebar}{bg=coso} 


\logo{\includegraphics[height=1.585cm]{"C:/Users/usuario/OneDrive/Documentos/U/logo-eci-azulito.png"}}


% -- Titlepage --

\title{Extrapolación de secuencias finitas}

\author[{David G. Laura R. Daniel P.}]{{David Gómez, Laura Rincón, Daniel Pérez}}

\institute[ECI]{
    Escuela Colombiana de Ingeniería\\
    Matemáticas
    }
\date[30/05/2023]{30 de Mayo del 2023}

% ---

\usepackage{scalerel}

\NewDocumentCommand{\flechita}{}{\scalerel*{
    \includegraphics{C:/Users/usuario/OneDrive/Documentos/U/flechita}
    }{X}}

\NewDocumentCommand{\vdflechita}{}{\rotatebox[origin=c]{270}{\flechita}}

\begin{document}
\begin{frame}
    \titlepage
\end{frame}

\section[Algor. recursivo]{Obtención general}

\begin{frame}{Ejemplo}
    \[S = \left<7, 0, 15\right>\]
    $a(n)$ es una función tal que:
    \[a(0) = 7\ ; a(1) = 0\ ; a(2) = 15\]
    \begin{columns}
        \column{0.5\textwidth}
        \begin{itemize}
            \item<1-> $a_0(n) = 7$
            
            \only<2>{\item $a_1(n) = y_1 \cdot n + a_0(n)$}
            \only<3->{\item $a_1(n) = -7n + 7$}

            \only<3>{\item $a_2(n) = y_2\cdot n(n-1) + a_1(n)$}
            \only<4>{\item $a_2(n) = 11n(n-1) - 7n + 7$}
        \end{itemize}
        \column{0.5\textwidth}
        \only<2>{
            $
            \begin{derivation}
                \res{ 0 = y_1(1) + 7 }\\
            \why*{}\\
                \res{ y_1 = -7 }
            \end{derivation}
            $
        }
        \only<3>{
            $
            \begin{derivation}
                \res{ 15 = y_2\cdot 2(2-1) + a_1(2)}\\
            \why*{}\\
                \res{ y_2 = 11 }
            \end{derivation}
            $
        }
    \end{columns}
\end{frame}

\begin{frame}{Si $S$ es arbitraria}
    \[S = \left<x_0, x_1, \dots , x_m\right>\]
    \begin{align*}
        \onslide<1->{a_0(n) &= x_0}\\
        \onslide<2->{a_1(n) &= (x_1 - x_0)n + x_0}\\
        \onslide<3>{a_2(n) &= \frac{1}{2}(x_2 - 2x_1 + x_0)n(n-1) + (x_1 - x_0)n + x_0}
    \end{align*}
\end{frame}

\begin{frame}{Fórma recursiva}
    \centering
    Si $S = \left<x_0, \dots, x_m\right>$, $0 \leq k < m$
    \begin{align*}
        a_0(n)      &= x_0\\
        a_{k+1}(n)  &= 
        \frac{1}{(k+1)!}[x_{k+1} - a_k(k+1)]\prod_{c=0}^{k}(n - c)
        + a_k(n)\\[10pt]
        &= \binom{n}{k+1}[x_{k+1} - a_k(k+1)] + a_k(n)
    \end{align*}
\end{frame}

\section[Algor. no recursivo]{Eliminación de recursión}

\begin{frame}{ ¿Qué hay que solucionar? }
    Forma general del algoritmo
    \begin{align*}
        \onslide<1->{a_k(n) &= \sum_{i=0}^{k}y_i \prod_{c=0}^{i-1}(n-c)}\\
        \onslide<2->{y_{b+1} &= \frac{1}{(b+1)!}[x_{b+1} - a_b(b+1)]}
    \end{align*}
    
    \onslide<3>{
        \begin{proofbox}{Valor a generalizar}
        \[x_{b+1} - a_b(b+1)\]
        \end{proofbox}
        }
\end{frame}

\begin{frame}{ Ejemplos hasta $k=6$ }
    \only<1>{
        \[
            \begin{matrix*}[l]
                k = 0 & \flechita& x_0\\
                k = 1 & \flechita& -x_0 + x_1\\
                k = 2 & \flechita& x_0 - 2x_1 + x_2\\
                k = 3 & \flechita& -x_0 + 3x_1 - 3x_2 + x_3\\
                k = 4 & \flechita& x_0 - 4x_1 + 6x_2 - 4x_3 + x_4\\
                k = 5 & \flechita& -x_0 + 5x_1 - 10x_2 + 10x_3 - 5x_4 + x_5\\
                k = 6 & \flechita& x_0 -6x_1 + 15x_2 - 20x_3 + 15x_4 - 6x_5 + x_6
            \end{matrix*}
        \]
        }
    \only<2>{
        \[
            \begin{blockarray}{cccccccc}
                & x_0     & x_1   & x_2   & x_3   & x_4   & x_5   & x_6\\
                \begin{block}{c(ccccccc)}
            0   &   1     & 0     & 0     & 0     & 0     & 0     & 0\\
            1   &   -1    & 1     & 0     & 0     & 0     & 0     & 0\\
            2   &   1     & -2    & 1     & 0     & 0     & 0     & 0\\
            3   &   -1    & 3     & -3    & 1     & 0     & 0     & 0\\
            4   &   1     & -4    & 6     & -4    & 1     & 0     & 0\\
            5   &   -1    & 5     & -10   & 10    & -5    & 1     & 0\\
            6   &   1     & -6    & 15    & -20   & 15    & -6    & 1\\
        \end{block}
    \end{blockarray}
    \]
    }
\end{frame}

\begin{frame}{Conjetura}
    \only<2->{Expresiones con binomios}
    \only<1>{
        \begin{gather*}
            \phi_j(i) = \dfrac{(-1)^{i + j}}{j!}\prod_{c=0}^{j-1}(i - c)\\
            \vdflechita\\
            x_i \text{ en la iteración } k = j
        \end{gather*}
    }
    \only<2>{
        \[
            \left\{
                \begin{aligned}
                    a_0(n)      &= x_0\\
                    a_{k+1}(n)  &= \binom{n}{k+1}\left(x_{k+1} - a_k(k+1)\right) + a_k(k+1)
                \end{aligned}
            \right\}
        \]
    }
    \only<3>{
        \begin{align*}
            a_k(n)  &= \sum_{i=0}^{k}\binom{n}{i}
            \left(x_0\binom{i}{0}(-1)^i + \cdots x_k \binom{i}{k}(-1)^{i+k}\right)\\
                    &= \sum_{i=0}^{k}\binom{n}{i}\sum_{j=0}^{i}\binom{i}{j}(-1)^{i+j}x_j            
        \end{align*}
    }
\end{frame}

\section[Extensión]{Extensión a otros dominios}

\begin{frame}{Extrapolación deseada}
    Siendo $f$ un conjunto de puntos finito para una extrapolación
    \only<1>{\[f:\Lambda \to \Upsilon\]
    \[|f| = k, k \in \mathbb{N}\]}
    \only<2>{
        Se definen sucesiones para las primeras y segundas coordenadas
        \[S_\Lambda\ ,\ S_\Upsilon\]
    }
\end{frame}

\begin{frame}{Ejemplo rápido}
    \[    
        \begin{aligned}
            f &= \left\{(0.1; 34); (2; -10); (12; 34)\right\}\\
            &\vdflechita\\
            S_\Lambda &= \left<0.1; 2; 12\right>\\
            S_\Upsilon &= \left<34; -10; 34\right>
        \end{aligned}
    \]
\end{frame}

\begin{frame}{Función general de las sucesiones}
    \onslide<1->{
        Definición de $S_\Lambda(n)$ y $S_\Upsilon(n)$
        \[
\left\{
    \begin{aligned}
        S_\Lambda(0) &= \texttt{min}(\Lambda)\\
        S_\Lambda(n + 1) &= 
        \texttt{min}\left(\Lambda - \bigcup_{i=0}^{n}\left\{S_\Lambda(i)\right\} \right)
    \end{aligned}
\right\}
\]
\vspace{20pt}
}
\onslide<2>{
    \[S_\Upsilon(n) = f(S_\Lambda(n))\]
}
\end{frame}

\begin{frame}{Extrapolación para cualquier función finita}
    Composición para obtener la extrapolación deseada
        \begin{gather*}
            x \in \texttt{dom}(f)\\
            f^{*}(x) = S_\Upsilon\left(S_\Lambda^{-1}(x)\right) 
        \end{gather*}
\end{frame}
\end{document}