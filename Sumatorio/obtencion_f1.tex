\subsect{Primer resultado}

\begin{proofbox}{Obtención}
    \begin{itemize}
        \item[(i)] Sabemos que al tomar naturales para generar los rectángulos, su base es de magnitud $1$, y si altura está definida por el valor de la función en el mayor de los naturales de la base. De esta forma, tenemos que:
        \begin{center}
            \begin{derivation}<1.6>
                    \res{ R(n) = A - \displaystyle\int_{n-1}^{n}x^P dx }\\
                \why{ Desarrollo del área $A$ y la integral }\\
                    \res{ R(n) = n^P - \dfrac{1}{P+1}\left[n^{P+1} - (n-1)^{P+1}\right] }\\
                \why*{}\\
                    \res{ R(n) = \dfrac{1}{P+1}\left[(P+1)n^P - n^{P+1} + (n-1)^{P+1}\right] }
            \end{derivation}
        \end{center}
        \item[(ii)] Se sabe que hasta $x=n$ hay $n$ rectángulos, por lo que el sumatorio se puede expresar mediante la suma de todos los restantes y la integral:
        \begin{center}
            \begin{derivation}<1.6>
                    \res{ \displaystyle\sum_{k=1}^{n} k^P = \displaystyle\int_{0}^{n} x^P dx + \displaystyle\sum_{k=1}^{n} R(k) }\\
                \why*{}\\
                    \res{ \displaystyle\sum_{k=1}^{n}k^P = n^{P+1} + \dfrac{1}{P+1}\displaystyle\sum_{k=1}^{n} (P+1)k^P - k^{P+1} + (k-1)^{P+1} }
            \end{derivation}
        \end{center}
    \end{itemize}
\end{proofbox}