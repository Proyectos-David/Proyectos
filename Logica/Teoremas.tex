\sect{Teoremas}

El desarrollo para la distribución de la negación en la conjunción se ve bastante saturado,\\
en comparación al uso de las herramientas de la lógica `normal'. Sin embargo, con el uso de
algunos teoremas, se puede facilitar bastante la forma de demostrar usando estas funciones lógicas.

\textbf{ Teorema: Igualdad}

\begin{adjustbox}{minipage=0.6\textwidth,margin=5pt, center}
    \textit{
        Dadas $h_0$, $h_1$ funciones lógicas (compuestas o simples).\\
        $E(h_0, h_1) = 1$ si y solo si $h_0 = h_1$
    }
\end{adjustbox}
\begin{proofbox}*[10]{\emph{Demostración}}
    \begin{enumerate}[label=(\roman*)]
        \item Suponiendo que $h_0 = h_1$
        
        \[
            \begin{derivation}<.9>
                    \res{ E(h_0, h_1) }\\
                \why*[=]{}\\
                    \res{ h_0h_1 + (1 - h_0)(1 - h_1) }\\
                \why[=]{ $h_0 = h_1$, propiedad fundamental }\\
                    \res{ h_0 + 1 - h_0 }\\
                \why*[=]{}\\
                    \res{ 1 }
            \end{derivation}
        \]

        \item Suponiendo que $E(h_0, h_1) = 1$
        
        \[
            \begin{derivation}[5pt]
                    \res{ E(h_0, h_1) = 1 }\\
                \why*{}\\
                    \res{ h_0h_1 + (1 - h_0)(1 - h_1) = 1 }\\
                \why*{}\\
                    \res{ h_0h_1 + 1 - h_1 - h_0 + h_0h_1 = 1 }\\
                \why*{}\\
                    \res{ 2h_0h_1 = h_0 + h_1 }\\
                \why{ Separando por casos (los posibles valores en $\mathbb{V}$) }\\[.5cm]
                    \res{
                        \left\{
                            \begin{aligned}
                                h_0 = 0 &\To 0 = h_1\\
                                h_0 = 1 &\To h_1 = 1
                            \end{aligned}
                        \right\}
                    }
            \end{derivation}
        \]
    \end{enumerate}
\end{proofbox}

\textbf{ Teorema: Implicación}

\begin{adjustbox}{minipage=0.6\textwidth,margin=5pt, center}
\textit{
    Dadas $\ h_0, h_1\ $ funciones lógicas (compuestas o simples).\\
    $I(h_0,h_1) = 1$ si y solo si $h_0\,h_1 = h_0$
}
\end{adjustbox}

\begin{proofbox}*[10]{\emph{Demostración}}
    \begin{enumerate}[label=(\roman*)]
        \item Suponiendo $h_0h_1 = h_0$
        
        \[
            \begin{derivation}<.9>
                    \res{ I(h_0, h_1) }\\
                \why*[=]{}\\
                    \res{ (h_1 - 1)h_0 + 1 }\\
                \why*[=]{}\\
                    \res{ h_0h_1 - h_0 + 1 }\\
                \why[=]{ Suposición }\\
                    \res{ h_0 - h_0 + 1 }\\
                \why*[=]{}\\
                    \res{ 1 }
            \end{derivation}
        \]

        \item Suponiendo $I(h_0, h_1) = 1$
        
        \[
            \begin{derivation}<.9>
                    \res{ I(h_0, h_1) = 1 }\\
                \why*{}\\
                    \res{ (h_1 - 1)h_0 + 1 = 1}\\
                \why*{}\\
                    \res{ h_0h_1 - h_0 + 1 = 1 }\\
                \why*{}\\
                    \res{ h_0h_1 = h_0 }
            \end{derivation}
        \]

    \end{enumerate}
\end{proofbox}