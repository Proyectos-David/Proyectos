\sect{Teoremas}

El desarrollo para la distribución de la negación en la conjunción se ve bastante saturado, en comparación al uso de las herramientas de la lógica `normal'. Sin embargo, con el uso de algunos teoremas, se puede facilitar bastante la forma de demostrar usando estas funciones lógicas.

\begin{proofbox}[10]{\textbf{Teorema: Igualdad}}\label{Teo_igualdad}
    \emph{
        Dadas proposiciones $\phi_1$ `y' $\phi_2$, las cuales se pueden representar como funciones lógicas (compuestas o simples) $h_0$ `y' $h_1$ , respectivamente y $\phi_1 \equiv \phi_2$. $E(h_0, h_1) = 1$ si y solo si $h_0 = h_1$ 
    }    
\end{proofbox}

\begin{subproofbox}[10]{\emph{Demostración}}
    \begin{itemize}
        \item[(i)] Suponiendo que $h_0 = h_1$
        
        \begin{center}
            \begin{derivation}<.9>
                    \res{ E(h_0, h_1) }\\
                \why*[=]{}\\
                    \res{ h_0h_1 + (1 - h_0)(1 - h_1) }\\
                \why[=]{ $h_0 = h_1$, propiedad fundamental }\\
                    \res{ h_0 + 1 - h_0 }\\
                \why*[=]{}\\
                    \res{ 1 }
            \end{derivation}
        \end{center}

        \item[(ii)] Suponiendo que $E(h_0, h_1) = 1$
        
        \begin{center}
            \begin{derivation}[5pt]
                    \res{ E(h_0, h_1) = 1 }\\
                \why*{}\\
                    \res{ h_0h_1 + (1 - h_0)(1 - h_1) = 1 }\\
                \why*{}\\
                    \res{ h_0h_1 + 1 - h_1 - h_0 + h_0h_1 = 1 }\\
                \why*{}\\
                    \res{ 2h_0h_1 = h_0 + h_1 }\\
                \why{ Separando por casos (los posibles valores en $\mathbb{V}$) }\\[.5cm]
                    \res{
                        \left\{
                            \begin{aligned}
                                h_0 = 0 &\To 0 = h_1\\
                                h_0 = 1 &\To h_1 = 1
                            \end{aligned}
                        \right\}
                    }
            \end{derivation}
        \end{center}
    \end{itemize}
\end{subproofbox}

\begin{proofbox}[10]{\textbf{Teorema: implicación}}
    \emph{
        Dadas dos proposiciones $\phi_1$ `y' $\phi_2$, las cuales se pueden representar como funciones lógicas (compuestas o simples) $h_0$ `y' $h_1$ , respectivamente y $\phi_1 \To \phi_2$. $h_0h_1 = h_0$ si y solo si $I(h_0, h_1) = 1$
    }
\end{proofbox}
\begin{subproofbox}[10]{\emph{Demostración}}
    \begin{itemize}
        \item[(i)] Suponiendo $h_0h_1 = h_0$
        
        \begin{center}
            \begin{derivation}<.9>
                    \res{ I(h_0, h_1) }\\
                \why*[=]{}\\
                    \res{ (h_1 - 1)h_0 + 1 }\\
                \why*[=]{}\\
                    \res{ h_0h_1 - h_0 + 1 }\\
                \why[=]{ Suposición }\\
                    \res{ h_0 - h_0 + 1 }\\
                \why*[=]{}\\
                    \res{ 1 }
            \end{derivation}
        \end{center}

        \item[(ii)] Suponiendo $I(h_0, h_1) = 1$
        
        \begin{center}
            \begin{derivation}<.9>
                    \res{ I(h_0, h_1) = 1 }\\
                \why*{}\\
                    \res{ (h_1 - 1)h_0 + 1 = 1}\\
                \why*{}\\
                    \res{ h_0h_1 - h_0 + 1 = 1 }\\
                \why*{}\\
                    \res{ h_0h_1 = h_0 }
            \end{derivation}
        \end{center}

    \end{itemize}
\end{subproofbox}