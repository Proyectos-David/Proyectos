\subsubsection{Hipergeométrica}
\label{dist:hip}
Supóngase que se tienen dos tipos de objetos, $a$ y $b$ en un total de
$N$ objetos exclusivamente de estos dos tipos. Sea $K$ el número de
objetos de tipo $a$ en el total de los $N$ objetos, es decir
hay $N-K$ objetos de tipo $b$. Supóngase que se toman ahora $n$
objetos del total ($N$). Se define una variable aleatoria $X$
correspondiente al número de objetos de tipo $a$ en los
$n$ objetos tomados. Entonces
\[P(X=x) = \dfrac{\binom{K}{x}\binom{N-K}{n-x}}{\binom{N}{n}}\]
\begin{Def}
  Sea $X$ una variable aleatoria discreta. $X$ sigue una distribución
  hipergeométrica con parámetros $N,K,n$ ($N,K,n\in\Z^+,K\leq N, n\leq N$),
  denotada por $Hg(N,K,n)$, cuando su función de masa es
  \[
    f(x)=P(X=x)=\dfrac{\binom{K}{x}\binom{N-K}{n-x}}{\binom{N}{n}}\qquad
    (\max\{0,n+K-N\} \leq x \leq \min\{K,n\})
  \]
\end{Def}

La razón de esta condición para $x$ está en que tenga sentido para lo
que se está representando. Por un lado, no tiene sentido pensar en la
probabilidad de tomar más objetos de tipo $a$ de los que hay en el total
de la muestra o tomar más objetos de tipo $a$ del total de estos.
De la misma forma, no tiene sentido tomar una cantidad negativa
de objetos tipo $a$, o tomar una cantidad de objetos tipo $a$
de forma que haya una cantidad negativa de objetos tipo $b$ para completar
los $n$ objetos o más objetos de tipo $b$ de los que hay en total.
De forma más concreta, se pueden ver las condiciones de $x$ dada la
expresión de la función de masa presentada.
\begin{longderivation}<0.8>
    & 0\leq x\leq K \quad\land\quad 0\leq n-x\leq N-K\\
  \iff\\
    & 0\leq x\leq K \quad\land\quad n+K-N\leq x\leq n\\
  \iff\\
    & \max\{0,n+K-N\}\leq x\leq\min\{K,n\}
\end{longderivation}
Sin embargo, tomando la convención de que $\binom{n}{k} = 0$ cuando $k>n$,
se puede tomar a $x$ entre $0$ y $n$.

Para demostrar la validez de esta función de masa, hace falta un resultado
sobre la combinatoria.
\begin{Lema}[Identidad de Vandermonde]
  Sean $m,n,k\in\Z$ no negativos. Entonces
  \[\binom{m+n}{k} = \sum_{r=0}^k\binom{m}{r}\binom{n}{k-r}\]
  Esta identidad tiene sentido tomando la convención mencionada anteriormente.
\end{Lema}
\begin{Demo}
  La demostración se hará por inducción sobre $m$, tomando $k,n$ como
  enteros no negativos arbitrarios. Caso base ($m=0$):
  \[\binom{0+n}{k} = \sum_{r=0}^k\binom{0}{r}\binom{n}{k-r}=\binom{n}{k}\]
  Paso inductivo: sea $k\in\Z$ con $k\geq1$ y supóngase que la propiedad se mantiene
  para todo entero no negativo hasta $k$. (Para $k=0$ la propiedad es trivial)
  \begin{longderivation}
      & \binom{m+n}{k+1}\\
    =\\
      & \binom{m+n}{k} + \binom{m+n}{k-1}\\
    =\\
      & \sum_{r=0}^k\binom{m}{r}\binom{n}{k-r} +
      \sum_{r=0}^{k-1}\binom{m}{r}\binom{n}{k-1-r}\\
    =\\
      & \binom{m}{k} + 
      \sum_{r=0}^{k-1}\binom{m}{r}\left(\binom{n}{k-r} + \binom{n}{k-1-r}\right)\\
    =\\
      & \binom{m}{n} + \sum_{k=0}^{k-1}\binom{m}{r}\binom{n}{k-r}\\
    =\\
      & \sum_{k=0}^k \binom{m}{k}\binom{n}{k-r}
  \end{longderivation}
\end{Demo}

\begin{Teo}
  Sea $X\sim Hg(N,K,n)$. Entonces,
  \begin{enumerate}
    \item para todo $x\in\Z$ con $0\leq x\leq n$, $P(X=x) \geq 0$.
    \item $\sum_{x\in\Z}P(X=x) = 1$.
    \item $\text{E}[X]=\frac{nK}{N}$.
    \item $\text{Var}[X] = \frac{n\,K(N-K)(N-n)}{N^2(N-1)}$.
  \end{enumerate}
\end{Teo}
\begin{Demo}~
  \begin{enumerate}
    \item Sea $x\in\Z$ con $0\leq x\leq n$. Recordando que
    \[P(X=x) = \dfrac{\binom{K}{x}\binom{N-K}{n-x}}{\binom{N}{n}}\]
    dado que todos los términos de la expresión son no negativos, se
    concluye que $P(X=x)\geq0$
    \item ~
    \begin{longderivation}
        & \sum_{x=0}^nP(X=x)\\
      =\\
        & \sum_{x=0}^n\dfrac{\binom{K}{x}\binom{N-K}{n-x}}{\binom{N}{n}}\\
      =\\
        & \dfrac{1}{\binom{N}{n}}\sum_{x=0}^n\binom{K}{x}\binom{N-K}{n-x}\\
      =\\
        & \dfrac{\binom{N}{n}}{\binom{N}{n}}\\
      =\\
        & 1
    \end{longderivation}
    \item~
    \begin{longderivation}
        & \text{E}[X]\\
      =\\
        & \sum_{x=0}^nx\dfrac{\binom{K}{x}\binom{N-K}{n-x}}{\binom{N}{n}}\\
      =\\
        & \dfrac{1}{\binom{N}{n}}\sum_{x=0}^nx\binom{K}{x}\binom{N-K}{n-x}\\
      =\\
        & \dfrac{K}{\binom{N}{n}}\sum_{x=1}^n\binom{K-1}{x-1}\binom{N-K}{n-x}\\
      =\\
        & \dfrac{K}{\binom{N}{n}}\sum_{x=0}^{n-1}\binom{K-1}{x}\binom{N-K}{n-x-1}\\
      =\\
        & \dfrac{K}{\binom{N}{n}}\binom{N-1}{n-1}\\
      =\\
        & \frac{K\,n}{N}
    \end{longderivation}
    \item~
    \begin{longderivation}
        & \text{Var}[X]\\
      =\\
        & \text{E}[X^2] - \text{E}[X] + \text{E}[X] - \text{E}^2[X]\\
      =\\
        & \dfrac{1}{\binom{N}{n}}\sum_{x=0}^nx^2\binom{K}{x}\binom{N-K}{n-x}
        - \dfrac{1}{\binom{N}{n}}\sum_{x=0}^nx\binom{K}{x}\binom{N-K}{n-x}
        + \frac{n\,K}{N} - \frac{n^2\,K^2}{N^2}\\
      =\\
        & \dfrac{K}{\binom{N}{n}}\left[
          \sum_{x=0}^{n-1}x\binom{K-1}{x-1}\binom{N-K}{n-x-1} -
          \sum_{x=0}^{n-1}\binom{K-1}{x-1}\binom{N-K}{n-x-1}
        \right]
        + \frac{n\,K}{N} - \frac{n^2\,K^2}{N^2}\\
      =\\
        & \dfrac{K(K-1)}{\binom{N}{n}}
          \sum_{x=0}^{n-2}\binom{K-2}{x}\binom{N-K}{n-x-2}
        + \frac{n\,K}{N} - \frac{n^2\,K^2}{N^2}\\
      =\\
        & \dfrac{K(K-1)}{\binom{N}{n}}\binom{N-2}{n-2}
        + \frac{n\,K}{N} - \frac{n^2\,K^2}{N^2}\\
      =\\
        & \frac{K(K-1)n(n-1)}{N(N-1)}
        + \frac{n\,K}{N} - \frac{n^2\,K^2}{N^2}\\
      =\\
        & \frac{n\,K(N-K)(N-n)}{N^2(N-1)}
    \end{longderivation}
  \end{enumerate}
\end{Demo}