\subsubsection{Binomial}
\label{dist:binom}
Supóngase que se realiza un experimento el cual tiene como posible
resultado $a$ o $b$ exclusivamente, y además, el resultado
de realizar nuevamente el experimento es independiente al
resultado anterior. Dado que $a$ y $b$ son los únicos resultados,
para un único experimento, se debe tener que
$P(a) = 1 - P(b)$. Sea $p=P(a)$. Supóngase que
este experimento es realizado $n$ veces. Se define una variable
aleatoria $X$ correspondiente a la cantidad de ocurrencias de $a$.
Entonces
\[P(X=x) = \binom{n}{x}p^x(1-p)^{n-x}\]
\begin{Def}
  Sea $X$ una variable aleatoria discreta. $X$ sigue una distribución
  binomial con parámetros $n$ y $p$ ($n\in\Z^+, p\in(0,1)$), denotada por $B(n,p)$,
  cuando su función de masa es
  \[f(x) = P(X=x) = B(n,p)(x) = \binom{n}{x}p^x(1-p)^{n-x}
  \qquad(0\leq x \leq n)\]
\end{Def}

\begin{Teo}
  Sea $X\sim B(n,p)$. Entonces
  \begin{enumerate}
    \item Para todo $x\in\Z$ con $0\leq x\leq n$, $0 \leq P(X=x) \leq 1$.
    \item $\sum_{x\in\Z}P(X=x) = 1$.
    \item $\text{E}[X] = np$.
    \item $\text{Var}[X]=np(1-p)$.
  \end{enumerate}
\end{Teo}
\begin{Demo}~
  \begin{enumerate}
    \item Sea $x\in\Z$ con $0\leq x\leq n$. Recordando que
    \[P(X=x) = \binom{n}{x}p^x(1-p)^{n-x}\]
    Dado todos los términos de la expresión son no negativos, se concluye que
    $P(X=x)\geq0$.
    Para la otra parte de la desigualdad,
    \begin{longderivation}
        & \binom{n}{x}p^x(1-p)^{n-x} \leq 1\\
      \iff\\
        & \dfrac{n!}{(n-x)!\,x!}p^x(1-p)^{n-x} \leq 1\\
      \iff\\
        & n!\,p^x(1-p)^n \leq (n-x)!\,x!\,(1-p)^x
    \end{longderivation}

    Procediendo por inducción sobre $n$

    Caso base, $n=1$: 
    \[p^x(1-p)^n \leq (1-x)!x!(1-p)^x\]
    En este caso $x\in\{0,1\}$ y las desigualdades
    respectivas son $1\leq1$ y $p\leq1$.

    Paso inductivo:\\
    Supóngase que la propiedad se mantiene para algún $n\in\Z^+$.
    Para $0\leq x\leq n$, se tiene entonces que
    \begin{longderivation}
        & n!\,p^x(1-p)^n \leq (n-x)!\,x!\,(1-p)^x\\
      \why{
        Usando el hecho de que $n+1\geq n+1-x$. Multiplicando miembro a
        miembro
      }\\
        & (n+1)!\,p^x(1-p)^n \leq (n+1-x)!\,x!\,(1-p)^x\\
      \why{Como $p\in(0,1)$, entonces, $(1-p)^{n+1}\leq (1-p)^n$. Por
      transitividad}\\
        & (n+1)!\,p^x(1-p)^{n+1} \leq (n+1-x)!\,x!\,(1-p)^x
    \end{longderivation}
    Para $x=n+1$, se obtiene que
    \[(n+1)!\leq0\]
    Así, para todo $x\in\Z$ con $0\leq x\leq n$, $0\leq P(X=x)\leq1$.
    \item~
    \begin{longderivation}
        & \sum_{x=0}^nP(X=x)\\
      =\\
        & \sum_{x=0}^n\binom{n}{x}p^x(1-p)^{n-x}\\
      =\\
        & (p + 1 - p)^n\\
      =\\
        1
    \end{longderivation}
    \item~
    \begin{longderivation}<0.8>
        & {\text{E}[X]}\\
      =\\
        & {\sum_{x=0}^nx\binom{n}{x}p^x(1-p)^{n-x}}\\
      =\\
        & {np\sum_{x=1}^n\frac{(n-1)!}{(n-k)!\,(x-1)!}p^x(1-p)^{n-x}}\\
      =\\
        & {np\sum_{x=0}^{n-1}\binom{n-1}{x}p^x(1-p)^{n-1-x}}\\
      =\\
        & {np\,(p+1-p)^{n-1}}\\
      =\\
        & {np}
    \end{longderivation}
    \item~
    \begin{longderivation}<0.8>
        & {\text{Var}[X]}\\
      =\\
        & {\text{E}[X^2] - \text{E}^2[X]}\\
      =\\
        & {\sum_{x=0}^nx^2\binom{n}{x}p^x(1-p)^{n-x} - (np)^2}\\
      =\\
        & {np\left[
          \sum_{x=1}^nx\binom{n-1}{x-1}p^{x-1}(1-p)^{n-x} - np
        \right]}\\
      =\\
        & {np\left[
          \sum_{x=0}^{n-1}(x+1)\binom{n-1}{x}p^x(1-p)^{n-1-x} - np
        \right]}\\
      =\\
        & {np\left[
          \sum_{x=0}^{n-1}x\binom{n-1}{x}p^x(1-p)^{n-1-x}
          +\sum_{x=0}^{n-1}\binom{n-1}{x}p^x(1-p)^{n-1-x}
          -np
        \right]}\\
      =\\
        & {np\left[
          p(n-1)\sum_{x=1}^{n-1}\binom{n-2}{x-1}p^{x-1}(1-p)^{n-1-x}+1-np
        \right]}\\
      =\\
        & {np\left[
          p(n-1)\sum_{x=0}^{n-2}\binom{n-2}{x}p^x(1-p)^{n-2-x}+1-np
        \right]}\\
      =\\
        & {np[p(n-1)+1-np]}\\
      =\\
        & {np(1-p)}
    \end{longderivation}
    Este argumento es válido siempre que $n\geq 2$. Si $n<2$, entonces $n=1$ y
    \begin{longderivation}<0.8>
        & \text{Var}[X]\\
      =\\
        & \text{E}[X^2] - \text{E}^2[X]\\
      =\\
        & \sum_{x=0}^1x^2\binom{1}{x}p^x(1-p)^{1-x} - p^2\\
      =\\
        & p - p^2\\
      =\\
        & p\,(1-p)
    \end{longderivation}
    Así, el resultado se mantiene para todo $n\in\Z^+$
  \end{enumerate}
\end{Demo}