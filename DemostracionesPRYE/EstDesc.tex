\section{Estadística Descriptiva}

Uno de los problemas no resueltos de esta primera parte del
curso fue hallar la probabilidad de una unión finita de eventos.

Esta probabilidad, sin embargo, se puede hallar mediante una
forma recursiva con la siguiente expresión
\begin{align*}
  P(A_1 \cup A_2) &= P(A_1) + P(A_2) - P(A_1 \cap A_2)\\[10pt]
  P\left(\bigcup_{i=1}^{n+1} A_i\right)
  &= P(A_{n+1}) + P\left(\bigcup_{i=1}^n A_i\right) -
    P\left(\bigcup_{i=1}^n (A_{n+1}\cap A_i)\right)
\end{align*}
Lo que se quiere es resolver esta función de recursión. Para esto,
se evaluaran unos cuantos de sus resultados.
\[
\begin{derivation}
    \wff{P(A_1 \cup A_2\cup A_3)}\\
  =\\
    \wff{P(A_3) + P(A_1 \cup A_2) - P((A_3\cap A_1) \cup (A_3\cap A_2))}\\
  =\\
    &P(A_3) + P(A_2) + P(A_1) - P(A_1\cap A_2)\\
    &\quad-[P(A_1\cap A_3) + P(A_2 \cap A_3) - P(A_1\cap A_2\cap A_3)]\\
  =\\
    &P(A_1) + P(A_2) + P(A_3) - P(A_1\cap A_2) - P(A_1\cap A_3)\\
    &\quad- P(A_2\cap A_3) + P(A_1\cap A_2\cap A_3)\\
  =\\
    \wff{
      \sum_{i=1}^3P(A_i) - \sum_{i=1}^2\sum_{j=i+1}^3P(A_i\cap A_j)
      + P\left(\bigcap_{i=1}^3 A_i\right)
    }
\end{derivation}
\]
Análogamente para cuatro eventos, usando lo obtenido
\[
\begin{derivation}
    \wff{P(A_1\cup A_2\cup A_3\cup A_4)}\\
  =\\
    \wff{P(A_4) + P(A_1\cup A_2\cup A_3) - P\left(\bigcup_{i=1}^3(A_4\cap A_i)\right)}\\
  =\\
    & P(A_4) + \sum_{i=1}^3P(A_i) - \sum_{i=1}^2\sum_{j=i+1}^3P(A_i\cap A_j)
    + P\left(\bigcap_{i=1}^3 A_i\right)\\
    &\quad-\left[\sum_{i=1}^3P(A_4\cap A_i) - \sum_{i=1}^2\sum_{j=i+1}^3P(A_4\cap A_i\cap A_j)
    + P\left(\bigcap_{i=1}^4 A_i\right)\right]\\
  =\\
    \wff{
      \sum_{i=1}^4P(A_i) - \sum_{i=1}^3\sum_{i=1}^4P(A_i\cap A_j) +
      \sum_{i=1}^2\sum_{j=i+1}^3\sum_{k=j+1}^4P(A_1\cap A_j\cap A_k) -
      P\left(\bigcap_{i=1}^4 A_i\right)
    }
\end{derivation}
\]
Por último, recordar que
\[\sum_{i=a}^b\sum_{j=i+1}^{b+1} f(i,j) = \sum_{\mathclap{a\leq i < j \leq b}}f(i,j)\]
\begin{Teo}
  Sean $A_1,A_2,\dots,A_n$ eventos de un espacio muestral. Entonces,
  \[
    P\left(\bigcup_{i=1}^nA_i\right) = \sum_{\mathclap{1\leq i\leq n}}P(A_i)
    -\sum_{\mathclap{1\leq i < j\leq n}}P(A_i\cap A_j) +
    \sum_{\mathclap{1\leq i<j<k\leq n}}P(A_i\cap A_j\cap A_k)
    -\dots+(-1)^nP\left(\bigcap_{i=1}^n A_i\right)
  \]
  De otra forma, la probabilidad de una unión finita es la suma de
  la probabilidad de cada evento menos las posibles intersecciones
  dos a dos, sumando las probabilidades tres a tres\dots
\end{Teo}
\begin{Demo}
  Siguiendo por inducción. Los caso base $n=1$ y $n=2$ caen en la
  definición recursiva y para $n=3$ fue el desarrollo anterior.
  Para el paso inductivo, supóngase que la propiedad se mantiene hasta
  un valor $n$.
  \[
  \begin{derivation}
      \wff{P\left(\bigcup_{i=1}^{n+1} A_i\right)}\\
    =\\
      \wff{
        P(A_{n+1}) + P\left(\bigcup_{i=1}^n A_i\right) -
        P\left(\bigcup_{i=1}^n (A_{n+1}\cap A_i)\right)
      }\\
    =\\
      &P(A_{n+1}) + P\left(\bigcup_{i=1}^n A_i\right)\\
      &\quad-\left[\hspace{3pt}
        \sum_{\mathclap{1\leq i\leq n}}(A_{n+1}\cap A_i)
        -\sum_{\mathclap{1\leq i<j\leq n}}(A_{n+1}\cap A_i\cap A_j) +
        \dots+(-1)^nP\left(\bigcap_{i=1}^{n+1}A_i\right)
      \right]\\
    =\\
      &\qquad\begin{matrix*}[l]
        \sum_{\mathclap{1\leq i\leq n+1}}P(A_i)
        -\sum_{\mathclap{1\leq i < j\leq n+1}}P(A_i\cap A_j) +\\
        \quad\sum_{\mathclap{1\leq i<j<k\leq n+1}}P(A_i\cap A_j\cap A_k)
        -\dots+(-1)^{n+1}P\left(\bigcap_{i=1}^{n+1} A_i\right)
      \end{matrix*}
  \end{derivation}
  \]
\end{Demo}

\begin{Def}[Eventos independientes]
  Sean $A$ y $B$ eventos de un espacio muestral. Se dice que $A$ y $B$
  son independientes, cuando $P(A\cap B)=P(A)\,P(B)$.
\end{Def}

\begin{Teo}
  Sean $A$ y $B$ eventos independientes. Entonces
  \begin{enumerate}
    \item $A^c$ y $B^c$ son independientes.
    \item $A^c$ y $B$ son independientes.
  \end{enumerate}
\end{Teo}
\begin{Demo}
  \begin{enumerate}
    Supóngase $A$ y $B$ eventos independientes de un espacio muestral.
    \item Partiendo de que $A^c \cap B^c = (A\cup B)^c$.
    \[
    \begin{derivation}<0.9>
        \wff{P(A^c\cap B^c)}\\
      =\\
        \wff{1 - P(A\cup B)}\\
      =\\
        \wff{1 - [P(A) + P(B) - P(A\cap B)]}\\
      =\\
        \wff{1-P(A)+P(B)-P(A)P(B)}\\
      =\\
        \wff{(1-P(A))(1-P(B))}\\
      =\\
        \wff{P(A^c)\,P(B^c)}
    \end{derivation}
    \]
    Así, los eventos $A^c$ y $B^c$ también son independientes.
    \item Partiendo de que $B=(B\cap A)\cup(B\cap A^c)$
    \[
    \begin{derivation}<0.9>
        \wff{P(B)}\\
      =\\
        \wff{P((B\cap A)\cup(B\cap A^c))}\\
      =\\
        \wff{P(B\cap A) + P(B\cap A^c) - P(\varnothing)}\\
      =\\
        \wff{P(B)\,P(A) + P(B\cap A^c)}
    \end{derivation}
    \]
    Tomando la primera y última igualdad
    \[
    \begin{derivation}<0.9>
        \wff{P(B\cap A^c)}\\
      =\\
        \wff{P(B) - P(B)P(A)}\\
      =\\
        \wff{P(B)(1-P(A))}\\
      =\\
        \wff{P(B)\,P(A^c)}
    \end{derivation}
    \]
    Así, los eventos $A^c$ y $B$ también son independientes.
  \end{enumerate}
\end{Demo}
