\documentclass{article}
\usepackage[arrows]{logicDG}
\usepackage{calcDG}
\usepackage{analysis}
\usepackage{amsfonts}
\usepackage{mathrsfs}
\usepackage{amsthm}
\usepackage{thmtools}
\usepackage{etoolbox}
\usepackage[hidelinks]{hyperref}
\usepackage{lmodern}
\usepackage[T1]{fontenc}
\usepackage[spanish, es-noquoting, es-lcroman, es-noshorthands]{babel}
\usepackage{fancyhdr}
\usepackage{graphicx}
\usepackage{setspace}
\usepackage{tikz}
\usetikzlibrary{positioning}
\usepackage{geometry}
\geometry{
  left=2cm,
  right=2cm,
  bottom=4cm,
  a4paper
}
\declaretheoremstyle[
    headpunct={},
    bodyfont={\normalfont\leftskip2.5em},
    headindent=-2.5em,
    headformat={\textbf{Demostración:}},
    qed={\qedsymbol}
]{Proof}

\begin{document}

\section{Propiedades Var y E}
\subsection{$Var(x) \geq 0$}
\begin{proof}
    Por definición $Var(X) = E(X - \mu)^2$ donde $\mu = E(X) = \int_{-\infty}^{\infty}xf(x)dx$ 
    Entonces $Var(X) = \int_{-\infty}^{\infty}(x-\mu)^2 f(x)dx$.

    Dado que $(x-\mu)^2 \geq 0$ y $f(x) \geq 0$ para toda función de densidad, entonces
    la integral siempre es mayor que cero
\end{proof}
\subsection{$Var(cX) = c^2Var(X)$}
\end{document}