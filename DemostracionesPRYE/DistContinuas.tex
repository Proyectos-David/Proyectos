\section{Distribuciones Continuas}
\subsection{Chi-cuadrada}
Sea X una variable aleatoria tal que $X\sim\chi^2(V)$, entonces su $fdd$
\[
    F(X) = \frac{1}{2^{\nicefrac{v}{2}}\Gamma\left(\frac{v}{2}\right)}x^{\left(\nicefrac{v}{2}\right)-1}e^{\nicefrac{-x}{2}}
\]
Entonces $\text{Var}[X] = 2V$ y $\text{E}[X] = V$

\begin{Demo}
    Primero se comprueba $\text{E}[X] = V$

    \begin{center}
        \begin{derivation}
            & \text{E}[x]\\
            =\\
            & \int_{0}^{\infty} xf(x)dx\\
            =\\
            & \frac{1}{2^{\nicefrac{v}{2}}\Gamma\left(\frac{v}{2}\right)}\int_{0}^{\infty}x^{\left(\nicefrac{v}{2}\right)}e^{\nicefrac{-x}{2}}dx\\
            = \\
            & \frac{1}{2^{\nicefrac{v}{2}}\Gamma\left(\frac{v}{2}\right)}2\int_{0}^{\infty}2^{\nicefrac{v}{2}}u^{\nicefrac{v}{2}}e^{-u}du\\
            =\\
            & \frac{2}{\Gamma\left(\frac{v}{2}\right)}\int_{0}^{\infty}u^{\nicefrac{v}{2}}e^{-u}du\\
            =\\
            & \frac{2}{\Gamma\left(\frac{v}{2}\right)}\Gamma\left(\frac{v}{2}+1\right)\\
            =\\
            &\frac{2}{\Gamma\left(\frac{v}{2}\right)}\left(\frac{v}{2}\right)\Gamma\left(\frac{v}{2}\right)\\
            =\\
            & v 
            \end{derivation}
    \end{center}

    Ahora, se demuestra que $Var(x) = 2n$
    
    \begin{center}
        \begin{derivation}
            & Var(x)\\
            =\\
            & E(X^2)-E^2(X)\\
            =\\
            &\frac{1}{2^{\nicefrac{v}{2}}\Gamma\left(\frac{v}{2}\right)}\int_{0}^{\infty}x^{2}x^{(\nicefrac{v}{2})-1}e^{\nicefrac{-x}{2}}dx -v^2\\
            =\\
            &\frac{1}{2^{\nicefrac{v}{2}}\Gamma\left(\frac{v}{2}\right)}\int_{0}^{\infty}x^{(\nicefrac{v}{2})+1}e^{-\nicefrac{x}{2}}dx - v^2\\
            =\\
            &\frac{1}{2^{\nicefrac{v}{2}}\Gamma\left(\frac{v}{2}\right)}2\int_{0}^{\infty}2^{1+\nicefrac{v}{2}}u^{1+\nicefrac{v}{2}}e^{-u}du - v^2\\
            =\\
            & \frac{4}{\Gamma\left(\frac{v}{2}\right)}\Gamma\left(\frac{v}{2}+2\right)-v^2\\
            =\\
            & \frac{4}{\Gamma\left(\frac{v}{2}\right)}\left(\frac{v}{2}+1\right)\left(\frac{v}{2}\right)\Gamma\left(\frac{v}{2}\right)-v^2\\
            =\\
            & 4\left(\frac{v^2}{4}\right)+4\left(\frac{v}{2}\right)-v^2\\
            =\\
            & v^2 +2v - v^2\\
            =\\
            & 2v
        \end{derivation}
    \end{center}
\end{Demo}
