\subsubsection{Distribución Normal}
\begin{Def}
  Sea $X$ una variable aleatoria continua. $X$ tiene distribución normal
  de parámetros $\mu\in\R$, $\sigma^2\in\R^+$ cuando su función de densidad es
  \[
    f(x) = \dfrac{1}{\sqrt{2\pi\sigma^2}}
    e^{\nicefrac{(x-\mu)^2}{\left(2\sigma^2\right)}}
    \qquad (x\in\R)
  \]
  Esto se denotará como $X\sim N(\mu,\sigma^2)$.
\end{Def}
\begin{Teo}
  Sean $X\sim N(\mu,\sigma^2)$ y $f$ la función de masa de $X$. Entonces,
  \begin{enumerate}
    \item Para todo $x\in\R$, $f(x) \geq 0$.
    \item $\int_{\R}{f(x)}\diff{x} = 1$.
    \item $\text{E}[X] = \mu$.
    \item $\text{Var}[X] = \sigma^2$.
  \end{enumerate}
\end{Teo}

Para las integrales involucradas en los resultados del teorema es
conveniente tener el siguiente resultado anterior a proceder con el teorema.
\[I = \int_\R e^{-x^2}\diff{x} = \sqrt{\pi}\]
\begin{longderivation}
    & I^2\\
  =\\
    & \left(\int_{\R}e^{-x^2}\diff{x}\right)\left(\int_{\R}e^{-x^2}\diff{x}\right)\\
  =\\
    & \left(\int_{\R}e^{-x^2}\diff{x}\right)\left(\int_{\R}e^{-y^2}\diff{y}\right)\\
  =\\
    & \int_{\R}\int_{\R}e^{-(x^2+y^2)}\diff{x,y}\\
  =\\
    & \int_{\R^2}e^{-(x^2+y^2)}\diff{x,y}
\end{longderivation}
  Parametrizando $\R^2$ con la función
  \[
    \vv{t(r,\theta)}=\left<r\cos(\theta),r\sin(\theta)\right>\qquad
    (r,\theta)\in \left(R^+\cup\{0\}\right)\times[-\pi,\pi]
  \]
  acomodando $\vv{t}$ para $\R^3$, el elemento de área es
\begin{longderivation}<0.9>
    & \left|\vv{t_r} \times \vec{t_\theta}\right|\\
  =\\
    & \left|
      \left<\cos(\theta),\sin(\theta),0\right>
      \times
      \left< -r\sin(\theta),r\cos(\theta),0\right>
    \right|\\
  =\\
    & \left|
      \left<0,0,r\cos^2(\theta) + r\sin^2(\theta)\right>
    \right|\\
  =\\
    & r
\end{longderivation}
Dado que
\[(x^2+y^2)\circ\vv{t(r,\theta)} = r^2\sin^2(\theta) + r^2\cos^2(\theta) = r^2\]
La integral presentada se puede reescribir y desarrollar de la siguiente manera
\begin{longderivation}
    & \int_0^\infty\int_{-\pi}^\pi re^{-r^2}\diff{\theta,r}\\
  =\\
    & 2\pi\int_0^\infty re^{-r^2}\diff{r}\\
  \why[=]{Tomando $u=r^2$}\\
    & \pi\int_0^\infty e^{-u}\diff{u}\\
  =\\
    & \pi
\end{longderivation}
Así, $I = \sqrt{\pi}$