\subsubsection{Chi-cuadrada}
\begin{Def}
    Sea $\chi^2$ una variable aleatoria continua. $\chi^2$ tiene una
    distribución $\chi^2$ de parámetro $v$ si su función de densidad es:
    \[
        f(x) = \frac{1}{2^{\nicefrac{v}{2}}\Gamma\left(\frac{v}{2}\right)}
        x^{\left(\nicefrac{v}{2}\right)-1}e^{\nicefrac{-x}{2}}
    \] 
\end{Def}

\begin{Teo}
    Sea $\chi^2 \sim \chi^2(v)$ y $f$ su función de densidad:
    \begin{enumerate}
        \item Para todo $x \geq 0$, $f(x) \geq 0$
        \item $\int_{0}^{\infty}f(x)dx$ = 1
        \item E[$\chi^2$]$= v$
        \item Var[$\chi^2$]$= 2v$
    \end{enumerate}
\end{Teo}

\begin{Demo}~
    \begin{enumerate}
        \item Todos los términos que componen la expresión de $f$ son
            no negativos para
            $x \in \R$, por lo tanto $f(x)\geq 0$ para todo $x$.
        \item Basta con demostrar la siguiente igualdad:
            \[
                \int_{0}^{\infty}x^{\nicefrac{v}{2}-1}
                e^{\nicefrac{-x}{2}}dx=2^{\nicefrac{v}{2}}
                \Gamma\left(\frac{v}{2}\right)
            \]
            Para esto, se hace el cambio de variable $\nicefrac{x}{2} = u$.
            Cuando $x=0$, $u=0$; cuando $x\to\infty$, $u\to\infty$ también,
            además $\diff{u} = $
            por lo qe la integral queda:
            \[
                \int_{0}^{\infty}(2u)^{\nicefrac{v}{2}-1}
                e^{-u}
            \]
    \end{enumerate}

    Ahora se comprueba $\text{E}[X] = v$

    \begin{center}
        \begin{derivation}
            & \text{E}[X]\\
            =\\
            & \int_{0}^{\infty} xf(x)dx\\
            =\\
            & \frac{1}{2^{\nicefrac{v}{2}}\Gamma\left(\frac{v}{2}\right)}
            \int_{0}^{\infty}x^{\left(\nicefrac{v}{2}\right)}e^{\nicefrac{-x}{2}}dx\\
            = \\
            & \frac{1}{2^{\nicefrac{v}{2}}\Gamma\left(\frac{v}{2}\right)}
            2\int_{0}^{\infty}2^{\nicefrac{v}{2}}u^{\nicefrac{v}{2}}e^{-u}du\\
            =\\
            & \frac{2}{\Gamma\left(\frac{v}{2}\right)}\int_{0}^{\infty}u^{\nicefrac{v}{2}}e^{-u}du\\
            =\\
            & \frac{2}{\Gamma\left(\frac{v}{2}\right)}\Gamma\left(\frac{v}{2}+1\right)\\
            =\\
            &\frac{2}{\Gamma\left(\frac{v}{2}\right)}\left(\frac{v}{2}\right)\Gamma\left(\frac{v}{2}\right)\\
            =\\
            & v 
            \end{derivation}
    \end{center}

    Por ultimo, se demuestra que $\text{Var}[x] = 2v$
    
    \begin{center}
        \begin{derivation}
            & \text{Var}[X]\\
            =\\
            & \text{E}[X^2]-\text{E}^2[X]\\
            =\\
            &\frac{1}{2^{\nicefrac{v}{2}}\Gamma\left(\frac{v}{2}\right)}\int_{0}^{\infty}x^{2}
            x^{(\nicefrac{v}{2})-1}e^{\nicefrac{-x}{2}}dx -v^2\\
            =\\
            &\frac{1}{2^{\nicefrac{v}{2}}\Gamma\left(\frac{v}{2}\right)}\int_{0}^{\infty}
            x^{(\nicefrac{v}{2})+1}e^{-\nicefrac{x}{2}}dx - v^2\\
            =\\
            &\frac{1}{2^{\nicefrac{v}{2}}\Gamma\left(\frac{v}{2}\right)}2\int_{0}^{\infty}
            2^{1+\nicefrac{v}{2}}u^{1+\nicefrac{v}{2}}e^{-u}du - v^2\\
            =\\
            & \frac{4}{\Gamma\left(\frac{v}{2}\right)}\Gamma\left(\frac{v}{2}+2\right)-v^2\\
            =\\
            & \frac{4}{\Gamma\left(\frac{v}{2}\right)}\left(\frac{v}{2}+1\right)\left(\frac{v}{2}\right)
            \Gamma\left(\frac{v}{2}\right)-v^2\\
            =\\
            & 4\left(\frac{v^2}{4}\right)+4\left(\frac{v}{2}\right)-v^2\\
            =\\
            & v^2 +2v - v^2\\
            =\\
            & 2v
        \end{derivation}
    \end{center}
\end{Demo}