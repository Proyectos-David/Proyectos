\subsubsection{Chi-cuadrada}
\begin{Def}
    Sea $X$ una variable aleatoria continua, $X$ tiene una
    distribución $\chi^2$ con parámetro $v$ si su función de densidad
    es:
    \[
        f(x) = \frac{1}{2^{\nicefrac{v}{2}}\Gamma\left(\frac{v}{2}\right)}
        x^{\left(\nicefrac{v}{2}\right)-1}e^{\nicefrac{-x}{2}} \qquad(x\geq0)
    \]  
\end{Def}

\begin{Teo}
    Si $X \sim \chi^2(v)$ tiene función de densidad $f$, entonces:
    \begin{enumerate}
        \item para todo $x\in \R$, $f(x)\geq0$
        \item $\int_{-\infty}^{\infty}f(x)\diff{x}\geq0$
        \item $\text{E}[X]= v \qquad(v\geq0)$
        \item $\text{Var}[X]= 2v \qquad(v\geq0)$
    \end{enumerate}
\end{Teo}

\begin{Demo}~
    \begin{enumerate}
        \item Dado que todos los factores del término que componen a
        $f$ son positivos, $f\geq0$ para todo $x\geq 0$
        \item Para demostrar esto, basta con demostrar que 
        \[
            \int_{0}^{\infty}x^{\left(\nicefrac{v}{2}\right)-1}
            e^{\nicefrac{-x}{2}}\diff{x}
            =2^{\nicefrac{v}{2}}\Gamma\left(\frac{v}{2}\right)
        \]
            El procedimiento inicia con el cambio de variable
            $u=\displaystyle\frac{x}{2}$. Si $x=0$, $u=0$ 
            y si $x\to\infty$, $u\to\infty$. De la misma 
            manera $\diff{u} = \displaystyle\frac{1}{2}\diff{x}$, por lo 
            tanto $\diff{x}=2\diff{u}$, así:
        \begin{center}
            \begin{longderivation}
                & \int_{0}^{\infty}x^{(\nicefrac{v}{2})-1}
                e^{\nicefrac{-x}{2}}\diff{x}\\
                =\\
                & \int_{0}^{\infty}(2u)^{(\nicefrac{v}{2})-1}e^{-u}2\diff{u}\\
                =\\
                & 2^{\nicefrac{v}{2}}\int_{0}^{\infty}
                u^{(\nicefrac{v}{2})-1}e^{-u}\diff{u}
            \end{longderivation}
        \end{center}
        Recordando que la definición de la función $\Gamma$ es 
        \[
            \Gamma(x)=\int_{0}^{\infty}t^{x-1}e^{-t}\diff{t}
        \]
        Por lo que se obtiene:
        \[
            \int_{0}^{\infty}x^{(\nicefrac{v}{2})-1}
            e^{\nicefrac{-x}{2}}\diff{x} =
            2^{\nicefrac{v}{2}}\Gamma\left(\frac{v}{2}\right)
        \]
        Que es lo que se quería demostrar.
        \item Para empezar la demostración, se hará la misma sustitución
        del item pasado.

        \begin{center}
            \begin{longderivation}
                & \text{E}[X]\\
                =\\
                & \int_{-\infty}^{\infty} xf(x)dx\\
                =\\
                & \frac{1}{2^{\nicefrac{v}{2}}
                \Gamma\left(\frac{v}{2}\right)}\int_{0}^{\infty}x^{
                \nicefrac{v}{2}}e^{\nicefrac{-x}{2}}\diff{x}\\
                \why[=]{\text{Cambio de variable}}\\
                & \frac{1}{2^{\nicefrac{v}{2}}
                \Gamma\left(\frac{v}{2}\right)}\int_{0}^{\infty}(2u)^{
                \nicefrac{v}{2}}e^{-u}2\diff{u}\\
                =\\
                & \frac{2}{\Gamma\left(\frac{v}{2}\right)}
                \int_{0}^{\infty}u^{\nicefrac{v}{2}}e^{-u}\diff{u}\\
                \why[=]{\text{Definición de $\Gamma$}}\\
                & \frac{2}{\Gamma\left(\frac{v}{2}\right)}
                \Gamma\left(\frac{v}{2}+1\right)
            \end{longderivation}
        \end{center}
        Por propiedad de la función $\Gamma$, $\Gamma(x+1)=x\Gamma(x)$, 
        la igualdad se transforma a lo siguiente:
        \begin{center}
            \begin{longderivation}
                &\text{E}[X]=\left(\frac{2}
                {\Gamma\left(\frac{v}{2}\right)}\right)\left(\frac{v}
                {2}\right)\Gamma\left(\frac{v}{2}\right)\\
                \iff\\
                &\text{E}[X]=v
            \end{longderivation}
        \end{center}
        Lo que demuestra la propiedad
        \item Nuevamente se utiliza el mismo cambio de variable para esta demostración.
        \begin{center}
            \begin{longderivation}
                & \text{Var}[X]\\
                =\\
                & \text{E}[X^2]-\text{E}^2[X]\\
                =\\
                &\frac{1}{2^{\nicefrac{v}{2}}\Gamma\left(\frac{v}{2}\right)}
                \int_{0}^{\infty}x^{2}x^{(\nicefrac{v}{2})-1}
                e^{\nicefrac{-x}{2}}\diff{x} -v^2\\
                =\\
                &\frac{1}{2^{\nicefrac{v}{2}}\Gamma\left(\frac{v}{2}\right)}
                \int_{0}^{\infty}x^{(\nicefrac{v}{2})+1}
                e^{-\nicefrac{x}{2}}\diff{x} - v^2\\
                \why[=]{\text{Cambio de variable}}\\
                &\frac{1}{2^{\nicefrac{v}{2}}\Gamma\left(\frac{v}{2}\right)}
                \int_{0}^{\infty}(2u)^{(\nicefrac{v}{2})+1}
                e^{-u}2\diff{u}-v^2\\
                =\\
                &\frac{1}{2^{\nicefrac{v}{2}}\Gamma\left(\frac{v}{2}\right)}
                2^{(\nicefrac{v}{2})+2}
                \int_{0}^{\infty}u^{(\nicefrac{v}{2})+1}e^{-u}\diff{u}-v^2\\
                \why[=]{\text{Definición de $\Gamma$}}\\
                &\frac{4}{\Gamma\left(\frac{v}{2}\right)}\Gamma
                \left(\frac{v}{2}+2\right)-v^2\\
                \why[=]{\text{propiedades de $\Gamma$}}\\
                &\frac{4}{\Gamma\left(\frac{v}{2}\right)}\left
                (\left(\frac{v}{2}+1\right)\left
                (\frac{v}{2}\right)\Gamma\left(\frac{v}{2}\right)\right)-v^2\\
                =\\
                &4\left(\frac{v+2}{2}\right)\left(\frac{v}{2}\right)-v^2\\
                =\\
                &(v+2)(v)-v^2\\
                =\\
                &v^2+2v-v^2\\
                =\\
                2v
            \end{longderivation}
        \end{center}    
        Así, se obtiene que Var$[X]=2v$, lo que concluye la demostración.     
    \end{enumerate}
\end{Demo}