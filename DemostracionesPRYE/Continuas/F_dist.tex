\subsubsection{Distribucion F}

\begin{Def}
  Sean $F$ una variable aleatoria continua. $F$ tiene una distribución
  $F$ de parámetros $u,v\in\R^+$, cuando su función de densidad es
  \[f(x)=
  \dfrac{\Gamma\left(\frac{u+v}{2}\right)\left(\frac{u}{v}\right)^{\nicefrac{u}{2}}}
  {\Gamma\left(\frac{u}{2}\right)\Gamma\left(\frac{v}{2}\right)}
  \dfrac{x^{(\nicefrac{u}{2})-1}}
  {\left(1 + \frac{u}{v}x\right)^{\nicefrac{(u+v)}{2}}}
  =
  \dfrac{\left(\frac{u}{v}\right)^{\nicefrac{u}{2}}}
    {B\left(\frac{u}{2},\frac{v}{2}\right)}
  \dfrac{x^{(\nicefrac{u}{2})-1}}
    {\left(1 + \frac{u}{v}x\right)^{\nicefrac{(u+v)}{2}}}
  \qquad (x\in\R^+)
  \]
  Recordando que $B$ hace referencia a la función beta.
\end{Def}
\begin{Teo}
  Sean $F\sim F(u,v)$ y $f$ su función de densidad. Entonces,
  \begin{enumerate}
    \item Para todo $x\in\R$, $f(x) \geq 0$.
    \item $\int_{\R}f(x)\mathrm{d}x=1$.
    \item $\text{E}[F]=\frac{v}{v-2} \qquad(v>2)$
    \item $\text{Var}[F]=\frac{2v^2(u+v-2)}{u(v-2)^2(v-4)}\qquad(v>4)$
  \end{enumerate}
\end{Teo}
\begin{Demo}~
  \begin{enumerate}
    \item Todos los términos de la expresión que define $f$ son no negativos para
    $x\in\R^+$, con lo que se concluye $f(x)\geq0$.
    \item Basta mostrar la siguiente igualdad
    \[
      \int_0^\infty{
      \dfrac{x^{(\nicefrac{u}{2})-1}}
      {\left(1 + \frac{u}{v}x\right)^{\nicefrac{(u+v)}{2}}}
      }\diff{x}
      =
      \left(\frac{u}{v}\right)^{-\nicefrac{u}{2}}
      B\left(\frac{u}{2},\frac{v}{2}\right)
    \]
    Recordando que
    \[
      B\left(\frac{u}{2},\frac{v}{2}\right) =
      \int_0^1{t^{(\nicefrac{u}{2})-1}(1-t)^{(\nicefrac{v}{2})-1}}\diff{t}
    \]

    El procedimiento inicia con el siguiente cambio de variable:
    \[\tan^2(\theta)=\frac{u}{v}x\]
    Cuando $x=0$, $\theta=0$ y cuando $x\to\infty$, $\theta\to\nicefrac{\pi}{2}$.
    Recordando que $\odv*{\tan^2(\theta)}{\theta}=2\tan(\theta)\sec^2(\theta)$,
    se obtienen las siguientes igualdades
    \begin{longderivation}
        & \int_0^\infty{
          \dfrac{x^{(\nicefrac{u}{2})-1}}
          {\left(1 + \frac{u}{v}x\right)^{\nicefrac{(u+v)}{2}}}
          }\diff{x}\\
      =\\
        & 2\frac{v}{u}\int_0^{\nicefrac{\pi}{2}}{
          \dfrac{\left(\frac{v}{u}\right)^{(\nicefrac{u}{2})-1}\tan^{u-2}(\theta)}
          {(1 + \tan^2(\theta))^{\nicefrac{(u+v)}{2}}}
          \tan(\theta)\sec^2(\theta)
        }\diff{\theta}\\
      =\\
        & 2\left(\frac{v}{u}\right)^{\nicefrac{u}{2}}
        \int_0^{\nicefrac{\pi}{2}}{
          \dfrac{\tan^{u-1}(\theta)\sec^2(\theta)}{\sec^{u+v}(\theta)}
        }\diff{\theta}\\
      =\\
        & 2\left(\frac{v}{u}\right)^{\nicefrac{u}{2}}
        \int_0^{\nicefrac{\pi}{2}}{
          \sin^{u-1}(\theta)\cos^{v-2}\cos(\theta)
        }\diff{\theta}
    \end{longderivation}
    Para continuar, se realiza el cambio de variable
    \[\tilde{n}=\sin(\theta)\]
    Cuando $\theta=0$, $\tilde{n}=0$ y cuando $\theta=\nicefrac{\pi}{2}$, $\tilde{n}=1$.
    Por otra parte, dado que $\cos^2(\theta) + \sin^2(\theta) = 1$, entonces
    $\cos(\theta) = (1-\tilde{n}^2)^{\nicefrac{1}{2}}$. Recordando que
    $\odv*{\sin(\theta)}{\theta} = \cos(\theta)$, se obtiene la siguiente igualdad
    \begin{longderivation}
        & 2\left(\frac{v}{u}\right)^{\nicefrac{u}{2}}
        \int_0^{\nicefrac{\pi}{2}}{
          \sin^{u-1}(\theta)\cos^{v-2}\cos(\theta)
        }\diff{\theta}\\
      =\\
        & 2\left(\frac{v}{u}\right)^{\nicefrac{u}{2}}
        \int_0^1{
          \tilde{n}^{u-2}(1-\tilde{n}^2)^{(\nicefrac{v}{2})-1}
          \tilde{n}
        }\,\diff{\tilde{n}}
    \end{longderivation}
    Por último, ser realiza el cambio de variable
    \[t = \tilde{n}^2\]
    Los límites de integración se mantienen. Recordando que
    $\odv*{\tilde{n}^2}{\tilde{n}}=2\tilde{n}$,
    se obtienen las siguientes igualdades
    \begin{longderivation}
        & 2\left(\frac{v}{u}\right)^{\nicefrac{u}{2}}
          \int_0^1{
            \tilde{n}^{u-2}(1-\tilde{n}^2)^{(\nicefrac{v}{2})-1}
            \tilde{n}
          }\,\diff{\tilde{n}}\\
      =\\
        & 2\left(\frac{v}{u}\right)^{\nicefrac{u}{2}}
        \int_0^1{
          \frac{1}{2}t^{(\nicefrac{u}{2})-1}(1 - t)^{(\nicefrac{v}{2})-1}
        }\diff{t}\\
      =\\
        & \left(\frac{v}{u}\right)^{\nicefrac{u}{2}}
        B\left(\frac{u}{2},\frac{v}{2}\right)\\
      =\\
        & \left(\frac{u}{v}\right)^{-\nicefrac{u}{2}}
        B\left(\frac{u}{2},\frac{v}{2}\right)
    \end{longderivation}
    \item Denotando:
    \[
      A = \displaystyle\frac{\left(\frac{u}{v}\right)^{\nicefrac{u}{2}}}
      {B\left(\frac{u}{2}\text{,}\frac{v}{2}\right)}
    \]
    Que es el término constante de $f$. Manteniendo el cambio de 
    variable se obtiene las siguientes igualdades:

    \begin{longderivation}
      &\text{E}[F]\\
      =\\
      &A\int_{-\infty}^{\infty}xf(x)\diff{x}\\
      =\\
      &A\int_{0}^{\infty}x\frac{x^{(\nicefrac{u}{2})-1}}
      {\left(1+\frac{u}{v}x\right)^{\nicefrac{(u+v)}{2}}}\diff{x}\\
      =\\
      &A\int_{0}^{\infty}\frac{x^{\nicefrac{u}{2}}}
      {\left(1+\frac{u}{v}x\right)^{\nicefrac{(u+v)}{2}}}\\
      \why[=]{\text{Cambio de variable}}\\
      &A\int_{0}^{\frac{\pi}{2}}\frac{\left(\frac{v}{u}\tan^2(\theta)\right)}
      {(1+\tan^2(\theta))^{\nicefrac{(u+v)}{2}}}
      \frac{2v}{u}\tan(\theta)\sec^2(\theta)\diff{\theta}\\
      =\\
      &2\left(\frac{v}{u}\right)^{(\nicefrac{u}{2})+1}A
      \int_{0}^{\frac{\pi}{2}}\frac{\tan^{u+1}\theta}
      {(\sec^2(\theta))^{\nicefrac{(u+1)}{2}}}
      \sec^2(\theta)\diff{\theta}\\
      =\\
      &\left(\frac{v}{u}\right)^{(\nicefrac{u}{2})+1}A
      \int_{0}^{\frac{\pi}{2}}\tan^{u+1}(\theta)
      \sec^{2-u-v}(\theta)\diff{\theta}\\
      =\\
      &\left(\frac{v}{u}\right)^{(\nicefrac{u}{2})+1}A
      \int_{0}^{\frac{\pi}{2}}\sin^{u+1}(\theta)
      \cos^{v-3}(\theta)\diff{\theta}\\
    \end{longderivation}
    Haciendo la sustitución presentada anteriormente $t=\sin(\theta)$, 
    se continua como sigue:
    \begin{longderivation}
      &2A\left(\frac{v}{u}\right)^{(\nicefrac{u}{2})+1}
      \int_{0}^{1}t^{u+1}(1-t^2)^{\frac{(v-4)}{2}}\diff{t}\\
    \end{longderivation}
    Ahora, con la sustitución $w=t^2$. Cuando $t=0$, $w=0$, cuando 
    $t=1$, $w=1$. Además, $\diff{w}=2t\diff{t}$. 
    Luego, la integral se transforma en:
    \begin{longderivation}
      &A\left(\frac{v}{u}\right)^{(\nicefrac{u}{2})+1}
      \int_{0}^{1}w^{\nicefrac{u}{2}}(1-w)^{(\nicefrac{v}{2})-2}\diff{w}\\
      \why[=]{\text{Definición de $B$}}\\
      &A\left(\frac{v}{u}\right)^{(\nicefrac{u}{2})+1}
      B\left(\frac{u}{2}+1\text{,}\frac{v}{2}-1\right)\\
    \end{longderivation}
    Recordando que $B(x\text{,}y)=\displaystyle\frac{\Gamma(x)\Gamma(y)}
    {\Gamma(x+y)}$, y expandiendo $A$ se obtiene la expresión:
    \begin{longderivation}
      \frac{\left(\frac{u}{v}\right)^{\nicefrac{u}{2}}}
      {B\left(\frac{u}{2}\text{,}\frac{v}{2}\right)}\left(\frac{v}{u}\right)^
      {(\nicefrac{u}{2})+1}
      \frac{\Gamma\left(\frac{u}{2}+1\right)\Gamma\left(\frac{v}{2}-1\right)}
      {\Gamma\left(\frac{u+v}{2}\right)}\\
      =\\
      \left(\frac{v}{u}\right)\frac{\Gamma\left(\frac{u+v}{2}\right)}
      {\Gamma\left(\frac{u}{2}\right)\Gamma\left(\frac{v}{2}\right)}
      \frac{\Gamma\left(\frac{u}{2}+1\right)\Gamma\left(\frac{v}{2}-1\right)}
      {\Gamma\left(\frac{u+v}{2}\right)}\\
      =\\
      \left(\frac{v}{u}\right)\frac{\frac{u}{2}\Gamma\left(\frac{u}{2}\right)
      \Gamma\left(\frac{v}{2}-1\right)}{\Gamma\left(\frac{u}{2}\right)
      \left(\frac{v}{2}-1\right)\Gamma\left(\frac{v}{2}-1\right)}\\
      =\\
      \left(\frac{v}{u}\right)\left(\frac{u}{2}\right)
      \frac{1}{\left(\frac{v}{2}-1\right)}\\
      =\\
      \left(\frac{v}{2}\right)\left(\frac{2}{v-2}\right)\\
      =\\
      \frac{v}{v-2}
    \end{longderivation}
    Así, Var$[F]=\displaystyle\frac{v}{v-2}$, y se concluye la demostración. 
  \end{enumerate}
\end{Demo}