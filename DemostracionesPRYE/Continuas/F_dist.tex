\subsubsection{Distribucion F}

\begin{Def}
  Sean $F$ una variable aleatoria continua. $F$ tiene una distribución
  $F$ de parámetros $u,v\in\R^+$, cuando su función de masa es
  \[f(x)=
  \dfrac{\Gamma\left(\frac{u+v}{2}\right)\left(\frac{u}{v}\right)^{\nicefrac{u}{2}}}
  {\Gamma\left(\frac{u}{2}\right)\Gamma\left(\frac{v}{2}\right)}
  \dfrac{x^{(\nicefrac{u}{2})-1}}
  {\left(1 + \frac{u}{v}x\right)^{\nicefrac{(u+v)}{2}}}
  =
  \dfrac{\left(\frac{u}{v}\right)^{\nicefrac{u}{2}}}
    {B\left(\frac{u}{2},\frac{v}{2}\right)}
  \dfrac{x^{(\nicefrac{u}{2})-1}}
    {\left(1 + \frac{u}{v}x\right)^{\nicefrac{(u+v)}{2}}}
  \qquad (x\in\R^+)
  \]
  Recordando que $B$ hace referencia a la función beta.
\end{Def}
\begin{Teo}
  Sea $F\sim F(u,v)$ y $f$ su función de masa. Entonces,
  \begin{enumerate}
    \item Para todo $x\in\R$, $f(x) \geq 0$.
    \item $\int_{\R}f(x)\mathrm{d}x=1$.
    \item $\text{E}[F]=\frac{v}{v-2} \qquad(v>2)$
    \item $\text{Var}[F]=\frac{2v^2(u+v-2)}{u(v-2)^2(v-4)}\qquad(v>4)$
  \end{enumerate}
\end{Teo}
\begin{Demo}~
  \begin{enumerate}
    \item Todos los términos de la expresión que define $f$ son no negativos para
    $x\in\R^+$, con lo que se concluye $f(x)\geq0$.
    \item Basta mostrar la siguiente igualdad
    \[
      \Int[0,\infty]{
      \dfrac{x^{(\nicefrac{u}{2})-1}}
      {\left(1 + \frac{u}{v}x\right)^{\nicefrac{(u+v)}{2}}}
      }{x}
      =
      \left(\frac{u}{v}\right)^{-\nicefrac{u}{2}}
      B\left(\frac{u}{2},\frac{v}{2}\right)
    \]
    Recordando que
    \[
      B\left(\frac{u}{2},\frac{v}{2}\right) =
      \Int[0,1]{t^{(\nicefrac{u}{2})-1}(1-t)^{(\nicefrac{v}{2})-1}}{t}
    \]

    El procedimiento inicia con el siguiente cambio de variable:
    \[\tan^2(\theta)=\frac{u}{v}x\]
    Cuando $x=0$, $\theta=0$ y cuando $x\to\infty$, $\theta\to\nicefrac{\pi}{2}$.
    Recordando que $\odv*{\tan^2(\theta)}{\theta}=2\tan(\theta)\sec^2(\theta)$,
    se obtienes las siguientes igualdades
    \begin{longderivation}
        & \Int[0,\infty]{
          \dfrac{x^{(\nicefrac{u}{2})-1}}
          {\left(1 + \frac{u}{v}x\right)^{\nicefrac{(u+v)}{2}}}
          }{x}\\
      =\\
        & 2\frac{n}{m}\Int[0,\nicefrac{\pi}{2}]{
          \dfrac{\left(\frac{n}{m}\right)^{(\nicefrac{m}{2})-1}\tan^{m-2}(\theta)}
          {(1 + \tan^2(\theta))^{\nicefrac{(n+m)}{2}}}
          \tan(\theta)\sec^2(\theta)
        }{\theta}\\
      =\\
        & 2\left(\frac{n}{m}\right)^{\nicefrac{m}{2}}
        \Int[0,\nicefrac{\pi}{2}]{
          \dfrac{\tan^{m-1}(\theta)\sec^2(\theta)}{\sec^{m+n}(\theta)}
        }{\theta}\\
      =\\
        & 2\left(\frac{n}{m}\right)^{\nicefrac{m}{2}}
        \Int[0,\nicefrac{\pi}{2}]{
          \sin^{m-1}(\theta)\cos^{n-2}\cos(\theta)
        }{\theta}
    \end{longderivation}
    Para continuar, se realiza el cambio de variable
    \[v=\sin(\theta)\]
    Cuando $\theta=0$, $v=0$ y cuando $\theta=\nicefrac{\pi}{2}$, $v=1$.
    Por otra parte, dado que $\cos^2(\theta) + \sin^2(\theta) = 1$, entonces
    $\cos(\theta) = (1-v^2)^{\nicefrac{1}{2}}$. Recordando que
    $\odv*{\sin(\theta)}{\theta} = \cos(\theta)$, se obtiene la siguiente igualdad
    \begin{longderivation}
        & 2\left(\frac{n}{m}\right)^{\nicefrac{m}{2}}
        \Int[0,\nicefrac{\pi}{2}]{
          \sin^{m-1}(\theta)\cos^{n-2}\cos(\theta)
        }{\theta}\\
      =\\
        & 2\left(\frac{n}{m}\right)^{\nicefrac{m}{2}}
        \Int[0,1]{
          v^{m-2}(1-v^2)^{(\nicefrac{n}{2})-1}v
        }{v}
    \end{longderivation}
    Por último, ser realiza el cambio de variable
    \[t = v^2\]
    Los límites de integración se mantienen. Recordando que $\odv*{v^2}{v}=2v$,
    se obtienen las siguientes igualdades
    \begin{longderivation}
        & 2\left(\frac{n}{m}\right)^{\nicefrac{m}{2}}
          \Int[0,1]{
            v^{m-2}(1-v^2)^{(\nicefrac{n}{2})-1}v
          }{v}\\
      =\\
        & 2\left(\frac{n}{m}\right)^{\nicefrac{m}{2}}
        \Int[0,1]{
          \frac{1}{2}t^{(\nicefrac{m}{2})-1}(1 - t)^{(\nicefrac{n}{2})-1}
        }{t}\\
      =\\
        & \left(\frac{n}{m}\right)^{\nicefrac{m}{2}}
        B\left(\frac{u}{2},\frac{v}{2}\right)
    \end{longderivation}
    \item a
  \end{enumerate}
\end{Demo}