\subsubsection{Teorema Central del Límite}
Es de interés conocer la distribución de una variable aleatoria la
cual se puede escribir como combinación de varias variables
aleatorias. Estas combinaciones son llamadas \emph{estadísticos}.
Un ejemplo puede ser simplemente la suma o producto de dos variables
aleatorias. Uno de los más simples pero a su vez el que llevó
a uno de los resultados más importantes es el de la media.
\begin{Teo}
  Sea $\{X_n\}_{n\in\Z^+}$ una colección de variables aleatorias
  independientes e igualmente distribuidas, donde $\mu$ es su media y $\sigma^2$
  su varianza. denotando $\overline{X_n}=\frac{1}{n}\sum_{k=1}^nX_k$, Entonces,
  \[\sqrt{n}\,\overline{X}_n\longrightarrow N(0,1)\]
  en distribución.
\end{Teo}
\begin{Demo}
  Se define $Z_n = \frac{X_n-\mu}{\sigma}$. Para todo $n\in\Z^+$,
  \begin{align*}
    \text{E}[Z_n]&= \text{E}\left[\frac{X_n-\mu}{\sigma}\right]
    = 0\\
    \text{Var}[Z_n] &= \text{E}\left[\frac{X_n-\mu}{\sigma}\right]
    = 1
  \end{align*}
  El enunciado del teorema resulta equivalente a mostrar que
  \[\sqrt{n}\,\overline{Z}_n\longrightarrow N(0,1)\]

  Sea $\phi$ la función característica de $Z_n$ con $n\in\Z^+$.
  Se mostrará una convergencia puntual de la función
  característica. Es decir,
  \[\lim_{n\to\infty}\phi_{\sqrt{n}\overline{Z}_n}(t) = e^{-t^2/2}\]
  \begin{longderivation}
      & \phi_{\sqrt{n}\overline{Z}_n}(t)\\
    =\\
      & \text{E}\left[
        \exp\left(it\frac{Z_1 + Z_2 +\dots+Z_n}{\sqrt{n}}\right)
      \right]\\
    =\\
      & \text{E}\left[
        \exp\left(
          \prod_{k=1}^n\exp\left(i\frac{t}{\sqrt{n}}Z_k\right)
        \right)
      \right]\\
    \why[=]{
      Dado que las variables aleatorias en $\{Z_k\}_{k\in\Z^+}$ son
      independientes entre si
    }\\
      & \prod_{k=1}^n\text{E}\left[
        \exp\left(i\frac{t}{\sqrt{n}}Z_k\right)
      \right]\\
    =\\
      & \prod_{k=1}^n\phi\left(\frac{t}{\sqrt{n}}\right)\\
    =\\
      & \left[\phi\left(\frac{t}{\sqrt{n}}\right)\right]^n\\
    =\\
      & \exp\left[
        n\text{Log}\left(\phi\left(\frac{t}{\sqrt{n}}\right)\right)
      \right]
  \end{longderivation}

  Considerando únicamente el argumento de la exponencial. Se garantiza
  la existencia del límite para $n\in\Z^+$ si se extiende a
  $n\in\R^+$. Tomando ahora $y=\frac{1}{\sqrt{n}}$, se obtiene
  \begin{longderivation}
      & \lim_{y\to0}\frac{\text{Log}\left(\phi(yt)\right)}{y^2}\\
    \why[=]{Recordando que $\phi(0)=0$, se puede aplicar L'Hôpital}\\
      & \lim_{y\to0}\frac{\phi'(yt)t}{2y\phi(yt)}\\
    \why[=]{Nuevamente, $\phi(0)=0$, con lo que se puede separar el límite
    y aplicando L'Hôpital}\\
      & \lim_{y\to0}\frac{t^2\phi''(yt)}{2}\\
    \why[=]{Recordando que $\phi''(0) = -\text{Var}[Z_n]+\text{E}^2[Z_n]$}\\
      & -\frac{t^2}{2}
  \end{longderivation}
\end{Demo}