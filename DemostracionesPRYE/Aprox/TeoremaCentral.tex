\subsubsection{Teorema Central del Límite}
Es de interés conocer la distribución de una variable aleatoria la
cual se puede escribir como combinación de varias variables
aleatorias. Estas combinaciones son llamadas \emph{estadísticos}.
Un ejemplo puede ser simplemente la suma o producto de dos variables
aleatorias. Uno de los más simples pero a su vez el que llevó
a uno de los resultados más importantes es el de la media.
\begin{Teo}
  Sean $\{X_n\}_{n\in\Z^+}$ una colección de variables aleatorias
  independientes e igualmente distribuidas, $\mu$ su media y $\sigma^2$
  su varianza. Sean $\{S_n\}_{n\in\Z^+}=\left\{\sum_{i=1}^nX_i\right\}$
  y $\{Y_n\}_{n\in\Z^+}=\left\{\dfrac{S_n-n\mu}{\sigma\sqrt{n}}\right\}$.
  Entonces
  \[Y_n\to N(0,1)\]
\end{Teo}
\begin{Demo}
  Sin pérdida de generalidad, supóngase que $\mu=0$. Sea $\phi$ la
  función característica de los elementos en $\{X_n\}_{n\in\Z^+}$.
  Entonces,
  \begin{longderivation}
      & \phi_{Y_n}(t)\\
    =\\
      & \text{E}\left[
        \exp\left(it\frac{X_1+X_2+\dots+X_n}{\sigma\sqrt{n}}\right)
      \right]\\
    =\\
      & \text{E}\left[
        \prod_{k=1}^n\exp\left(
          \frac{itX_k}{\sigma\sqrt{n}}
        \right)
      \right]\\
    \why[=]{Dado que las variables $X_k$ son independientes}\\
      & \prod_{k=1}^n\text{E}\left[
        \exp\left(\frac{itX_k}{\sigma\sqrt{n}}\right)
      \right]\\
    =\\
      & \prod_{k=1}^n\phi\left(\frac{t}{\sigma\sqrt{n}}\right)\\
    =\\
      & \left[\phi\left(\frac{t}{\sigma\sqrt{n}}\right)\right]^n
  \end{longderivation}

  Recordando las \hyperref[Teo:prop_carac]{propiedades de $\phi$} y cómo
  se calcula el valor esperado y varianza, se pueden obtener los
  primeros términos de su expansión en serie de Mclaurin.
  \[\phi(t) = 1 + 0 + i^2\frac{\sigma^2t^2}{2} + \dots\]
\end{Demo}