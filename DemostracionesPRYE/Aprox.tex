\subsection{Teoremas de Aproximación}
Se puede ver una similitud entre la \hyperref[dist:binom]{distribución binomial}
y la \hyperref[dist:hip]{distribución hipergeométrica}, pues si en esta última, manteniendo
un tamaño de muestra ($n$) fijo, a medida que aumentan el total de objetos ($N$ y $K$) bajo
ciertas condiciones, los eventos que esta distribución describe tienen a ser independientes.
Esto lleva al siguiente teorema de aproximación.

\begin{Teo}
  Sea $X$ una variable aleatoria con distribución hipergeométrica de
  parámetros $N,K,n$. Si para un $\epsilon>0$ , $n > 1$ , $x > 0$,
  se tiene que
  \begin{align*}
    \frac{x-1}{K}     &< \epsilon\\
    \frac{n-x-1}{N-K} &< \epsilon\\
    \frac{n-1}{N}     &< \epsilon
  \end{align*}
  entonces,
  \[\left|P(X=x) - B\left(n,\frac{K}{N}\right)(x)\right| < (\epsilon + 1)^{2n} - 1\]
\end{Teo}

Antes de comenzar con la demostración de este teorema, se presenta el siguiente lema, el
cual será de utilidad para obtener el resultado presentado.
\begin{Lema}
  Sean $r\in\Z^+$ y $\left\{S_{k,n}\right\}_{1}^{r}$ una colección de $r$ sucesiones
  en función de $n$ las cuales convergen a $1$. Dado $\epsilon>0$,
  existe un $N\in\N$ tal que para todo $1\leq k\leq r$
  \[n\geq N \To |S_{k,n} - 1| < \epsilon\]
  entonces,
  \[n\geq N \To \left|\prod_{k=1}^r S_{k,n} -1\right| < (\epsilon + 1)^r - 1\]
\end{Lema}
\begin{Demo}
  La existencia de este $N$ consta de tomar el máximo entre los diferentes $N$'s
  dados por la convergencia de cada sucesión en la colección. La demostración del
  acotamiento se hará por inducción.

  Caso base ($r=1$): se tiene que $|S_{1,n}| < \epsilon$, con lo que se cumple
  la propiedad para una sucesión.

  Paso inductivo: supóngase que para el producto de $r$ sucesiones, la propiedad se mantiene.
  Para $r+1$ sucesiones

  \begin{longderivation}
      & \left|\prod_{k=1}^{r+1}S_{k,n} - 1\right|\\
    =\\
      & \left|S_{r+1,n}\prod_{k=1}^{r}S_{k,n} - 1\right|\\
    =\\
      & \left|
        (S_{r+1,n} - 1)\left(\prod_{k=1}^{r}S_{k,n} - 1\right)
        + (S_{r+1,n} - 1)
        + \left(\prod_{k=1}^{r}S_{k,n} - 1\right)
      \right|\\
    \leq\\
      &\left|(S_{r+1,n} - 1)\left(\prod_{k=1}^{r}S_{k,n} - 1\right)\right|
      + \left|(S_{r+1,n} - 1)\right|
      + \left|\left(\prod_{k=1}^{r}S_{k,n} - 1\right)\right|\\
    <\\
      &\epsilon\left|\left(\prod_{k=1}^{r}S_{k,n} - 1\right)\right|
      + \epsilon + \left|\left(\prod_{k=1}^{r}S_{k,n} - 1\right)\right|\\
    =\\
      &\left|\left(\prod_{k=1}^{r}S_{k,n} - 1\right)\right|(\epsilon + 1) + \epsilon
  \end{longderivation}

  Con esto, se define $f$ de la siguiente manera: $f(1) = \epsilon$ y
  $f(r+1) = f(r)(\epsilon + 1) + \epsilon$. Nótese que, por el caso base y lo
  desarrollado, $f$ cumple que
  $\left|\prod_{k=1}^r S_{k,r}-1\right| < f(r)$.
  Para resolver la ecuación de recurrencia, se define $g$ como
  $g(r) = \frac{f(r)}{(\epsilon + 1)^r}$, es decir,
  $g(1) = \frac{\epsilon}{\epsilon + 1}$ y $g(r+1) = \frac{f(r+1)}{(\epsilon + 1)^r+1}$.
  Desarrollando la última expresión, se obtiene
  \[
    g(r+1) = \frac{f(r)}{(\epsilon + 1)^r} + \frac{\epsilon}{(\epsilon+1)^{r+1}}
    = g(r) + \frac{\epsilon}{(\epsilon+1)^{r+1}}
  \]
  con lo que, por definición,
  \begin{longderivation}
      & g(r)\\
    =\\
      & \sum_{k=1}^{r}\frac{\epsilon}{(\epsilon + 1)^k}\\
    =\\
      & \frac{\epsilon}{\epsilon + 1}\sum_{k=0}^{r-1}\frac{1}{(\epsilon+1)^r}\\
    =\\
      & \frac{\epsilon}{\epsilon+1}
      \left(\frac{1 - \dfrac{1}{(\epsilon + 1)^r}}{1 - \dfrac{1}{\epsilon+1}}\right)\\
    =\\
      & \frac{\epsilon}{\epsilon+1}
      \left(\frac{(\epsilon + 1)^r - 1}{(\epsilon+1)^r}\frac{\epsilon+1}{\epsilon}\right)\\
    =\\
      & \frac{(\epsilon + 1)^r - 1}{(\epsilon+1)^r}
  \end{longderivation}

  Así, $f(r) = (\epsilon+1)^r\,g(r) = (\epsilon+1)^r - 1$.
\end{Demo}
Siguiendo ahora con el teorema\dots
\begin{Demo}
  Inicialmente, se expresará la función de masa de $X$ en términos más prácticos para
  esta demostración:
  \begin{longderivation}
      & \dfrac{\binom{K}{x}\binom{N-K}{n-x}}{\binom{N}{n}}\\
    =\\
      & \frac{K!}{(K-x)!\,x!}\,\frac{(N-K)!}{(N-K-n+x)!\,(n-x)!}\,\frac{(N-n)!\,n!}{N!}\\
    =\\
      & \binom{n}{x}\dfrac{\prod_{i=1}^{K}i}{\prod_{i=1}^{K-x}i}
      \dfrac{\prod_{j=1}^{N-K}j}{\prod_{j=1}^{N-K-n+x}\mspace{-20mu}j\mspace{20mu}}
      \dfrac{\prod_{s=1}^{N-n}s}{\prod_{s=1}^{N}s}\\
    =\\
      & \binom{n}{x}\smashoperator[r]{\prod_{i=K-x+1}^{K}}i\hspace{20pt}
      \smashoperator[r]{\prod_{j=N-K-n+x+1}^{N-K}}j\hspace{30pt}
      \dfrac{1}{\prod_{s=N-n+1}^{N}\mspace{-25mu}s\mspace{25mu}}\\
    =\\
      & \binom{n}{x}\prod_{i=0}^{x-1}(K-i)
      \prod_{j=0}^{n-x-1}\mspace{-9mu}(N-K-j)\mspace{9mu}
      \dfrac{1}{\prod_{s=0}^{n-1}(N-s)}\\
    =\\
      &\binom{n}{x}\left(\frac{K}{N}\right)^x\left(\frac{N-K}{N}\right)^{n-x}
      \dfrac{\prod_{i=0}^{x-1}(K-i)}{K^x}
      \dfrac{\prod_{j=0}^{n-x-1}\mspace{-9mu}(N-K-j)\mspace{9mu}}{(N-K)^{n-x}}
      \dfrac{N^n}{\prod_{s=0}^{n-1}(N-s)}\\
    =\\
      & B\left(n,\frac{K}{N}\right)(x)\,\prod_{i=0}^{x-1}\left(1-\frac{i}{K}\right)
      \prod_{j=0}^{n-x-1}\left(1 - \frac{j}{N-K}\right)
      \prod_{s=0}^{n-1}\left(1 + \frac{s}{N}\right)
  \end{longderivation}

  En este proceso no se toma en cuenta el caso en el que $x=0$ o $x=n$. Estos casos se
  resolverán posterior a tratar con la última expresión.

  Tomando en cuenta este resultado,
  \begin{longderivation}<1.5>
      & \left|\dfrac{\binom{K}{x}\binom{N-K}{n-x}}{\binom{N}{n}} -
      B\left(n,\frac{K}{N}\right)(x)\right|\\
    =\\
      & B\left(n,\frac{K}{N}\right)(x)
      \left|\prod_{i=0}^{x-1}\left(1-\frac{i}{K}\right)
      \prod_{j=0}^{n-x-1}\left(1 - \frac{j}{N-K}\right)
      \prod_{s=0}^{n-1}\left(1 + \frac{s}{N}\right) - 1\right|\\
    \why[\leq]{$B(n,p)(x)$ es una probabilidad, con lo que es siempre menor o igual a $1$}\\
    &\left|\prod_{i=0}^{x-1}\left(1-\frac{i}{K}\right)
    \prod_{j=0}^{n-x-1}\left(1 - \frac{j}{N-K}\right)
    \prod_{s=0}^{n-1}\left(1 + \frac{s}{N}\right) - 1\right|
  \end{longderivation}

  Nótese que cada término en cada productorio tiende a $1$ cuando $K,N,N-K$
  tienden a infinito. Con esto basta para demostrar la convergencia.

  En el los productorios se ven involucradas sucesiones las cuales convergen
  a $0$ y además, son sencillas de acotar. Entonces, como
  \begin{align*}
    \left|1 - \frac{i}{K} - 1\right| &= \frac{i}{K} \leq \frac{x-1}{K}\\[10pt]
    \left|1 - \frac{j}{N-K} - 1\right| &= \frac{j}{N-K} \leq \frac{n-x-1}{N-K}\\[10pt]
    \left|1 + \frac{s}{N} - 1\right| &= \frac{s}{N} \leq \frac{n-1}{N}
  \end{align*}
  solo hace falta hallar una cota para los miembros derechos de las desigualdades para
  acotar todos los términos de los productorios.

  Sea $\epsilon>0$ tal que
  \begin{align*}
    \frac{x-1}{K}     &< \epsilon\\
    \frac{n-x-1}{N-K} &< \epsilon\\
    \frac{n-1}{N}     &< \epsilon
  \end{align*}
  Entonces, por el lema,
  \begin{align*}
    \left|\prod_{i=0}^{x-1}\left(1-\frac{i}{K}\right) - 1\right| 
    &< (\epsilon + 1)^x - 1\\[10pt]
    \left|\prod_{j=0}^{n-x-1}\left(1 - \frac{j}{N-K}\right) - 1\right|
    &< (\epsilon + 1)^{n-x} - 1\\[10pt]
    \left|\prod_{s=0}^{n-1}\left(1 + \frac{s}{N}\right) - 1\right|
    &< (\epsilon + 1)^n - 1
  \end{align*}

  Denotando cada uno de estos productos como $P_1$, $P_2$ y $P_3$ respectivamente,

  \begin{longderivation}
      & |P_1\,P_2\,P_3 - 1|\\
    =\\
      & |(P_1 - 1)(P_2\,P_3 - 1) + P_1 - 1 + P_2\,P_3 - 1|\\
    \leq\\
      & |P_1 - 1||P_2\,P_3 - 1| + |P_1 - 1| + |P_2\,P_3 - 1|\\
    =\\
      & |P_2\,P_3 - 1|(|P_1 - 1| + 1) + |P_1 - 1|\\
    =\\
      & |(P_2 - 1)(P_3 - 1) + P_2 - 1 + P_3 - 1|
      (|P_1 - 1| + 1) + |P_1 - 1|\\
    \leq\\
      &(|P_2 - 1||P_3 - 1| + |P_2 - 1| + |P_3 - 1|)(|P_1 - 1| + 1) + |P_1 - 1|\\
    <\\
      &[
        ((\epsilon + 1)^{n-x} - 1)((\epsilon + 1)^n - 1) +
        (\epsilon + 1)^{n-x} - 1 + (\epsilon + 1)^n - 1
      ]((\epsilon + 1)^x - 1 + 1) + (\epsilon + 1)^x - 1\\
    =\\
      & [
        (\epsilon + 1)^{2n-x} - (\epsilon + 1)^{n-x} - (\epsilon + 1)^n + 1
        + (\epsilon + 1)^{n-x} - 1 + (\epsilon + 1)^n - 1
      ](\epsilon + 1)^x + (\epsilon + 1)^x - 1\\
    =\\
      & [(\epsilon + 1)^{2n-x} - 1](\epsilon + 1)^x + (\epsilon + 1)^x - 1\\
    =\\
      & (\epsilon + 1)^{2n} - 1
  \end{longderivation}
  
  Recordando que todo lo anterior se hizo bajo la suposición de que $x > 0$ y
  $ x \not= n$. Para $x = 0$
  \begin{longderivation}
      &\left|
        \dfrac{\binom{K}{x}\binom{N-K}{n-x}}{\binom{N}{n}}
        - \binom{n}{x}\left(\frac{K}{N}\right)^x\left(\frac{N-K}{N}\right)^{n-x}
      \right|\\
    =\\
      &\left|
        \dfrac{\binom{N-K}{n}}{\binom{N}{n}}
        - \left(\frac{N-K}{N}\right)^n
      \right|\\
    =\\
      & B\left(n,\frac{K}{N}\right)(0)\left|
        \prod_{j=0}^{n-1}\left(1 - \frac{j}{N-K}\right)
        \prod_{s=0}^{n-1}\left(1 + \frac{s}{N}\right) - 1
      \right|\\
    \leq\\
      & \left|\prod_{j=0}^{n-1}\left(1 - \frac{j}{N-K}\right) - 1\right|
      \left|\prod_{s=0}^{n-1}\left(1 + \frac{s}{N}\right) - 1\right| + 
      \left|\prod_{j=0}^{n-1}\left(1 - \frac{j}{N-K}\right) - 1\right| +
      \left|\prod_{s=0}^{n-1}\left(1 + \frac{s}{N}\right) - 1\right|\\
    <\\
      & [(\epsilon + 1)^n - 1](\epsilon + 1)^n + (\epsilon + 1)^n - 1\\
    =\\
      & (\epsilon + 1)^{2n} - 1
  \end{longderivation}
  Para $x=n$
  \begin{longderivation}
      &\left|
          \dfrac{\binom{K}{x}\binom{N-K}{n-x}}{\binom{N}{n}}
          - \binom{n}{x}\left(\frac{K}{N}\right)^x\left(\frac{N-K}{N}\right)^{n-x}
        \right|\\
    =\\
      &\left|
        \dfrac{\binom{K}{n}}{\binom{N}{n}} - \left(\frac{K}{N}\right)^n
      \right|\\
    =\\
      & B\left(n,\frac{K}{N}\right)(n)
      \left|
        \prod_{i=0}^{n-1}\left(1 - \frac{i}{K}\right)
        \prod_{s=0}^{n-1}\left(1 + \frac{s}{N}\right) - 1
      \right|\\
    \leq\\
        & \left|\prod_{i=0}^{n-1}\left(1 - \frac{i}{K}\right) - 1\right|
        \left|\prod_{s=0}^{n-1}\left(1 + \frac{s}{N}\right) - 1\right| +
        \left|\prod_{i=0}^{n-1}\left(1 - \frac{i}{K}\right) - 1\right| +
        \left|\prod_{s=0}^{n-1}\left(1 + \frac{s}{N}\right) - 1\right|\\
    <\\
      & [(\epsilon + 1)^n - 1](\epsilon + 1)^n + (\epsilon + 1)^n - 1\\
    =\\
      & (\epsilon + 1)^{2n} - 1 
  \end{longderivation}

  Para $n=1$ o $n=0$, la diferencia presentada es nula.
\end{Demo}