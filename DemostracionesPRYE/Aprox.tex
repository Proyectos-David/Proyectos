\subsection{Teoremas de Aproximación}
Se puede ver una similitud entre la \hyperref[dist:binom]{distribución binomial}
y la \hyperref[dist:hip]{distribución hipergeométrica}, pues si en esta última, manteniendo
un tamaño de muestra ($n$) fijo, a medida que aumentan el total de objetos ($N$ y $K$) bajo
ciertas condiciones, los eventos que esta distribución describe tienen a ser independientes.
Esto lleva al siguiente teorema de aproximación.

\begin{Teo}
  Sea $X$ una variable aleatoria con distribución hipergeométrica de
  parámetros $N,K,n$. Si para $\epsilon_1,\epsilon_2,\epsilon_3\in\R^+$ , $n > 1$ , $x > 0$,
  se tiene que
  \begin{align*}
    \frac{x-1}{K}         &< \epsilon_1\\
    \frac{n-x-1}{N-K}     &< \epsilon_2\\
    \frac{n-1}{N - n + 1} &< \epsilon_3
  \end{align*}
  entonces,
  % Falta revisar cómo queda
  \[
    \left|\dfrac{Hg(N,K,n)(x)}{B\left(n,\frac{K}{N}\right)(x)} - 1\right| 
    < (\epsilon + 1)^{2n} - 1
  \]
  Nótese que la expresión en valor absoluto corresponde al error entre la
  función de masa de una distribución hipergeométrica y una binomial con ciertos
  parámetros.
\end{Teo}

Antes de comenzar con la demostración de este teorema, se presenta el siguiente lema, el
cual será de utilidad para obtener el resultado presentado.
\begin{Lema}
  Sean $r\in\Z^+$, $\left\{S_{k,n}\right\}_{1}^{r}$ una colección de $r$ sucesiones
  en función de $n$ las cuales convergen a $1$ y $\{\epsilon_k\}_1^r$ una colección
  de $r$ reales positivos. Si para un $N\in\N$, se tiene que
  \[1\leq k\leq r\quad\land\quad n\geq N \quad\To\quad |S_{k,n} - 1| \leq \epsilon_k\]
  entonces,
  \[n\geq N \To \left|\prod_{k=1}^r S_{k,n} -1\right| < \prod_{k=1}^r(\epsilon_k + 1) - 1\]
\end{Lema}
\begin{Demo}
  Supóngase la existencia de este $N$. Tomando una colección con $r+1$
  sucesiones con una colección respectiva de cotas $\{\epsilon_k\}_1^{r+1}$ para la diferencia
  de cada una con $1$, se tiene lo siguiente
  \begin{longderivation}
    & \left|\prod_{k=1}^{r+1}S_{k,n} - 1\right|\\
  =\\
    & \left|S_{r+1,n}\prod_{k=1}^{r}S_{k,n} - 1\right|\\
  =\\
    & \left|
      (S_{r+1,n} - 1)\left(\prod_{k=1}^{r}S_{k,n} - 1\right)
      + (S_{r+1,n} - 1)
      + \left(\prod_{k=1}^{r}S_{k,n} - 1\right)
    \right|\\
  \leq\\
    &\left|(S_{r+1,n} - 1)\left(\prod_{k=1}^{r}S_{k,n} - 1\right)\right|
    + \left|(S_{r+1,n} - 1)\right|
    + \left|\left(\prod_{k=1}^{r}S_{k,n} - 1\right)\right|\\
  <\\
    & \left|\prod_{k=1}^{r}S_{k,n} - 1\right|(\epsilon_{r+1}+1) + \epsilon_{r+1}
  \end{longderivation}

  Se define entonces la siguiente función recursiva
  \begin{align*}
    f(1) &= \epsilon_1\\
    f(n+1) &= f(n)(\epsilon_{n+1} + 1) + \epsilon_{n+1}
  \end{align*}

  Por el procedimiento anterior, es fácil ver que esta función cumple acotar la diferencia
  del producto de $n$ sucesiones y $1$ con las condiciones del enunciado. Se afirma que
  \[f(n) = \prod_{k=1}^n(\epsilon_k + 1) - 1\]
  Caso base: $n=1$, efectivamente $f(1) = \epsilon_1$. Para $n=2$, por definición
  \[f(2) = \epsilon_1(\epsilon_2+1)+\epsilon_2 = (\epsilon_1+1)(\epsilon_2+1)-1\]
  Paso inductivo: supóngase que, para algún $n\geq2$,
  \[f(n) = \prod_{k=1}^n(\epsilon_k + 1) - 1\]
  Entonces,
  \begin{longderivation}
      & f(n+1)\\
    =\\
      & f(n)(\epsilon_{n+1} + 1) + \epsilon_{n+1}\\
    =\\
      & \left(\prod_{k=1}^n(\epsilon_k + 1) - 1\right)(\epsilon_{n+1} + 1)
      + \epsilon_{n+1}\\
    =\\
      & \prod_{k=1}^{n+1}(\epsilon_k + 1) - \epsilon_{n+1} - 1 + \epsilon_{n+1}\\
    =\\
      & \prod_{k=1}^{n+1}(\epsilon_k + 1) - 1
  \end{longderivation}
  Con lo que
  \[n\geq N \To \left|\prod_{k=1}^r S_{k,n} -1\right| < \prod_{k=1}^r(\epsilon_k + 1) - 1\]
\end{Demo}
Siguiendo ahora con el teorema\dots
\begin{Demo}
  Inicialmente, se expresará la función de masa de $X$ en otros términos
  \begin{longderivation}
      & \dfrac{\binom{K}{x}\binom{N-K}{n-x}}{\binom{N}{n}}\\
    =\\
      & \frac{K!}{(K-x)!\,x!}\,\frac{(N-K)!}{(N-K-n+x)!\,(n-x)!}\,\frac{(N-n)!\,n!}{N!}\\
    =\\
      & \binom{n}{x}\dfrac{\prod_{i=1}^{K}i}{\prod_{i=1}^{K-x}i}
      \dfrac{\prod_{j=1}^{N-K}j}{\prod_{j=1}^{N-K-n+x}\mspace{-20mu}j\mspace{20mu}}
      \dfrac{\prod_{s=1}^{N-n}s}{\prod_{s=1}^{N}s}\\
    =\\
      & \binom{n}{x}\smashoperator[r]{\prod_{i=K-x+1}^{K}}i\mspace{20mu}
      \smashoperator[r]{\prod_{j=N-K-n+x+1}^{N-K}}j\mspace{35mu}
      \dfrac{1}{\smashoperator[r]{\prod_{s=N-n+1}^{N}}s\mspace{20mu}}\\
    =\\
      & \binom{n}{x}\prod_{i=0}^{x-1}(K-i)
      \smashoperator{\prod_{j=0}^{n-x-1}}(N-K-j)
      \dfrac{1}{\prod_{s=0}^{n-1}(N-s)}\\
    =\\
      &\binom{n}{x}\left(\frac{K}{N}\right)^x\left(\frac{N-K}{N}\right)^{n-x}
      \dfrac{\prod_{i=0}^{x-1}(K-i)}{K^x}
      \dfrac{\smashoperator[r]{\prod_{j=0}^{n-x-1}}(N-K-j)}{(N-K)^{n-x}}
      \dfrac{N^n}{\prod_{s=0}^{n-1}(N-s)}\\
    =\\
      & B\left(n,\frac{K}{N}\right)(x)\,\prod_{i=0}^{x-1}\left(1-\frac{i}{K}\right)
      \prod_{j=0}^{n-x-1}\left(1 - \frac{j}{N-K}\right)
      \prod_{s=0}^{n-1}\left(1 + \frac{s}{N-s}\right)
  \end{longderivation}

  En este proceso no se toma en cuenta el caso en el que $x=0$ o $x=n$. Estos casos se
  resolverán posterior a tratar con la última expresión.

  Tomando en cuenta este resultado,
  \begin{longderivation}<1.5>
      & \left|\dfrac{Hg(N,K,n)(x)}{B\left(n,\frac{K}{N}\right)(x)} - 1\right|\\
    =\\
      & \left|\prod_{i=0}^{x-1}\left(1-\frac{i}{K}\right)
      \prod_{j=0}^{n-x-1}\left(1 - \frac{j}{N-K}\right)
      \prod_{s=0}^{n-1}\left(1 + \frac{s}{N-s}\right) - 1\right|
  \end{longderivation}

  Nótese que cada término en cada productorio tiende a $1$ cuando $K,N,N-K$
  tienden a infinito. Con esto basta para demostrar la convergencia, debido a que
  estas condiciones de tendencia para $N,K$ y $N-K$ se deben a que
  $\frac{K}{N}$ debe ser un número entre $0$ y $1$.

  En los productorios, se ven involucradas sucesiones las cuales convergen
  a $0$ y además, son sencillas de acotar. Entonces, como
  \begin{align*}
    \left|1 - \frac{i}{K} - 1\right| &= \frac{i}{K} \leq \frac{x-1}{K}\\[10pt]
    \left|1 - \frac{j}{N-K} - 1\right| &= \frac{j}{N-K} \leq \frac{n-x-1}{N-K}\\[10pt]
    \left|1 + \frac{s}{N-s} - 1\right| &= \frac{s}{N-s} \leq \frac{n-1}{N-n+1}
  \end{align*}
  
  Dado que las expresiones a la derecha de cada desigualdad representan sucesiones
  decrecientes en función de $N$, $K$ y $N-K$ respectivamente, se tiene que, si para
  valores de estas variables, se toman $\epsilon_1, \epsilon_2, \epsilon_3\in\R^+$
  tales que
  \begin{align*}
    \frac{x-1}{K}     &< \epsilon_1\\
    \frac{n-x-1}{N-K} &< \epsilon_2\\
    \frac{n-1}{N-n+1} &< \epsilon_3
  \end{align*}
  Entonces, por el lema,
  \begin{align*}
    \left|\prod_{i=0}^{x-1}\left(1-\frac{i}{K}\right) - 1\right| 
    &< (\epsilon_1 + 1)^x - 1\\[10pt]
    \left|\prod_{j=0}^{n-x-1}\left(1 - \frac{j}{N-K}\right) - 1\right|
    &< (\epsilon_2 + 1)^{n-x} - 1\\[10pt]
    \left|\prod_{s=0}^{n-1}\left(1 + \frac{s}{N-s}\right) - 1\right|
    &< (\epsilon_3 + 1)^n - 1
  \end{align*}

  Denotando cada uno de estos productos como $P_1$, $P_2$ y $P_3$ respectivamente,
  aplicando nuevamente el lema, se obtiene que
  \[
    |P_1\,P_2\,P_3 - 1|<
    (\epsilon_1+1)^x(\epsilon_2+1)^{n-x}(\epsilon_3+1)^n - 1
  \]
  Recordando que todo lo anterior se hizo bajo la suposición de que $x > 0$ y
  $ x \not= n$. Para $x = 0$
  \begin{longderivation}
      &\left|
        \dfrac{Hg(N,K,n)(0)}{B\left(n,\frac{K}{N}\right)(0)}-1
      \right|\\
    =\\
      &\left|
        \dfrac{\binom{N-K}{n}}{\binom{N}{n}}
        \left(\frac{N-K}{N}\right)^{-n}
        -1
      \right|\\
    =\\
      & \left|
        \prod_{j=0}^{n-1}\left(1 - \frac{j}{N-K}\right)
        \prod_{s=0}^{n-1}\left(1 + \frac{s}{N-s}\right) - 1
      \right|\\
    \why[<]{Aplicando el lema únicamente para las suceciones en estos productorios}\\
      & (\epsilon_2+1)^n(\epsilon_3+1)^n-1
  \end{longderivation}
  Para $x=n$
  \begin{longderivation}
      &\left|
        \dfrac{Hg(N,K,n)(n)}{B\left(n,\frac{K}{N}\right)(n)}-1
      \right|\\
    =\\
      &\left|
        \dfrac{\binom{K}{n}}{\binom{N}{n}}\left(\frac{K}{N}\right)^{-n}
        -1
      \right|\\
    =\\
      &\left|
        \prod_{i=0}^{n-1}\left(1 - \frac{i}{K}\right)
        \prod_{s=0}^{n-1}\left(1 + \frac{s}{N-s}\right) - 1
      \right|\\
    \why[<]{Aplicando el lema únicamente para las sucesiones en estos productos}\\
      & (\epsilon_1+1)^n(\epsilon_3+1)^n-1
  \end{longderivation}
  Para $n=1$ o $n=0$, el error es nulo.
\end{Demo}