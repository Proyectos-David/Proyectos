\documentclass{beamer}

\usepackage[arrows]{logicDG}
\usepackage{calcDG}
\usepackage{analysis}
\usepackage{nicefrac}
\usepackage{amsfonts}
\usepackage{upgreek}
\usepackage{mathrsfs}
\usepackage{thmtools}
\usepackage{marvosym}
\usepackage[spanish, es-noquoting, es-lcroman, es-noshorthands]{babel}
\usetheme{Warsaw}
\setbeamercovered{transparent}


\title{Demostraciones de PRYE}
\author[David G., Laura R.]{David Gómez, Laura Rincón}
\date[15/11/2024]{15 de Noviembre de 2024}

\begin{document}
  \begin{frame}
    \titlepage
  \end{frame}

  \section{Probabilidad}
  \begin{frame}{Probabilidad de una unión finita}
    \only<1>{La probabilidad de una unión finita está dada por
    \begin{align*}
      P\left(A_1\cup A_2\right) &= P(A_1) + P(A_2) - P(A_1\cap A_2)\\
      P\left(\bigcup_{i=1}^{n+1}A_i\right) &=
      P(A_{n+1}) + P\left(\bigcup_{i=1}^nA_i\right)
      - P\left(\bigcup_{i=1}^n(A_{n+1}\cap A_i)\right)
    \end{align*}}
    \only<2>{
      \begin{longderivation}
        & P\left(\bigcup_{i=1}^nA_i\right)\\
        =\\
        & \smashoperator[r]{\sum_{1\leq i\leq n}}P(A_i)
        -\smashoperator[r]{\sum_{1\leq i<j\leq n}}P(A_i\cap A_j) \\
        & + \smashoperator[r]{\sum_{1\leq i<j<k\leq n}}P(A_i\cap A_j\cap A_k)
        -\dots+(-1)^nP\left(\bigcap_{i=1}^n A_i\right)
      \end{longderivation}
      \begin{center}
        \Coffeecup
      \end{center}
    }
  \end{frame}
  \begin{frame}{Sobre eventos independientes}
    Dos eventos, $A$ y $B$, de un espacio muestral, son llamados
    independientes cuando $P(A\cap B)=P(A)P(B)$.

    \onslide<2->{
      Si $A$ y $B$ son independientes, entonces
      $A^c$ y $B$ son independientes;
      $A^c y B^c$ son independientes.
    }
    \only<4>{
      \begin{longderivation}<1>
          & {P(A^c\cap B^c)}\\
      =\\
        & {1 - P(A\cup B)}\\
      =\\
        & {1 - [P(A) + P(B) - P(A\cap B)]}\\
      =\\
        & {(1-P(A))(1-P(B))}\\
      =\\
        & {P(A^c)\,P(B^c)}
      \end{longderivation}
      }
      \only<3>{
        \\Usando la igualdad $P(B) = P(B)\,P(A) + P(B\cap A^c)$
        \begin{longderivation}<1>
            & {P(B\cap A^c)}\\
          =\\
            & {P(B) - P(B)P(A)}\\
          =\\
            & {P(B)(1-P(A))}\\
          =\\
            & {P(B)\,P(A^c)}
        \end{longderivation}
        }
  \end{frame}
  \section{Varianza y Valor esperado}

  \section{Distribuciones Discretas}
  \begin{frame}{Binomial}
    \[X\sim B(n,p)\]
    \[f(x)=P(X=x)=\binom{n}{x}p^x(1-p)^{n-x}\qquad(0\leq x\leq n)\]
    \onslide<2>{
      \begin{enumerate}
        \item Para todo $x\in\Z$ con $0\leq x\leq n$, $P(X=x) \geq 0$.
        \item $\sum_{x\in\Z}P(X=x) = 1$.
        \item $\text{E}[X] = np$.
        \item $\text{Var}[X]=np(1-p)$.
      \end{enumerate}
    }
  \end{frame}

  \section{Distribuciones Continuas}

  \section{Teoremas de Aproximación}
\end{document}