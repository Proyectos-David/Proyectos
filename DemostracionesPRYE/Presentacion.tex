\documentclass{beamer}

\usepackage[arrows]{logicDG}
\usepackage{calcDG}
\usepackage{analysis}
\usepackage{nicefrac}
\usepackage{amsfonts}
\usepackage{upgreek}
\usepackage{mathrsfs}
\usepackage{thmtools}
\usepackage{marvosym}
\usepackage{letltxmacro}
\usepackage[spanish, es-noquoting, es-lcroman, es-noshorthands]{babel}
\usetheme{Warsaw}
\setbeamercovered{transparent}

\makeatletter
\let\oldr@@t\r@@t
\def\r@@t#1#2{%
\setbox0=\hbox{$\oldr@@t#1{#2\,}$}\dimen0=\ht0
\advance\dimen0-0.2\ht0
\setbox2=\hbox{\vrule height\ht0 depth -\dimen0}%
{\box0\lower0.4pt\box2}}
\LetLtxMacro{\oldsqrt}{\sqrt}
\renewcommand*{\sqrt}[2][\ ]{\oldsqrt[#1]{#2}}
\makeatother
\renewcommand{\epsilon}{\upvarepsilon}

\title{Demostraciones de PRYE}
\author[David G., Laura R.]{David Gómez, Laura Rincón}
\date[15/11/2024]{15 de Noviembre de 2024}

\begin{document}
  \begin{frame}
    \titlepage
  \end{frame}

  \section{Probabilidad}
  \begin{frame}{Probabilidad de una unión finita}
    \only<1>{La probabilidad de una unión finita está dada por
    \begin{align*}
      P\left(A_1\cup A_2\right) &= P(A_1) + P(A_2) - P(A_1\cap A_2)\\
      P\left(\bigcup_{i=1}^{n+1}A_i\right) &=
      P(A_{n+1}) + P\left(\bigcup_{i=1}^nA_i\right)
      - P\left(\bigcup_{i=1}^n(A_{n+1}\cap A_i)\right)
    \end{align*}}
    \only<2>{
      \begin{longderivation}
        & P\left(\bigcup_{i=1}^nA_i\right)\\
        =\\
        & \smashoperator[r]{\sum_{1\leq i\leq n}}P(A_i)
        -\smashoperator[r]{\sum_{1\leq i<j\leq n}}P(A_i\cap A_j) \\
        & + \smashoperator[r]{\sum_{1\leq i<j<k\leq n}}P(A_i\cap A_j\cap A_k)
        -\dots+(-1)^nP\left(\bigcap_{i=1}^n A_i\right)
      \end{longderivation}
      \begin{center}
        \Coffeecup
      \end{center}
    }
  \end{frame}
  \begin{frame}{Sobre eventos independientes}
    Dos eventos, $A$ y $B$, de un espacio muestral, son llamados
    independientes cuando $P(A\cap B)=P(A)P(B)$.

    \onslide<2->{
      Si $A$ y $B$ son independientes, entonces
      $A^c$ y $B$ son independientes;
      $A^c y B^c$ son independientes.
    }
    \only<4>{
      \begin{longderivation}<1>
          & {P(A^c\cap B^c)}\\
      =\\
        & {1 - P(A\cup B)}\\
      =\\
        & {1 - [P(A) + P(B) - P(A\cap B)]}\\
      =\\
        & {(1-P(A))(1-P(B))}\\
      =\\
        & {P(A^c)\,P(B^c)}
      \end{longderivation}
      }
      \only<3>{
        \\Usando la igualdad $P(B) = P(B)\,P(A) + P(B\cap A^c)$
        \begin{longderivation}<1>
            & {P(B\cap A^c)}\\
          =\\
            & {P(B) - P(B)P(A)}\\
          =\\
            & {P(B)(1-P(A))}\\
          =\\
            & {P(B)\,P(A^c)}
        \end{longderivation}
        }
  \end{frame}
  \section{Varianza y Valor esperado}

  \section{Distribuciones Discretas}
  \begin{frame}{Binomial}
    \only<1-2>{
      \[X\sim B(n,p)\]
    }
    \only<3>{
      \[(x+y)^n = \sum_{k=0}^n\binom{n}{k}x^ky^{n-k}\]
      \begin{center}
        \Coffeecup
      \end{center}
    }
    \[f(x)=P(X=x)=\binom{n}{x}p^x(1-p)^{n-x}\qquad(0\leq x\leq n)\]
    \onslide<2>{
      \begin{enumerate}
        \item Para todo $x\in\Z$ con $0\leq x\leq n$, $P(X=x) \geq 0$.
        \item $\sum_{x=0}^nP(X=x) = 1$.
        \item $\text{E}[X] = np$.
        \item $\text{Var}[X]=np(1-p)$.
      \end{enumerate}
    }
  \end{frame}
  \begin{frame}{Geométrica}
    \only<1-2>{
      \[X\sim\text{Geom}(p)\]
    }
    \only<3>{
      \[\sum_{k=0}^np^k = \frac{1-p^{n+1}}{1-p}\]
      \begin{center}
        \Coffeecup
      \end{center}
    }
    \[f(x)=P(X=x)=p(1-p)^{x-1}\qquad(x\in\Z^+)\]
    \onslide<2>{
      \begin{enumerate}
        \item Para todo $x\in\Z^+$, $P(X=x)\geq0$.
        \item $\sum_{x\in\Z^+}P(X=x)=1$.
        \item $\text{E}[X]=\frac{1}{p}$.
        \item $\text{Var}[X]=\frac{1-p}{p^2}$.
      \end{enumerate}
    }
  \end{frame}
  
  \begin{frame}{Hipergeométrica}
    \only<1-2>{
      \[X\sim Hg(N,K,n)\]
    }
    \only<3>{
      \[\binom{m+n}{k}=\sum_{r=0}^k\binom{m}{r}\binom{n}{k-r}\]
      \begin{center}
        \Coffeecup
      \end{center}
    }
    \begin{align*}
      f(x)=P(X=x)=\dfrac{\binom{K}{x}\binom{N-K}{n-x}}{\binom{N}{n}}\\
      (\max\{0,n+K-N\} \leq x \leq \min\{K,n\})
    \end{align*}
    \onslide<2>{
      \begin{enumerate}
        \item para todo $x\in\Z$ con $0\leq x\leq n$, $P(X=x) \geq 0$.
        \item $\sum_{x\in\Z}P(X=x) = 1$.
        \item $\text{E}[X]=\frac{nK}{N}$.
        \item $\text{Var}[X] = \frac{n\,K(N-K)(N-n)}{N^2(N-1)}$.
      \end{enumerate}
    }
  \end{frame}
  \begin{frame}{Poisson}
    \only<1-2>{
      \[X\sim\text{Pois}(\lambda)\]
      \begin{center}
        \Coffeecup
      \end{center}
    }
    \only<3>{
      \[e^x=\sum_{n=0}^{\infty}\frac{x^n}{n!}\]
      \begin{center}
        \Coffeecup
      \end{center}
      }
      \[f(x)=P(X=x)=\frac{\lambda^xe^{-\lambda}}{x!}\qquad(x\in\N)\]
    \onslide<2>{
      \begin{enumerate}
        \item Para todo $x\in\N$, $P(X=x)\geq0$.
        \item $\sum_{x\in\N}P(X=x)=1$.
        \item $\text{E}[X] = \lambda$.
        \item $\text{Var}[X]=\lambda$.
      \end{enumerate}
    }
  \end{frame}
\section{Distribuciones Continuas}
  \begin{frame}{Normal}
    \only<1-2>{
      \[X\sim N(\mu,\sigma^2)\]
    }
    \only<3>{
      \[\int_\R e^{-x^2}\diff{x} = \sqrt{\pi}\]
      \begin{center}
        \Coffeecup
      \end{center}
    }
  \[
    f(x) = \dfrac{1}{\sqrt{2\pi\sigma^2}}
    e^{-\nicefrac{(x-\mu)^2}{\left(2\sigma^2\right)}}
    \qquad (x\in\R)
  \]
    \onslide<2>{
      \begin{enumerate}
        \item Para todo $x\in\R$, $f(x) \geq 0$.
        \item $\int_{\R}{f(x)}\diff{x} = 1$.
        \item $\text{E}[X] = \mu$.
        \item $\text{Var}[X] = \sigma^2$.
      \end{enumerate}
    }
  \end{frame}
  \begin{frame}{Gamma}
    \only<1-2>{
      \[X\sim\text{Gamma}(\alpha,\beta)\]
    }
    \only<3>{
      \[\Gamma(\alpha) = \int_0^{\infty}t^{\alpha-1}e^{-t}\diff{t}\]
      \begin{center}
        \Coffeecup
      \end{center}
    }
    \[
    f(x)=\frac{\beta^{\alpha}}{\Gamma(\alpha)}x^{\alpha-1}e^{-\beta x}
    \qquad (x\in\R^+)
    \]
    \onslide<2>{
      \begin{enumerate}
        \item Para todo $x\in\R$, $f(x) \geq 0$.
        \item $\int_{\R}f(x)\mathrm{d}x=1$.
        \item $\text{E}[X]=\frac{\alpha}{\beta}$
        \item $\text{Var}[X]=\frac{\alpha}{\beta^2}$
      \end{enumerate}
    }
  \end{frame}
  \begin{frame}{Chi-Cuadrado}
    \only<1-2>{
      \[X\sim\chi^2(v)\]
    }
    \only<3>{
      \[\Gamma(x+1)=x\Gamma(x)\]
      \begin{center}
        \Coffeecup
      \end{center}
    }
    \[
        f(x) = \frac{1}{2^{\nicefrac{v}{2}}\Gamma\left(\frac{v}{2}\right)}
        x^{\left(\nicefrac{v}{2}\right)-1}e^{\nicefrac{-x}{2}} \qquad(x\geq0)
    \]
    \onslide<2>{
      \begin{enumerate}
        \item para todo $x\in \R$, $f(x)\geq0$
        \item $\int_{-\infty}^{\infty}f(x)\diff{x}\geq0$
        \item $\text{E}[X]= v \qquad(v\geq0)$
        \item $\text{Var}[X]= 2v \qquad(v\geq0)$
      \end{enumerate}
    }
  \end{frame}
  \begin{frame}{f}
    \only<1-3>{
      \[F\sim\mathbf{f}(u,v)\]
    }
    \only<4>{
      \[B(x,y)=\frac{\Gamma(x)\Gamma(y)}{\Gamma(x+y)}\]
      \begin{center}
        \Coffeecup
      \end{center}
    }
    \only<1>{
      \[f(x)=
      \dfrac{\Gamma\left(\frac{u+v}{2}\right)\left(\frac{u}{v}\right)^{\nicefrac{u}{2}}}
      {\Gamma\left(\frac{u}{2}\right)\Gamma\left(\frac{v}{2}\right)}
      \dfrac{x^{(\nicefrac{u}{2})-1}}
      {\left(1 + \frac{u}{v}x\right)^{\nicefrac{(u+v)}{2}}}
      \qquad (x\in\R^+)\]
    }
    \only<2->{
      \[f(x)=
      \dfrac{\left(\frac{u}{v}\right)^{\nicefrac{u}{2}}}
      {B\left(\frac{u}{2},\frac{v}{2}\right)}
      \dfrac{x^{(\nicefrac{u}{2})-1}}
      {\left(1 + \frac{u}{v}x\right)^{\nicefrac{(u+v)}{2}}}
      \qquad (x\in\R^+)
      \]
    }
    \onslide<3>{
      \begin{enumerate}
        \item Para todo $x\in\R$, $f(x) \geq 0$.
        \item $\int_{\R}f(x)\mathrm{d}x=1$.
        \item $\text{E}[F]=\frac{v}{v-2} \qquad(v>2)$
        \item $\text{Var}[F]=\frac{2v^2(u+v-2)}{u(v-2)^2(v-4)}\qquad(v>4)$
      \end{enumerate}
    }
  \end{frame}

\section{Teoremas de Aproximación}
\begin{frame}{Hipergeométrica a Binomial}
  Sea $X\sim Hg(N,K,n)$. Para $n$ fijo, si $N,K$ cumplen que
  \begin{align*}
    \frac{x-1}{K}         &< \epsilon_1\\
    \frac{n-x-1}{N-K}     &< \epsilon_2\\
    \frac{n-1}{N - n + 1} &< \epsilon_3
  \end{align*}
  Entonces,
  \[
    \left|\dfrac{Hg(N,K,n)(x)}{B\left(n,\frac{K}{N}\right)(x)} - 1\right| 
    < (\epsilon_1 + 1)^{x}(\epsilon_2 + 1)^{n-x}(\epsilon_3 + 1)^{n} - 1
    \qquad\text{\Coffeecup}
  \]
\end{frame}
\begin{frame}{Teorema Central del Límite}
  Sea $\{X_n\}_{n\in\Z^+}$ una colección de variables aleatorias independientes,
  igualmente distribuidas, con media y varianza iguales ($\mu$ y $\sigma$
  respectivamente). Entonces, cuando $n\longrightarrow\infty$,
  \[\sqrt{n}\,\overline{X}_n\longrightarrow N(0,1)\]
  \only<2>{
    \begin{center}
      \begin{tabular}{ccc}
        $
        \begin{matrix*}
          |x_n| &\leq M\\
          |y_n| &\leq M
        \end{matrix*}
        $
        &
        $\To$
        &
        \(
          \left|\prod_{k=1}^n x_k - \prod_{j=1}^n y_j\right| \leq 
          M^{n-1}\sum_{k=1}^n |x_k - y_k|
        \)
      \end{tabular}
    \end{center}
    \begin{center}
      \Coffeecup
    \end{center}
  }
\end{frame}
\end{document}