\subsection{Distribuciones Discretas}
\subsubsection{Binomial}
Supóngase se realiza un experimento el cual tiene como posible
resultado $a$ o $b$ exclusivamente, y además, el resultado
de realizar nuevamente el experimento es independiente al
resultado anterior. Dado que $a$ y $b$ son los únicos resultados,
para un único experimento, se debe tener que
$P(a) = 1 - P(b)$. Sea $p=P(a)$. Supóngase que
este experimento es realizado $n$ veces. Se define una variable
aleatoria $X$ correspondiente a la cantidad de ocurrencias de $a$.
Entonces
\[P(X=x) = \binom{n}{x}p^x(1-p)^{n-x}\]  
\begin{Def}
  Sea $X$ una variable aleatoria discreta. $X$ sigue una distribución
  binomial con parámetros $n$ y $p$ ($n\in\Z^+, p\in[0,1]$), denotada por $B(n,p)$,
  cuando su función de masa es
  \[f(x) = P(X=x) = B(n,p)(x) = \binom{n}{x}p^x(1-p)^{n-x}\]
\end{Def}

\begin{Teo}
  Sea $X$ una variable aleatoria la cual sigue una distribución
  binomial $B(n,p)$. Entonces
  \begin{enumerate}
    \item $E(X) = np$
    \item $\text{Var}(X)=np(1-p)$
  \end{enumerate}
\end{Teo}
\begin{Demo}
  Desarrollando ambos valores, y recordando el teorema del binomio\dots
  \begin{enumerate}
    \item~
    \begin{longderivation}<0.8>
        \wff{E(X)}\\
      =\\
        \wff{\sum_{x=0}^nx\binom{n}{x}p^x(1-p)^{n-x}}\\
      =\\
        \wff{np\sum_{x=1}^n\frac{(n-1)!}{(n-k)!\,(x-1)!}p^x(1-p)^{n-x}}\\
      =\\
        \wff{np\sum_{x=0}^{n-1}\binom{n-1}{x}p^x(1-p)^{n-1-x}}\\
      =\\
        \wff{np\,(p+1-p)^{n-1}}\\
      =\\
        \wff{np}
    \end{longderivation}
    \item~
    \begin{longderivation}<0.8>
        \wff{\text{Var}(X)}\\
      =\\
        \wff{E(X^2) - E^2(X)}\\
      =\\
        \wff{\sum_{x=0}^nx^2\binom{n}{x}p^x(1-p)^{n-x} - (np)^2}\\
      =\\
        \wff{np\left[
          \sum_{x=1}^nx\binom{n-1}{x-1}p^{x-1}(1-p)^{n-x} - np
        \right]}\\
      =\\
        \wff{np\left[
          \sum_{x=0}^{n-1}(x+1)\binom{n-1}{x}p^x(1-p)^{n-1-x} - np
        \right]}\\
      =\\
        \wff{np\left[
          \sum_{x=0}^{n-1}x\binom{n-1}{x}p^x(1-p)^{n-1-x}
          +\sum_{x=0}^{n-1}\binom{n-1}{x}p^x(1-p)^{n-1-x}
          -np
        \right]}\\
      =\\
        \wff{np\left[
          p(n-1)\sum_{x=1}^{n-1}\binom{n-2}{x-1}p^{x-1}(1-p)^{n-1-x}+1-np
        \right]}\\
      =\\
        \wff{np\left[
          p(n-1)\sum_{x=0}^{n-2}\binom{n-2}{x}p^x(1-p)^{n-2-x}+1-np
        \right]}\\
      =\\
        \wff{np[p(n-1)+1-np]}\\
      =\\
        \wff{np(1-p)}
    \end{longderivation}
    Este argumento es válido siempre que $n\geq 2$. Si $n<2$, entonces $n=1$ y
    \begin{longderivation}<0.8>
        & \text{Var}(X)\\
      =\\
        & E(X^2) - E^2(X)\\
      =\\
        & \sum_{x=0}^1x^2\binom{n}{x}p^x(1-p)^{n-x} - p^2\\
      =\\
        & p - p^2\\
      =\\
        & p\,(1-p)
    \end{longderivation}
    Así, el resultado se mantiene para todo $n\in\Z^+$
  \end{enumerate}
\end{Demo}
\subsubsection{Hipergeométrica}
Supóngase se tienen dos tipos de objetos, $a$ y $b$ en un total de
$N$ objetos exclusivamente de estos dos tipos. Sea $K$ el número de
objetos de tipo $a$ en el total de los $N$ objetos, es decir
hay $N-K$ objetos de tipo $b$. Supóngase que se toman ahora $n$
objetos del total ($N$). Se define una variable aleatoria $X$
correspondiente al número de objetos de tipo $a$ en los
$n$ objetos tomados. Entonces
\[P(X=x) = \dfrac{\binom{K}{x}\binom{N-K}{n-x}}{\binom{N}{n}}\]
\begin{Def}
  Sea $X$ una variable aleatoria discreta. $X$ sigue una distribución
  hipergeométrica con parámetros $N,K,n$ ($N,K,n\in\Z^+,K\leq N, n\leq N$),
  cuando su función de masa es
  \[
    f(x)=P(X=x)=\dfrac{\binom{K}{x}\binom{N-K}{n-x}}{\binom{N}{n}},\qquad
    \max\{0,n+K-N\} \leq x \leq \min\{K,n\}
  \]
\end{Def}

La razón de esta condición para $x$ está en que tenga sentido para lo
que se está representando. Por un lado, no tiene sentido pensar en la
probabilidad de tomar más objetos de tipo $a$ de los que hay en el total
de la muestra o tomar más objetos de tipo $a$ del total de estos.
De la misma forma, no tiene sentido tomar una cantidad negativa
de objetos tipo $a$, o tomar una cantidad de objetos tipo $a$
de forma que haya una cantidad negativa de objetos tipo $b$ para completar
los $n$ objetos.
De forma más concreta, se pueden ver las condiciones de $x$ dada la
expresión de la función de masa presentada.
\begin{longderivation}<0.8>
    & 0\leq x\leq K \quad\land\quad 0\leq n-x\leq N-K\\
  \iff\\
    & 0\leq x\leq K \quad\land\quad n+K-N\leq x\leq n\\
  \iff\\
    & \max\{0,n+K-N\}\leq x\leq\min\{K,n\}
\end{longderivation}
\clearpage