\subsection{Propiedades da valor esperado}
\begin{Teo}
    Sea $X$ una variable aleatoria real entonces:
    \begin{enumerate}
        \item Si $P(X \geq 0) = 1$ y E$[X]$ existe entonces E[$X$]$\geq 0$
        \item E[$\alpha$]$= \alpha$ para $\alpha$ constante
        \item Si existe $M \geq 0$ tal que P($|X| \leq M$)$=1$ entonces E[$X$] existe. 
        \item Si $\alpha$ y $\beta$ son constantes, y si $g$ y $h$ son funciones tales que 
              $g(X)$ y $h(X)$ son variables aleatorias cuyos valores esperados existen, 
              entonces E[$\alpha g(X) + \beta h(X)$]$= \alpha$E[$g(X)$]$+ \beta$E[$h(X)$]
        \item Si $g$ y $h$ son funciones tales que $g(X)$ y $h(X)$ son variables aleatorias
              cuyos valores esperados existen y $g(x)\leq h(x)$ para todo $x$, entonces 
              E$[g(X)$]$\leq$E[$X$]  
    \end{enumerate}
\end{Teo}