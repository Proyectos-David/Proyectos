\subsection{Propiedades de varianza}
\begin{Teo}
    Sea $X$ una variable aleatoria cuyo valor esperado existe y 
    $\alpha$, $\beta \in \R$ constantes. Entonces:

    \begin{enumerate}
        \item Var$[X] \geq 0$ 
        \item Var$[\alpha]$ = 0
        \item Var$[\alpha X]$ = $\alpha^2$Var[X]
        \item Var$[X + \beta]$ = Var$[X]$
        \item Var$[X] = 0$ si y solo si $P(X=E(X))=1$ 
    \end{enumerate}
\end{Teo}
\begin{Demo}
    La demostración se hará en base a que Var$[X] =$ 
    E$[X^2] -$$\text{E}^2[X] =$E$[(X-E(X))^2]$
    \begin{enumerate}
        \item Por propiedad de valor esperado, como $(X-E(x))^2\geq 0$, 
        entonces E$[(X-E(X))^2]\geq 0$.  
        \item $\text{Var}[\alpha] = \text{E}[\alpha^2]-\text{E}^2[\alpha]
        =\alpha^2-\alpha^2=0$
        \item Se usa la misma propiedad del item anterior.
        
        \begin{longderivation}
        &\text{Var}[\alpha X]\\
        =\\
        &\text{E}[\alpha^2X^2] - \text{E}^2[\alpha X]\\
        =\\
        &\alpha^2\text{E}[X^2] - (\alpha\text{E}[X])^2\\
        =\\
        &\alpha^2(\text{E}[X^2]-\text{E}^2[X])\\
        =\\
        &\alpha^2\text{Var}[X]
        \end{longderivation}
        
        \item Por definición de varianza y propiedades del valor esperado
        \begin{longderivation}
            &\text{Var}[X+\beta]\\
            =\\
            &\text{E}[(X+\beta)^2]-\text{E}^2[x+\beta]\\
            =\\
            &\text{E}[X^2+2\beta X + \beta^2]-(\text{E}[X]-\text{E}[\beta])^2\\
            =\\
            &\text{Var}[X] + 2\beta\text{E}[X] + \beta^2
            -2\beta\text{E}[X]-\beta^2\\
            =\\
            &\text{Var}[X]
        \end{longderivation}
        \item $P(X=\text{E}[X]) =1$ es lo mismo que decir que 
        $X=\text{E}[X]$, esto es,$X$ es constate. Por el teorema del 
        valor esperado, E$[\text{E}[X]^2]=\text{E}^2[X]$, y por lo tanto
        $\text{Var}[X]=\text{E}^2[X]-\text{E}^2[X]=0$
    \end{enumerate}
\end{Demo}