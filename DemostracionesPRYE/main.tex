\documentclass{article}

\usepackage[arrows]{logicDG}
\usepackage{calcDG}
\usepackage{analysis}
\usepackage{nicefrac}
\usepackage{amsfonts}
\usepackage{upgreek}
\usepackage{mathrsfs}
\usepackage{amsthm}
\usepackage{thmtools}
\usepackage{etoolbox}
\usepackage[hidelinks]{hyperref}
\usepackage{lmodern}
\usepackage[T1]{fontenc}
\usepackage[spanish, es-noquoting, es-lcroman, es-noshorthands]{babel}
\usepackage{fancyhdr}
\usepackage{graphicx}
\usepackage{setspace}
\usepackage{enumerate}
\usepackage{tikz}
\usepackage{ifthen}
\usetikzlibrary{positioning}
\usepackage{geometry}
\geometry{
  left=2cm,
  right=2cm,
  bottom=4cm,
  a4paper
}
\hypersetup{
  colorlinks=false,
  pdftitle={DemostracionesPRYE},
  pdfauthor={David Gómez y Laura Rincón}
}
\renewcommand{\labelenumi}{(\roman{enumi})}

\pagestyle{fancy}
\setlength{\headheight}{13pt}
\fancyhf{}
\fancyhead[R]{\textit{David G., Laura R.}}
\fancyhead[L]{\leftmark}
\fancyfoot[C]{\thepage}

\declaretheoremstyle[
  spaceabove=10pt,
  spacebelow=20pt,
  bodyfont={\normalfont},
  notefont={\normalfont},
  notebraces={(}{)},
  headpunct={:},
  headfont={\bfseries},
  ]{definition}
  
  \declaretheoremstyle[
    headpunct={:},
    headfont={\bfseries},
    bodyfont={\normalfont\leftskip2.5em},
    headindent=-2.5em,
    qed={\qedsymbol}
]{Proof}

% Entornos de definiciones, teoremas y demás %
%%%%%%%%%%%%%%%%%%%%%%%%%%%%%%%%%%%%%%%%%%%%%%
\declaretheorem[style=definition,name=Definición]{Def}
\declaretheorem[style=definition,name=Teorema]{Teo}
\declaretheorem[style=Proof, numbered=no, name=Demostración]{Demo}
\declaretheorem[style=definition, numbered=no, name=Lema]{Lema}
% Portada %
\newcommand*{\titleTMB}{\begingroup
\def\drop{0.1\textheight}
\centering
\vspace*{\baselineskip}
{\large\scshape David Gómez, Laura Rincón}\\[\baselineskip]
\rule{\textwidth}{1.6pt}\\[0pt]
\vspace*{-\baselineskip}
\vspace*{2pt}
\rule{\textwidth}{0.4pt}\\[\baselineskip]
{\LARGE DEMOSTRACIONES DE PRYE}\\[\baselineskip]
\rule{\textwidth}{0.4pt}\\[0pt]
\vspace*{-\baselineskip}\vspace{3.2pt}
\rule{\textwidth}{1.6pt}\\[\baselineskip]
\vfill
{\large\scshape Matemáticas}\\[\baselineskip]
{\small\scshape 2024}\par
\vspace*{\drop}
\endgroup}

% Comandos y demás %
%%%%%%%%%%%%%%%%%%%%%%%%%%
\everymath{\displaystyle}
\RenewDocumentCommand{\dfrac}{mm}{\frac{\displaystyle#1}{\displaystyle#2}}
\renewcommand{\epsilon}{\upvarepsilon}
\renewcommand{\iff}{\Leftrightarrow}
\doublespacing
\hyphenpenalty=1000

\begin{document}
\begin{titlepage}
  \titleTMB
\end{titlepage}
\tableofcontents
\clearpage

\section{Introduccion}

Cuando Newton y Leibniz trabajaron en la fundamentación del cálculo, una
de sus diferencias fue la definición de límite. Mientras Newton lo
definió de la forma en la que se ha enseñado principalmente, la
definición $\e$ y $\delta$. Leibniz, lo definia de una forma que
incluso parece una versión más amigable que la de Newton. Leibniz
consideraba números infinitamente pequeños, de tal forma que fueran
menores que cualquier número positivo pero mayores a $0$. Junto con
números infinitamente grandes, mayores que cualquier número positivo.

El problema de esta definición, se encontró cuando se intentó fundamentar
formalmente. Cosa que Leibniz ni sus discípulos lograron demostrar.
La definición de Newton recurre a los mismos números reales ya usados.
La definición de Leibniz, recurre a una nueva especie de números, los cuales
deben ser comparables y se deben poder operar con los reales. La idea entonces
con esta nueva especie de números, es poder operar con estos, para
posteriormente tomar el resultado y recuperar la información que interesa,
la que corresponde a un valor real estándar. Esta nueva especie de números
resulta tener aplicaciones en más áreas que el cálculo de límites, sin
embargo, no hacen parte del objetivo de este proyecto, el cual consta de
presentar esta idea en relación al análisis estándar, específicamente,
el análisis diferencial.

Los pasos para la construcción de estos números, recurre a los filtros,
un objeto de la teoría de conjuntos sobre el que se hablará en el
documento. Con estos, se puede lograr la construcción de esta nueva especie
de números sobre los reales. 

\section{Probabilidad}

Uno de los problemas no resueltos de esta primera parte del
curso fue hallar la probabilidad de una unión finita de eventos.

Esta probabilidad, sin embargo, se puede hallar mediante una
forma recursiva con la siguiente expresión
\begin{align*}
  P(A_1 \cup A_2) &= P(A_1) + P(A_2) - P(A_1 \cap A_2)\\[10pt]
  P\left(\bigcup_{i=1}^{n+1} A_i\right)
  &= P(A_{n+1}) + P\left(\bigcup_{i=1}^n A_i\right) -
    P\left(\bigcup_{i=1}^n (A_{n+1}\cap A_i)\right)
\end{align*}
Lo que se quiere es resolver esta función de recursión. Para esto,
se evaluaran unos cuantos de sus resultados.

\begin{derivation}
    & {P(A_1 \cup A_2\cup A_3)}\\
  =\\
    & {P(A_3) + P(A_1 \cup A_2) - P((A_3\cap A_1) \cup (A_3\cap A_2))}\\
  =\\
    &P(A_3) + P(A_2) + P(A_1) - P(A_1\cap A_2)\\
    &-[P(A_1\cap A_3) + P(A_2 \cap A_3) - P(A_1\cap A_2\cap A_3)]\\
  =\\
    &P(A_1) + P(A_2) + P(A_3) - P(A_1\cap A_2) - P(A_1\cap A_3)\\
    &- P(A_2\cap A_3) + P(A_1\cap A_2\cap A_3)\\
  =\\
    & {
      \sum_{i=1}^3P(A_i) - \sum_{i=1}^2\sum_{j=i+1}^3P(A_i\cap A_j)
      + P\left(\bigcap_{i=1}^3 A_i\right)
    }
\end{derivation}

Análogamente para cuatro eventos, usando lo obtenido

\begin{derivation}
    & {P(A_1\cup A_2\cup A_3\cup A_4)}\\
  =\\
    & {P(A_4) + P(A_1\cup A_2\cup A_3) - P\left(\bigcup_{i=1}^3(A_4\cap A_i)\right)}\\
  =\\
    & P(A_4) + \sum_{i=1}^3P(A_i) - \sum_{i=1}^2\sum_{j=i+1}^3P(A_i\cap A_j)
    + P\left(\bigcap_{i=1}^3 A_i\right)\\
    &-\left[\sum_{i=1}^3P(A_4\cap A_i) - \sum_{i=1}^2\sum_{j=i+1}^3P(A_4\cap A_i\cap A_j)
    + P\left(\bigcap_{i=1}^4 A_i\right)\right]\\
  =\\
    & {
      \sum_{i=1}^4P(A_i) - \sum_{i=1}^3\sum_{i=1}^4P(A_i\cap A_j) +
      \sum_{i=1}^2\sum_{j=i+1}^3\sum_{k=j+1}^4P(A_1\cap A_j\cap A_k) -
      P\left(\bigcap_{i=1}^4 A_i\right)
    }
\end{derivation}

Por último, recordar que
\[\sum_{i=a}^b\sum_{j=i+1}^{b+1} f(i,j) = \sum_{\mathclap{a\leq i < j \leq b}}f(i,j)\]
\begin{Teo}
  Sean $A_1,A_2,\dots,A_n$ eventos de un espacio muestral. Entonces,
  \[
    P\left(\bigcup_{i=1}^nA_i\right) = \smashoperator[r]{\sum_{1\leq i\leq n}}P(A_i)
    -\smashoperator[lr]{\sum_{1\leq i<j\leq n}}P(A_i\cap A_j) +
    \smashoperator[lr]{\sum_{1\leq i<j<k\leq n}}P(A_i\cap A_j\cap A_k)
    -\dots+(-1)^nP\left(\bigcap_{i=1}^n A_i\right)
  \]
  De otra forma, la probabilidad de una unión finita es la suma de
  la probabilidad de cada evento menos las posibles intersecciones
  dos a dos, sumando las probabilidades tres a tres\dots
\end{Teo}
\begin{Demo}
  Siguiendo por inducción. Los caso base $n=1$ y $n=2$ caen en la
  definición recursiva y para $n=3$ fue el desarrollo anterior.
  Para el paso inductivo, supóngase que la propiedad se mantiene hasta
  un valor $n$.
\begin{center}
  \begin{derivation}
      & {P\left(\bigcup_{i=1}^{n+1} A_i\right)}\\
    =\\
      & {
        P(A_{n+1}) + P\left(\bigcup_{i=1}^n A_i\right) -
        P\left(\bigcup_{i=1}^n (A_{n+1}\cap A_i)\right)
      }\\
    =\\
      &P(A_{n+1}) + P\left(\bigcup_{i=1}^n A_i\right)\\
      &-\left[
        \smashoperator[r]{\sum_{1\leq i\leq n}}(A_{n+1}\cap A_i)
        -\sum_{\mathclap{1\leq i<j\leq n}}(A_{n+1}\cap A_i\cap A_j) +
        \dots+(-1)^nP\left(\bigcap_{i=1}^{n+1}A_i\right)
      \right]\\
    =\\
      & 
          \smashoperator[r]{\sum_{1\leq i\leq n+1}}P(A_i)
          -\sum_{\mathclap{1\leq i < j\leq n+1}}P(A_i\cap A_j)\\
      &   +\\
      &   \smashoperator[r]{\sum_{1\leq i<j<k\leq n+1}}P(A_i\cap A_j\cap A_k)
          -\dots+(-1)^{n+1}P\left(\bigcap_{i=1}^{n+1} A_i\right)
  \end{derivation}
\end{center}
\end{Demo}

\begin{Def}
  Sean $A$ y $B$ eventos de un espacio muestral. Se dice que $A$ y $B$
  son independientes, cuando $P(A\cap B)=P(A)\,P(B)$.
\end{Def}

\begin{Teo}
  Sean $A$ y $B$ eventos independientes. Entonces
  \begin{enumerate}
    \item $A^c$ y $B^c$ son independientes.
    \item $A^c$ y $B$ son independientes.
  \end{enumerate}
\end{Teo}
\clearpage
\begin{Demo}
  \begin{enumerate}
    Supóngase $A$ y $B$ eventos independientes de un espacio muestral.
    \item Partiendo de que $A^c \cap B^c = (A\cup B)^c$.
    
    \begin{derivation}<0.9>
        & {P(A^c\cap B^c)}\\
      =\\
        & {1 - P(A\cup B)}\\
      =\\
        & {1 - [P(A) + P(B) - P(A\cap B)]}\\
      =\\
        & {1-P(A)+P(B)-P(A)P(B)}\\
      =\\
        & {(1-P(A))(1-P(B))}\\
      =\\
        & {P(A^c)\,P(B^c)}
    \end{derivation}
    
    Así, los eventos $A^c$ y $B^c$ también son independientes.
    \item Partiendo de que $B=(B\cap A)\cup(B\cap A^c)$
    
    \begin{derivation}<0.9>
        & {P(B)}\\
      =\\
        & {P((B\cap A)\cup(B\cap A^c))}\\
      =\\
        & {P(B\cap A) + P(B\cap A^c) - P(\varnothing)}\\
      =\\
        & {P(B)\,P(A) + P(B\cap A^c)}
    \end{derivation}
    
    Tomando la primera y última igualdad
    
    \begin{derivation}<0.9>
        & {P(B\cap A^c)}\\
      =\\
        & {P(B) - P(B)P(A)}\\
      =\\
        & {P(B)(1-P(A))}\\
      =\\
        & {P(B)\,P(A^c)}
    \end{derivation}
    
    Así, los eventos $A^c$ y $B$ también son independientes.
  \end{enumerate}
\end{Demo}

\clearpage
\section{Propiedades Var y E}
\begin{Demo}
    Por definición $Var(X) = E(X - \mu)^2$ donde $\mu = E(X) = \int_{-\infty}^{\infty}xf(x)dx$ 
    entonces $Var(X) = \int_{-\infty}^{\infty}(x-\mu)^2 f(x)dx$.

    Dado que $(x-\mu)^2 \geq 0$ y $f(x) \geq 0$ para toda función de densidad, entonces
    la integral siempre es mayor que cero
\end{Demo}
\clearpage

\section{Distribuciones}

En esta sección se repasarán las definiciones de algunas distribuciones
de la probabilidad.

\subsection{Distribuciones Discretas}
\subsubsection{Binomial}
\label{dist:binom}
Supóngase que se realiza un experimento el cual tiene como posible
resultado $a$ o $b$ exclusivamente, y además, el resultado
de realizar nuevamente el experimento es independiente al
resultado anterior. Dado que $a$ y $b$ son los únicos resultados,
para un único experimento, se debe tener que
$P(a) = 1 - P(b)$. Sea $p=P(a)$. Supóngase que
este experimento es realizado $n$ veces. Se define una variable
aleatoria $X$ correspondiente a la cantidad de ocurrencias de $a$.
Entonces
\[P(X=x) = \binom{n}{x}p^x(1-p)^{n-x}\]  
\begin{Def}
  Sea $X$ una variable aleatoria discreta. $X$ sigue una distribución
  binomial con parámetros $n$ y $p$ ($n\in\Z^+, p\in[0,1]$), denotada por $B(n,p)$,
  cuando su función de masa es
  \[f(x) = P(X=x) = B(n,p)(x) = \binom{n}{x}p^x(1-p)^{n-x}\]
\end{Def}

\begin{Teo}
  Sea $X$ una variable aleatoria la cual sigue una distribución
  binomial $B(n,p)$. Entonces
  \begin{enumerate}
    \item $\text{E}[X] = np$
    \item $\text{Var}[X]=np(1-p)$
  \end{enumerate}
\end{Teo}
\begin{Demo}
  Desarrollando ambos valores, y recordando el teorema del binomio\dots
  \begin{enumerate}
    \item~
    \begin{longderivation}<0.8>
        & {\text{E}[X]}\\
      =\\
        & {\sum_{x=0}^nx\binom{n}{x}p^x(1-p)^{n-x}}\\
      =\\
        & {np\sum_{x=1}^n\frac{(n-1)!}{(n-k)!\,(x-1)!}p^x(1-p)^{n-x}}\\
      =\\
        & {np\sum_{x=0}^{n-1}\binom{n-1}{x}p^x(1-p)^{n-1-x}}\\
      =\\
        & {np\,(p+1-p)^{n-1}}\\
      =\\
        & {np}
    \end{longderivation}
    \item~
    \begin{longderivation}<0.8>
        & {\text{Var}[X]}\\
      =\\
        & {E[X^2] - E^2[X]}\\
      =\\
        & {\sum_{x=0}^nx^2\binom{n}{x}p^x(1-p)^{n-x} - (np)^2}\\
      =\\
        & {np\left[
          \sum_{x=1}^nx\binom{n-1}{x-1}p^{x-1}(1-p)^{n-x} - np
        \right]}\\
      =\\
        & {np\left[
          \sum_{x=0}^{n-1}(x+1)\binom{n-1}{x}p^x(1-p)^{n-1-x} - np
        \right]}\\
      =\\
        & {np\left[
          \sum_{x=0}^{n-1}x\binom{n-1}{x}p^x(1-p)^{n-1-x}
          +\sum_{x=0}^{n-1}\binom{n-1}{x}p^x(1-p)^{n-1-x}
          -np
        \right]}\\
      =\\
        & {np\left[
          p(n-1)\sum_{x=1}^{n-1}\binom{n-2}{x-1}p^{x-1}(1-p)^{n-1-x}+1-np
        \right]}\\
      =\\
        & {np\left[
          p(n-1)\sum_{x=0}^{n-2}\binom{n-2}{x}p^x(1-p)^{n-2-x}+1-np
        \right]}\\
      =\\
        & {np[p(n-1)+1-np]}\\
      =\\
        & {np(1-p)}
    \end{longderivation}
    Este argumento es válido siempre que $n\geq 2$. Si $n<2$, entonces $n=1$ y
    \begin{longderivation}<0.8>
        & \text{Var}[X]\\
      =\\
        & E[X^2] - E^2[X]\\
      =\\
        & \sum_{x=0}^1x^2\binom{1}{x}p^x(1-p)^{1-x} - p^2\\
      =\\
        & p - p^2\\
      =\\
        & p\,(1-p)
    \end{longderivation}
    Así, el resultado se mantiene para todo $n\in\Z^+$
  \end{enumerate}
\end{Demo}
\subsubsection{Hipergeométrica}
\label{dist:hip}
Supóngase que se tienen dos tipos de objetos, $a$ y $b$ en un total de
$N$ objetos exclusivamente de estos dos tipos. Sea $K$ el número de
objetos de tipo $a$ en el total de los $N$ objetos, es decir
hay $N-K$ objetos de tipo $b$. Supóngase que se toman ahora $n$
objetos del total ($N$). Se define una variable aleatoria $X$
correspondiente al número de objetos de tipo $a$ en los
$n$ objetos tomados. Entonces
\[P(X=x) = \dfrac{\binom{K}{x}\binom{N-K}{n-x}}{\binom{N}{n}}\]
\begin{Def}
  Sea $X$ una variable aleatoria discreta. $X$ sigue una distribución
  hipergeométrica con parámetros $N,K,n$ ($N,K,n\in\Z^+,K\leq N, n\leq N$),
  denotada por $Hg(N,K,n)$, cuando su función de masa es
  \[
    f(x)=P(X=x)=\dfrac{\binom{K}{x}\binom{N-K}{n-x}}{\binom{N}{n}},\qquad
    \max\{0,n+K-N\} \leq x \leq \min\{K,n\}
  \]
\end{Def}

La razón de esta condición para $x$ está en que tenga sentido para lo
que se está representando. Por un lado, no tiene sentido pensar en la
probabilidad de tomar más objetos de tipo $a$ de los que hay en el total
de la muestra o tomar más objetos de tipo $a$ del total de estos.
De la misma forma, no tiene sentido tomar una cantidad negativa
de objetos tipo $a$, o tomar una cantidad de objetos tipo $a$
de forma que haya una cantidad negativa de objetos tipo $b$ para completar
los $n$ objetos o más objetos de tipo $b$ de los que hay en total.
De forma más concreta, se pueden ver las condiciones de $x$ dada la
expresión de la función de masa presentada.
\begin{longderivation}<0.8>
    & 0\leq x\leq K \quad\land\quad 0\leq n-x\leq N-K\\
  \iff\\
    & 0\leq x\leq K \quad\land\quad n+K-N\leq x\leq n\\
  \iff\\
    & \max\{0,n+K-N\}\leq x\leq\min\{K,n\}
\end{longderivation}
Sin embargo, tomando la convención de que $\binom{n}{k} = 0$ cuando $k>n$,
se puede tomar a $x$ entre $0$ y $n$.

Para demostrar la validez de esta función de masa, hace falta un resultado
sobre la combinatoria.
\begin{Lema}[Identidad de Vandermonde]
  Sean $m,n,k\in\Z$ no negativos. Entonces
  \[\binom{m+n}{k} = \sum_{r=0}^k\binom{m}{r}\binom{n}{k-r}\]
  Esta identidad tiene sentido tomando la convención mencionada anteriormente.
\end{Lema}
\begin{Demo}
  La demostración se hará por inducción sobre $m$, tomando $k,n$ como
  enteros no negativos arbitrarios. Caso base ($m=0$):
  \[\binom{0+n}{k} = \sum_{r=0}^k\binom{0}{r}\binom{n}{k-r}=\binom{n}{k}\]
  Paso inductivo: sea $k\in\Z$ con $k\geq1$ y supóngase que la propiedad se mantiene
  para todo entero no negativo hasta $k$. (Para $k=0$ la propiedad es trivial)
  \begin{longderivation}
      & \binom{m+n}{k+1}\\
    =\\
      & \binom{m+n}{k} + \binom{m+n}{k-1}\\
    =\\
      & \sum_{r=0}^k\binom{m}{r}\binom{n}{k-r} +
      \sum_{r=0}^{k-1}\binom{m}{r}\binom{n}{k-1-r}\\
    =\\
      & \binom{m}{k} + 
      \sum_{r=0}^{k-1}\binom{m}{r}\left(\binom{n}{k-r} + \binom{n}{k-1-r}\right)\\
    =\\
      & \binom{m}{n} + \sum_{k=0}^{k-1}\binom{m}{r}\binom{n}{k-r}\\
    =\\
      & \sum_{k=0}^k \binom{m}{k}\binom{n}{k-r}
  \end{longderivation}
\end{Demo}

\begin{Teo}
  Sea $X\sim Hg(N,K,n)$. Entonces,
  \begin{enumerate}
    \item para todo $x\in\Z$ con $0\leq x\leq n$, $P(X=x) \geq 0$.
    \item $\sum_{x=0}^nP(X=x) = 1$.
    \item $\text{E}[X]=\frac{nK}{N}$.
    \item $\text{Var}[X] = \frac{n\,K(N-K)(N-n)}{N^2(N-1)}$.
  \end{enumerate}
\end{Teo}
\begin{Demo}~
  \begin{enumerate}
    \item Sea $x\in\Z$ con $0\leq x\leq n$. Recordando que
    \[P(X=x) = \dfrac{\binom{K}{x}\binom{N-K}{n-x}}{\binom{N}{n}}\]
    dado que todos los términos de la expresión son no negativos, se
    concluye que $P(X=x)\geq0$
    \item ~
    \begin{longderivation}
        & \sum_{x=0}^nP(X=i)\\
      =\\
        & \sum_{x=0}^n\dfrac{\binom{K}{x}\binom{N-K}{n-x}}{\binom{N}{n}}\\
      =\\
        & \dfrac{1}{\binom{N}{n}}\sum_{x=0}^n\binom{K}{x}\binom{N-K}{n-x}\\
      =\\
        & \dfrac{\binom{N}{n}}{\binom{N}{n}}\\
      =\\
        & 1
    \end{longderivation}
    \item~
    \begin{longderivation}
        & \text{E}[X]\\
      =\\
        & \sum_{x=0}^nx\dfrac{\binom{K}{x}\binom{N-K}{n-x}}{\binom{N}{n}}\\
      =\\
        & \dfrac{1}{\binom{N}{n}}\sum_{x=0}^nx\binom{K}{x}\binom{N-K}{n-x}\\
      =\\
        & \dfrac{K}{\binom{N}{n}}\sum_{x=1}^n\binom{K-1}{x-1}\binom{N-K}{n-x}\\
      =\\
        & \dfrac{K}{\binom{N}{n}}\sum_{x=0}^{n-1}\binom{K-1}{x}\binom{N-K}{n-x-1}\\
      =\\
        & \dfrac{K}{\binom{N}{n}}\binom{N-1}{n-1}\\
      =\\
        & \frac{K\,n}{N}
    \end{longderivation}
    \item~
    \begin{longderivation}
        & \text{Var}[X]\\
      =\\
        & \text{E}[X^2] - \text{E}[X] + \text{E}[X] - \text{E}^2[X]\\
      =\\
        & \dfrac{1}{\binom{N}{n}}\sum_{x=0}^nx^2\binom{K}{x}\binom{N-K}{n-x}
        - \dfrac{1}{\binom{N}{n}}\sum_{x=0}^nx\binom{K}{x}\binom{N-K}{n-x}
        + \frac{n\,K}{N} - \frac{n^2\,K^2}{N^2}\\
      =\\
        & \dfrac{K}{\binom{N}{n}}\left[
          \sum_{x=0}^{n-1}x\binom{K-1}{x-1}\binom{N-K}{n-x-1} -
          \sum_{x=0}^{n-1}\binom{K-1}{x-1}\binom{N-K}{n-x-1}
        \right]
        + \frac{n\,K}{N} - \frac{n^2\,K^2}{N^2}\\
      =\\
        & \dfrac{K(K-1)}{\binom{N}{n}}
          \sum_{x=0}^{n-2}\binom{K-2}{x}\binom{N-K}{n-x-2}
        + \frac{n\,K}{N} - \frac{n^2\,K^2}{N^2}\\
      =\\
        & \dfrac{K(K-1)}{\binom{N}{n}}\binom{N-2}{n-2}
        + \frac{n\,K}{N} - \frac{n^2\,K^2}{N^2}\\
      =\\
        & \frac{K(K-1)n(n-1)}{N(N-1)}
        + \frac{n\,K}{N} - \frac{n^2\,K^2}{N^2}\\
      =\\
        & \frac{n\,K(N-K)(N-n)}{N^2(N-1)}
    \end{longderivation}
  \end{enumerate}
\end{Demo}
\clearpage
\subsection{Teoremas de Aproximación}
Se puede ver una similitud entre la \hyperref[dist:binom]{distribución binomial}
y la \hyperref[dist:hip]{distribución hipergeométrica}, pues si en esta última, manteniendo
un tamaño de muestra ($n$) fijo, a medida que aumentan el total de objetos ($N$ y $K$) bajo
ciertas condiciones, los eventos que esta distribución describe tienen a ser independientes.
Esto lleva al siguiente teorema de aproximación.

\begin{Teo}
  Sea $X$ una variable aleatoria con distribución hipergeométrica de
  parámetros $N,K,n$. Si para un $\epsilon>0$ , $n > 1$ , $x > 0$,
  se tiene que
  \begin{align*}
    \frac{x-1}{K}     &< \epsilon\\
    \frac{n-x-1}{N-K} &< \epsilon\\
    \frac{n-1}{N - n + 1}     &< \epsilon
  \end{align*}
  entonces,
  \[\left|P(X=x) - B\left(n,\frac{K}{N}\right)(x)\right| < (\epsilon + 1)^{2n} - 1\]
\end{Teo}

Antes de comenzar con la demostración de este teorema, se presenta el siguiente lema, el
cual será de utilidad para obtener el resultado presentado.
\begin{Lema}
  Sean $r\in\Z^+$ y $\left\{S_{k,n}\right\}_{1}^{r}$ una colección de $r$ sucesiones
  en función de $n$ las cuales convergen a $1$. Dado $\epsilon>0$,
  existe un $N\in\N$ tal que para todo $1\leq k\leq r$
  \[n\geq N \To |S_{k,n} - 1| < \epsilon\]
  entonces,
  \[n\geq N \To \left|\prod_{k=1}^r S_{k,n} -1\right| < (\epsilon + 1)^r - 1\]
\end{Lema}
\begin{Demo}
  La existencia de este $N$ consta de tomar el máximo entre los diferentes $N$'s
  dados por la convergencia de cada sucesión en la colección. La demostración del
  acotamiento se hará por inducción.

  Caso base ($r=1$): se tiene que $|S_{1,n}-1| < \epsilon$, con lo que se cumple
  la propiedad para una sucesión.

  para $r=2$

  \begin{longderivation}
      & \left|S_{1,n}\,S_{2,n} - 1\right|\\
    =\\
      & \left|(S_{1,n} - 1)(S_{2,n} - 1) + S_{1,n} - 1 + S_{2,n} - 1\right|\\
    \leq\\
      & \left|S_{1,n}-1\right||S_{2,n}-1| + |S_{1,n}-1| + |S_{2,n}-1|\\
    <\\
      & \epsilon^2 + 2\epsilon\\
    =\\
      & (\epsilon + 1)^2 - 1
  \end{longderivation}

  Paso inductivo: supóngase que para el producto de $r$ sucesiones, la propiedad se mantiene.
  Para $r+1$ sucesiones

  \begin{longderivation}
      & \left|\prod_{k=1}^{r+1}S_{k,n} - 1\right|\\
    =\\
      & \left|S_{r+1,n}\prod_{k=1}^{r}S_{k,n} - 1\right|\\
    =\\
      & \left|
        (S_{r+1,n} - 1)\left(\prod_{k=1}^{r}S_{k,n} - 1\right)
        + (S_{r+1,n} - 1)
        + \left(\prod_{k=1}^{r}S_{k,n} - 1\right)
      \right|\\
    \leq\\
      &\left|(S_{r+1,n} - 1)\left(\prod_{k=1}^{r}S_{k,n} - 1\right)\right|
      + \left|(S_{r+1,n} - 1)\right|
      + \left|\left(\prod_{k=1}^{r}S_{k,n} - 1\right)\right|\\
    <\\
      &\epsilon\left|\left(\prod_{k=1}^{r}S_{k,n} - 1\right)\right|
      + \epsilon + \left|\left(\prod_{k=1}^{r}S_{k,n} - 1\right)\right|\\
    =\\
      &\left|\left(\prod_{k=1}^{r}S_{k,n} - 1\right)\right|(\epsilon + 1) + \epsilon\\
    <\\
      & [(\epsilon + 1)^r - 1](\epsilon + 1) + \epsilon\\
    =\\
      &(\epsilon + 1)^{r+1} - 1
  \end{longderivation}

  Con esto ya estaría demostrado. Sin embargo, para entender de dónde
  surgió originalmente el lema se presenta esta demostración alternativa:
  Se define $f$ de la siguiente manera: $f(1) = \epsilon$ y
  $f(r+1) = f(r)(\epsilon + 1) + \epsilon$. Nótese que, por el caso base y lo
  desarrollado, $f$ cumple que
  $\left|\prod_{k=1}^r S_{k,r}-1\right| < f(r)$.
  Para resolver la ecuación de recurrencia, se define $g$ como
  $g(r) = \frac{f(r)}{(\epsilon + 1)^r}$, es decir,
  $g(1) = \frac{\epsilon}{\epsilon + 1}$ y $g(r+1) = \frac{f(r+1)}{(\epsilon + 1)^{r+1}}$.
  Desarrollando la última expresión, se obtiene
  \[
    g(r+1) = \frac{f(r)}{(\epsilon + 1)^r} + \frac{\epsilon}{(\epsilon+1)^{r+1}}
    = g(r) + \frac{\epsilon}{(\epsilon+1)^{r+1}}
  \]
  con lo que, por definición,
  \begin{longderivation}
      & g(r)\\
    =\\
      & \sum_{k=1}^{r}\frac{\epsilon}{(\epsilon + 1)^k}\\
    =\\
      & \frac{\epsilon}{\epsilon + 1}\sum_{k=0}^{r-1}\frac{1}{(\epsilon+1)^r}\\
    =\\
      & \frac{\epsilon}{\epsilon+1}
      \left(\frac{1 - \dfrac{1}{(\epsilon + 1)^k}}{1 - \dfrac{1}{\epsilon+1}}\right)\\
    =\\
      & \frac{\epsilon}{\epsilon+1}
      \left(\frac{(\epsilon + 1)^r - 1}{(\epsilon+1)^r}\frac{\epsilon+1}{\epsilon}\right)\\
    =\\
      & \frac{(\epsilon + 1)^r - 1}{(\epsilon+1)^r}
  \end{longderivation}

  Así, $f(r) = (\epsilon+1)^r\,g(r) = (\epsilon+1)^r - 1$.
\end{Demo}
Siguiendo ahora con el teorema\dots
\begin{Demo}
  Inicialmente, se expresará la función de masa de $X$ en términos más prácticos para
  esta demostración:
  \begin{longderivation}
      & \dfrac{\binom{K}{x}\binom{N-K}{n-x}}{\binom{N}{n}}\\
    =\\
      & \frac{K!}{(K-x)!\,x!}\,\frac{(N-K)!}{(N-K-n+x)!\,(n-x)!}\,\frac{(N-n)!\,n!}{N!}\\
    =\\
      & \binom{n}{x}\dfrac{\prod_{i=1}^{K}i}{\prod_{i=1}^{K-x}i}
      \dfrac{\prod_{j=1}^{N-K}j}{\prod_{j=1}^{N-K-n+x}\mspace{-20mu}j\mspace{20mu}}
      \dfrac{\prod_{s=1}^{N-n}s}{\prod_{s=1}^{N}s}\\
    =\\
      & \binom{n}{x}\smashoperator[r]{\prod_{i=K-x+1}^{K}}i\hspace{20pt}
      \smashoperator[r]{\prod_{j=N-K-n+x+1}^{N-K}}j\hspace{30pt}
      \dfrac{1}{\prod_{s=N-n+1}^{N}\mspace{-25mu}s\mspace{25mu}}\\
    =\\
      & \binom{n}{x}\prod_{i=0}^{x-1}(K-i)
      \prod_{j=0}^{n-x-1}\mspace{-9mu}(N-K-j)\mspace{9mu}
      \dfrac{1}{\prod_{s=0}^{n-1}(N-s)}\\
    =\\
      &\binom{n}{x}\left(\frac{K}{N}\right)^x\left(\frac{N-K}{N}\right)^{n-x}
      \dfrac{\prod_{i=0}^{x-1}(K-i)}{K^x}
      \dfrac{\prod_{j=0}^{n-x-1}\mspace{-9mu}(N-K-j)\mspace{9mu}}{(N-K)^{n-x}}
      \dfrac{N^n}{\prod_{s=0}^{n-1}(N-s)}\\
    =\\
      & B\left(n,\frac{K}{N}\right)(x)\,\prod_{i=0}^{x-1}\left(1-\frac{i}{K}\right)
      \prod_{j=0}^{n-x-1}\left(1 - \frac{j}{N-K}\right)
      \prod_{s=0}^{n-1}\left(1 + \frac{s}{N-s}\right)
  \end{longderivation}

  En este proceso no se toma en cuenta el caso en el que $x=0$ o $x=n$. Estos casos se
  resolverán posterior a tratar con la última expresión.

  Tomando en cuenta este resultado,
  \begin{longderivation}<1.5>
      & \left|\dfrac{\binom{K}{x}\binom{N-K}{n-x}}{\binom{N}{n}} -
      B\left(n,\frac{K}{N}\right)(x)\right|\\
    =\\
      & B\left(n,\frac{K}{N}\right)(x)
      \left|\prod_{i=0}^{x-1}\left(1-\frac{i}{K}\right)
      \prod_{j=0}^{n-x-1}\left(1 - \frac{j}{N-K}\right)
      \prod_{s=0}^{n-1}\left(1 + \frac{s}{N-s}\right) - 1\right|\\
    \why[\leq]{$B(n,p)(x)$ es una probabilidad, con lo que es siempre menor o igual a $1$}\\
    &\left|\prod_{i=0}^{x-1}\left(1-\frac{i}{K}\right)
    \prod_{j=0}^{n-x-1}\left(1 - \frac{j}{N-K}\right)
    \prod_{s=0}^{n-1}\left(1 + \frac{s}{N-s}\right) - 1\right|
  \end{longderivation}

  Nótese que cada término en cada productorio tiende a $1$ cuando $K,N,N-K$
  tienden a infinito. Con esto basta para demostrar la convergencia.

  En los productorios, se ven involucradas sucesiones las cuales convergen
  a $0$ y además, son sencillas de acotar. Entonces, como
  \begin{align*}
    \left|1 - \frac{i}{K} - 1\right| &= \frac{i}{K} \leq \frac{x-1}{K}\\[10pt]
    \left|1 - \frac{j}{N-K} - 1\right| &= \frac{j}{N-K} \leq \frac{n-x-1}{N-K}\\[10pt]
    \left|1 + \frac{s}{N-s} - 1\right| &= \frac{s}{N-s} \leq \frac{n-1}{N-n+1}
  \end{align*}
  solo hace falta hallar una cota para los miembros derechos de las desigualdades para
  acotar todos los términos de los productorios.

  Sea $\epsilon>0$ tal que
  \begin{align*}
    \frac{x-1}{K}     &< \epsilon\\
    \frac{n-x-1}{N-K} &< \epsilon\\
    \frac{n-1}{N-n+1} &< \epsilon
  \end{align*}
  Entonces, por el lema,
  \begin{align*}
    \left|\prod_{i=0}^{x-1}\left(1-\frac{i}{K}\right) - 1\right| 
    &< (\epsilon + 1)^x - 1\\[10pt]
    \left|\prod_{j=0}^{n-x-1}\left(1 - \frac{j}{N-K}\right) - 1\right|
    &< (\epsilon + 1)^{n-x} - 1\\[10pt]
    \left|\prod_{s=0}^{n-1}\left(1 + \frac{s}{N-s}\right) - 1\right|
    &< (\epsilon + 1)^n - 1
  \end{align*}

  Denotando cada uno de estos productos como $P_1$, $P_2$ y $P_3$ respectivamente,

  \begin{longderivation}
      & |P_1\,P_2\,P_3 - 1|\\
    =\\
      & |(P_1 - 1)(P_2\,P_3 - 1) + P_1 - 1 + P_2\,P_3 - 1|\\
    \leq\\
      & |P_1 - 1||P_2\,P_3 - 1| + |P_1 - 1| + |P_2\,P_3 - 1|\\
    =\\
      & |P_2\,P_3 - 1|(|P_1 - 1| + 1) + |P_1 - 1|\\
    =\\
      & |(P_2 - 1)(P_3 - 1) + P_2 - 1 + P_3 - 1|
      (|P_1 - 1| + 1) + |P_1 - 1|\\
    \leq\\
      &(|P_2 - 1||P_3 - 1| + |P_2 - 1| + |P_3 - 1|)(|P_1 - 1| + 1) + |P_1 - 1|\\
    <\\
      &[
        ((\epsilon + 1)^{n-x} - 1)((\epsilon + 1)^n - 1) +
        (\epsilon + 1)^{n-x} - 1 + (\epsilon + 1)^n - 1
      ]((\epsilon + 1)^x - 1 + 1) + (\epsilon + 1)^x - 1\\
    =\\
      & [
        (\epsilon + 1)^{2n-x} - (\epsilon + 1)^{n-x} - (\epsilon + 1)^n + 1
        + (\epsilon + 1)^{n-x} - 1 + (\epsilon + 1)^n - 1
      ](\epsilon + 1)^x + (\epsilon + 1)^x - 1\\
    =\\
      & [(\epsilon + 1)^{2n-x} - 1](\epsilon + 1)^x + (\epsilon + 1)^x - 1\\
    =\\
      & (\epsilon + 1)^{2n} - 1
  \end{longderivation}
  
  Recordando que todo lo anterior se hizo bajo la suposición de que $x > 0$ y
  $ x \not= n$. Para $x = 0$
  \begin{longderivation}
      &\left|
        \dfrac{\binom{K}{x}\binom{N-K}{n-x}}{\binom{N}{n}}
        - \binom{n}{x}\left(\frac{K}{N}\right)^x\left(\frac{N-K}{N}\right)^{n-x}
      \right|\\
    =\\
      &\left|
        \dfrac{\binom{N-K}{n}}{\binom{N}{n}}
        - \left(\frac{N-K}{N}\right)^n
      \right|\\
    =\\
      & B\left(n,\frac{K}{N}\right)(0)\left|
        \prod_{j=0}^{n-1}\left(1 - \frac{j}{N-K}\right)
        \prod_{s=0}^{n-1}\left(1 + \frac{s}{N-s}\right) - 1
      \right|\\
    \leq\\
      & \left|\prod_{j=0}^{n-1}\left(1 - \frac{j}{N-K}\right) - 1\right|
      \left|\prod_{s=0}^{n-1}\left(1 + \frac{s}{N-s}\right) - 1\right| + 
      \left|\prod_{j=0}^{n-1}\left(1 - \frac{j}{N-K}\right) - 1\right| +
      \left|\prod_{s=0}^{n-1}\left(1 + \frac{s}{N-s}\right) - 1\right|\\
    <\\
      & [(\epsilon + 1)^n - 1](\epsilon + 1)^n + (\epsilon + 1)^n - 1\\
    =\\
      & (\epsilon + 1)^{2n} - 1
  \end{longderivation}
  Para $x=n$
  \begin{longderivation}
      &\left|
          \dfrac{\binom{K}{x}\binom{N-K}{n-x}}{\binom{N}{n}}
          - \binom{n}{x}\left(\frac{K}{N}\right)^x\left(\frac{N-K}{N}\right)^{n-x}
        \right|\\
    =\\
      &\left|
        \dfrac{\binom{K}{n}}{\binom{N}{n}} - \left(\frac{K}{N}\right)^n
      \right|\\
    =\\
      & B\left(n,\frac{K}{N}\right)(n)
      \left|
        \prod_{i=0}^{n-1}\left(1 - \frac{i}{K}\right)
        \prod_{s=0}^{n-1}\left(1 + \frac{s}{N-s}\right) - 1
      \right|\\
    \leq\\
        & \left|\prod_{i=0}^{n-1}\left(1 - \frac{i}{K}\right) - 1\right|
        \left|\prod_{s=0}^{n-1}\left(1 + \frac{s}{N-s}\right) - 1\right| +
        \left|\prod_{i=0}^{n-1}\left(1 - \frac{i}{K}\right) - 1\right| +
        \left|\prod_{s=0}^{n-1}\left(1 + \frac{s}{N-s}\right) - 1\right|\\
    <\\
      & [(\epsilon + 1)^n - 1](\epsilon + 1)^n + (\epsilon + 1)^n - 1\\
    =\\
      & (\epsilon + 1)^{2n} - 1
  \end{longderivation}

  Para $n=1$ o $n=0$, la diferencia presentada es nula.
\end{Demo}

\end{document}